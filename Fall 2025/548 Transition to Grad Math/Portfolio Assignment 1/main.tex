\documentclass[10pt,twoside,openany]{memoir}
%\usepackage{mlmodern}
%\usepackage{tgpagella} % text only
%\usepackage{mathpazo}  % math & text
\usepackage[T1]{fontenc}
\usepackage[hidelinks]{hyperref}
\usepackage{amsmath}
\usepackage[fixamsmath]{mathtools}  % Extension to amsmath
\usepackage{amsthm}
\usepackage{amssymb}

\usepackage{newpxtext}
\usepackage{eulerpx}
\usepackage{eucal}
\usepackage{datetime}
    \newdateformat{specialdate}{\THEYEAR\ \monthname\ \THEDAY}
\usepackage[margin=1.5in]{geometry}
\usepackage{fancyhdr}
    \pagestyle{fancy}
    \renewcommand{\headrulewidth}{0pt}
    \cfoot{\scriptsize \thepage}
\usepackage{thmtools}
    \declaretheoremstyle[
        spaceabove=10pt,
        spacebelow=10pt,
        headfont=\normalfont\bfseries,
        notefont=\mdseries, notebraces={(}{)},
        bodyfont=\normalfont,
        postheadspace=0.5em
        %qed=\qedsymbol
        ]{defs}

    \declaretheoremstyle[ 
        spaceabove=10pt, % space above the theorem
        spacebelow=10pt,
        headfont=\normalfont\bfseries,
        bodyfont=\normalfont\itshape,
        postheadspace=0.5em
        ]{thmstyle}
    
    \declaretheorem[
        style=thmstyle,
        numberwithin=section
    ]{theorem}

    \declaretheorem[
        style=thmstyle,
        sibling=theorem,
    ]{proposition}

    \declaretheorem[
        style=thmstyle,
        sibling=theorem,
    ]{lemma}

    \declaretheorem[
        style=thmstyle,
        sibling=theorem,
    ]{corollary}

    \declaretheorem[
        numberwithin=section,
        style=defs,
    ]{example}

    \declaretheorem[
        numberwithin=section,
        style=defs,
    ]{definition}

    \declaretheorem[
        style=defs,
        sibling=theorem,
        numberwithin=section,
    ]{exercise}

    \declaretheorem[
        style=thmstyle
        numbered=unless unique
    ]{axiom}

    \declaretheorem[numbered=unless unique, style=defs]{problem}
    \declaretheorem[numberwithin=section,style=defs]{note}
    \declaretheorem[numbered=no,style=defs]{question}
    \declaretheorem[numbered=no,style=defs]{recall}
    \declaretheorem[numbered=no,style=remark]{answer}
    \declaretheorem[numbered=no,style=remark]{solution}
    \declaretheorem[numbered=no,style=defs]{remark}
\usepackage{enumitem}
\usepackage[utf8x]{inputenc}
\usepackage{tikz}
\usepackage{tikz-cd}
\usetikzlibrary{patterns}
\usetikzlibrary{positioning,arrows.meta}
\linespread{1.00}
%%%%%%%%%%%%%%%%%%%%%%%%%%%%%%%%%%%%%%%%%%%%%%%%%%%%%%%%%%%%%
%%%%%%%%%%%%%%%%%%%%%%%%%%%%%%%%%%%%%%%%%%%%%%%%%%%%%%%%%%%%%
\input{/Users/gcrescenzo/Documents/School/LaTeX Documents/latex-class-notes/makros.tex}

\begin{document}
\begin{center}
    {\Large Math 548 \\[0.1in]Portfolio Assignment 1}\\[.175in]
    {Name:} {\underline{Gianluca Crescenzo\hspace*{2in}}}\\[0.15in]
    \end{center}
    \vspace{4pt}
%%%%%%%%%%%%%%%%%%%%%%%%%%%%%%%%%%%%%%%%%%%%%%%%%%
\begin{problem}
Let $T:\bfR^4 \rightarrow \bfR^4$ be an orthogonal projection to the 2-dimensional plane $P$ spanned by the vectors $\vec{v} = (2,0,1,0)$ and $\vec{w} = (-1,0,2,0)$.
    \begin{enumerate}[label = (\arabic*),itemsep=1pt,topsep=3pt]
        \item Find (with proof) all the eigenvalues and eigenvectors, along with their geometric and algebraic multiplicities.
        \item Find the matrix $T$ with respect to the standard basis. Is this matrix diagonalizable? Why or why not?
    \end{enumerate}
\end{problem}
    \begin{proof}
        (1) Since $T$ is a projection onto $P$, note that for any vector $v \in \bfR^4$, we have that $T(v) \in P$. Since $T$ acts as the identity map on any vector in $P$, it must be the case that $T^2(v) = T(v)$. 

        Let $w$ be an eigenvector of $T$. Since $T^2(v) = T(v)$ for all $v \in \bfR^4$, we get:
            \begin{equation*}
            \begin{split}
                T(w) &= \lambda w, \\
                T(w) & = T^2(w)  = \lambda^2 w.
            \end{split}
            \end{equation*}
        It must be the case that $\lambda^2 = \lambda$; i.e., $\lambda = 0$ or $\lambda = 1$. 
        
    \end{proof}
%%%%%%%%%%%%%%%%%%%%%%%%%%%%%%%%%%%%%%%%%%%%%%%%%%
\begin{problem}

\end{problem}
%%%%%%%%%%%%%%%%%%%%%%%%%%%%%%%%%%%%%%%%%%%%%%%%%%
\end{document}