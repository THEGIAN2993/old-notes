\documentclass[10pt,twoside,openany]{memoir}
%\usepackage{mlmodern}
%\usepackage{tgpagella} % text only
%\usepackage{mathpazo}  % math & text
\usepackage[T1]{fontenc}
\usepackage[hidelinks]{hyperref}
\usepackage{amsmath}
\usepackage[fixamsmath]{mathtools}  % Extension to amsmath
\usepackage{amsthm}
\usepackage{amssymb}

\usepackage{newpxtext}
\usepackage{eulerpx}
\usepackage{eucal}
\usepackage{datetime}
    \newdateformat{specialdate}{\THEYEAR\ \monthname\ \THEDAY}
\usepackage[margin=1.5in]{geometry}
\usepackage{fancyhdr}
    \pagestyle{fancy}
    \renewcommand{\headrulewidth}{0pt}
    \cfoot{\scriptsize \thepage}
\usepackage{thmtools}
    \declaretheoremstyle[
        spaceabove=10pt,
        spacebelow=10pt,
        headfont=\normalfont\bfseries,
        notefont=\mdseries, notebraces={(}{)},
        bodyfont=\normalfont,
        postheadspace=0.5em
        %qed=\qedsymbol
        ]{defs}

    \declaretheoremstyle[ 
        spaceabove=10pt, % space above the theorem
        spacebelow=10pt,
        headfont=\normalfont\bfseries,
        bodyfont=\normalfont\itshape,
        postheadspace=0.5em
        ]{thmstyle}
    
    \declaretheorem[
        style=thmstyle,
        numberwithin=section
    ]{theorem}

    \declaretheorem[
        style=thmstyle,
        sibling=theorem,
    ]{proposition}

    \declaretheorem[
        style=thmstyle,
        sibling=theorem,
    ]{lemma}

    \declaretheorem[
        style=thmstyle,
        sibling=theorem,
    ]{corollary}

    \declaretheorem[
        numberwithin=section,
        style=defs,
    ]{example}

    \declaretheorem[
        numberwithin=section,
        style=defs,
    ]{definition}

    \declaretheorem[
        style=thmstyle
        numbered=unless unique
    ]{axiom}

    \declaretheorem[numbered=unless unique, style=defs]{exercise}
    \declaretheorem[numberwithin=section,style=defs]{note}
    \declaretheorem[numbered=no,style=defs]{question}
    \declaretheorem[numbered=no,style=defs]{recall}
    \declaretheorem[numbered=no,style=remark]{answer}
    \declaretheorem[numbered=no,style=remark]{solution}
    \declaretheorem[numbered=no,style=defs]{remark}
\usepackage{enumitem}
\usepackage[utf8x]{inputenc}
\usepackage{tikz}
\usepackage{tikz-cd}
\usetikzlibrary{patterns}
\usetikzlibrary{positioning,arrows.meta}
\linespread{1.00}
%%%%%%%%%%%%%%%%%%%%%%%%%%%%%%%%%%%%%%%%%%%%%%%%%%%%%%%%%%%%%
%%%%%%%%%%%%%%%%%%%%%%%%%%%%%%%%%%%%%%%%%%%%%%%%%%%%%%%%%%%%%
\input{/Users/gcrescenzo/Documents/School/LaTeX Documents/latex-class-notes/makros.tex}

\begin{document}
\begin{center}
    {\Large Math 550 \\[0.1in]Homework 1}\\[.175in]
    {Name:} {\underline{Gianluca Crescenzo\hspace*{2in}}}\\[0.15in]
    \end{center}
    \vspace{4pt}
%%%%%%%%%%%%%%%%%%%%%%%%%%%%%%%%%%%%%%%%%%%%%%%%%%
\begin{exercise}
    Show that $\cB_\bfR$ is generated by each of the following:
        \begin{enumerate}[label = \alph*.,itemsep=1pt,topsep=3pt]
            \item the open intervals: $\cE_1 = \{(a,b) \mid a < b\}$,
            \item the closed intervals: $\cE_2 = \{[a,b] \mid a < b\}$,
            \item the half-open intervals: $\cE_3 = \{(a,b] \mid a < b\}$ or $\cE_4 = \{[a,b) \mid a < b\}$,
            \item the open rays: $\cE_5 = \{(a,\infty) \mid a \in \bfR\}$ or $\cE_6 = \{(-\infty,a) \mid a \in \bfR\}$,
            \item the closed rays: $\cE_7 = \{(a,\infty) \mid a \in \bfR\}$ or $\cE_8 = \{(-\infty,a) \mid a \in \bfR\}$.
        \end{enumerate}
\end{exercise}
    {\color{blue} \begin{proof}
    
    \end{proof}}
%%%%%%%%%%%%%%%%%%%%%%%%%%%%%%%%%%%%%%%%%%%%%%%%%%
\begin{exercise}
    Let $\cM$ be an infinite $\sigma$-algebra.
        \begin{enumerate}[label = \alph*.,itemsep=1pt,topsep=3pt]
            \item $\cM$ contains an infinite sequence of disjoint sets.
            \item $\card(\cM) \geq \fc$.
        \end{enumerate}
\end{exercise}

%%%%%%%%%%%%%%%%%%%%%%%%%%%%%%%%%%%%%%%%%%%%%%%%%%
\begin{exercise}
    An algebra $\cA$ is a $\sigma$-algebra if and only if $\cA$ is closed under countable increasing unions.
\end{exercise}
    {\color{blue} \begin{proof}
        If $\cA$ is a $\sigma$-algebra, then $\cA$ is closed under countable unions, hence it is closed under countable \textit{increasing} unions. Conversely, suppose $\cA$ is closed under countable increasing unions. Let $\{E_n\}_{n = 1}^\infty$ be a family of sets in $\cA$. Define:
            \begin{equation*}
            \begin{split}
                F_1 &= E_1, \\
                F_2 &= E_1 \cup E_2, \\
                &\vdots \\
                F_n &= E_1 \cup E_2 \cup ... \cup E_n.
            \end{split}
            \end{equation*}
        Then clearly $F_1 \subset F_2 \subset F_3 ...$ Moreover, it is clear that $\bigcup_{n = 1}^\infty E_n \subset \bigcup_{n = 1}^\infty F_n$. Conversely, let $x \in \bigcup_{n = 1}^\infty F_n$. Then $x \in F_i$ for some $i \in \bfN$. By construction, $x \in E_1 \cup ... \cup E_i$, hence $x \in E_j$ for some $j < i$. Whence $x \in \bigcup_{n=1}^\infty E_n$, giving the inclusion $\bigcup_{n = 1}^\infty F_n \subset \bigcup_{n = 1}^\infty E_n$. Together, we have  $\bigcup_{n = 1}^\infty E_n = \bigcup_{n = 1}^\infty F_n \in \cA$. Since $\{E_n\}_{n = 1}^\infty$ was an arbitrary family of sets in $\cA$, we can conclude $\cA$ is a $\sigma$-algebra.
    \end{proof}}
%%%%%%%%%%%%%%%%%%%%%%%%%%%%%%%%%%%%%%%%%%%%%%%%%%
\end{document}