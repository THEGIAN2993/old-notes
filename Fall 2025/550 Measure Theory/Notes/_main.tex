\documentclass[10pt,twoside,openany]{memoir}
%\usepackage{mlmodern}
%\usepackage{tgpagella} % text only
%\usepackage{mathpazo}  % math & text
\usepackage[T1]{fontenc}
\usepackage[hidelinks]{hyperref}
\usepackage{amsmath}
\usepackage[fixamsmath]{mathtools}  % Extension to amsmath
\usepackage{amsthm}
\usepackage{amssymb}
\renewcommand*{\mathbf}[1]{\varmathbb{#1}}
\usepackage{newpxtext}
\usepackage{eulerpx}
\usepackage{eucal}
\usepackage{datetime}
    \newdateformat{specialdate}{\THEYEAR\ \monthname\ \THEDAY}
\usepackage[margin=1.5in]{geometry}
\usepackage{fancyhdr}
    \fancyhf{}
    \pagestyle{fancy}
    \cfoot{\scriptsize \thepage}
    \fancyhead[R]{\scalebox{0.7}{\rightmark}}
    \fancyhead[L]{\scalebox{0.7}{\leftmark}}
\usepackage{thmtools}
    \declaretheoremstyle[
        spaceabove=10pt,
        spacebelow=10pt,
        headfont=\normalfont\bfseries,
        notefont=\mdseries, notebraces={(}{)},
        bodyfont=\normalfont,
        postheadspace=0.5em
        %qed=\qedsymbol
        ]{defs}

    \declaretheoremstyle[ 
        spaceabove=10pt, % space above the theorem
        spacebelow=10pt,
        headfont=\normalfont\bfseries,
        bodyfont=\normalfont\itshape,
        postheadspace=0.5em
        ]{thmstyle}
    
    \declaretheorem[
        style=thmstyle,
        numberwithin=section
    ]{theorem}

    \declaretheorem[
        style=thmstyle,
        sibling=theorem,
    ]{proposition}

    \declaretheorem[
        style=thmstyle,
        sibling=theorem,
    ]{lemma}

    \declaretheorem[
        style=thmstyle,
        sibling=theorem,
    ]{corollary}

    \declaretheorem[
        numberwithin=section,
        style=defs,
    ]{example}

    \declaretheorem[
        numberwithin=section,
        style=defs,
    ]{definition}

    \declaretheorem[
        style=defs,
        sibling=theorem,
        numberwithin=section,
    ]{exercise}

    \declaretheorem[
        style=thmstyle
        numbered=unless unique
    ]{axiom}

    \declaretheorem[numberwithin=section,style=defs]{note}
    \declaretheorem[numbered=no,style=defs]{question}
    \declaretheorem[numbered=no,style=defs]{recall}
    \declaretheorem[numbered=no,style=remark]{answer}
    \declaretheorem[numbered=no,style=remark]{solution}
    \declaretheorem[numbered=no,style=defs]{remark}
\usepackage{enumitem}
\usepackage{titlesec}
    \titleformat{\chapter}[display]
    {\bfseries\Huge\raggedright}
    {Chapter {\thechapter}}
    {1ex minus .1ex}
    {\HUGE}
    \titlespacing{\chapter}
    {3pc}{*3}{40pt}[3pc]

    \titleformat{\section}[block]
    {\normalfont\bfseries\LARGE}
    {\S\ \thesection.}{.5em}{}[]
    \titlespacing{\section}
    {0pt}{3ex plus .1ex minus .2ex}{3ex plus .1ex minus .2ex}
\usepackage[utf8x]{inputenc}
\usepackage{tikz}
\usepackage{tikz-cd}
\usetikzlibrary{patterns}
\usetikzlibrary{positioning,arrows.meta}
\usepackage{wasysym}
\linespread{1.00}
%%%%%%%%%%%%%%%%%%%%%%%%%%%%%%%%%%%%%%%%%%%%%%%%%%%%%%%%%%%%%
%%%%%%%%%%%%%%%%%%%%%%%%%%%%%%%%%%%%%%%%%%%%%%%%%%%%%%%%%%%%%
%to make the correct symbol for Sha
%\newcommand\cyr{%
%\renewcommand\rmdefault{wncyr}%
%\renewcommand\sfdefault{wncyss}%
%\renewcommand\encodingdefault{OT2}%
%\normalfont \selectfont} \DeclareTextFontCommand{\textcyr}{\cyr}


\DeclareMathOperator{\ab}{ab}
\newcommand{\absgal}{\G_{\bbQ}}
\DeclareMathOperator{\ad}{ad}
\DeclareMathOperator{\adj}{adj}
\DeclareMathOperator{\alg}{alg}
\DeclareMathOperator{\Alt}{Alt}
\DeclareMathOperator{\Ann}{Ann}
\DeclareMathOperator{\arith}{arith}
\DeclareMathOperator{\Aut}{Aut}
\DeclareMathOperator{\Be}{B}
\DeclareMathOperator{\Bd}{Bd}
\DeclareMathOperator{\card}{card}
\DeclareMathOperator{\Char}{char}
\DeclareMathOperator{\csp}{csp}
\DeclareMathOperator{\codim}{codim}
\DeclareMathOperator{\coker}{coker}
\DeclareMathOperator{\coh}{H}
\DeclareMathOperator{\compl}{compl}
\DeclareMathOperator{\conj}{conj}
\DeclareMathOperator{\cont}{cont}
\DeclareMathOperator{\crys}{crys}
\DeclareMathOperator{\Crys}{Crys}
\DeclareMathOperator{\cusp}{cusp}
\DeclareMathOperator{\diag}{diag}
\DeclareMathOperator{\diam}{diam}
\DeclareMathOperator{\Dom}{Dom}
\DeclareMathOperator{\disc}{disc}
\DeclareMathOperator{\dist}{dist}
\DeclareMathOperator{\dR}{dR}
\DeclareMathOperator{\Eis}{Eis}
\DeclareMathOperator{\End}{End}
\DeclareMathOperator{\ev}{ev}
\DeclareMathOperator{\eval}{eval}
\DeclareMathOperator{\Eq}{Eq}
\DeclareMathOperator{\Ext}{Ext}
\DeclareMathOperator{\Fil}{Fil}
\DeclareMathOperator{\Fitt}{Fitt}
\DeclareMathOperator{\Frob}{Frob}
\DeclareMathOperator{\G}{G}
\DeclareMathOperator{\Gal}{Gal}
\DeclareMathOperator{\GL}{GL}
\DeclareMathOperator{\Gr}{Gr}
\DeclareMathOperator{\Graph}{Graph}
\DeclareMathOperator{\GSp}{GSp}
\DeclareMathOperator{\GUn}{GU}
\DeclareMathOperator{\Hom}{Hom}
\DeclareMathOperator{\id}{id}
\DeclareMathOperator{\Id}{Id}
\DeclareMathOperator{\Ik}{Ik}
\DeclareMathOperator{\IM}{Im}
\DeclareMathOperator{\Image}{im}
\DeclareMathOperator{\Ind}{Ind}
\DeclareMathOperator{\Inf}{inf}
\DeclareMathOperator{\Isom}{Isom}
\DeclareMathOperator{\J}{J}
\DeclareMathOperator{\Jac}{Jac}
\DeclareMathOperator{\lcm}{lcm}
\DeclareMathOperator{\length}{length}
\DeclareMathOperator*{\limit}{limit}
\DeclareMathOperator{\Log}{Log}
\DeclareMathOperator{\M}{M}
\DeclareMathOperator{\Mat}{Mat}
\DeclareMathOperator{\N}{N}
\DeclareMathOperator{\Nm}{Nm}
\DeclareMathOperator{\NIk}{N-Ik}
\DeclareMathOperator{\NSK}{N-SK}
\DeclareMathOperator{\new}{new}
\DeclareMathOperator{\obj}{obj}
\DeclareMathOperator{\old}{old}
\DeclareMathOperator{\ord}{ord}
\DeclareMathOperator{\Or}{O}
\DeclareMathOperator{\op}{op}
\DeclareMathOperator{\PGL}{PGL}
\DeclareMathOperator{\PGSp}{PGSp}
\DeclareMathOperator{\rank}{rank}
\DeclareMathOperator{\Ran}{Ran}
\DeclareMathOperator{\Rel}{Rel}
\DeclareMathOperator{\Real}{Re}
\DeclareMathOperator{\RES}{res}
\DeclareMathOperator{\Res}{Res}
%\DeclareMathOperator{\Sha}{\textcyr{Sh}}
\DeclareMathOperator{\Sel}{Sel}
\DeclareMathOperator{\semi}{ss}
\DeclareMathOperator{\sgn}{sign}
\DeclareMathOperator{\SK}{SK}
\DeclareMathOperator{\SL}{SL}
\DeclareMathOperator{\SO}{SO}
\DeclareMathOperator{\Sp}{Sp}
\DeclareMathOperator{\Span}{span}
\DeclareMathOperator{\Spec}{Spec}
\DeclareMathOperator{\spin}{spin}
\DeclareMathOperator{\st}{st}
\DeclareMathOperator{\St}{St}
\DeclareMathOperator{\SUn}{SU}
\DeclareMathOperator{\supp}{supp}
\DeclareMathOperator{\Sup}{sup}
\DeclareMathOperator{\Sym}{Sym}
\DeclareMathOperator{\Tam}{Tam}
\DeclareMathOperator{\tors}{tors}
\DeclareMathOperator{\tr}{tr}
\DeclareMathOperator{\Tr}{Tr}
\DeclareMathOperator{\un}{un}
\DeclareMathOperator{\Un}{U}
\DeclareMathOperator{\val}{val}
\DeclareMathOperator{\vol}{vol}

\DeclareMathOperator{\Sets}{S \mkern1.04mu e \mkern1.04mu t \mkern1.04mu s}
    \newcommand{\cSets}{\scalebox{1.02}{\contour{black}{$\Sets$}}}
    
\DeclareMathOperator{\Groups}{G \mkern1.04mu r \mkern1.04mu o \mkern1.04mu u \mkern1.04mu p \mkern1.04mu s}
    \newcommand{\cGroups}{\scalebox{1.02}{\contour{black}{$\Groups$}}}

\DeclareMathOperator{\TTop}{T \mkern1.04mu o \mkern1.04mu p}
    \newcommand{\cTop}{\scalebox{1.02}{\contour{black}{$\TTop$}}}

\DeclareMathOperator{\Htp}{H \mkern1.04mu t \mkern1.04mu p}
    \newcommand{\cHtp}{\scalebox{1.02}{\contour{black}{$\Htp$}}}

\DeclareMathOperator{\Mod}{M \mkern1.04mu o \mkern1.04mu d}
    \newcommand{\cMod}{\scalebox{1.02}{\contour{black}{$\Mod$}}}

\DeclareMathOperator{\Ab}{A \mkern1.04mu b}
    \newcommand{\cAb}{\scalebox{1.02}{\contour{black}{$\Ab$}}}

\DeclareMathOperator{\Rings}{R \mkern1.04mu i \mkern1.04mu n \mkern1.04mu g \mkern1.04mu s}
    \newcommand{\cRings}{\scalebox{1.02}{\contour{black}{$\Rings$}}}

\DeclareMathOperator{\ComRings}{C \mkern1.04mu o \mkern1.04mu m \mkern1.04mu R \mkern1.04mu i \mkern1.04mu n \mkern1.04mu g \mkern1.04mu s}
    \newcommand{\cComRings}{\scalebox{1.05}{\contour{black}{$\ComRings$}}}

\DeclareMathOperator{\hHom}{H \mkern1.04mu o \mkern1.04mu m}
    \newcommand{\cHom}{\scalebox{1.02}{\contour{black}{$\hHom$}}}

         %  \item $\cGroups$
          %  \item $\cTop$
          %  \item $\cHtp$
          %  \item $\cMod$




\renewcommand{\k}{\kappa}
\newcommand{\Ff}{F_{f}}
%\newcommand{\ts}{\,^{t}\!}


%Mathcal

\newcommand{\cA}{\mathcal{A}}
\newcommand{\cB}{\mathcal{B}}
\newcommand{\cC}{\mathcal{C}}
\newcommand{\cD}{\mathcal{D}}
\newcommand{\cE}{\mathcal{E}}
\newcommand{\cF}{\mathcal{F}}
\newcommand{\cG}{\mathcal{G}}
\newcommand{\cH}{\mathcal{H}}
\newcommand{\cI}{\mathcal{I}}
\newcommand{\cJ}{\mathcal{J}}
\newcommand{\cK}{\mathcal{K}}
\newcommand{\cL}{\mathcal{L}}
\newcommand{\cM}{\mathcal{M}}
\newcommand{\cN}{\mathcal{N}}
\newcommand{\cO}{\mathcal{O}}
\newcommand{\cP}{\mathcal{P}}
\newcommand{\cQ}{\mathcal{Q}}
\newcommand{\cR}{\mathcal{R}}
\newcommand{\cS}{\mathcal{S}}
\newcommand{\cT}{\mathcal{T}}
\newcommand{\cU}{\mathcal{U}}
\newcommand{\cV}{\mathcal{V}}
\newcommand{\cW}{\mathcal{W}}
\newcommand{\cX}{\mathcal{X}}
\newcommand{\cY}{\mathcal{Y}}
\newcommand{\cZ}{\mathcal{Z}}


%mathfrak (missing \fi)

\newcommand{\fa}{\mathfrak{a}}
\newcommand{\fA}{\mathfrak{A}}
\newcommand{\fb}{\mathfrak{b}}
\newcommand{\fB}{\mathfrak{B}}
\newcommand{\fc}{\mathfrak{c}}
\newcommand{\fC}{\mathfrak{C}}
\newcommand{\fd}{\mathfrak{d}}
\newcommand{\fD}{\mathfrak{D}}
\newcommand{\fe}{\mathfrak{e}}
\newcommand{\fE}{\mathfrak{E}}
\newcommand{\ff}{\mathfrak{f}}
\newcommand{\fF}{\mathfrak{F}}
\newcommand{\fg}{\mathfrak{g}}
\newcommand{\fG}{\mathfrak{G}}
\newcommand{\fh}{\mathfrak{h}}
\newcommand{\fH}{\mathfrak{H}}
\newcommand{\fI}{\mathfrak{I}}
\newcommand{\fj}{\mathfrak{j}}
\newcommand{\fJ}{\mathfrak{J}}
\newcommand{\fk}{\mathfrak{k}}
\newcommand{\fK}{\mathfrak{K}}
\newcommand{\fl}{\mathfrak{l}}
\newcommand{\fL}{\mathfrak{L}}
\newcommand{\fm}{\mathfrak{m}}
\newcommand{\fM}{\mathfrak{M}}
\newcommand{\fn}{\mathfrak{n}}
\newcommand{\fN}{\mathfrak{N}}
\newcommand{\fo}{\mathfrak{o}}
\newcommand{\fO}{\mathfrak{O}}
\newcommand{\fp}{\mathfrak{p}}
\newcommand{\fP}{\mathfrak{P}}
\newcommand{\fq}{\mathfrak{q}}
\newcommand{\fQ}{\mathfrak{Q}}
\newcommand{\fr}{\mathfrak{r}}
\newcommand{\fR}{\mathfrak{R}}
\newcommand{\fs}{\mathfrak{s}}
\newcommand{\fS}{\mathfrak{S}}
\newcommand{\ft}{\mathfrak{t}}
\newcommand{\fT}{\mathfrak{T}}
\newcommand{\fu}{\mathfrak{u}}
\newcommand{\fU}{\mathfrak{U}}
\newcommand{\fv}{\mathfrak{v}}
\newcommand{\fV}{\mathfrak{V}}
\newcommand{\fw}{\mathfrak{w}}
\newcommand{\fW}{\mathfrak{W}}
\newcommand{\fx}{\mathfrak{x}}
\newcommand{\fX}{\mathfrak{X}}
\newcommand{\fy}{\mathfrak{y}}
\newcommand{\fY}{\mathfrak{Y}}
\newcommand{\fz}{\mathfrak{z}}
\newcommand{\fZ}{\mathfrak{Z}}


%mathbf
\newcommand{\bfA}{\mathbf{A}}
\newcommand{\bfB}{\mathbf{B}}
\newcommand{\bfC}{\mathbf{C}}
\newcommand{\bfD}{\mathbf{D}}
\newcommand{\bfE}{\mathbf{E}}
\newcommand{\bfF}{\mathbf{F}}
\newcommand{\bfG}{\mathbf{G}}
\newcommand{\bfH}{\mathbf{H}}
\newcommand{\bfI}{\mathbf{I}}
\newcommand{\bfJ}{\mathbf{J}}
\newcommand{\bfK}{\mathbf{K}}
\newcommand{\bfL}{\mathbf{L}}
\newcommand{\bfM}{\mathbf{M}}
\newcommand{\bfN}{\mathbf{N}}
\newcommand{\bfO}{\mathbf{O}}
\newcommand{\bfP}{\mathbf{P}}
\newcommand{\bfQ}{\mathbf{Q}}
\newcommand{\bfR}{\mathbf{R}}
\newcommand{\bfS}{\mathbf{S}}
\newcommand{\bfT}{\mathbf{T}}
\newcommand{\bfU}{\mathbf{U}}
\newcommand{\bfV}{\mathbf{V}}
\newcommand{\bfW}{\mathbf{W}}
\newcommand{\bfX}{\mathbf{X}}
\newcommand{\bfY}{\mathbf{Y}}
\newcommand{\bfZ}{\mathbf{Z}}

\newcommand{\bfa}{\mathbf{a}}
\newcommand{\bfb}{\mathbf{b}}
\newcommand{\bfc}{\mathbf{c}}
\newcommand{\bfd}{\mathbf{d}}
\newcommand{\bfe}{\mathbf{e}}
\newcommand{\bff}{\mathbf{f}}
\newcommand{\bfg}{\mathbf{g}}
\newcommand{\bfh}{\mathbf{h}}
\newcommand{\bfi}{\mathbf{i}}
\newcommand{\bfj}{\mathbf{j}}
\newcommand{\bfk}{\mathbf{k}}
\newcommand{\bfl}{\mathbf{l}}
\newcommand{\bfm}{\mathbf{m}}
\newcommand{\bfn}{\mathbf{n}}
\newcommand{\bfo}{\mathbf{o}}
\newcommand{\bfp}{\mathbf{p}}
\newcommand{\bfq}{\mathbf{q}}
\newcommand{\bfr}{\mathbf{r}}
\newcommand{\bfs}{\mathbf{s}}
\newcommand{\bft}{\mathbf{t}}
\newcommand{\bfu}{\mathbf{u}}
\newcommand{\bfv}{\mathbf{v}}
\newcommand{\bfw}{\mathbf{w}}
\newcommand{\bfx}{\mathbf{x}}
\newcommand{\bfy}{\mathbf{y}}
\newcommand{\bfz}{\mathbf{z}}

%blackboard bold

\newcommand{\bbA}{\mathbb{A}}
\newcommand{\bbB}{\mathbb{B}}
\newcommand{\bbC}{\mathbb{C}}
\newcommand{\bbD}{\mathbb{D}}
\newcommand{\bbE}{\mathbb{E}}
\newcommand{\bbF}{\mathbb{F}}
\newcommand{\bbG}{\mathbb{G}}
\newcommand{\bbH}{\mathbb{H}}
\newcommand{\bbI}{\mathbb{I}}
\newcommand{\bbJ}{\mathbb{J}}
\newcommand{\bbK}{\mathbb{K}}
\newcommand{\bbL}{\mathbb{L}}
\newcommand{\bbM}{\mathbb{M}}
\newcommand{\bbN}{\mathbb{N}}
\newcommand{\bbO}{\mathbb{O}}
\newcommand{\bbP}{\mathbb{P}}
\newcommand{\bbQ}{\mathbb{Q}}
\newcommand{\bbR}{\mathbb{R}}
\newcommand{\bbS}{\mathbb{S}}
\newcommand{\bbT}{\mathbb{T}}
\newcommand{\bbU}{\mathbb{U}}
\newcommand{\bbV}{\mathbb{V}}
\newcommand{\bbW}{\mathbb{W}}
\newcommand{\bbX}{\mathbb{X}}
\newcommand{\bbY}{\mathbb{Y}}
\newcommand{\bbZ}{\mathbb{Z}}
\newcommand{\jota}{\jmath}

\newcommand{\bmat}{\left( \begin{matrix}}
\newcommand{\emat}{\end{matrix} \right)}

\newcommand{\pmat}{\left( \begin{smallmatrix}}
\newcommand{\epmat}{\end{smallmatrix} \right)}

\newcommand{\lat}{\mathscr{L}}
\newcommand{\mat}[4]{\begin{pmatrix}{#1}&{#2}\\{#3}&{#4}\end{pmatrix}}
\newcommand{\ov}[1]{\overline{#1}}
\newcommand{\res}[1]{\underset{#1}{\RES}\,}
\newcommand{\up}{\upsilon}

\newcommand{\tac}{\textasteriskcentered}

%mahesh macros
\newcommand{\tm}{\textrm}

%Comments
\newcommand{\com}[1]{\vspace{5 mm}\par \noindent
\marginpar{\textsc{Comment}} \framebox{\begin{minipage}[c]{0.95
\textwidth} \tt #1 \end{minipage}}\vspace{5 mm}\par}

\newcommand{\Bmu}{\mbox{$\raisebox{-0.59ex}
  {$l$}\hspace{-0.18em}\mu\hspace{-0.88em}\raisebox{-0.98ex}{\scalebox{2}
  {$\color{white}.$}}\hspace{-0.416em}\raisebox{+0.88ex}
  {$\color{white}.$}\hspace{0.46em}$}{}}  %need graphicx and xcolor. this produces blackboard bold mu 

\newcommand{\hooktwoheadrightarrow}{%
  \hookrightarrow\mathrel{\mspace{-15mu}}\rightarrow
}

\makeatletter
\newcommand{\xhooktwoheadrightarrow}[2][]{%
  \lhook\joinrel
  \ext@arrow 0359\rightarrowfill@ {#1}{#2}%
  \mathrel{\mspace{-15mu}}\rightarrow
}
\makeatother

\renewcommand{\geq}{\geqslant}
\renewcommand{\leq}{\leqslant}
\newcommand{\midd}{\hspace{4pt}\middle|\hspace{4pt}}
    
    \newcommand{\bone}{\mathbf{1}}
    \newcommand{\sign}{\mathrm{sign}}
    \newcommand{\eps}{\varepsilon}
    \newcommand{\textui}[1]{\uline{\textit{#1}}}
    
    %\newcommand{\ov}{\overline}
    %\newcommand{\un}{\underline}
    \newcommand{\fin}{\mathrm{fin}}
    
    \newcommand{\chnum}{\titleformat
    {\chapter} % command
    [display] % shape
    {\centering} % format
    {\Huge \color{black} \shadowbox{\thechapter}} % label
    {-0.5em} % sep (space between the number and title)
    {\LARGE \color{black} \underline} % before-code
    }
    
    \newcommand{\chunnum}{\titleformat
    {\chapter} % command
    [display] % shape
    {} % format
    {} % label
    {0em} % sep
    { \begin{flushright} \begin{tabular}{r}  \Huge \color{black}
    } % before-code
    [
    \end{tabular} \end{flushright} \normalsize
    ] % after-code
    }

\newcommand{\nl}{\newline \mbox{}}

\newcommand{\h}[1]{\hspace{#1pt}}

\newcommand{\littletaller}{\mathchoice{\vphantom{\big|}}{}{}{}}
\newcommand\restr[2]{{% we make the whole thing an ordinary symbol
  \left.\kern-\nulldelimiterspace % automatically resize the bar with \right
  #1 % the function
  \littletaller % pretend it's a little taller at normal size
  \right|_{#2} % this is the delimiter
  }}

\newcommand{\mtext}[1]{\hspace{6pt}\text{#1}\hspace{6pt}}

\newcommand{\lnorm}{\left\lVert}
\newcommand{\rnorm}{\right\rVert}

\newcommand{\ds}{\displaystyle}
\newcommand{\ts}{\textstyle}

%This adds a "front cover" page.
%{\thispagestyle{empty}
%\vspace*{\fill}
%\begin{tabular}{l}
%\begin{tabular}{l}
%\includegraphics[scale=0.24]{oxy-logo.png}
%\end{tabular} \\
%\begin{tabular}{l}
%\Large \color{black} Module Theory, Linear Algebra, and Homological Algebra \\ \Large \color{black} Gianluca Crescenzo
%\end{tabular}
%\end{tabular}
%\newpage

\newcommand{\sfrac}[2]{{}^{#1}\mskip -5mu/\mskip -3mu_{#2}}


\makeatletter
\newcommand*{\da@rightarrow}{\mathchar"0\hexnumber@\symAMSa 4B }
\newcommand*{\da@leftarrow}{\mathchar"0\hexnumber@\symAMSa 4C }
\newcommand*{\xdashrightarrow}[2][]{%
  \mathrel{%
    \mathpalette{\da@xarrow{#1}{#2}{}\da@rightarrow{\,}{}}{}%
  }%
}
\newcommand{\xdashleftarrow}[2][]{%
  \mathrel{%
    \mathpalette{\da@xarrow{#1}{#2}\da@leftarrow{}{}{\,}}{}%
  }%
}
\newcommand*{\da@xarrow}[7]{%
  % #1: below
  % #2: above
  % #3: arrow left
  % #4: arrow right
  % #5: space left 
  % #6: space right
  % #7: math style 
  \sbox0{$\ifx#7\scriptstyle\scriptscriptstyle\else\scriptstyle\fi#5#1#6\m@th$}%
  \sbox2{$\ifx#7\scriptstyle\scriptscriptstyle\else\scriptstyle\fi#5#2#6\m@th$}%
  \sbox4{$#7\dabar@\m@th$}%
  \dimen@=\wd0 %
  \ifdim\wd2 >\dimen@
    \dimen@=\wd2 %   
  \fi
  \count@=2 %
  \def\da@bars{\dabar@\dabar@}%
  \@whiledim\count@\wd4<\dimen@\do{%
    \advance\count@\@ne
    \expandafter\def\expandafter\da@bars\expandafter{%
      \da@bars
      \dabar@ 
    }%
  }%  
  \mathrel{#3}%
  \mathrel{%   
    \mathop{\da@bars}\limits
    \ifx\\#1\\%
    \else
      _{\copy0}%
    \fi
    \ifx\\#2\\%
    \else
      ^{\copy2}%
    \fi
  }%   
  \mathrel{#4}%
}
\makeatother


\begin{document}
%This adds a "front cover" page.
%{\thispagestyle{empty}
%\vspace*{\fill}
%\begin{tabular}{l}
%\begin{tabular}{l}
%\includegraphics[scale=0.24]{oxy-logo.png}
%\end{tabular} \\
%\begin{tabular}{l}
%\Large \color{black} Module Theory, Linear Algebra, and Homological Algebra \\ \Large \color{black} Gianluca Crescenzo
%\end{tabular}
%\end{tabular}
%\newpage
    \pagenumbering{roman}
    \chapter*{Preface}
    At the root of measure theory is a rigorous treatment of how to extend the notions of length (on the real line), area (on the Euclidean plane), and volume (in three-space) to a larger collection of sets than those seen in, say, courses on Calculus. We have seen in a first course in analysis that there exist sets with some exceptionally strange properties (the Cantor set being a classic example). We would like to be able to discuss these types of sets in the context of measuring their sizes (be it a length, area, volume, etc). While the historical impetus was to solve this problem for sets in $\bfR^n$, one finds that it is not hard to abstract the theory to more general sets. Accordingly we take up this abstract study. This is not just for the sake of abstraction. In fact one can find applications for abstract measure theory in, for example, probability theory (where the measure of a set corresponds to the likelihood of an element in that set being chosen under some random process) and physics (where one might want the measure of a set to correspond to the mass of some object occupying that space).

    \vfill
    \specialdate
    Last update: \today

    \newpage
    \tableofcontents

    \chapter{Preliminaries}
\pagenumbering{arabic}
\section{Set Theory}

    \begin{definition}
        A \textit{set} (typically denoted with capital letters) is a collection of different things; the things are called \textit{elements} (typically denoted with lowercase letters) of the set.
    \end{definition}

    Zermelo–Fraenkel set theory introduces eight different axioms which mathematics builds upon. Below is an introduction to the ones which are relevant for the material covered in this course.

    \begin{axiom}[Axiom of Extension]
        If the sets $X$ and $Y$ have the same members, then they are the same set. In other words:
            \begin{equation*}
            \begin{split}
                (\forall X)(\forall Y)\Bigl((\forall z)\bigl(z \in X \Leftrightarrow z \in Y\bigr) \implies X = Y\Bigr)
            \end{split}
            \end{equation*}
    \end{axiom}

    \begin{axiom}[Axiom of Existence]
        There is a set such that no element is a member of it. In other words:
            \begin{equation*}
            \begin{split}
                (\exists A):(\forall x)(x \not\in A)
            \end{split}
            \end{equation*}
    \end{axiom}

    Note that the Axiom of Extension immediately proves that the set which contains no elements is unique; two sets are equal if they have the same members, hence two sets which contain no elements are equal.

    \begin{definition}
        The set which contains no elements is called the \textit{empty set}, and is denoted by $\emptyset$.
    \end{definition}

    \begin{axiom}[Axiom of Pairing]
        Let $A$ and $B$ be sets. There exists the set $\cC := \{A,B\}$. In other words:
            \begin{equation*}
            \begin{split}
                (\forall A)(\forall B)(\exists \cC):(A \in \cC \land B \in \cC).
            \end{split}
            \end{equation*}
    \end{axiom}

    \begin{axiom}[Axiom of Union]
        Let $\cC$ be a set. Then there exists a set $U$ consisting of the elements $x$ that are contained in at least one element of the set $\cC$, that is:
            \begin{equation*}
            \begin{split}
                (\forall \cC)(\exists U):(\forall x)(x \in U \Leftrightarrow (\exists C)(u \in C \land C \in \cC))
            \end{split}
            \end{equation*}
    \end{axiom}

    \begin{definition}
        Let $\cC$ be a set, and let $U$ be the set consisting of the elements $x$ that are contained in at least one element of $\cC$. 
        \begin{enumerate}[label = (\arabic*),itemsep=1pt,topsep=3pt]
            \item The set $U$ is called the \textit{union of the set $\cC$} and is denoted:
                \begin{equation*}
                \begin{split}
                    U := \bigcup \cC.
                \end{split}
                \end{equation*}

            \item If the set $\cC$ consists of only two sets $A$ and $B$, then we write $U:= A \cup B$.
                \begin{center}
                \begin{tikzpicture}
                % Fill left circle
                \fill[pattern=north east lines] (0,0) circle (1.5cm);
                % Fill right circle
                \fill[pattern=north east lines] (2,0) circle (1.5cm);

                % Draw outlines
                \draw[thick] (0,0) circle (1.5cm) node[left=1.6cm] {$A$};
                \draw[thick] (2,0) circle (1.5cm) node[right=1.6cm] {$B$};

                % Label
                \node at (1,-2) {$A \cup B$};
                \end{tikzpicture}
                \end{center}
        \end{enumerate}
    \end{definition}

    \begin{axiom}[Axiom Schema of Specification]
        Let $A$ be a set, and let $\varphi$ be a predicate containing the free variable $x$. There exists a subset $B$ of the set $A$ whose members are precisely the elements $x$ of the set $A$ such that the sentence $\varphi(x)$ is true. In other words:
            \begin{equation*}
            \begin{split}
                (\forall A)(\forall \varphi)(\exists B):(B = \{x \in A \mid \varphi(x)\})
            \end{split}
            \end{equation*}
    \end{axiom}

    \begin{proposition}
        Let $\cC$ be a nonempty set. There exists a set $D$ consisting of the elements $x$ that are contained in all elements of the set $\cC$.
    \end{proposition}
        \begin{proof}
            Let $E \in \cC$. By the Axiom Schema of Specification we have $D := \{c \in E \mid (\forall X)(X \in \cC \Rightarrow c \in X)\}$.
        \end{proof}

    \begin{definition}
        Let $\cC$ be a nonempty set, and let $D$ be the set consisting of the elements $x$ that are contained in all elements of the set $\cC$.
            \begin{enumerate}[label = (\arabic*),itemsep=1pt,topsep=3pt]
                \item The set $D$ is called the \textit{intersection of the set $\cC$} and is denoted:
                    \begin{equation*}
                    \begin{split}
                        D := \bigcap \cC.
                    \end{split}
                    \end{equation*}
                \item If the set $\cC$ contains only two sets $A$ and $B$, then we write $D:= A \cap B$.
                    \begin{center}
                    \begin{tikzpicture}
                    \draw[thick] (0,0) circle (1.5cm) node[left=1.6cm] {$A$};
                    \draw[thick] (2,0) circle (1.5cm) node[right=1.6cm] {$B$};
                    \begin{scope}
                        \clip (0,0) circle (1.5cm);
                        \fill[pattern=north east lines] (2,0) circle (1.5cm);
                    \end{scope}
                    \begin{scope}
                        \clip (2,0) circle (1.5cm);
                        \fill[pattern=north east lines] (0,0) circle (1.5cm);
                    \end{scope}
                    \node at (1,-2) {$A \cap B$};
                    \end{tikzpicture}
                    \end{center}
            \end{enumerate}
    \end{definition}

    \begin{axiom}
        Let $X$ be a set. Then the set of all subsets of $X$ exists.
    \end{axiom}
    
    \begin{definition}
        Let $X$ be a set. Then the set of all subsets of the set $X$ is called the \textit{power set of the set $X$} and is denoted $\cP(X)$.
    \end{definition}

    \begin{axiom}[Axiom of Infinity]
        There is a set that contains all natural numbers.
    \end{axiom}

    \begin{definition}
        An \textit{index set} is a nonempty set whose elements label the elements of another set. We denote the collection of objects labeled by elements of $I$ as $\{A_\alpha\}_{\alpha \in I}$.
    \end{definition}

    With this definition and the Axiom of Infinity, we can extend the concept of unions and intersections to families of sets of arbitrary size.

    \begin{definition}
        Let $I$ be an index set, let $\{X_\alpha\}_{\alpha \in I}$ be an indexed family of sets, and write $\cX := \{X_\alpha \mid \alpha \in I\}$. Define:
            \begin{equation*}
            \begin{split}
                \bigcup_{\alpha \in I}X_i &:= \bigcup \cX, \\
                \bigcap_{\alpha \in I}X_i &:= \bigcap \cX.
            \end{split}
            \end{equation*}
    \end{definition}

    \begin{definition}
        Let $\{X_n\}_{n \in \bfN}$ be a family of sets indexed by the natural numbers.
        \begin{enumerate}[label = (\arabic*),itemsep=1pt,topsep=3pt]
            \item The \textit{limit superior} of $\{X_n\}_{n \in \bfN}$ is:
                \begin{equation*}
                \begin{split}
                    \limsup X_n 
                    & := \bigcap_{k = 1}^\infty \left( \bigcup_{n = k}^\infty X_n\right) \\
                    & = \{x \mid x \in E_n \h3\text{for infinitely many $n$}\h1\}. 
                \end{split}
                \end{equation*}
            \item The \textit{limit inferior} of $\{X_n\}_{n \in \bfN}$ is:
                \begin{equation*}
                \begin{split}
                    \limsup X_n 
                    & := \bigcup_{k = 1}^\infty \left( \bigcap_{n = k}^\infty X_n\right) \\
                    & \{x \mid x \not\in E_n \h3\text{for only finitely many $n$}\h1\}
                \end{split}
                \end{equation*}
        \end{enumerate}
    \end{definition}

    \begin{definition}
        Let $A$ and $B$ be two sets. Define the \textit{difference of $A$ and $B$} by $A \setminus B := \{x \in A \mid x \not\in B\}$.
    \end{definition}

    Note that the difference of two sets exists by the Axiom Schema of Specification.

    \begin{definition}
        Let $a$ and $b$ be two objects. The \textit{ordered pair} $(a,b)$ is defined by $(a,b) := \{\{a\},\{a,b\}\}$.
    \end{definition}

    \begin{proposition}
        Let $A$ and $B$ be sets. There exists a set $E$ containing all ordered pairs $(a,b)$, where $a \in A$ and $b \in B$.
    \end{proposition}
        \begin{proof}
            Let $(a,b)$ be any ordered pair such that $a \in A$ and $b \in B$. By definition $(a,b) = \{\{a\},\{a,b\}\}$. Now $\{a\} \subseteq A$, so $\{a\} \subseteq A \cup B$, hence $\{a\} \in \cP(A \cup B)$. Similarly, $\{a,b\} \subseteq A \cup B$, so $\{a,b\} \in \cP(A \cup B)$. It follows that $\{\{a\},\{a,b\}\} \subseteq \cP(A \cup B)$, hence $\{\{a\},\{a,b\}\} \in \cP(\cP(A \cup B))$. We know all these sets are known to exist by the Axiom of Union and Axiom of Power Set. Then by the Axiom Schema of Specification, there exists a set:
                \begin{equation*}
                \begin{split}
                    E := \{t \in \cP(\cP(A \cup B)) \mid (\exists a)(\exists b)(a \in A \land b \in B \land t = (a,b))\}.
                \end{split}
                \end{equation*}
            Moreover, this set is unique by the Axiom of Extension, and is the set of all ordered pairs $(a,b)$ with $a \in A$ and $b \in B$.
        \end{proof}

    \begin{definition}
        Let $A$ and $B$ be two sets, and let $E$ be the set containing all ordered pairs $(a,b)$ where $a \in A$ and $b \in B$. The set $E$ is called the \textit{Cartesian product of $A$ and $B$}, and is denoted $E:= A \times B = \{(a,b) \mid a \in A \land b \in B\}$.
    \end{definition}

    \begin{definition}
        Let $X$ and $Y$ be sets. A \textit{relation} from $X$ to $Y$ is a subset $R$ of $X \times Y$. We write $xRy$ to mean $(x,y) \in R$.
            \begin{enumerate}[label = (\arabic*),itemsep=1pt,topsep=3pt]
                \item An \textit{equivalence relation} on $X$ is a relation which is reflexive, transitive, and symmetric. We write $x \sim y$ to mean $xRy$.
                \item A \textit{function} (or \textit{map}) from $X$ to $Y$, denoted $f:X \rightarrow Y$, is a relation from $X$ to $Y$ such that, for every $x \in X$, there exists a unique $y \in Y$ such that $xRy$. We write $f(x) = y$ to mean $xRy$.
                \item A \textit{partial order} on $X$ is a relation which is reflexive, transitive, and antisymmetric. We write $x \leq y$ to mean $xRy$, and say that $X$ is an \textit{ordered set}.
            \end{enumerate}
    \end{definition}

    \begin{example}
        An equivalent relation on $X$ partitions $X$ into equivalence classes.
    \end{example}

    \begin{example}
        Recall that, if $\cC$ and $\cD$ are categories, then a \textit{covariant functor} $T:\cC \rightarrow \cD$ is a map such that:
            \begin{enumerate}[label = (\roman*),itemsep=1pt,topsep=3pt]
                \item If $A \in \obj(\cC)$, then $T(A) \in \obj(\cD)$.
                \item If $f \in \Hom_{\cC}(A,A')$, then $T(f) \in \Hom_{\cD}(T(A),T(A'))$.
                \item If $f \in \Hom_{\cC}(A,A')$ and $g \in \Hom_{\cC}(A',A'')$, then $T(f) \in \Hom_{\cD}(T(A),T(A'))$ and $T(g) \in \Hom_{\cD}(T(A'),T(A''))$. In particular:
                    \begin{equation*}
                    \begin{split}
                        T(gf) = T(g)T(f).
                    \end{split}
                    \end{equation*}
                \item For every $A \in \obj(\cC)$, $T(\id_A) = \id_{T(A)}$.
            \end{enumerate}
        A \textit{contravariant functor} $T:\cC \rightarrow \cD$ is similarly defined, but $(ii)$ and $(iii)$ are instead defined as:
            \begin{enumerate}[label = (\roman*),itemsep=1pt,topsep=3pt]
                \addtocounter{enumi}{1}
                \item If $f \in \Hom_{\cC}(A,A')$, then $T(f) \in \Hom_{\cD}(T(A'),T(A))$.
                \item If $f \in \Hom_{\cC}(A,A')$ and $g \in \Hom_{\cC}(A',A'')$, then $T(f) \in \Hom_{\cD}(T(A'),T(A))$ and $T(g) \in \Hom_{\cD}(T(A''),T(A'))$. In particular:
                    \begin{equation*}
                    \begin{split}
                        T(gf) = T(f)T(g).
                    \end{split}
                    \end{equation*}
            \end{enumerate}
        Given $f:X \rightarrow Y$, the power set gives rise to two functors, the \textit{contravariant power set functor} $\mathbf{Set}^{\text{op}} \rightarrow \mathbf{Set}$ and the \textit{covariant power set functor} $\mathbf{Set} \rightarrow \mathbf{Set}$. The first sends $f:X \rightarrow Y$ to the \textit{preimage} function $f^{-1}:\cP(Y) \rightarrow \cP(X)$, whereas the second sends $f$ to the \textit{image} function $f_\ast:\cP(X) \rightarrow \cP(Y)$. The preimage is more well-behaved, commuting with boolean operations:
            \begin{equation*}
            \begin{split}
                f^{-1}\left( E^c \right) &= \left( f^{-1}(E) \right)^c, \\
                f^{-1}\left( \bigcup_{\alpha \in I}E_\alpha \right) &= \bigcup_{\alpha \in I} f^{-1}(E_\alpha), \\
                f^{-1}\left( \bigcap_{\alpha \in I}E_\alpha \right) &= \bigcap_{\alpha \in I} f^{-1}(E_\alpha).  \\
            \end{split}
            \end{equation*}
        This is not necessarily the case for the image function.
    \end{example}

    \begin{example}
        Consider the directed acyclic graph:
            \begin{center}
            \begin{tikzpicture}[->,>=Stealth]
                \tikzstyle{vertex}=[circle,fill,inner sep=2pt]

                \node[vertex,label=above:A] (A) {};
                \node[vertex,label=above:B] (B) [right=2cm of A] {};
                \node[vertex,label=left:C] (C) [below left=1.2cm and 1cm of A] {};
                \node[vertex,label=right:D] (D) [below right=1.2cm and 1cm of A] {};
                \node[vertex,label=right:E] (E) [below=2.5cm of A] {};
                \node[vertex,label=right:F] (F) [right=2cm of E] {};
                \node[vertex,label=below:G] (G) [below=1.5cm of E] {};
                \node[vertex,label=below:H] (H) [below=1.5cm of F] {};

                \draw (A) -- (C);
                \draw (A) -- (D);
                \draw (B) -- (D);
                \draw (C) -- (E);
                \draw (D) -- (E);
                \draw (D) -- (F);
                \draw (E) -- (G);
                \draw (F) -- (H);
            \end{tikzpicture}
            \end{center}
        Let $G = \{A,B,C,D,E,F,G,H\}$ be the set of vertices of our graph. We can define a partial ordering on $G$ as follows: for any $x,y \in G$, we have $x \leq y$ if and only if there exists a directed path from $x$ to $y$. However, note that not every element of $G$ can be compared.
    \end{example}

    \begin{definition}
        An ordering on a set $X$ is said to be \textit{linear} (or \textit{total}) if for every $x,y \in X$, $x \leq y$ or $y \leq x$.
    \end{definition}

    \begin{definition}
        Let $X$ be an ordered set. Let $A \subseteq X$.
        \begin{enumerate}[label = (\arabic*),itemsep=1pt,topsep=3pt]
            \item $A$ is called \textit{bounded above} if there exists an element $u \in X$ with $a \leq u$ for all $a \in A$. Such a $u$ is called an \textit{upper bound} for $A$. The set of upper bounds of $A$ is denoted $\cU_A = \{u \in X \mid \text{$u$ is an upper bound of $A$}\}$.
            \item $A$ is called \textit{bounded below} if there exists an element $v \in X$ with $v \leq a$ for all $a \in A$. Such a $v$ is called a \textit{lower bound} for $A$. The set of lower bounds of $A$ is defined as $\mathscr{L}_A = \{v \in X \mid \text{$v$ is a lower bound of $A$}\}$.
            \item If $A$ admits an upper bound $u$ with $u \in A$, then $u$ is called \textit{the greatest element of $A$}.
            \item If $A$ admits a lower bound $v$ with $v \in A$, then $v$ is called \textit{the least element of $A$}.
            \item If $l$ is the least element of $\cU_A$, we write $l = \sup{(A)}$ and call it the \textit{supremum of $A$}.
            \item If $g$ is the greatest element of $\mathscr{L}_A$, we write $g = \inf{(A)}$ and call it the \textit{infimum of $A$}.
            \item A \textit{maximal element of $A$} is an element $m \in A$ such that if $a \geq m$, then $a = m$ (not necessarily unique).
            \item A \textit{minimal element of $A$} is an element $n \in A$ such that if $a \leq n$, then $a = n$ (not necessarily unique).
        \end{enumerate}
    \end{definition}

    \begin{definition}
        If a linear order on a set $X$ satisfies the property that every nonempty subset has a minimal element, then we say it is a \textit{well-ordering}.
    \end{definition}

    \begin{center}
        \begin{tikzpicture}
            \draw[thick] (0.3,0) -- (2.3,0);
            \node at (2.39, 0) {$/\,$};
            \node at (2.56, 0) {$/\,$};
            \draw[thick] (2.6,0) -- (4.6,0);
        \end{tikzpicture}
    \end{center}

    include stuff HERE 

    \begin{definition}
        
    \end{definition}


    do general cartesian product
    
    question is it nonempty?

    axiom of choice

    answer: no, axiom of choice says this choice function exists.

    'shoes vs. socks' formulation of the axiom

    \begin{center}
        \begin{tikzpicture}
            \draw[thick] (0.3,0) -- (2.3,0);
            \node at (2.39, 0) {$/\,$};
            \node at (2.56, 0) {$/\,$};
            \draw[thick] (2.6,0) -- (4.6,0);
        \end{tikzpicture}
    \end{center}


\end{document}