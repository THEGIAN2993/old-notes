\documentclass[11pt,twoside,openany]{memoir}
\usepackage[T1]{fontenc}
\usepackage[hidelinks]{hyperref}
\usepackage{amsmath}
\usepackage[fixamsmath]{mathtools}  % Extension to amsmath
\usepackage{amsthm}
\usepackage{amssymb}
%\renewcommand*{\mathbf}[1]{\varmathbb{#1}}


%fonts

\usepackage{datetime}
    \newdateformat{specialdate}{\THEYEAR\ \monthname\ \THEDAY}
\usepackage[margin=1in]{geometry}
\usepackage{fancyhdr}
\fancyhf{} % clear all headers/footers

% update marks for chapter and section
\renewcommand{\chaptermark}[1]{\markboth{#1}{}}
\renewcommand{\sectionmark}[1]{\markright{#1}}

% page style for normal pages
\pagestyle{fancy}
\cfoot{\scriptsize \thepage}
\fancyhead[L]{\scalebox{0.8}{Algebra Test Bank}}
\fancyhead[R]{\scalebox{0.8}{\leftmark, \rightmark}}

% page style for chapter starting pages (plain)
\fancypagestyle{plain}{%
    \fancyhf{}
    \cfoot{\scriptsize \thepage}
    \fancyhead[L]{\scalebox{0.8}{Algebra Test Bank}}
    \fancyhead[R]{\scalebox{0.8}{\leftmark, \rightmark}}
}
\usepackage{thmtools}
    \declaretheoremstyle[
        spaceabove=10pt,
        spacebelow=10pt,
        headfont=\normalfont\bfseries,
        notefont=\mdseries, notebraces={(}{)},
        bodyfont=\normalfont,
        postheadspace=0.5em
        %qed=\qedsymbol
        ]{defs}

    \declaretheoremstyle[ 
        spaceabove=10pt, % space above the theorem
        spacebelow=10pt,
        headfont=\normalfont\bfseries,
        bodyfont=\normalfont\itshape,
        postheadspace=0.5em
        ]{thmstyle}
    
    \declaretheorem[
        style=thmstyle,
    ]{theorem}

    \declaretheorem[
        style=thmstyle,
        sibling=theorem,
    ]{proposition}

    \declaretheorem[
        style=thmstyle,
    ]{lemma}

    \declaretheorem[
        style=thmstyle,
        sibling=theorem,
    ]{corollary}

    \declaretheorem[
        style=defs,
    ]{example}

    \declaretheorem[
        style=defs,
    ]{definition}

    \declaretheoremstyle[
    spaceabove=15pt,   % space above each exercise
    spacebelow=15pt,   % space below each exercise
    headfont=\normalfont\bfseries,
    notefont=\mdseries, notebraces={(}{)},
    bodyfont=\normalfont,
    postheadspace=0.5em
]{exerciseStyle}

\declaretheorem[
    style=exerciseStyle,
    sibling=theorem
]{exercise}

    \declaretheorem[
        numbered=unless unique,
        shaded={rulecolor=black,
    rulewidth=1pt, bgcolor={rgb}{1,1,1}}
    ]{axiom}

    \declaretheorem[numberwithin=section,style=defs]{note}
    \declaretheorem[numbered=no,style=defs]{question}
    \declaretheorem[numbered=no,style=defs]{recall}
    \declaretheorem[numbered=no,style=remark]{answer}
    \declaretheorem[numbered=no,style=remark]{solution}
    \declaretheorem[numbered=no,style=defs]{remark}
\usepackage{enumitem}
\usepackage{titlesec}
    \titleformat{\chapter}[display]
    {\bfseries\Huge\raggedright}
    {Chapter {\thechapter}}
    {1ex minus .1ex}
    {\Huge}
    \titlespacing{\chapter}
    {3pc}{*3}{15pt}[3pc]

    \titleformat{\section}[block]
    {\normalfont\bfseries\LARGE}
    {\S\ \thesection.}{.5em}{}[]
    \titlespacing{\section}
    {0pt}{3ex plus .1ex minus .2ex}{3ex plus .1ex minus .2ex}
\usepackage[utf8x]{inputenc}
\usepackage{tikz}
\usepackage{tikz-cd}
\usepackage{wasysym}
\linespread{1.00}
%%%%%%%%%%%%%%%%%%%%%%%%%%%%%%%%%%%%%%%%%%%%%%%%%%%%%%%%%%%%%
%%%%%%%%%%%%%%%%%%%%%%%%%%%%%%%%%%%%%%%%%%%%%%%%%%%%%%%%%%%%%
%to make the correct symbol for Sha
%\newcommand\cyr{%
%\renewcommand\rmdefault{wncyr}%
%\renewcommand\sfdefault{wncyss}%
%\renewcommand\encodingdefault{OT2}%
%\normalfont \selectfont} \DeclareTextFontCommand{\textcyr}{\cyr}


\DeclareMathOperator{\ab}{ab}
\newcommand{\absgal}{\G_{\bbQ}}
\DeclareMathOperator{\ad}{ad}
\DeclareMathOperator{\adj}{adj}
\DeclareMathOperator{\alg}{alg}
\DeclareMathOperator{\Alt}{Alt}
\DeclareMathOperator{\Ann}{Ann}
\DeclareMathOperator{\arith}{arith}
\DeclareMathOperator{\Aut}{Aut}
\DeclareMathOperator{\Be}{B}
\DeclareMathOperator{\Bd}{Bd}
\DeclareMathOperator{\card}{card}
\DeclareMathOperator{\Char}{char}
\DeclareMathOperator{\csp}{csp}
\DeclareMathOperator{\codim}{codim}
\DeclareMathOperator{\coker}{coker}
\DeclareMathOperator{\coh}{H}
\DeclareMathOperator{\compl}{compl}
\DeclareMathOperator{\conj}{conj}
\DeclareMathOperator{\cont}{cont}
\DeclareMathOperator{\crys}{crys}
\DeclareMathOperator{\Crys}{Crys}
\DeclareMathOperator{\cusp}{cusp}
\DeclareMathOperator{\diag}{diag}
\DeclareMathOperator{\diam}{diam}
\DeclareMathOperator{\Dom}{Dom}
\DeclareMathOperator{\disc}{disc}
\DeclareMathOperator{\dist}{dist}
\DeclareMathOperator{\dR}{dR}
\DeclareMathOperator{\Eis}{Eis}
\DeclareMathOperator{\End}{End}
\DeclareMathOperator{\ev}{ev}
\DeclareMathOperator{\eval}{eval}
\DeclareMathOperator{\Eq}{Eq}
\DeclareMathOperator{\Ext}{Ext}
\DeclareMathOperator{\Fil}{Fil}
\DeclareMathOperator{\Fitt}{Fitt}
\DeclareMathOperator{\Frob}{Frob}
\DeclareMathOperator{\G}{G}
\DeclareMathOperator{\Gal}{Gal}
\DeclareMathOperator{\GL}{GL}
\DeclareMathOperator{\Gr}{Gr}
\DeclareMathOperator{\Graph}{Graph}
\DeclareMathOperator{\GSp}{GSp}
\DeclareMathOperator{\GUn}{GU}
\DeclareMathOperator{\Hom}{Hom}
\DeclareMathOperator{\id}{id}
\DeclareMathOperator{\Id}{Id}
\DeclareMathOperator{\Ik}{Ik}
\DeclareMathOperator{\IM}{Im}
\DeclareMathOperator{\Image}{im}
\DeclareMathOperator{\Ind}{Ind}
\DeclareMathOperator{\Inf}{inf}
\DeclareMathOperator{\Isom}{Isom}
\DeclareMathOperator{\J}{J}
\DeclareMathOperator{\Jac}{Jac}
\DeclareMathOperator{\lcm}{lcm}
\DeclareMathOperator{\length}{length}
\DeclareMathOperator*{\limit}{limit}
\DeclareMathOperator{\Log}{Log}
\DeclareMathOperator{\M}{M}
\DeclareMathOperator{\Mat}{Mat}
\DeclareMathOperator{\N}{N}
\DeclareMathOperator{\Nm}{Nm}
\DeclareMathOperator{\NIk}{N-Ik}
\DeclareMathOperator{\NSK}{N-SK}
\DeclareMathOperator{\new}{new}
\DeclareMathOperator{\obj}{obj}
\DeclareMathOperator{\old}{old}
\DeclareMathOperator{\ord}{ord}
\DeclareMathOperator{\Or}{O}
\DeclareMathOperator{\op}{op}
\DeclareMathOperator{\PGL}{PGL}
\DeclareMathOperator{\PGSp}{PGSp}
\DeclareMathOperator{\rank}{rank}
\DeclareMathOperator{\Ran}{Ran}
\DeclareMathOperator{\Rel}{Rel}
\DeclareMathOperator{\Real}{Re}
\DeclareMathOperator{\RES}{res}
\DeclareMathOperator{\Res}{Res}
%\DeclareMathOperator{\Sha}{\textcyr{Sh}}
\DeclareMathOperator{\Sel}{Sel}
\DeclareMathOperator{\semi}{ss}
\DeclareMathOperator{\sgn}{sign}
\DeclareMathOperator{\SK}{SK}
\DeclareMathOperator{\SL}{SL}
\DeclareMathOperator{\SO}{SO}
\DeclareMathOperator{\Sp}{Sp}
\DeclareMathOperator{\Span}{span}
\DeclareMathOperator{\Spec}{Spec}
\DeclareMathOperator{\spin}{spin}
\DeclareMathOperator{\st}{st}
\DeclareMathOperator{\St}{St}
\DeclareMathOperator{\SUn}{SU}
\DeclareMathOperator{\supp}{supp}
\DeclareMathOperator{\Sup}{sup}
\DeclareMathOperator{\Sym}{Sym}
\DeclareMathOperator{\Tam}{Tam}
\DeclareMathOperator{\tors}{tors}
\DeclareMathOperator{\tr}{tr}
\DeclareMathOperator{\Tr}{Tr}
\DeclareMathOperator{\un}{un}
\DeclareMathOperator{\Un}{U}
\DeclareMathOperator{\val}{val}
\DeclareMathOperator{\vol}{vol}

\DeclareMathOperator{\Sets}{S \mkern1.04mu e \mkern1.04mu t \mkern1.04mu s}
    \newcommand{\cSets}{\scalebox{1.02}{\contour{black}{$\Sets$}}}
    
\DeclareMathOperator{\Groups}{G \mkern1.04mu r \mkern1.04mu o \mkern1.04mu u \mkern1.04mu p \mkern1.04mu s}
    \newcommand{\cGroups}{\scalebox{1.02}{\contour{black}{$\Groups$}}}

\DeclareMathOperator{\TTop}{T \mkern1.04mu o \mkern1.04mu p}
    \newcommand{\cTop}{\scalebox{1.02}{\contour{black}{$\TTop$}}}

\DeclareMathOperator{\Htp}{H \mkern1.04mu t \mkern1.04mu p}
    \newcommand{\cHtp}{\scalebox{1.02}{\contour{black}{$\Htp$}}}

\DeclareMathOperator{\Mod}{M \mkern1.04mu o \mkern1.04mu d}
    \newcommand{\cMod}{\scalebox{1.02}{\contour{black}{$\Mod$}}}

\DeclareMathOperator{\Ab}{A \mkern1.04mu b}
    \newcommand{\cAb}{\scalebox{1.02}{\contour{black}{$\Ab$}}}

\DeclareMathOperator{\Rings}{R \mkern1.04mu i \mkern1.04mu n \mkern1.04mu g \mkern1.04mu s}
    \newcommand{\cRings}{\scalebox{1.02}{\contour{black}{$\Rings$}}}

\DeclareMathOperator{\ComRings}{C \mkern1.04mu o \mkern1.04mu m \mkern1.04mu R \mkern1.04mu i \mkern1.04mu n \mkern1.04mu g \mkern1.04mu s}
    \newcommand{\cComRings}{\scalebox{1.05}{\contour{black}{$\ComRings$}}}

\DeclareMathOperator{\hHom}{H \mkern1.04mu o \mkern1.04mu m}
    \newcommand{\cHom}{\scalebox{1.02}{\contour{black}{$\hHom$}}}

         %  \item $\cGroups$
          %  \item $\cTop$
          %  \item $\cHtp$
          %  \item $\cMod$




\renewcommand{\k}{\kappa}
\newcommand{\Ff}{F_{f}}
%\newcommand{\ts}{\,^{t}\!}


%Mathcal

\newcommand{\cA}{\mathcal{A}}
\newcommand{\cB}{\mathcal{B}}
\newcommand{\cC}{\mathcal{C}}
\newcommand{\cD}{\mathcal{D}}
\newcommand{\cE}{\mathcal{E}}
\newcommand{\cF}{\mathcal{F}}
\newcommand{\cG}{\mathcal{G}}
\newcommand{\cH}{\mathcal{H}}
\newcommand{\cI}{\mathcal{I}}
\newcommand{\cJ}{\mathcal{J}}
\newcommand{\cK}{\mathcal{K}}
\newcommand{\cL}{\mathcal{L}}
\newcommand{\cM}{\mathcal{M}}
\newcommand{\cN}{\mathcal{N}}
\newcommand{\cO}{\mathcal{O}}
\newcommand{\cP}{\mathcal{P}}
\newcommand{\cQ}{\mathcal{Q}}
\newcommand{\cR}{\mathcal{R}}
\newcommand{\cS}{\mathcal{S}}
\newcommand{\cT}{\mathcal{T}}
\newcommand{\cU}{\mathcal{U}}
\newcommand{\cV}{\mathcal{V}}
\newcommand{\cW}{\mathcal{W}}
\newcommand{\cX}{\mathcal{X}}
\newcommand{\cY}{\mathcal{Y}}
\newcommand{\cZ}{\mathcal{Z}}


%mathfrak (missing \fi)

\newcommand{\fa}{\mathfrak{a}}
\newcommand{\fA}{\mathfrak{A}}
\newcommand{\fb}{\mathfrak{b}}
\newcommand{\fB}{\mathfrak{B}}
\newcommand{\fc}{\mathfrak{c}}
\newcommand{\fC}{\mathfrak{C}}
\newcommand{\fd}{\mathfrak{d}}
\newcommand{\fD}{\mathfrak{D}}
\newcommand{\fe}{\mathfrak{e}}
\newcommand{\fE}{\mathfrak{E}}
\newcommand{\ff}{\mathfrak{f}}
\newcommand{\fF}{\mathfrak{F}}
\newcommand{\fg}{\mathfrak{g}}
\newcommand{\fG}{\mathfrak{G}}
\newcommand{\fh}{\mathfrak{h}}
\newcommand{\fH}{\mathfrak{H}}
\newcommand{\fI}{\mathfrak{I}}
\newcommand{\fj}{\mathfrak{j}}
\newcommand{\fJ}{\mathfrak{J}}
\newcommand{\fk}{\mathfrak{k}}
\newcommand{\fK}{\mathfrak{K}}
\newcommand{\fl}{\mathfrak{l}}
\newcommand{\fL}{\mathfrak{L}}
\newcommand{\fm}{\mathfrak{m}}
\newcommand{\fM}{\mathfrak{M}}
\newcommand{\fn}{\mathfrak{n}}
\newcommand{\fN}{\mathfrak{N}}
\newcommand{\fo}{\mathfrak{o}}
\newcommand{\fO}{\mathfrak{O}}
\newcommand{\fp}{\mathfrak{p}}
\newcommand{\fP}{\mathfrak{P}}
\newcommand{\fq}{\mathfrak{q}}
\newcommand{\fQ}{\mathfrak{Q}}
\newcommand{\fr}{\mathfrak{r}}
\newcommand{\fR}{\mathfrak{R}}
\newcommand{\fs}{\mathfrak{s}}
\newcommand{\fS}{\mathfrak{S}}
\newcommand{\ft}{\mathfrak{t}}
\newcommand{\fT}{\mathfrak{T}}
\newcommand{\fu}{\mathfrak{u}}
\newcommand{\fU}{\mathfrak{U}}
\newcommand{\fv}{\mathfrak{v}}
\newcommand{\fV}{\mathfrak{V}}
\newcommand{\fw}{\mathfrak{w}}
\newcommand{\fW}{\mathfrak{W}}
\newcommand{\fx}{\mathfrak{x}}
\newcommand{\fX}{\mathfrak{X}}
\newcommand{\fy}{\mathfrak{y}}
\newcommand{\fY}{\mathfrak{Y}}
\newcommand{\fz}{\mathfrak{z}}
\newcommand{\fZ}{\mathfrak{Z}}


%mathbf
\newcommand{\bfA}{\mathbf{A}}
\newcommand{\bfB}{\mathbf{B}}
\newcommand{\bfC}{\mathbf{C}}
\newcommand{\bfD}{\mathbf{D}}
\newcommand{\bfE}{\mathbf{E}}
\newcommand{\bfF}{\mathbf{F}}
\newcommand{\bfG}{\mathbf{G}}
\newcommand{\bfH}{\mathbf{H}}
\newcommand{\bfI}{\mathbf{I}}
\newcommand{\bfJ}{\mathbf{J}}
\newcommand{\bfK}{\mathbf{K}}
\newcommand{\bfL}{\mathbf{L}}
\newcommand{\bfM}{\mathbf{M}}
\newcommand{\bfN}{\mathbf{N}}
\newcommand{\bfO}{\mathbf{O}}
\newcommand{\bfP}{\mathbf{P}}
\newcommand{\bfQ}{\mathbf{Q}}
\newcommand{\bfR}{\mathbf{R}}
\newcommand{\bfS}{\mathbf{S}}
\newcommand{\bfT}{\mathbf{T}}
\newcommand{\bfU}{\mathbf{U}}
\newcommand{\bfV}{\mathbf{V}}
\newcommand{\bfW}{\mathbf{W}}
\newcommand{\bfX}{\mathbf{X}}
\newcommand{\bfY}{\mathbf{Y}}
\newcommand{\bfZ}{\mathbf{Z}}

\newcommand{\bfa}{\mathbf{a}}
\newcommand{\bfb}{\mathbf{b}}
\newcommand{\bfc}{\mathbf{c}}
\newcommand{\bfd}{\mathbf{d}}
\newcommand{\bfe}{\mathbf{e}}
\newcommand{\bff}{\mathbf{f}}
\newcommand{\bfg}{\mathbf{g}}
\newcommand{\bfh}{\mathbf{h}}
\newcommand{\bfi}{\mathbf{i}}
\newcommand{\bfj}{\mathbf{j}}
\newcommand{\bfk}{\mathbf{k}}
\newcommand{\bfl}{\mathbf{l}}
\newcommand{\bfm}{\mathbf{m}}
\newcommand{\bfn}{\mathbf{n}}
\newcommand{\bfo}{\mathbf{o}}
\newcommand{\bfp}{\mathbf{p}}
\newcommand{\bfq}{\mathbf{q}}
\newcommand{\bfr}{\mathbf{r}}
\newcommand{\bfs}{\mathbf{s}}
\newcommand{\bft}{\mathbf{t}}
\newcommand{\bfu}{\mathbf{u}}
\newcommand{\bfv}{\mathbf{v}}
\newcommand{\bfw}{\mathbf{w}}
\newcommand{\bfx}{\mathbf{x}}
\newcommand{\bfy}{\mathbf{y}}
\newcommand{\bfz}{\mathbf{z}}

%blackboard bold

\newcommand{\bbA}{\mathbb{A}}
\newcommand{\bbB}{\mathbb{B}}
\newcommand{\bbC}{\mathbb{C}}
\newcommand{\bbD}{\mathbb{D}}
\newcommand{\bbE}{\mathbb{E}}
\newcommand{\bbF}{\mathbb{F}}
\newcommand{\bbG}{\mathbb{G}}
\newcommand{\bbH}{\mathbb{H}}
\newcommand{\bbI}{\mathbb{I}}
\newcommand{\bbJ}{\mathbb{J}}
\newcommand{\bbK}{\mathbb{K}}
\newcommand{\bbL}{\mathbb{L}}
\newcommand{\bbM}{\mathbb{M}}
\newcommand{\bbN}{\mathbb{N}}
\newcommand{\bbO}{\mathbb{O}}
\newcommand{\bbP}{\mathbb{P}}
\newcommand{\bbQ}{\mathbb{Q}}
\newcommand{\bbR}{\mathbb{R}}
\newcommand{\bbS}{\mathbb{S}}
\newcommand{\bbT}{\mathbb{T}}
\newcommand{\bbU}{\mathbb{U}}
\newcommand{\bbV}{\mathbb{V}}
\newcommand{\bbW}{\mathbb{W}}
\newcommand{\bbX}{\mathbb{X}}
\newcommand{\bbY}{\mathbb{Y}}
\newcommand{\bbZ}{\mathbb{Z}}
\newcommand{\jota}{\jmath}

\newcommand{\bmat}{\left( \begin{matrix}}
\newcommand{\emat}{\end{matrix} \right)}

\newcommand{\pmat}{\left( \begin{smallmatrix}}
\newcommand{\epmat}{\end{smallmatrix} \right)}

\newcommand{\lat}{\mathscr{L}}
\newcommand{\mat}[4]{\begin{pmatrix}{#1}&{#2}\\{#3}&{#4}\end{pmatrix}}
\newcommand{\ov}[1]{\overline{#1}}
\newcommand{\res}[1]{\underset{#1}{\RES}\,}
\newcommand{\up}{\upsilon}

\newcommand{\tac}{\textasteriskcentered}

%mahesh macros
\newcommand{\tm}{\textrm}

%Comments
\newcommand{\com}[1]{\vspace{5 mm}\par \noindent
\marginpar{\textsc{Comment}} \framebox{\begin{minipage}[c]{0.95
\textwidth} \tt #1 \end{minipage}}\vspace{5 mm}\par}

\newcommand{\Bmu}{\mbox{$\raisebox{-0.59ex}
  {$l$}\hspace{-0.18em}\mu\hspace{-0.88em}\raisebox{-0.98ex}{\scalebox{2}
  {$\color{white}.$}}\hspace{-0.416em}\raisebox{+0.88ex}
  {$\color{white}.$}\hspace{0.46em}$}{}}  %need graphicx and xcolor. this produces blackboard bold mu 

\newcommand{\hooktwoheadrightarrow}{%
  \hookrightarrow\mathrel{\mspace{-15mu}}\rightarrow
}

\makeatletter
\newcommand{\xhooktwoheadrightarrow}[2][]{%
  \lhook\joinrel
  \ext@arrow 0359\rightarrowfill@ {#1}{#2}%
  \mathrel{\mspace{-15mu}}\rightarrow
}
\makeatother

\renewcommand{\geq}{\geqslant}
\renewcommand{\leq}{\leqslant}
\newcommand{\midd}{\hspace{4pt}\middle|\hspace{4pt}}
    
    \newcommand{\bone}{\mathbf{1}}
    \newcommand{\sign}{\mathrm{sign}}
    \newcommand{\eps}{\varepsilon}
    \newcommand{\textui}[1]{\uline{\textit{#1}}}
    
    %\newcommand{\ov}{\overline}
    %\newcommand{\un}{\underline}
    \newcommand{\fin}{\mathrm{fin}}
    
    \newcommand{\chnum}{\titleformat
    {\chapter} % command
    [display] % shape
    {\centering} % format
    {\Huge \color{black} \shadowbox{\thechapter}} % label
    {-0.5em} % sep (space between the number and title)
    {\LARGE \color{black} \underline} % before-code
    }
    
    \newcommand{\chunnum}{\titleformat
    {\chapter} % command
    [display] % shape
    {} % format
    {} % label
    {0em} % sep
    { \begin{flushright} \begin{tabular}{r}  \Huge \color{black}
    } % before-code
    [
    \end{tabular} \end{flushright} \normalsize
    ] % after-code
    }

\newcommand{\nl}{\newline \mbox{}}

\newcommand{\h}[1]{\hspace{#1pt}}

\newcommand{\littletaller}{\mathchoice{\vphantom{\big|}}{}{}{}}
\newcommand\restr[2]{{% we make the whole thing an ordinary symbol
  \left.\kern-\nulldelimiterspace % automatically resize the bar with \right
  #1 % the function
  \littletaller % pretend it's a little taller at normal size
  \right|_{#2} % this is the delimiter
  }}

\newcommand{\mtext}[1]{\hspace{6pt}\text{#1}\hspace{6pt}}

\newcommand{\lnorm}{\left\lVert}
\newcommand{\rnorm}{\right\rVert}

\newcommand{\ds}{\displaystyle}
\newcommand{\ts}{\textstyle}

%This adds a "front cover" page.
%{\thispagestyle{empty}
%\vspace*{\fill}
%\begin{tabular}{l}
%\begin{tabular}{l}
%\includegraphics[scale=0.24]{oxy-logo.png}
%\end{tabular} \\
%\begin{tabular}{l}
%\Large \color{black} Module Theory, Linear Algebra, and Homological Algebra \\ \Large \color{black} Gianluca Crescenzo
%\end{tabular}
%\end{tabular}
%\newpage

\newcommand{\sfrac}[2]{{}^{#1}\mskip -5mu/\mskip -3mu_{#2}}


\makeatletter
\newcommand*{\da@rightarrow}{\mathchar"0\hexnumber@\symAMSa 4B }
\newcommand*{\da@leftarrow}{\mathchar"0\hexnumber@\symAMSa 4C }
\newcommand*{\xdashrightarrow}[2][]{%
  \mathrel{%
    \mathpalette{\da@xarrow{#1}{#2}{}\da@rightarrow{\,}{}}{}%
  }%
}
\newcommand{\xdashleftarrow}[2][]{%
  \mathrel{%
    \mathpalette{\da@xarrow{#1}{#2}\da@leftarrow{}{}{\,}}{}%
  }%
}
\newcommand*{\da@xarrow}[7]{%
  % #1: below
  % #2: above
  % #3: arrow left
  % #4: arrow right
  % #5: space left 
  % #6: space right
  % #7: math style 
  \sbox0{$\ifx#7\scriptstyle\scriptscriptstyle\else\scriptstyle\fi#5#1#6\m@th$}%
  \sbox2{$\ifx#7\scriptstyle\scriptscriptstyle\else\scriptstyle\fi#5#2#6\m@th$}%
  \sbox4{$#7\dabar@\m@th$}%
  \dimen@=\wd0 %
  \ifdim\wd2 >\dimen@
    \dimen@=\wd2 %   
  \fi
  \count@=2 %
  \def\da@bars{\dabar@\dabar@}%
  \@whiledim\count@\wd4<\dimen@\do{%
    \advance\count@\@ne
    \expandafter\def\expandafter\da@bars\expandafter{%
      \da@bars
      \dabar@ 
    }%
  }%  
  \mathrel{#3}%
  \mathrel{%   
    \mathop{\da@bars}\limits
    \ifx\\#1\\%
    \else
      _{\copy0}%
    \fi
    \ifx\\#2\\%
    \else
      ^{\copy2}%
    \fi
  }%   
  \mathrel{#4}%
}
\makeatother


\begin{document}

\begin{abstract}
    This is the Cal Poly Algebra Test Bank. Beginning with the September 2025 exam, all problems will be drawn from a public "problem bank." This bank contains two types of problems: template problems and pool problems. \textbf{Template problems} are generally computational with easily adjustable specifics. These types of problems are especially prevalent in linear algebra. \textbf{Pool problems} make up the rest of the problem bank, and include all problems that are not easily adjustable. These problems, when chosen, will usually be asked as is.
\end{abstract}
\begingroup
\let\clearpage\relax
\chapter*{Group Theory}

    \iffalse
    \begin{exercise}
        Let $G$ be a group. Prove that $G$ is non-cyclic if and only if $G$ is the union of its proper subgroups.
    \end{exercise} 
        {\color{blue} \begin{proof}
            ($\Rightarrow$) Suppose that $G$ is non-cyclic. Let $a \in G$. Since $G \neq \langle a \rangle$, there exists $a_2 \in G$ such that $a_2 \in G \setminus \langle a_1 \rangle$. If $G = \langle a_1 \rangle \cup \langle a_2 \rangle$, then we are done. If not, then find $a_3 \in G$ such that $a_3 \in G \setminus \left(\langle a_1 \rangle \cup \langle a_2 \rangle\right)$. Inductively, we have that $G = \bigcup_{i \in I}\langle a_i \rangle$

            ($\Leftarrow$) Let $\{H_i\}_{i \in I}$ be the family of proper subgroups of $G$. Suppose $G = \bigcup_{i \in I}H_i$. Let $a \in G$ be arbitrary. Then $a \in H_i$ for some $i \in I$. It must be the case that $\langle a \rangle \subseteq H_i$; i.e., $G \neq \langle a \rangle$. Thus $G$ is non-cyclic.
        \end{proof}}

    \begin{exercise}
        Let $G$ be a group, and $G\times G$ the direct product. The set $D=\{(g,g)\mid g\in G\}$ is a subgroup of $G\times G$. Prove that if $D$ is normal in $G\times G$ then $G$ is abelian.
    \end{exercise}
        {\color{blue} \begin{proof}
            If $D$ is normal in $G \times G$, then for all $(a,b) \in G$ and $(g,g) \in D$:
                \begin{equation*}
                \begin{split}
                    (a,b)(g,g)(a,b)^{-1} \in D
                    & \iff (aga^{-1},bgb^{-1}) \in D \\
                    & \iff aga^{-1} = bgb^{-1}.
                \end{split}
                \end{equation*}
            Take $g = b$. Then $aba^{-1} = bbb^{-1}$, which simplifies to $ab = ba$. Thus $G$ is abelian.
        \end{proof}}

    \begin{exercise}
        The dihedral group, $D_8$, is the group of eight rigid symmetries of a square. Prove that $D_8$ is not the internal direct product of two of its proper subgroups.
    \end{exercise}

    \begin{exercise}
        Let $G$ be a finite group and $H,K\mathrel{\unlhd}G$ be normal subgroups of relatively prime order. Prove that $G$ is isomorphic to a subgroup of $G/H\times G/K$.
    \end{exercise}

    \begin{exercise}
        Suppose $G$ is a group that contains normal subgroups $H,K\unlhd G$ with $H\cap K=\{e\}$ and $HK=G$. Prove that $G\cong H\times K$.
    \end{exercise}
        \begingroup\color{blue}
        \begin{proof}
            We will first show that if $H,K \unlhd G$ with $H\cap K=\{e\}$, then $hk = kh$ for all $h \in H$ and $k \in K$. Since $K \unlhd G$, $hkh^{-1} \in K$. Thus $hkh^{-1}k^{-1} \in K$. Since $H \unlhd G$, $kh^{-1}k^{-1} \in H$. Thus $hkh^{-1}k^{-1} \in H$. Since $hkh^{-1}k^{-1} \in H \cap K$, it must be the case that $hkh^{-1}k^{-1} = e_G$. Thus $hk = kh$.

            Define $\varphi:HK \rightarrow H \times K$ by $hk \mapsto (h,k)$. We can see $\varphi$ is a homomorphism since:
                \begin{equation*}
                \begin{split}
                    \varphi(h_1k_1 h_2 k_2)
                    & = \varphi(h_1 h_2 k_1 k_2) \\
                    & = (h_1 h_2 , k_1 k_2) \\
                    & = (h_1,k_1)(h_2,k_2) \\
                    & = \varphi(h_1k_1)\varphi(h_2 k_2).
                \end{split}
                \end{equation*}
            Now suppose $\varphi(h_1 k_1) = \varphi(h_2 k_2)$. Then $(h_1,k_1) = (h_2,k_2)$. It must be the case that $h_1 = h_2$ and $k_1 = k_2$. Thus $h_1k_1 = h_2k_2$; i.e., $\varphi$ is injective. Furthermore, given any $(h,k) \in H \times K$, we have $\varphi(hk) = (h,k)$, hence $\varphi$ is surjective. Thus $\varphi$ is an isomorphism, so $HK = G \cong H \times K$. \qedhere


        \end{proof}\endgroup
    \begin{exercise}
        Let $G$ be the group of upper-triangular real matrices $\bmat a & b \\ 0 & d\emat$ with $a,d\neq 0$, under matrix multiplication. Let $S$ be the subset of $G$ defined by $d=1$. Show that $S$ is normal and that $G/S\cong {\bfR}^{\times}$, the multiplicative group of nonzero real numbers.
    \end{exercise}

    \begin{exercise}
        Let $G$ be a group and suppose $\operatorname{Aut}(G)$ is trivial.
            \begin{enumerate}[label=(\alph*)]
                \item Show that $G$ is abelian.
                \item Show that for any abelian group $H$, the {\bfseries inversion map} $\phi(h)=h^{-1}$ is an automorphism.
                \item Use parts (a) and (b) above to show that $g^2$ is the identity element for every $g\in G$.
            \end{enumerate}
    \end{exercise}
        {\color{blue} \begin{proof}
            (a) Let $h \in G$ be arbitrary. Define $\varphi_h:G \rightarrow G$ by $\varphi_h(g) = hgh^{-1}$. Let $g_1,g_2 \in G$. The map $\varphi_h$ is a homomorphism since:
                \begin{equation*}
                \begin{split}
                    \varphi_h(g_1 g_2)
                    & = h g_1 g_2 h^{-1} \\
                    & = h g_1 h^{-1} h g_2 h^{-1} \\
                    & = \varphi(g_1)\varphi(g_2).
                \end{split}
                \end{equation*}
            Now suppose $\varphi(g_1) = \varphi(g_2)$. Then $hg_1h^{-1} = hg_2h^{-1}$, which simplifies to $g_1 = g_2$. Hence $\varphi$ is injective. Furthermore, given $g \in G$, $\varphi(h^{-1}gh) = hh^{-1}ghh^{-1} = g$; i.e., $\varphi$ is surjective. Thus $\varphi_h \in \Aut(G)$. Since $\Aut(G)$ is trivial, $\varphi_h$ must be the identity map. Hence $g = hgh^{-1}$, or equivalently $gh = hg$. Thus $G$ is abelian.

            (b) The map $\Phi:H \rightarrow H$ defined by $h \mapsto h^{-1}$ is a homomorphism since:
                \begin{equation*}
                \begin{split}
                    \Phi(h_1 h_2)
                    & = (h_1h_2)^{-1} \\
                    & = h_2^{-1}h_1^{-1} \\
                    & = h_1^{-1}h_2^{-1}.
                \end{split}
                \end{equation*}
            Furthermore, if $\Phi(h_1) = \Phi(h_2)$, then $h_1^{-1} = h_2^{-1}$ Equivalently, $h_1^{-1}h_2 = e_H$, which yields $h_2 = h_1$. Thus $\Phi$ is injective. Furthermore, for any $h \in H$, $\Phi(h^{-1}) = (h^{-1})^{-1} = h$. Thus $\Phi \in \Aut(H)$.

            (c) Let $\Phi:G \rightarrow G$ be defined by $\Phi(g) = g^{-1}$. Since $G$ is abelian, part (b) gives $\Phi \in \Aut(G)$. Since $\Aut(G)$ is trivial, $\Phi$ is the identity map. Hence $g = g^{-1}$, which is equivalent to $g^2 = e_G$
        \end{proof}}

    \begin{exercise}
        \phantom{a}
        \begin{enumerate}[label=(\alph*)]
            \item Suppose $N$ is a normal subgroup of a group $G$ and $\pi_N:G\to G/N$ is the usual projection homomorphism, defined by $\pi_N(g)=gN$. Prove that if $\phi:G\to H$ is any homomorphism with $N\leq \ker(\phi)$, then there exists a unique homomorphism $\psi:G/N\to H$ such that $\phi = \psi\circ \pi_N$. (You must explicitly define $\psi$, show it is well-defined, show $\phi=\psi\circ\pi_N$, and show that $\psi$ is uniquely determined.)
            \item Prove the Third Isomorphism Theorem: if $M, N\unlhd G$ with $N\leq M$, then $(G/N)/(M/N)\cong G/M$.
        \end{enumerate}
    \end{exercise}

    \begin{exercise}
        Let $G$ be a group and $a \in G$ be an element. Let $n \in \bfN$ be the smallest positive number such that $a^n = e$, where $e$ is the identity element. Show that the set $\{e,a,a^2,...,a^{n-1}\}$ contains no repetitions.
    \end{exercise}
        {\color{blue} \begin{proof}
            Suppose $a^i = a^j$ where $0 \leq i < n$ and $0 \leq j < n$. Then $a^{i-j} = e$. Since $0 \leq i < n$ and $0 \leq j < n$, it must be that $i-j < n$. But since $n$ is the smallest positive number such that $a^n = e$, it must be the case that $i-j = 0$. Hence $i = j$.
        \end{proof}}

    \begin{exercise}
        Let $G$ be a finite abelian group of odd order. Let $\phi:G\to G$ be the function defined by $\phi(g)=g^2$ for all $g\in G$. Prove that $\phi$ is an automorphism.
    \end{exercise}
        {\color{blue} \begin{proof}
            Let $g,h \in G$. The map $\phi$ is a homomorphism since:
                \begin{equation*}
                \begin{split}
                    \phi(gh)
                    & = (gh)^2 \\
                    & = g^2 h^2 \h9\text{\tiny Since $G$ is abelian} \\
                    & = \phi(g)\phi(h).
                \end{split}
                \end{equation*}
            Now suppose $\phi(g) = \phi(h)$. Then $g^2 = h^2$, or equivalently $(gh^{-1})^2 = e_G$. Since $G$ is of odd order, it must be the case that $gh^{-1} = e_G$; i.e., $g = h$. Thus $\phi$ is injective.

            Suppose the order of $G$ is $2k-1$ for some $k \geq 1$. Let $g \in G$. Then $g^k \in G$. Observe that $\phi(g^k) = g^{2k} = g g^{2k-1} = g$, hence $\phi$ is surjective. Thus $\phi \in \Aut(G)$.
        \end{proof}}

    \begin{exercise}
        Let $\bfZ/n\bfZ$ denote the cyclic group of order $n$. Suppose $m \in \bfN$ is relatively prime to $n$. Define the function $\mu_m:\bfZ/n\bfZ \rightarrow \bfZ/n\bfZ$ by $\mu([a]_n) = [ma]_n$.
        \begin{enumerate}[label = (\arabic*),itemsep=1pt,topsep=3pt]
            \item Prove that the map $\mu_m$ is a well-defined automorphism of $\bfZ/n\bfZ$.
            \item Prove that any automorphism of $\bfZ/n\bfZ$ has the form $\mu_m$ for some $m$.
        \end{enumerate}
    \end{exercise}

    \begin{exercise}
        For a group $G$ and an element $g \in G$, the \textit{centralizer} of $g$ in $G$ is the subgroup:
            \begin{equation*}
            \begin{split}
                C_G(g) = \{h \in G \mid hgh^{-1} = g\}.
            \end{split}
            \end{equation*}
        We say $g$ and $g'$ are \textit{conjugate in $G$} if there exists an element $h \in G$ such that $g' = hgh^{-1}$.

        Suppose $S_n$ is a symmetric group with $n \geq 4$, and $\sigma$ is one of the $(n-2)$-cycles in $S_n$ (There are $\frac{n!}{2(n-2)}$ such cycles). 
            \begin{enumerate}[label = (\roman*),itemsep=1pt,topsep=3pt]
                \item Prove that $[S_n : C_{S_n}(\sigma)] = [A_n:C_{A_n}(\sigma)]$.
                \item Determine whether all $(n-2)$-cycles are conjugate in $A_n$.
            \end{enumerate}
    \end{exercise}

    \begin{exercise}
        Let $G$ be a finite group and $n > 1$ an integer such that $(ab)^n = a^n b^n$ for all $a,b \in G$. Let 
            \begin{equation*}
            \begin{split}
                G_n = \{c \in G \mid c^n = e\}, \h9 G^n = \{c^n \mid c \in G\}.
            \end{split}
            \end{equation*}
        You may take for granted that these are subgroups. Prove that $G_n$ and $G^n$ are normal in $G$, and $|G^n| = [G : G_n]$.
    \end{exercise}
        {\color{blue} \begin{proof}
            Let $g \in G$ and $c \in G_n$. Observe that:
                \begin{equation*}
                \begin{split}
                    (gcg^{-1})^n 
                    & = g^n c^n (g^{-1})^n \\
                    & = g^n (g^{-1})^n \\
                    & = (g g^{-1})^n \\
                    & = e.
                \end{split}
                \end{equation*}
            Thus $gcg^{-1} \in G_n$, so $G_n \unlhd G$. Let $g \in G$ and $h \in G^n$. Then $h = c^n$ for some $c \in G$. Observe that:
                \begin{equation*}
                \begin{split}
                    ghg^{-1} 
                    & = gc^n g^{-1} \\
                    & = g\underbrace{c...c}_{\text{$n$ times}}g^{-1} \\
                    & = \underbrace{(gcg^{-1})(gcg^{-1})...(gcg^{-1})}_{\text{$n$ times}} \\
                    & = (gcg^{-1})^n \\
                \end{split}
                \end{equation*}
            Thus $ghg^{-1} \in G^n$, so $G^n \unlhd G$.

            Define $\varphi:G \rightarrow G$ by $\varphi(g) = g^n$. The First Isomorphism Theorem says that $G/\ker{\varphi} \cong \Image{\varphi}$. Observe that:
                \begin{equation*}
                \begin{split}
                    \ker \varphi
                    & = \{g \in G \mid \varphi(g) = e\} \\
                    & = \{g \in G \mid g^n = e\} \\
                    & = G_n. \\
                    &\phantom = \\
                    \Image \varphi &= \{\varphi(g) \mid g \in G\} \\
                    & = \{g^n \mid g \in G\} \\
                    & = G^n.
                \end{split}
                \end{equation*}
            Thus $G/G_n \cong G^n$. Hence $|G^n| = |G/G_n| = [G:G_n]$.
        \end{proof}}

    \begin{exercise}
        Suppose $G$ is a group, $H \leq G$ a subgroup, and $a,b \in G$. Prove that the following are equivalent:
            \begin{enumerate}[label = (\arabic*),itemsep=1pt,topsep=3pt]
                \item $aH = bH$;
                \item $b \in aH$;
                \item $b^{-1}a \in H$.
            \end{enumerate}
    \end{exercise}
        {\color{blue} \begin{proof}
            (1)$\Rightarrow$(2) Note that $b = b e_G \in bH$. Since $aH = bH$, $b \in aH$.

            (2)$\Rightarrow(3)$ If $b \in aH$, then $b = ah$ for some $h \in H$. Equivalently, $h^{-1} = b^{-1}a$. Since $h^{-1} \in H$, $b^{-1}a \in H$.

            (3)$\Rightarrow$(1) First, note that $b^{-1}a \in H$ implies $b^{-1}a = h$ for some $h \in H$. Then $(b^{-1}a)^{-1} = h^{-1}$, which simplifies to $a^{-1}b = h^{-1}$. Hence $a^{-1}b \in H$. Let $x \in aH$. Then $x = ah$ for some $h \in H$. This is equivalent to $x = bb^{-1}ah$. Since $b^{-1}ah \in H$, $x \in bH$. Now let $y \in bH$. Then $y = bh$ for some $h \in H$. This is equivalent to $y = aa^{-1}bh$. Since $a^{-1}bh \in H$, $y \in aH$. Thus $aH = bH$.
        \end{proof}}
    
    --

    \begin{exercise}
        Suppose $G$ is a nontrivial finite group and $H,K \unlhd G$ are normal subgroups with $\gcd \left( |H|,|K| \right) = 1$.
            \begin{enumerate}[label = (\arabic*),itemsep=1pt,topsep=3pt]
                \item Define a nontrivial group homomorphism $\phi:G \rightarrow G/H \times G/K$.
                \item Prove $G$ is isomorphic to a subgroup of $G/H \times G/K$
                \item Suppose $\gcd(m,n) = 1$. Prove $\bfZ/mn \bfZ \cong \bfZ/m\bfZ \times \bfZ/n\bfZ$.
            \end{enumerate}
    \end{exercise}
        {\color{blue} \begin{proof}
            (1) Define $\phi:G \rightarrow G/H \rightarrow G/K$ by $\phi(g) = (gH,gK)$. \nl
            
            \noindent (2) Suppose $x \in H \cap K$. Then $x \in H$ and $x \in K$. So $\div{o(x)}{|H|}$ and $\div{o(x)}{|K|}$. This means $\div{o(x)}{\gcd(|H|,|K|)}$. Since $\gcd(|H|,|K|) = 1$, it must be the case that $o(x) = 1$; i.e., $x = e$. Thus $H \cap K = \{e\}$.
            
            By the First Isomorphism Theorem, $G/\ker\phi \cong \Image \phi$. Observe that:
                \begin{equation*}
                \begin{split}
                    \ker \phi  
                    & = \{g \in G \mid \phi(g) = (H,K)\} \\
                    & = \{g \in G \mid (gH,gK) = (H,K)\} \\
                    & = \{g \in G \mid g \in H, g \in K\} \\
                    & = H \cap K \\
                    & = \{e\}.
                \end{split}
                \end{equation*}
            Since $\ker \phi$ is trivial and $\Image \phi \leq G/H \times G/K$, $G$ is isomorphic to a subgroup of $G/H \times G/K$. \nl
            
            \noindent (3) Define $\varphi : \bfZ \rightarrow \bfZ/m\bfZ \times \bfZ/n\bfZ$ by $x \mapsto ([x]_m,[x]_n)$. This is a homomorphism since:
                \begin{equation*}
                \begin{split}
                    \varphi(x+y)
                    & = ([x+y]_m,[x+y]_n) \\
                    & = ([x]_m + [y]_m,[x]_n + [y]_n) \\
                    & = ([x]_m,[x]_n) + ([y]_m,[y]_n) \\
                    & = \varphi(x) + \varphi(y).
                \end{split}
                \end{equation*}
            Let $([a]_m,[b]_n) \in \bfZ/m\bfZ \times \bfZ/n\bfZ$. Since $\gcd(m,n)=1$, by the Chinese Remainder Theorem, for all $a,b \in \bfZ$ with $0 \leq a < n$ and $0 \leq b < m$, there exists a unique $x \in \bfZ$ such that $x \equiv a \pmod{m}$ and $x \equiv b \pmod{n}$. This is equivalent $x = a + sm$ and $x = b + tn$ for some $s,t \in \bfZ$. Using these equations for $x$, we can see:
                \begin{equation*}
                \begin{split}
                    \varphi(x)
                    & = ([x]_m,[x]_n) \\
                    & = ([a+sm]_m,[b+tn]_n) \\
                    & = ([a]_m,[b]_n).
                \end{split}
                \end{equation*}
            Whence $\varphi$ is surjective. Furthermore:
                \begin{equation*}
                \begin{split}
                    \ker \phi 
                    & = \{x \in \bfZ \mid \varphi(x) = ([0]_m,[0]_n) \} \\
                    & = \{x \in \bfZ \mid ([x]_m,[x]_n) = ([0]_m,[0]_n)\} \\
                    & = \{x \in \bfZ \mid [x]_m = [0]_m, [x]_n = [0]_n\} \\
                    & = \{x \in \bfZ \mid x \equiv 0\pmod{m}, x \equiv 0 \pmod{n} \} \\
                    & = \{x \in \bfZ \mid x = sm, x = tn, s,t \in \bfZ \} \\
                    & = \{x \in \bfZ \mid \div{m}{x}, \div{n}{x} \} \\
                    & = \{x \in \bfZ \mid \div{mn}{x} \} \h9\text{\tiny Since $\gcd(m,n) = 1$} \\
                    & = \{x \in \bfZ \mid x = (mn)s, s \in \bfZ\} \\
                    & = mn\bfZ.
                \end{split}
                \end{equation*}
            By the First Isomorphism Theorem, $\bfZ/mn\bfZ \cong \bfZ/m\bfZ \times \bfZ/n\bfZ$.
        \end{proof}}

    ---

    \begin{exercise}
        Let $G$ be a finite group and $Z(G)$ denote its center.
        \begin{enumerate}[label = (\arabic*),itemsep=1pt,topsep=3pt]
            \item Prove that if $G/Z(G)$ is cyclic, then $G$ is abelian.
            \item Prove that if $G$ is nonabelian, then $|Z(G)| \leq \frac{1}{4}|G|$.
        \end{enumerate}
    \end{exercise}
        {\color{blue} \begin{proof}
            (1) If $G/Z(G)$ is cyclic, there exists $aZ(G) \in G/Z(G)$ such that $\langle aZ(G) \rangle = G/Z(G)$. Let $gZ(G) \in G/Z(G)$. Then $gZ(G) = a^{k_1}Z(G)$ for some $k_1 \in \bfZ$. Equivalently, $(a^{k_1})^{-1}g \in Z(G)$, so there exists $z_1 \in Z(G)$ such that $(a^{k_1})^{-1}g = z_1$. Thus $g = z_1a^{k_1}$. Similarly for $hZ(G) \in G/Z(G)$, there exists $k_2 \in \bfZ$ and $z_2 \in Z(G)$ so that $h = z_2a^{k_2}$. Observe that:
                \begin{equation*}
                \begin{split}
                    gh 
                    & = z_1a^{k_1}z_2a^{k_2} \\
                    & = z_1 a^{k_1}a^{k_2}z_2 \\
                    & = z_1 a^{k_1 + k_2}z_2 \\
                    & = z_2 a^{k_1 + k_2}z_1 \\
                    & = z_2 a^{k_2}a^{k_1}z_1 \\
                    & = z_2a^{k_2}z_1a^{k_1} \\
                    & = hg.
                \end{split}
                \end{equation*}
            Thus $G$ is abelian. \nl
            
            \noindent (2) By Lagrange's Theorem, $4 \leq \frac{|G|}{|Z(G)|} = |G/Z(G)|$. Suppose towards contradiction $G$ is nonabelian and $|G/Z(G)| < 4$. If $|G/Z(G)| = 1$, then $G/Z(G)$ is trivial, so $G/Z(G) = Z(G)$. That means for every $g \in G$, $gZ(G) = Z(G)$, hence $g \in Z(G)$. But because $G = Z(G)$, $G$ must be abelian, which is a contradiction. For the cases where $|G/Z(G)| = 2$ and $|G/Z(G)| = 3$, note that $G/Z(G)$ will be cyclic since their orders are prime. Since $G/Z(G)$ is cyclic, $G$ is abelian, which again is a contradiction. It must be the case that $|G/Z(G)| \geq 4$; i.e., $|Z(G)| \leq \frac{1}{4}|G|$.
        \end{proof}}


    ---

    \begin{exercise}
        Let $G$ be a group, $m \in \bfN$, and $g \in G$ be an element such that $g^m = e$. Prove that $o(g) \mid m$, where $o(g)$ is the order of $g$.
    \end{exercise}
        {\color{blue} \begin{proof}
            Suppose $o(g) = k$. Then $g^k = e$. Apply the division algorithm to $m$ and $k$. There exists $q,r \in \bfZ$ so that $m = kq + r$, $0 \leq r < k$. This means:
                \begin{equation*}
                \begin{split}
                    e
                    & = g^m \\ 
                    & = g^{kq+r} \\
                    & = g^{kq}g^r \\
                    & = g^r.
                \end{split}
                \end{equation*}
            But since $o(g) = k$ and $0\leq r < k$, it must be the case that $r = 0$. Thus $m = kq$; i.e., $o(g)\mid m$.
        \end{proof}}
    \fi
%%%%%%%%%%%%%%%%%%%%%%%%%%%%%%%%%%%
%%%%%%%%%%%%%%%%%%%%%%%%%%%%%%%%%%%
%%%%%%%%%%%%%%%%%%%%%%%%%%%%%%%%%%%
%%%%%%%%%%%%%%%%%%%%%%%%%%%%%%%%%%%
%%%%%%%%%%%%%%%%%%%%%%%%%%%%%%%%%%%
%%%%%%%%%%%%%%%%%%%%%%%%%%%%%%%%%%%
%%%%%%%%%%%%%%%%%%%%%%%%%%%%%%%%%%%
%%%%%%%%%%%%%%%%%%%%%%%%%%%%%%%%%%%
%%%%%%%%%%%%%%%%%%%%%%%%%%%%%%%%%%%
%%%%%%%%%%%%%%%%%%%%%%%%%%%%%%%%%%%
%%%%%%%%%%%%%%%%%%%%%%%%%%%%%%%%%%%
%%%%%%%%%%%%%%%%%%%%%%%%%%%%%%%%%%%
%%%%%%%%%%%%%%%%%%%%%%%%%%%%%%%%%%%
%%%%%%%%%%%%%%%%%%%%%%%%%%%%%%%%%%%
%%%%%%%%%%%%%%%%%%%%%%%%%%%%%%%%%%%
%%%%%%%%%%%%%%%%%%%%%%%%%%%%%%%%%%%
%%%%%%%%%%%%%%%%%%%%%%%%%%%%%%%%%%%
%%%%%%%%%%%%%%%%%%%%%%%%%%%%%%%%%%%
    \section*{\S\ Pool Problems}
    \markboth{Group Theory}{Pool Problems}
    \endgroup

    \begin{exercise}
        Let $G$ be a group. Prove that $G$ is non-cyclic if and only if $G$ is the union of its proper subgroups.
        \end{exercise}
        
        \begin{exercise}
        Let $G$ be a group, and $G\times G$ the direct product. The set $D=\{(g,g)\mid g\in G\}$ is a subgroup of $G\times G$. Prove that if $D$ is normal in $G\times G$ then $G$ is abelian.
        \end{exercise}
        
        \begin{exercise}
        The dihedral group, $D_8$, is the group of eight rigid symmetries of a square. Prove that $D_8$ is not the internal direct product of two of its proper subgroups.
        \end{exercise}
        
        \begin{exercise}
        Let $G$ be a finite group and $H,K\mathrel{\unlhd}G$ be normal subgroups of relatively prime order. Prove that $G$ is isomorphic to a subgroup of $G/H\times G/K$.
        \end{exercise}
        
        \begin{exercise}
        Suppose $G$ is a group that contains normal subgroups $H,K\unlhd G$ with $H\cap K=\{e\}$ and $HK=G$. Prove that $G\cong H\times K$.
        \end{exercise}
        
        \begin{exercise}
        Let $G$ be the group of upper-triangular real matrices 
        \begin{equation*}
        \begin{split}
        \begin{bmatrix} a & b \\ 0 & d\end{bmatrix}
        \end{split}
        \end{equation*}
        with $a,d\neq 0$, under matrix multiplication. Let $S$ be the subset of $G$ defined by $d=1$. Show that $S$ is normal and that $G/S\cong \mathbb{R}^{\times}$, the multiplicative group of nonzero real numbers.
        \end{exercise}
        
        \begin{exercise}
        Let $G$ be a group and suppose $\operatorname{Aut}(G)$ is trivial.
        \begin{enumerate}[label=(\alph*)]
            \item Show that $G$ is abelian.
            \item Show that for any abelian group $H$, the \textbf{inversion map} $\phi(h)=h^{-1}$ is an automorphism.
            \item Use parts (a) and (b) above to show that $g^2$ is the identity element for every $g\in G$.
        \end{enumerate}
        \end{exercise}
        
        \newpage
        \begin{exercise}
            \phantom{a}
        \begin{enumerate}[label=(\alph*)]
            \item Suppose $N$ is a normal subgroup of a group $G$ and $\pi_N:G\to G/N$ is the usual projection homomorphism, defined by $\pi_N(g)=gN$. Prove that if $\phi:G\to H$ is any homomorphism with $N\leq \ker(\phi)$, then there exists a unique homomorphism $\psi:G/N\to H$ such that $\phi = \psi\circ \pi_N$. (You must explicitly define $\psi$, show it is well defined, show $\phi=\psi\circ\pi_N$, and show that $\psi$ is uniquely determined.)
            \item Prove the 
            \textbf{Third Isomorphism Theorem}: if $M, N\unlhd G$ with $N\le M$, then $(G/N)/(M/N)\cong G/M$.
        \end{enumerate}
        \end{exercise}
        
        \begin{exercise}
        Let $G$ be a group and $a\in G$ be an element. Let $n\in \mathbb{N}$ be the smallest positive number such that $a^n=e$, where $e$ is the identity element. Show that the set
        \begin{equation*}
        \begin{split}
        \{e,a,a^2,\ldots, a^{n-1}\}
        \end{split}
        \end{equation*}
        contains no repetitions.
        \end{exercise}
        
        \begin{exercise}
        Let $G$ be a finite abelian group of odd order. Let $\phi:G\to G$ be the function defined by $\phi(g)=g^2$ for all $g\in G$. Prove that $\phi$ is an automorphism.
        \end{exercise}
        
        \begin{exercise}
        Let $G$ be a group with exactly two conjugacy classes. Prove that $G$ is abelian, and describe all such groups (with proof).
        \end{exercise}
        
        \begin{exercise}
        Let $\mathbb{Z}_n$ denote the cyclic group of order $n$. Suppose $m\in \mathbb{N}$ is relatively prime to $n$. Define the function $\mu_m:\mathbb{Z}_n\to \mathbb{Z}_n$ by $m[a]_n=[ma]_n$.
        \begin{enumerate}[label=(\alph*)]
            \item Prove that the map $\mu_m$ is a well-defined automorphism of $\mathbb{Z}_n$.
            \item Prove that any automorphism of $\mathbb{Z}_n$ has the form $\mu_m$ for some $m$.
        \end{enumerate}
        \end{exercise}
        
        \begin{exercise}
        For a group $G$ and an element $g\in G$, the \textbf{centralizer} of $g$ in $G$ is the subgroup
        \begin{equation*}
        \begin{split}
        C_G(g)=\{h\in G:hgh^{-1}=g\}.
        \end{split}
        \end{equation*}
        We say $g$ and $g'$ are \textbf{conjugate in $G$} if there exists an element $h\in G$ such that $g'=hgh^{-1}$.
        
        Suppose $S_n$ is a symmetric group with $n\ge 4$, and $\sigma$ is one of the $(n-2)$-cycles in $S_n$. 
        \begin{enumerate}[label=(\alph*)]
            \item Prove that $[S_n:C_{S_n}(\sigma)]=[A_n:C_{A_n}(\sigma)]$.
            \item Determine whether all $(n-2)$-cycles are conjugate in $A_n$.
        \end{enumerate}
        \end{exercise}
        
        \begin{exercise}
        Let $G$ be a finite group and $n>1$ an integer such that $(ab)^n=a^n b^n$ for all $a,b\in G$. Let
        \begin{equation*}
        \begin{split}
        G_n=\{c\in G\mid c^n=e\},\qquad G^n=\{c^n\mid c\in G\}.
        \end{split}
        \end{equation*}
        You may take for granted that these are subgroups. Prove that both $G_n$ and $G^n$ are normal in $G$, and $|G^n|=[G:G_n]$.
        \end{exercise}
        
        \begin{exercise}
        Suppose $G$ is a group, $H\le G$ a subgroup, and $a,b\in G$. Prove that the following are equivalent:
        \begin{enumerate}[label=(\alph*)]
            \item $aH=bH$
            \item $b\in aH$
            \item $b^{-1}a\in H$
        \end{enumerate}
        \end{exercise}
        
        \begin{exercise}
        Let $G$ be a group, and let $\operatorname{Aut}(G)$ denote the group of automorphisms of $G$. There is a homomorphism $\gamma:G\to \operatorname{Aut}(G)$ that takes $s\in G$ to the automorphism $\gamma_s$ defined by $\gamma_s(t)=sts^{-1}$.
        \begin{enumerate}[label=(\alph*)]
            \item Prove rigorously, possibly with induction, that if $\gamma_s(t)=t^b$, then $\gamma_{s^n}(t)=t^{b^n}$.
            \item Suppose $s\in G$ has order 5, and $sts^{-1}=t^2$. Find the order of $t$. Justify your answer.
        \end{enumerate}
        \end{exercise}
        
        \begin{exercise}
        Let $G$ be an abelian group and $G_T$ be the set of elements of finite order in $G$.
        \begin{enumerate}[label=(\alph*)]
            \item Prove that $G_T$ is a subgroup of $G$.
            \item Prove that every non-identity element of $G/G_T$ has infinite order.
            \item Characterize the elements of $G_T$ when $G=\mathbb{R}/\mathbb{Z}$, where $\mathbb{R}$ is the additive group of real numbers.
        \end{enumerate}
        \end{exercise}
        
        \begin{exercise}
        Suppose $G$ is a finite group of even order.
        \begin{enumerate}[label=(\alph*)]
            \item Prove that an element in $G$ has order dividing 2 if and only if it is its own inverse.
            \item Prove that the number of elements in $G$ of order 2 is odd.
            \item Use (b) to show $G$ must contain a subgroup of order 2.
        \end{enumerate}
        \end{exercise}
        
        \begin{exercise}
        Let $N$ be a finite normal subgroup of $G$. Prove there is a normal subgroup $M$ of $G$ such that $[G:M]$ is finite and $nm=mn$ for all $n\in N$ and $m\in M$.
        
        \medskip
        \noindent ({Hint:} You may use the fact that the centralizer $C(h):=\{g\in G\mid ghg^{-1}=h\}$ is a subgroup of $G$.)
        \end{exercise}
        
        \begin{exercise}
        Show that every finite group with more than two elements has a nontrivial automorphism.
        \end{exercise}
        
        \begin{exercise}
        Suppose $G_1$ and $G_2$ are groups, with identity elements $e_1$ and $e_2$, respectively. Prove that if $\phi:G_1\to G_2$ is an isomorphism, then $\phi(e_1)=e_2$.
        \end{exercise}
        
        \begin{exercise}
        Suppose $A$ and $B$ are subgroups of a group $G$, and suppose $B$ is of finite index in $G$.
        \begin{enumerate}[topsep=0.1in,label=(\alph*)]
            \item Show that the index of $A\cap B\le A$ is finite, and in fact $|A:A\cap B|\le |G:B|$. {Hint:} Find a set map $A/A\cap B\to G/B$.
            \item Prove that equality holds in (a) if and only if $G=AB$.
        \end{enumerate}
        \end{exercise}
        
        \begin{exercise}
        Let $G$ be a group. For each $a\in G$, let $\gamma_a$ denote the automorphism of $G$ defined by $\gamma_a(b)=aba^{-1}$ for all $b\in G$. The set $\operatorname{Inn}(G)=\{\gamma_a:a\in G\}$ is a subgroup of the automorphism group of $G$, called the subgroup of \textbf{inner automorphisms}.
        
        \medskip
        Prove that $\operatorname{Inn}(G)$ is isomorphic to $G/Z(G)$, where $Z(G)$ is the center of $G$.
        \end{exercise}
        
        \begin{exercise}
        Let $G$ be a group of order $2p$, where $p$ is an odd prime. Prove $G$ contains a nontrivial, proper normal subgroup.
        \end{exercise}
        
        \begin{exercise}
        Prove from the definition along that there are no nonabelian groups of order less than $5$.
        \end{exercise}
        
        \begin{exercise}
        Let $G$ be a group and $H,K\mathrel{\unlhd}G$ be normal subgroups with $H\cap K=\{e\}$. Show that each element in $H$ commutes with every element in $K$.
        \end{exercise}
        
        \begin{exercise}
        Let $G$ be a group and $N$ a normal subgroup of $G$. Let $aN$ denote the left coset defined by $a\in G$, and consider the binary operation
        \[
        G/N\times G/N\to G/N
        \]
        given by $(aN, bN)\mapsto abN$.
        \begin{enumerate}[label=(\alph*)]
            \item Show the operation is well defined.
            \item Show the operation is well defined only if the subgroup $N$ is normal.
        \end{enumerate}
        \end{exercise}
        
        \begin{exercise}
        Let $G$ be a group, $H\le G$ a subgroup that is not normal. Prove there exist cosets $Ha$ and $Hb$ such that $HaHb\neq Hab$.
        \end{exercise}
        
        \begin{exercise}
        Let $H$ be a subgroup of a group $G$. The \textbf{normalizer} of $H$ in $G$ is the set $\mathbb{N}_G(H)=\{g\in G\,\mid\, gH=Hg\}$.
        \begin{enumerate}[label=(\alph*)]
            \item Prove $\mathbb{N}_G(H)$ is a subgroup of $G$ containing $H$.
            \item Prove $\mathbb{N}_G(H)$ is the largest subgroup of $G$ in which $H$ is normal.
        \end{enumerate}
        \end{exercise}
        
        \begin{exercise}
        Let $G$ be a group and suppose $H\le G$. The \textbf{normalizer} of $H$ in $G$ is defined to be $N(H)=\{g\in G\,|\, gH=Hg\}$ and the \textbf{centralizer} of $H$ in $G$ is defined to be $C(H)=\{g\in G\,|\,  gh=hg\text{ for all }h\in H\}$.
        \begin{enumerate}[label=(\alph*)]
            \item Prove that $N(H)$ is a subgroup of $G$.
            \item Prove that $C(H)$ is a normal subgroup of $N(H)$ and that $N(H)/C(H)$ is isomorphic to a subgroup of $\operatorname{Aut}(H)$.
        \end{enumerate}
        \end{exercise}

        \begin{exercise}
            Suppose $G$ is a cyclic group of order $n$, and $t\in G$ is a generator.
            \begin{enumerate}[label=(\alph*)]
                \item Give a positive integer $d$ such that $t^{-1}=t^d$.
                \item Let $c$ be an integer and let $m=\gcd(n,c)$. Prove that the order of $t^c$ is $\frac{n}{m}$.
            \end{enumerate}
            \end{exercise}
            
            \begin{exercise}
            Let $G$ be a finite group. Prove \textit{from the definitions} that there exists a number $N$ such that $a^N=e$ for all $a\in G$.
            \end{exercise}
            
            \begin{exercise}
            Suppose $G$ is a group and $N\unlhd G$ is a finite normal subgroup. Prove that if $G/N$ contains an element of order $n$, then $G$ also contains an element of order $n$.
            \end{exercise}
            
            \begin{exercise}
            Suppose $\phi:G\to G'$ is a surjective homomorphism, $H\le G$ is a subgroup containing $\ker(\phi)$, and $H'=\phi(H)$. Prove $\phi^{-1}(H')=H$, where $\phi^{-1}(H')=\{g\in G\,\mid\, \phi(g)\in H'\}$. Make sure to state explicitly where each hypothesis is used.
            \end{exercise}
            
            \begin{exercise}
            Let $G$ be a group, and $H, K$ be subgroups of $G$. Let $HK=\{hk\,\mid \, h\in H, k\in K\}$ denote the set product. Prove that $HK$ is a group if and only if $HK=KH$.
            \end{exercise}
            
            \begin{exercise}
            Suppose $G$ is a nontrivial finite group and $H,K\mathrel{\unlhd}G$ are normal subgroups with $\gcd(|H|,|K|)=1$.
            \begin{enumerate}[label=(\alph*)]
                \item Define a nontrivial group homomorphism $\phi:G\to G/H\times G/K$
                \item Prove $G$ is isomorphic to a subgroup of $G/H\times G/K$.
                \item Suppose $\gcd(m,n)=1$. Prove $\mathbb{Z}_{mn}\cong \mathbb{Z}_m\times \mathbb{Z}_n$.
            \end{enumerate}
            \end{exercise}
            
            \begin{exercise}
            Suppose $G$ is a group, $H$ and $K$ are normal subgroups of $G$, and $H\le K$.
            \begin{enumerate}[label=(\alph*)]
                \item Define a group homomorphism from $K$ to $G/H$.
                \item Compute the kernel of the homomorphism in (a), and apply the First Isomorphism Theorem.
            \end{enumerate}
            \end{exercise}
            
            \begin{exercise}
            Let $G$ be a finite group and $\operatorname{Z}(G)$ denote its center.
            \begin{enumerate}[label=(\alph*)]
                \item Prove that if $G/\operatorname{Z}(G)$ is cyclic, then $G$ is abelian.
                \item Prove that if $G$ is nonabelian, then $|\operatorname{Z}(G)|\le \frac{1}{4}|G|$.
            \end{enumerate}
            \end{exercise}
            
            \begin{exercise}
            Let $G$ be a group, $m\in \mathbb{N}$, and $g\in G$ be an element such that $g^m=e$. Prove that $\operatorname{o}(g)\mid m$, where $\operatorname{o}(g)$ is the order of $g$.
            \end{exercise}
            
            \begin{exercise}
                \phantom{a}
            \begin{enumerate}[label=(\alph*)]
                \item Show that if $G$ is any group (not necessarily finite) and $H$ is a subgroup, then $G$ is a disjoint union of left cosets of $H$.
                \item State and prove Lagrange's Theorem for finite groups.
            \end{enumerate}
            \end{exercise}
            
            \newpage
            \begin{exercise}
            Let $G$ be a group and $H\le G$ a subgroup. For each coset $aH$ of $H$ in $G$, define the set
            \begin{equation*}
            \begin{split}
            G_{aH}=\{b\in G\,|\, baH=aH\}.
            \end{split}
            \end{equation*}
            \begin{enumerate}[label=(\alph*)]
                \item Prove that $G_{aH}$ is a subgroup of $G$.
                \item Suppose that $H$ is normal in $G$. Prove that $G_{aH}=H$.
            \end{enumerate}
            \end{exercise}
            
            \begin{exercise}
            Let $G$ be a group of order $2n$ for some positive integer $n>1$.
            \begin{enumerate}[label=(\alph*)]
                \item Prove there exists a subgroup $K$ of $G$ of order $2$.
                \item Suppose $K$ in (a) is a \textit{normal} subgroup. Prove that $K$ is contained in the center $\operatorname{Z}(G)$. (Recall $\operatorname{Z}(G)=\{a\in G\mid ab=ba\text{ for all }b\in G\}$.)
            \end{enumerate}
            \end{exercise}
            
            \begin{exercise}
            The additive group $\mathbb{Z}=(\mathbb{Z},+)$ of rational integers is a subgroup of the additive group $\mathbb{Q}=(\mathbb{Q},+)$. Show that $\mathbb{Z}$ has infinite index in $\mathbb{Q}$.
            \end{exercise}
%%%%%%%%%%%%%%%%%%%%%%%%%%%%%%%%%%%
%%%%%%%%%%%%%%%%%%%%%%%%%%%%%%%%%%%
%%%%%%%%%%%%%%%%%%%%%%%%%%%%%%%%%%%
%%%%%%%%%%%%%%%%%%%%%%%%%%%%%%%%%%%
%%%%%%%%%%%%%%%%%%%%%%%%%%%%%%%%%%%
%%%%%%%%%%%%%%%%%%%%%%%%%%%%%%%%%%%
%%%%%%%%%%%%%%%%%%%%%%%%%%%%%%%%%%%
%%%%%%%%%%%%%%%%%%%%%%%%%%%%%%%%%%%
%%%%%%%%%%%%%%%%%%%%%%%%%%%%%%%%%%%
%%%%%%%%%%%%%%%%%%%%%%%%%%%%%%%%%%%
%%%%%%%%%%%%%%%%%%%%%%%%%%%%%%%%%%%
%%%%%%%%%%%%%%%%%%%%%%%%%%%%%%%%%%%
%%%%%%%%%%%%%%%%%%%%%%%%%%%%%%%%%%%
%%%%%%%%%%%%%%%%%%%%%%%%%%%%%%%%%%%
%%%%%%%%%%%%%%%%%%%%%%%%%%%%%%%%%%%
%%%%%%%%%%%%%%%%%%%%%%%%%%%%%%%%%%%
%%%%%%%%%%%%%%%%%%%%%%%%%%%%%%%%%%%
%%%%%%%%%%%%%%%%%%%%%%%%%%%%%%%%%%%
\section*{\S\ Template Problems}
\markboth{Group Theory}{Template Problems}

        \begin{exercise}
            Let $G$ and $H$ be groups of order 10 and 15, respectively. Prove that if there is a nontrivial homomorphism $\phi:G\to H$, then $G$ is abelian.
            \end{exercise}
            
            \begin{exercise}
            Let $n$ be a number between $0$ and $10$. Compute $n^{111}\pmod{11}$, expressing your answer as a number between $0$ and $10$. Give as detailed a proof as you can, justifying every step, no matter how trivial you think it is.
            \end{exercise}
            
            \begin{exercise}
            Let $S_n$ denote the symmetric group on $n$ letters.
            \begin{enumerate}[label=(\alph*)]
                \item Is the element $(1\,2\,3\,4)(2\,5\,3\,4\,6)(1\,5\,3\,2\,4\,7)\in S_7$ even or odd? Indicate your reasoning.
                \item Find the order of $(1\,3\,4)(2\,4\,3)(1\,3\,4)\in S_4$. Show all work.
                \item Write $(1\,5\,2\,3)(2\,1\,3\,4)(1\,5\,2\,3)^{-1}\in S_5$ in disjoint cycle form. Show all work.
            \end{enumerate}
            \end{exercise}
            
            \begin{exercise}
            Determine with proof the automorphism group $\operatorname{Aut}(V)$ of the Klein 4-group $V=\{e,a,b,ab\}$. To what familiar group is it isomorphic?
            \end{exercise}
            
            \begin{exercise}
            Determine the number of group homomorphisms $\phi$ between the given groups. Here $K_4$ denotes the Klein four-group (also known as $\mathbb{Z}/2\mathbb{Z}\times \mathbb{Z}/2\mathbb{Z}$) and $S_3$ denotes the symmetric group on three elements.
            \medskip
            \begin{enumerate}[label=(\alph*)]
                \item $\phi:K_4\to \mathbb{Z}/2\mathbb{Z}$
                \item $\phi:\mathbb{Z}/2\mathbb{Z}\to K_4$
                \item $\phi:S_3\to K_4$
                \item $\phi:K_4\to S_3$
            \end{enumerate}
            \end{exercise}
            
            \begin{exercise}
            Explicitly list all group homomorphisms $f:\mathbb{Z}/6\mathbb{Z} \to \mathbb{Z}/12\mathbb{Z}$.
            \end{exercise}
            
            \begin{exercise}
            Let $C$ be a (possibly infinite) cyclic group, and let $\operatorname{Aut}(C)$ and $\operatorname{Inn}(C)$ be the groups of automorphisms and inner automorphisms, respectively. (Recall an automorphism $\gamma$ is \textbf{inner} if it is given by conjugation: $\gamma(b)=aba^{-1}$ for some $a\in C$.)
            \begin{enumerate}[label=(\alph*)]
                \item Describe $\operatorname{Aut}(C)$ and $\operatorname{Inn}(C)$ in familiar terms, as groups you would study in a first algebra course. Prove your result. (\textit{Hint:} Where do generators go?)
                \item Write $\operatorname{Aut}(\mathbb{Z}_{12})$ down explicitly, giving its generic name and computing the order of every element. Show all work.
            \end{enumerate}
            \end{exercise}
            
            \begin{exercise}
            Let $A_5$ denote the alternating group on a $5$-element set $\{1,2,3,4,5\}$. The set of automorphisms of $A_5$ form a group, denoted $\operatorname{Aut}(A_5)$. The group of \textbf{conjugations} of $A_5$, denoted $\operatorname{Conj}(A_5)$, is the subgroup of $\operatorname{Aut}(A_5)$ consisting of automorphisms of the form $\gamma_s:=s(-)s^{-1}$ where $s\in A_5$. Explicitly, $\gamma_s(x)=sxs^{-1}$ for any $x\in A_5$.
            \begin{enumerate}[label=(\alph*)]
                \item Prove that the function $\gamma:A_5\to \operatorname{Conj}(A_5)$, taking $s\in A_5$ to $\gamma_s$, is a surjective homomorphism.
                \item Prove that $A_5$ is isomorphic to $\operatorname{Conj}(A_5)$.
            \end{enumerate}
            \end{exercise}
            
            \begin{exercise}
            Suppose $H$ is a group of order 15. Prove there does not exist a nontrivial group homomorphism $\phi:D_5\to H$, where $D_5$ is the dihedral group with ten elements.
            \end{exercise}
            
            \begin{exercise}
            Let $S_7$ denote the symmetric group.
            \begin{enumerate}[label=(\alph*)]
                \item Give an example of two non-conjugate elements of $S_7$ that have the same order.
                \item If $g\in S_7$ has maximal order, what is the order of $g$?
                \item Does the element $g$ that you found in part (b) lie in $A_7$? Fully justify your answer.
                \item Determine whether the set $\{h\in S_7 \mid |h|=|g|\}$ is a single conjugacy class in $S_7$, where $g$ is the element you found in part (b).
            \end{enumerate}
            \end{exercise}
            
            \begin{exercise}
            Let $G$ be the additive group $\mathbb{Z}_{2020}$ and let $H\subseteq G$ be the subset consisting of those elements with order dividing 20.
            \begin{enumerate}[label=(\alph*)]
                \item Prove $H$ is a subgroup of $G$.
                \item Find an explicit generator for $H$ and determine its order.
            \end{enumerate}
            \end{exercise}
            
            \begin{exercise}
            Let $G$ denote the set of invertible $2\times 2$ matrices with values in a field. Prove $G$ is a group by defining a group law, identity element, and verifying the axioms. Credit is based on completeness.
            \end{exercise}
%%%%%%%%%%%%%%%%%%%%%%%%%%%%%%%%%%%
%%%%%%%%%%%%%%%%%%%%%%%%%%%%%%%%%%%
%%%%%%%%%%%%%%%%%%%%%%%%%%%%%%%%%%%
%%%%%%%%%%%%%%%%%%%%%%%%%%%%%%%%%%%
%%%%%%%%%%%%%%%%%%%%%%%%%%%%%%%%%%%
%%%%%%%%%%%%%%%%%%%%%%%%%%%%%%%%%%%
%%%%%%%%%%%%%%%%%%%%%%%%%%%%%%%%%%%
%%%%%%%%%%%%%%%%%%%%%%%%%%%%%%%%%%%
%%%%%%%%%%%%%%%%%%%%%%%%%%%%%%%%%%%
%%%%%%%%%%%%%%%%%%%%%%%%%%%%%%%%%%%
%%%%%%%%%%%%%%%%%%%%%%%%%%%%%%%%%%%
%%%%%%%%%%%%%%%%%%%%%%%%%%%%%%%%%%%
%%%%%%%%%%%%%%%%%%%%%%%%%%%%%%%%%%%
%%%%%%%%%%%%%%%%%%%%%%%%%%%%%%%%%%%
%%%%%%%%%%%%%%%%%%%%%%%%%%%%%%%%%%%
%%%%%%%%%%%%%%%%%%%%%%%%%%%%%%%%%%%
%%%%%%%%%%%%%%%%%%%%%%%%%%%%%%%%%%%
%%%%%%%%%%%%%%%%%%%%%%%%%%%%%%%%%%%
\chapter*{Ring Theory}
\section*{\S\ Pool Problems}
\markboth{Ring Theory}{Pool Problems}
\begin{exercise}
    Consider the additive group of integers $\mathbf{Z}$.
    \begin{enumerate}[label=(\alph*)]
        \item Prove that every subgroup of $\mathbf{Z}$ is cyclic.
        \item Prove that every homomorphic image of $\mathbf{Z}$ is cyclic.
        \item Consider the \textit{ring} $\mathbf{Z}$. Exhibit a prime ideal of $\mathbf{Z}$ that is not maximal.
    \end{enumerate}
    \end{exercise}
    
    \begin{exercise}
    Let $R$ be an integral domain. Suppose $a$ and $b$ are non-associate irreducible elements in $R$, and the ideal $(a,b)$ generated by $a$ and $b$ is a proper ideal. Show that $R$ is not a principal ideal domain (PID).
    \end{exercise}
    
    \begin{exercise}
    Let $R$ be a commutative ring with $1$. Suppose that for every $a\in R$ there exists $n\ge 2$ such that $a^n=a$. Show that every prime ideal of $R$ is maximal.
    \end{exercise}
    
    \begin{exercise}
    Let $R$ be a commutative ring with $1$, and $\sigma:R\to R$ be a ring automorphism.
    \begin{enumerate}[label=(\alph*)]
        \item Show that $F=\{r\in R\mid \sigma(r)=r\}$ is a subring of $R$ with $1$.
        \item Show that if $\sigma^2$ is the identity map on $R$, then each element of $R$ is the root of a monic polynomial of degree 2 in $F[x]$, where $F$ is as in (a).
    \end{enumerate}
    \end{exercise}
    
    \begin{exercise}
    Suppose $R$ is a ring such that $r^2=r$ for every $r\in R$.
    \begin{enumerate}[label=(\alph*)]
        \item Prove that $r=-r$ for every $r\in R$.
        \item Show that $R$ must be commutative. \textit{Hint: Consider $(a+b)^2$.}
    \end{enumerate}
    \end{exercise}
    
    \begin{exercise}
    Let $R$ be a commutative ring with $1$. The \textbf{characteristic} $\operatorname{char}(R)$ of $R$ is the unique integer $n\ge 0$ such that $\langle n\rangle \subset \mathbf{Z}$ is the kernel of the homomorphism $\theta:\mathbf{Z}\to R$ defined by
    \[
    \theta(m)=\begin{cases} \underbrace{1_R+\cdots +1_R}_{m}, & m\ge 0\\
    \underbrace{-1_R+\cdots + -1_R}_{|m|}, & m<0 \end{cases}.
    \]
    \begin{enumerate}[label=(\alph*)]
        \item Prove that if $f:R\to S$ is a monomorphism of commutative rings with $1$, then $\operatorname{char}(R)=\operatorname{char}(S)$.
        \item Give an example showing that $\operatorname{char}(R)$ is not always preserved by ring homomorphisms.
    \end{enumerate}
    \end{exercise}
    
    \begin{exercise}
    \begin{enumerate}[label=(\alph*)]
        \item Prove that for every commutative ring with unity $R$, there is a unique ring homomorphism $\phi_R:\mathbf{Z}\to R$, and that $\ker(\phi_R)=\langle d_R\rangle$ for a unique nonnegative integer $d_R$. The number $d_R$ is called the \textbf{characteristic} of $R$, denoted $\operatorname{char}(R)$.
        \item Suppose $F_1$ and $F_2$ are fields for which there exists a ring homomorphism $f:F_1\to F_2$. Prove that $\operatorname{char}(F_1)=\operatorname{char}(F_2)$.
    \end{enumerate}
    \end{exercise}
    
    \begin{exercise}
    Let $A$ be a commutative ring with $1$. The \textbf{dimension} of $A$ is the maximal length $d$ of a chain of prime ideals $\mathfrak{p}_0\subsetneq \mathfrak{p}_1\subsetneq \cdots \subsetneq \mathfrak{p}_d$. Prove that if $A$ is a PID, then $\dim(A)\le 1$.
    \end{exercise}
    
    \begin{exercise}
    Prove that every Euclidean domain is a principal ideal domain.
    \end{exercise}
    
    \begin{exercise}
    Let $F$ be a field and let $\alpha$ generate a field extension of $F$ of degree 5. Prove that $\alpha^2$ generates the same extension.
    \end{exercise}
    
    \begin{exercise}
    Let $R$ be a commutative ring with $1$. Use theorems in ring theory to prove:
    \begin{enumerate}[label=(\alph*)]
        \item $\langle x\rangle$ is a prime ideal in $R[x]$ if and only if $R$ is an integral domain.
        \item $\langle x\rangle$ is a maximal ideal in $R[x]$ if and only if $R$ is a field.
    \end{enumerate}
    \end{exercise}
    
    \begin{exercise}
    Let $F$ be a field and $F[x]$ the polynomial ring, which is a PID. Let
    \[
    R=\{f\in F[x]: f'\in (x)\},
    \]
    where $(x)\subset F[x]$ and $f'$ is the formal derivative.
    \begin{enumerate}[label=(\alph*)]
        \item Prove that $x^2$ and $x^3$ are irreducible elements of $R$.
        \item Let $(x^2,x^3)$ be the ideal generated by $x^2$ and $x^3$. Prove it is a proper ideal of $R$.
        \item Prove that $(x^2,x^3)$ is not a principal ideal of $R$.
    \end{enumerate}
    \end{exercise}
    
    \begin{exercise}
    Let $R$ be a commutative ring with $1$ and $e\in R$ an idempotent element ($e^2=e$).
    \begin{enumerate}[label=(\alph*)]
        \item Prove that $1-e$ is idempotent.
        \item If $e\neq 0,1$, show that $Re$ and $R(1-e)$ are proper ideals of $R$.
        \item Prove that $R\cong Re\times R(1-e)$.
    \end{enumerate}
    \end{exercise}
    
    \begin{exercise}
    An element $r$ of a ring $R$ is idempotent if $r^2=r$. Suppose $R$ is commutative with $1$ and contains an idempotent $e$.
    \begin{enumerate}[label=(\alph*)]
        \item Prove $1-e$ is idempotent.
        \item Prove $eR$ and $(1-e)R$ are ideals and $R\cong eR\times (1-e)R$.
        \item Prove that if $R$ has a unique maximal ideal, the only idempotents are $0$ and $1$.
    \end{enumerate}
    \end{exercise}
    
    \begin{exercise}
    Prove that if $\phi:R\to S$ is a surjective ring homomorphism between commutative rings with $1$, then $\phi(1_R)=1_S$.
    \end{exercise}
    
    \begin{exercise}
    Suppose $R$ is a PID. Prove that an ideal $I\subset R$ is maximal if and only if $I=\langle p\rangle$ for a prime $p\in R$.
    \end{exercise}
    
    \begin{exercise}
    Let $R$ be a commutative ring.
    \begin{enumerate}[label=(\alph*)]
        \item Prove that the set $N$ of all nilpotent elements of $R$ is an ideal.
        \item Prove that $R/N$ has no nonzero nilpotent elements.
        \item Show that $N$ is contained in every prime ideal of $R$.
    \end{enumerate}
    \end{exercise}
    
    \begin{exercise}
    Let $R$ be a commutative ring with $1$. An element $n\in R$ is nilpotent if $n^k=0$ for some $k\in \mathbf{N}$.
    \begin{enumerate}[label=(\alph*)]
        \item Show that if $n$ is nilpotent, then $1-n$ is a unit.
        \item Give an example of a commutative ring with $1$ with no nonzero nilpotent elements, but not an integral domain.
    \end{enumerate}
    \end{exercise}
    
    \begin{exercise}
    Let $D$ be a PID. Prove that every proper nonzero prime ideal is maximal.
    \end{exercise}
    
    \begin{exercise}
    Let $I\subseteq \mathbf{Z}[x]$ be the set of polynomials with even constant term.
    \begin{enumerate}[label=(\alph*)]
        \item Prove that $I$ is an ideal.
        \item Prove that $I$ is not a principal ideal.
    \end{enumerate}
    \end{exercise}
    
    \begin{exercise}
    Let $R$ be a commutative ring with $1$.
    \begin{enumerate}[label=(\alph*)]
        \item Define what it means for an element to be \textbf{prime} and \textbf{irreducible}.
        \item Prove that if $R$ is an integral domain, every prime element is irreducible.
    \end{enumerate}
    \end{exercise}
    
    \begin{exercise}
    Let $R$ be a commutative ring with $1$, $I\subseteq R$ an ideal, and $\pi:R\to R/I$ the natural projection.
    \begin{enumerate}[label=(\alph*)]
        \item Show that if $\wp$ is a prime ideal of $R/I$, then $\pi^{-1}(\wp)$ is a prime ideal of $R$.
        \item Show that the map $\wp\mapsto \pi^{-1}(\wp)$ is injective on prime ideals of $R/I$.
    \end{enumerate}
    \end{exercise}
    
    \begin{exercise}
    Let $A$ be a commutative ring with $1$. We say $A$ is \textbf{Boolean} if $a^2=a$ for every $a\in A$. Prove that in a Boolean ring:
    \begin{enumerate}[label=(\alph*)]
        \item $2a=0$ for all $a\in A$.
        \item If $I$ is a prime ideal, then $A/I$ has two elements, so $I$ is maximal.
        \item If $I=(a,b)$, then $I$ can be generated by $a+b+ab$. Conclude every finitely generated ideal is principal.
    \end{enumerate}
    \end{exercise}
    
    \newpage
    \begin{exercise}
    Let $R$ be a commutative ring. For $X\subseteq R$ nonempty, define the \textbf{annihilator} $\operatorname{ann}(X)=\{a\in R\mid ax=0\text{ for all }x\in X\}$.
    \begin{enumerate}[label=(\alph*)]
        \item Prove that $\operatorname{ann}(X)$ is an ideal.
        \item Prove that $X\subseteq \operatorname{ann}(\operatorname{ann}(X))$.
    \end{enumerate}
    \end{exercise}
    
    \begin{exercise}
    Suppose $\phi:R\to S$ is a ring homomorphism, and $S$ has no nonzero zero-divisors. Prove that $\ker(\phi)$ is a prime ideal.
    \end{exercise}
    
    \begin{exercise}
    Let $R$ be a commutative ring with $1$, and $N=\{a\in R\mid a^n=0 \text{ for some }n\}$. Let $[b]$ be the image of $b\in R$ in $R/N$. Prove that if $[a]^m=0$ in $R/N$, then $[a]=[0]$.
    \end{exercise}
    
    \begin{exercise}
    Let $R$ be a commutative ring. The \textbf{nilradical} of $R$ is $N=\{r\in R\mid r^n=0 \text{ for some } n\in \mathbf{N}\}$.
    \begin{enumerate}[label=(\alph*)]
        \item Prove that $N$ is an ideal of $R$.
        \item Prove that $N$ is contained in the intersection of all prime ideals of $R$.
    \end{enumerate}
    \end{exercise}
    

%%%%%%%%%%%%%%%%%%%%%%%%%%%%%%%%%%%
%%%%%%%%%%%%%%%%%%%%%%%%%%%%%%%%%%%
%%%%%%%%%%%%%%%%%%%%%%%%%%%%%%%%%%%
%%%%%%%%%%%%%%%%%%%%%%%%%%%%%%%%%%%
%%%%%%%%%%%%%%%%%%%%%%%%%%%%%%%%%%%
%%%%%%%%%%%%%%%%%%%%%%%%%%%%%%%%%%%
%%%%%%%%%%%%%%%%%%%%%%%%%%%%%%%%%%%
%%%%%%%%%%%%%%%%%%%%%%%%%%%%%%%%%%%
%%%%%%%%%%%%%%%%%%%%%%%%%%%%%%%%%%%
%%%%%%%%%%%%%%%%%%%%%%%%%%%%%%%%%%%
%%%%%%%%%%%%%%%%%%%%%%%%%%%%%%%%%%%
%%%%%%%%%%%%%%%%%%%%%%%%%%%%%%%%%%%
\section*{\S\ Template Problems}
\markboth{Ring Theory}{Template Problems}
\begin{exercise}
    Suppose $I$ and $J$ are ideals in a commutative ring $R$ such that $R=I+J$. 
    \begin{enumerate}[label=(\alph*)]
        \item Prove that the map $f:R\to R/I\times R/J$ given by $f(x)=(x+I,x+J)$ induces the isomorphism
        \[
            R/IJ \cong R/I \times R/J.
        \]
        \item Prove that $\left(\mathbb{Z}/3\mathbb{Z}\right)[x]/(x^3-x^2-1)\cong \left(\mathbb{Z}/3\mathbb{Z}\right)[x]/(x^3+x+1)$. (\textit{Hint: Use part (a).})
    \end{enumerate}
    \end{exercise}
    
    \begin{exercise}
    Let $\mathcal{C}([0,1])$ be the commutative ring of continuous real-valued functions on $[0,1]$, and let
    \[
        M=\{f\in \mathcal{C}([0,1]) \mid f(1/2)=0\}.
    \]
    Prove that $M$ is a maximal ideal.
    \end{exercise}
    
    \begin{exercise}
    Let $z\in \mathbb{C}$ and $\epsilon_z:\mathbb{R}[x]\to \mathbb{C}$ be the evaluation homomorphism $\epsilon_z(p)=p(z)$.
    \begin{enumerate}[label=(\alph*)]
        \item Show that $\ker(\epsilon_z)$ is a prime ideal.
        \item Compute $\ker(\epsilon_{1+i})$, $\operatorname{im}(\epsilon_{1+i})$ and state the conclusion of the First Isomorphism Theorem applied to $\epsilon_{1+i}$.
    \end{enumerate}
    \end{exercise}
    
    \begin{exercise}
    Let $\mathbb{Z}[\sqrt{2}]=\{a+b\sqrt{2} \mid a,b\in \mathbb{Z}\}$, and let $R\subset \operatorname{M}_2(\mathbb{Z})$ be the ring of all $2\times 2$ matrices of the form $\begin{bmatrix} a & b \\ 2b & a\end{bmatrix}$. Prove that $\mathbb{Z}[\sqrt{2}]$ is isomorphic to $R$.
    \end{exercise}
    
    \newpage
    \begin{exercise}
    Let $f(x)=x^3+x+1\in \mathbb{Z}_5[x]$.
    \begin{enumerate}[label=(\alph*)]
        \item Prove that $f(x)$ is irreducible.
        \item Prove that $\langle f(x)\rangle$ is a maximal ideal.
        \item Determine the cardinality of $\mathbb{Z}_5[x]/\langle f(x)\rangle$ and justify.
    \end{enumerate}
    \end{exercise}
    
    \begin{exercise}
    \begin{enumerate}[label=(\alph*)]
        \item Write down an irreducible cubic polynomial over $\mathbb{F}_2$.
        \item Construct a field with exactly 8 elements and write its multiplication table.
    \end{enumerate}
    \end{exercise}
    
    \begin{exercise}
    Let $\varepsilon:\mathbb{R}[x]\to \mathbb{C}$ be evaluation at $i$.
    \begin{enumerate}[label=(\alph*)]
        \item Prove that $\ker(\varepsilon)=(x^2+1)\subseteq \mathbb{R}[x]$.
        \item Prove that $(x^2+1)$ is a maximal ideal in $\mathbb{R}[x]$.
    \end{enumerate}
    \end{exercise}
    
    \begin{exercise}
    Let $\mathbb{Z}[i]=\{a+bi \mid a,b\in \mathbb{Z},\, i^2=-1\}$.
    \begin{enumerate}[label=(\alph*)]
        \item Prove there is no ring homomorphism $\mathbb{Z}[i]\to \mathbb{Z}_{19}$, but there is one to $\mathbb{Z}_{13}$.
    \end{enumerate}
    \end{exercise}
    
    \begin{exercise}
    Let $\mathbb{Z}_n$ be integers modulo $n$, and consider the ring homomorphism
    \[
        \mathbb{Z}_{28} \to \mathbb{Z}_4 \times \mathbb{Z}_7, \quad [m]_{28} \mapsto ([m]_4,[m]_7),
    \]
    which is an isomorphism by the Chinese Remainder Theorem. Let $\mathbb{Z}_n^\times$ denote the group of units. Prove that $\mathbb{Z}_{28}^\times \cong \mathbb{Z}_4^\times \times \mathbb{Z}_7^\times$.
    \end{exercise}
    
    \begin{exercise}
    Let $k\subset K$ be fields, and $k[X]$ the polynomial ring. The \textbf{evaluation} at $z\in K$ is $\varepsilon:k[X]\to K$, $\varepsilon(f(X))=f(z)$. Prove that if $\varepsilon$ is not injective, then $\varepsilon(k[X])$ is a field.
    \end{exercise}
    
    \begin{exercise}
    Let $i$ be the imaginary unit, $\mathbb{Z}[i]$ the Gaussian integers, and $\mathbb{Z}_2$ the finite field with two elements.
    \begin{enumerate}[label=(\alph*)]
        \item Define a ring homomorphism $\mathbb{Z}[i]\to \mathbb{Z}_2$ and prove it is a homomorphism.
        \item Find a generator for the kernel of your homomorphism with proof.
    \end{enumerate}
    \end{exercise}
    
    \begin{exercise}
    Let $\mathbb{Z}[i]$ be the Gaussian integers.
    \begin{enumerate}[label=(\alph*)]
        \item Prove there exists a nonzero ring homomorphism $\mathbb{Z}[i]\to \mathbb{Z}_5$.
        \item Compute the kernel explicitly and state the conclusion given by the First Isomorphism Theorem. (\textit{Hint: The kernel requires two generators.})
    \end{enumerate}
    \end{exercise}
    
    \begin{exercise}
    Let $\mathbb{Z}_2[x]/\langle x^3+x+1\rangle$ be a field of 8 elements with natural projection $\pi:\mathbb{Z}_2[x]\to \mathbb{Z}_2[x]/\langle x^3+x+1\rangle$.
    \begin{enumerate}[label=(\alph*)]
        \item Write down eight distinct coset representatives.
        \item Determine the multiplicative inverse of $\pi(x)$ in terms of your coset representatives.
    \end{enumerate}
    \end{exercise}
    
    \begin{exercise}
    Let $\mathbb{F}_9$ be the field of nine elements.
    \begin{enumerate}[label=(\alph*)]
        \item Show that each nonzero $a\in \mathbb{F}_9$ is a root of $X^3-1=(X-1)(X^2+1)(X^4+1)\in \mathbb{F}_3[X]$.
        \item Use the Pigeonhole Principle to prove that $\mathbb{F}_9$ has an element of multiplicative order 8, including a justification for applying the principle.
    \end{enumerate}
    \end{exercise}
    
    \begin{exercise}
    Let $\mathbb{Z}[X]$ be the ring of polynomials with integer coefficients, and let $K\subset \mathbb{Z}[X]$ be the kernel of the ``evaluation at 1'' map.
    \end{exercise}
    
%%%%%%%%%%%%%%%%%%%%%%%%%%%%%%%%%%%
%%%%%%%%%%%%%%%%%%%%%%%%%%%%%%%%%%%
%%%%%%%%%%%%%%%%%%%%%%%%%%%%%%%%%%%
%%%%%%%%%%%%%%%%%%%%%%%%%%%%%%%%%%%
%%%%%%%%%%%%%%%%%%%%%%%%%%%%%%%%%%%
%%%%%%%%%%%%%%%%%%%%%%%%%%%%%%%%%%%
%%%%%%%%%%%%%%%%%%%%%%%%%%%%%%%%%%%
%%%%%%%%%%%%%%%%%%%%%%%%%%%%%%%%%%%
%%%%%%%%%%%%%%%%%%%%%%%%%%%%%%%%%%%
%%%%%%%%%%%%%%%%%%%%%%%%%%%%%%%%%%%
%%%%%%%%%%%%%%%%%%%%%%%%%%%%%%%%%%%
%%%%%%%%%%%%%%%%%%%%%%%%%%%%%%%%%%%
\chapter*{Linear Algebra}
\markboth{Linear Algebra}{Pool Problems}
    \section*{\S\ Pool Problems}
    \begin{exercise}
        \phantom{a}
        \begin{enumerate}[label=(\alph*)]
            \item Give an explicit example (with proof) showing that the union of two subspaces (of a given vector space) is not necessarily a subspace.
            \item Suppose $U_1$ and $U_2$ are subspaces of a vector space $V$. Recall that their \textbf{sum} is defined to be the set $U_1+U_2 = \left\{ u_1 + u_2 \mid u_1 \in U_1, u_2 \in U_2 \right\}$. Prove $U_1 + U_2$ is a subspace of $V$ containing $U_1$ and $U_2$.
        \end{enumerate}
        \end{exercise}
        
        \begin{exercise}
        Suppose $F$ is a field and $A$ is an $n\times n$ matrix over $F$. Suppose further that $A$ possesses distinct eigenvalues $\lambda_1$ and $\lambda_2$ with $\dim \operatorname{Null}(A-\lambda_1 I_n)=n-1$. Prove $A$ is diagonalizable.
        \end{exercise}
        
        \begin{exercise}
        Let $\phi: V \to W$ be a surjective linear transformation of finite-dimensional linear spaces. Show that there is a $U \subseteq V$ such that $V = \ker(\phi) \oplus U$ and $\phi\mid_U : U \to W$ is an isomorphism. (Note that $V$ is not assumed to be an inner-product space; also note that $\ker(\phi)$ is sometimes referred to as the \textbf{null space} of $\phi$; finally, $\phi\mid_U$ denotes the restriction of $\phi$ to $U$.)
        \end{exercise}
        
        \begin{exercise}
        Suppose $V$ is a finite-dimensional real vector space and $T: V \to V$ is a linear transformation. Prove that $T$ has at most $\dim(\operatorname{range} \, T)$ distinct nonzero eigenvalues.
        \end{exercise}
        
        \begin{exercise}
        Let $T: V \to V$ be a linear transformation on a finite-dimensional vector space. Prove that if $T^2 = T$, then
        \[
        V = \ker(T) \oplus \operatorname{im}(T).
        \]
        \end{exercise}
        
        \begin{exercise}
        Let $\mathbb{R}^3$ denote the $3$-dimensional vector space, and let $\mathbf{v}=(a,b,c)$ be a fixed nonzero vector. The maps $C: \mathbb{R}^3 \to \mathbb{R}^3$ and $D: \mathbb{R}^3 \to \mathbb{R}^3$ defined by $C(\mathbf{w}) = \mathbf{v} \times \mathbf{w}$ and $D(\mathbf{w}) = (\mathbf{v} \cdot \mathbf{w}) \mathbf{v}$ are linear transformations.
        \begin{enumerate}[label=(\alph*)]
            \item Determine the eigenvalues of $C$ and $D$.
            \item Determine the eigenspaces of $C$ and $D$ as subspaces of $\mathbb{R}^3$, in terms of $a, b, c$.
            \item Find a matrix for $C$ with respect to the standard basis.
        \end{enumerate}
        Show all work and explain reasoning.
        \end{exercise}
        
        \begin{exercise}
        Suppose $A$ is a real $n \times n$ matrix that satisfies $A^2 \mathbf{v} = 2 A \mathbf{v}$ for every $\mathbf{v} \in \mathbb{R}^n$.
        \begin{enumerate}[label=(\alph*)]
            \item Show that the only possible eigenvalues of $A$ are 0 and 2.
            \item For each $\lambda \in \mathbb{R}$, let $E_\lambda$ denote the $\lambda$-eigenspace of $A$, i.e., $E_\lambda = \{ \mathbf{v} \in \mathbb{R}^n \mid A \mathbf{v} = \lambda \mathbf{v} \}$. Prove that $\mathbb{R}^n = E_0 \oplus E_2$. (\textit{Hint:} For every vector $\mathbf{v}$ one can write $\mathbf{v} = (\mathbf{v} - \frac{1}{2} A \mathbf{v}) + \frac{1}{2} A \mathbf{v}$.)
        \end{enumerate}
        \end{exercise}
        
        \begin{exercise}
        Suppose $T: \mathbb{R}^n \to \mathbb{R}^n$ is a linear transformation with distinct eigenvalues $\lambda_1, \lambda_2, \ldots, \lambda_m$, and let $\mathbf{v}_1, \mathbf{v}_2, \ldots, \mathbf{v}_m$ be corresponding eigenvectors. Prove $\mathbf{v}_1, \mathbf{v}_2, \ldots, \mathbf{v}_m$ are linearly independent.
        \end{exercise}
        
        \begin{exercise}
        Let $S: V \to V$ and $T: V \to V$ be linear transformations that commute, i.e., $S \circ T = T \circ S$. Let $\mathbf{v} \in V$ be an eigenvector of $S$ such that $T(\mathbf{v}) \ne 0$. Prove that $T(\mathbf{v})$ is also an eigenvector of $S$.
        \end{exercise}
        
        \begin{exercise}
        Suppose $A$ is a $5 \times 5$ matrix and $\mathbf{v}_1, \mathbf{v}_2, \mathbf{v}_3$ are eigenvectors of $A$ with distinct eigenvalues. Prove $\{\mathbf{v}_1, \mathbf{v}_2, \mathbf{v}_3\}$ is a linearly independent set. \textit{Hint:} Consider a minimal linear dependence relation.
        \end{exercise}
        
        \begin{exercise}
        Suppose $V$ is a vector space, and $\mathbf{v}_1, \mathbf{v}_2, \ldots, \mathbf{v}_n$ are in $V$. Prove that either $\mathbf{v}_1, \ldots, \mathbf{v}_n$ are linearly independent, or there exists a number $k \le n$ such that $\mathbf{v}_k$ is a linear combination of $\mathbf{v}_1, \ldots, \mathbf{v}_{k-1}$.
        \end{exercise}
        
        \begin{exercise}
        Let $M_4(\mathbb{R})$ denote the 16-dimensional real vector space of $4 \times 4$ matrices with real entries, in which the vectors are represented as matrices. Let $T: M_4(\mathbb{R}) \to M_4(\mathbb{R})$ be the linear transformation defined by $T(A) = A - A^\top$.
        \begin{enumerate}[label=(\alph*)]
            \item Determine the dimension of $\operatorname{ker}(T)$.
            \item Determine the dimension of $\operatorname{im}(T)$.
        \end{enumerate}
        \end{exercise}
        
        \begin{exercise}
        Let $A$ be a real $n \times n$ matrix and let $A^\top$ denote its transpose.
        \begin{enumerate}[label=(\alph*)]
            \item Prove that $(A \mathbf{v}) \cdot \mathbf{w} = \mathbf{v} \cdot (A^\top \mathbf{w})$ for all vectors $\mathbf{v}, \mathbf{w} \in \mathbb{R}^n$. \textit{Hint:} Recall that the dot product $\mathbf{u} \cdot \mathbf{v}$ equals the matrix product $\mathbf{u}^\top \mathbf{v}$.
            \item Suppose now $A$ is also symmetric, i.e., that $A^\top = A$. Also suppose $\mathbf{v}$ and $\mathbf{w}$ are eigenvectors of $A$ with different eigenvalues. Prove that $\mathbf{v}$ and $\mathbf{w}$ are orthogonal.
        \end{enumerate}
        \end{exercise}
        
        \begin{exercise}
        A real $n \times n$ matrix $A$ is called \textbf{skew-symmetric} if $A^\top = -A$. Let $V_n$ be the set of all skew-symmetric matrices in $\operatorname{M}_n(\mathbb{R})$. Recall that $\operatorname{M}_n(\mathbb{R})$ is an $n^2$-dimensional $\mathbb{R}$-vector space with standard basis $\{ e_{ij} \mid 1 \le i,j \le n\}$, where $e_{ij}$ is the $n \times n$ matrix with a 1 in the $(i,j)$-position and zeros everywhere else.
        \begin{enumerate}[label=(\alph*)]
            \item Show $V_n$ is a subspace of $\operatorname{M}_n(\mathbb{R})$.
            \item Find an ordered basis $\mathcal{B}$ for the space $V_3$ of all skew-symmetric $3 \times 3$ matrices.
        \end{enumerate}
        \end{exercise}

%%%%%%%%%%%%%%%%%%%%%%%%%%%%%%%%%%%
%%%%%%%%%%%%%%%%%%%%%%%%%%%%%%%%%%%
%%%%%%%%%%%%%%%%%%%%%%%%%%%%%%%%%%%
%%%%%%%%%%%%%%%%%%%%%%%%%%%%%%%%%%%
%%%%%%%%%%%%%%%%%%%%%%%%%%%%%%%%%%%
%%%%%%%%%%%%%%%%%%%%%%%%%%%%%%%%%%%
\newpage
\section*{\S\ Template Problems}
\markboth{Linear Algebra}{Template Problems}
\begin{exercise}
    Let $T:\mathbf{R}^3\to \mathbf{R}^3$ be the linear transformation that rotates counterclockwise around the $z$-axis by $\frac{2\pi}{3}$.
    \begin{enumerate}[label=(\alph*)]
        \item Write the matrix for $T$ with respect to the standard basis $\left\{\begin{bmatrix} 1 \\ 0 \\ 0 \end{bmatrix},\begin{bmatrix} 0 \\ 1 \\ 0\end{bmatrix},\begin{bmatrix} 0 \\ 0 \\ 1\end{bmatrix}\right\}$.
        \item Write the matrix for $T$ with respect to the basis $\left\{\begin{bmatrix} \frac{\sqrt{3}}{2} \\ -\frac{1}{2} \\ 0 \end{bmatrix},\begin{bmatrix} 0 \\ 1 \\ 0\end{bmatrix},\begin{bmatrix} 0 \\ 0 \\ 1\end{bmatrix}\right\}$.
        \item Determine all (complex) eigenvalues of $T$.
        \item Is $T$ diagonalizable over $\mathbf{C}$? Justify your answer.
    \end{enumerate}
    \end{exercise}
    
    \begin{exercise}
    Let $V$ denote the real vector space of polynomials in $x$ of degree at most 3. Let $\mathcal{B}=\{1, x, x^2, x^3\}$ be a basis for $V$ and $T:V\to V$ be the function defined by $T(f(x))=f(x)+f'(x)$.
    \begin{enumerate}[label=(\alph*)]
        \item Prove that $T$ is a linear transformation.
        \item Find $[T]_{\mathcal{B}}$, the matrix representation for $T$ in terms of the basis $\mathcal{B}$.
        \item Is $T$ diagonalizable? If yes, find a matrix $A$ so that $A[T]_{\mathcal{B}}A^{-1}$ is diagonal; otherwise explain why $T$ is not diagonalizable.
    \end{enumerate}
    \end{exercise}
    
    \begin{exercise}
    Let $\operatorname{M}_n(\mathbf{R})$ be the vector space of all $n \times n$ matrices with real entries. We say that $A, B \in \operatorname{M}_n(\mathbf{R})$ commute if $AB = BA$.
    \begin{enumerate}[label=(\alph*)]
        \item Fix $A \in \operatorname{M}_n(\mathbf{R})$. Prove that the set of all matrices in $\operatorname{M}_n(\mathbf{R})$ that commute with $A$ is a subspace of $\operatorname{M}_n(\mathbf{R})$.
        \item Let $A=\begin{bmatrix} 1 & 1 \\ 1 & 1  \end{bmatrix}\in \operatorname{M}_2(\mathbf{R})$ and let $W\subseteq \operatorname{M}_2(\mathbf{R})$ be the subspace of all matrices that commute with $A$. Find a basis of $W$. 
    \end{enumerate}
    \end{exercise}
    
    \begin{exercise}
    Let $V\subset \mathbf{R}^5$ be the subspace defined by the equation
    \[
    x_1-2x_2+3x_3-4x_4+5x_5=0.
    \]
    \begin{enumerate}[label=(\alph*)]
        \item Find (with justification) a basis for $V$.
        \item Find (with justification) a basis for $V^{\perp}$, the subspace of $\mathbf{R}^5$ orthogonal to $V$ under the usual dot product.
    \end{enumerate}
    \end{exercise}
    
    \newpage
    \begin{exercise}
    Let $T:\mathbf{R}^3\to \mathbf{R}^3$ be the linear transformation defined by
    \[
    T\left(\begin{bmatrix} x \\ y \\ z\end{bmatrix}\right) = \begin{bmatrix} x+y \\ 2z-x \\ y+2z\end{bmatrix}.
    \]
    \begin{enumerate}[label=(\alph*)]
        \item Find the matrix that represents $T$ with respect to the standard basis for $\mathbf{R}^3$.
        \item Find a basis for the kernel of $T$.
        \item Determine the rank of $T$.
    \end{enumerate}
    \end{exercise}
    
    \begin{exercise}
    Let $A=\begin{bmatrix} 0 & 0 & -2 \\ 1 & 2 & 1 \\ 1 & 0 & 3\end{bmatrix}$.
    \begin{enumerate}[label=(\alph*)]
        \item Determine whether $A$ is diagonalizable, and if so, give its diagonal form along with a diagonalizing matrix.
        \item Compute $A^{42}$. Remember to show all work.
    \end{enumerate}
    \end{exercise}
    
    \begin{exercise}
    Let $A=\begin{bmatrix} 2 & -1 & -1 \\ 1 & 0 & -1 \\ 1 & -1 & 0\end{bmatrix}$.
    \begin{enumerate}[label=(\alph*)]
        \item Compute the characteristic polynomial $p_A(x)$ of $A$.
        \item For each eigenvalue $\lambda$ of $A$, find a basis for the eigenspace $E_{\lambda}$.
        \item Determine if $A$ is diagonalizable. If so, give matrices $P$ and $B$ such that $P^{-1}AP=B$ and $B$ is diagonal. If no, explain carefully why $A$ is not diagonalizable.
    \end{enumerate}
    \end{exercise}
    
    \begin{exercise}
    Let $A=\begin{bmatrix} 6 & -2 & -1 \\ 10 & -3 & -2 \\ 0 & 0 & 1\end{bmatrix}$.
    \begin{enumerate}[label=(\alph*)]
        \item Find bases for the eigenspaces of $A$.
        \item Determine if $A$ is diagonalizable. If so, give an invertible matrix $P$ and diagonal matrix $D$ such that $P^{-1}AP=D$. If not, explain why not.
    \end{enumerate}
    \end{exercise}
    
    \begin{exercise}
    Let $W\subset \mathbf{R}^5$ be the subspace spanned by the set of vectors $$\{\langle 1,-2,0,2,-1\rangle,\langle -2,4,-1,1,2\rangle,\langle 0,1,2,-2,1\rangle\}.$$
    \begin{enumerate}[label=(\alph*)]
        \item Compute the dimension of $W$.
        \item Determine the dimension of $W^\perp$, the perpendicular subspace in $\mathbf{R}^5$.
        \item Find a basis for $W^\perp$.
    \end{enumerate}
    \end{exercise}
    
    \newpage
    \begin{exercise}
    Let $P_3$ be the real vector space of all real polynomials of degree three or less. Define $L:P_3\to P_3$ by $L(p(x))=p(x)+p(-x)$.
    \begin{enumerate}[label=(\alph*)]
        \item Prove $L$ is a linear transformation.
        \item Find a basis for the null space of $L$.
        \item Compute the dimension of the image of $L$.
    \end{enumerate}
    \end{exercise}
    
    \begin{exercise}
    Let $V=\{a_0+a_1\sqrt[3]{2}+a_2\sqrt[3]{4}\mid a_0, a_1, a_2\in \mathbf{Q}\}\subseteq \mathbf{R}$. This set is a vector space over $\mathbf{Q}$.
    \begin{enumerate}[label=(\alph*)]
        \item Verify $V$ is closed under product (using the usual product operation in $\mathbf{R}$).
        \item Let $T:V\to V$ be the linear transformation defined by $T(v)=(\sqrt[3]{2}+\sqrt[3]{4}) v$. Find the matrix that represents $T$ with respect to the basis $\{1,\sqrt[3]{2},\sqrt[3]{4}\}$ for $V$.
        \item Determine the characteristic polynomial for $T$.
    \end{enumerate}
    \end{exercise}

    \begin{exercise}
        Suppose $\{\mathbf{v}_1,\mathbf{v}_2,\mathbf{v}_3\}$ is a basis for $\mathbf{R}^3$ and $T:\mathbf{R}^3\to \mathbf{R}^3$ is a linear transformation satisfying
        \[
        T(\mathbf{v}_1) = 4\mathbf{v}_1+2\mathbf{v}_2,\quad
        T(\mathbf{v}_2) = 5\mathbf{v}_2,\quad
        T(\mathbf{v}_3) = -2\mathbf{v}_1+4\mathbf{v}_2+5\mathbf{v}_3.
        \]
        Determine the eigenvalues of $T$ and find a basis for each eigenspace.
        \end{exercise}
        
        \begin{exercise}
        Let $W\subset \mathbf{R}^5$ be the space spanned by the vectors
        \[
        \left\{\begin{bmatrix} 1 \\ -2 \\ 0 \\ 2 \\ 1 \end{bmatrix},\begin{bmatrix} -2 \\ 4 \\ -1 \\ 1 \\ 2\end{bmatrix}, \begin{bmatrix} 0 \\ 1 \\ 2 \\ -2 \\1\end{bmatrix}\right\}.\]

        \begin{enumerate}[label=(\alph*)]
            \item Compute the dimension of $W$.
            \item Let $W^{\perp}=\{\mathbf{v}\in \mathbf{R}^5\mid \mathbf{v}\cdot \mathbf{w}=0 \text{ for all } \mathbf{w}\in W\}$. Determine the dimension of $W^{\perp}$, and explain why this follows immediately from (a) using a theorem.
            \item Find a basis for $W^{\perp}$.
        \end{enumerate}
        \end{exercise}
        
        \begin{exercise}
        Let $L$ be the line in $\mathbf{R}^2$ defined by $y=-3x$, and let $T:\mathbf{R}^2\to \mathbf{R}^2$ be the linear transformation that orthogonally projects onto $L$ and then stretches along $L$ by a factor of two.
        \begin{enumerate}[label=(\alph*)]
            \item Find the eigenvalues and an eigenbasis $\mathcal{B}$ for $T$.
            \item Determine the matrix for $T$ with respect to the basis $\mathcal{B}$.
            \item Determine the matrix for $T$ with respect to the standard basis.
        \end{enumerate}
        \end{exercise}
        
        \newpage
        \begin{exercise}
        Let $T:\mathbf{R}^3\to \mathbf{R}^3$ be the orthogonal projection to a $1$-dimensional linear subspace $L\subset \mathbf{R}^3$.
        \begin{enumerate}[label=(\alph*)]
            \item List the eigenvalues of $T$.
            \item Write the characteristic polynomial $p_T(x)$ for $T$.
            \item Is $T$ diagonalizable? Justify your answer.
        \end{enumerate}
        \end{exercise}
        
        \begin{exercise}
        Let $L$ be the line parameterized by $L(t)=(2t,-3t,t)$ for $t\in \mathbf{R}$, and let $T:\mathbf{R}^3\to \mathbf{R}^3$ be the linear transformation that is orthogonal projection onto $L$.
        \begin{enumerate}[label=(\alph*)]
            \item Describe $\operatorname{ker}(T)$ and $\operatorname{im}(T)$, either implicitly (using equations in $x,y,z$) or parametrically.
            \item List the eigenvalues of $T$ and their geometric multiplicities.
            \item Find a basis for each eigenspace of $T$.
            \item Let $A$ be the matrix for $T$ with respect to the standard basis. Find a diagonal matrix $B$ and an invertible matrix $S$ such that $B=S^{-1}AS$. (You do not have to compute $A$.)
        \end{enumerate}
        \end{exercise}
        
        \begin{exercise}
        Let $T:\mathbf{R}^4\to \mathbf{R}^4$ be orthogonal projection to the $2$-dimensional plane $P$ spanned by the vectors $\mathbf{v}=(2,0,1,0)$ and $\mathbf{w}=(-1,0,2,0)$.
        \begin{enumerate}[label=(\alph*)]
            \item Find (with proof) all eigenvalues and eigenvectors, along with their geometric and algebraic multiplicities.
            \item Find the matrix representing $T$ with respect to the standard basis. Is this matrix diagonalizable? Why or why not?
        \end{enumerate}
        \end{exercise}
        
        \begin{exercise}
        Let $a, b \in \mathbf{R}$ and $T: \mathbf{R}^3 \to \mathbf{R}^3$ be the linear transformation that is orthogonal projection onto the plane $z=ax+by$ (with respect to the usual Euclidean inner-product on $\mathbf{R}^3$).
        \begin{enumerate}[label=(\alph*)]
            \item Find the eigenvalues of $T$ and bases for the corresponding eigenspaces.
            \item Is $T$ diagonalizable? Justify.
            \item What is the characteristic polynomial of $T$?
        \end{enumerate}
        \end{exercise}
        
        \begin{exercise}
        Let $T:\mathbf{R}^3\to \mathbf{R}^3$ be the orthogonal projection onto the plane $z=x+y$, with respect to the standard Euclidean inner product.
        \begin{enumerate}[label=(\alph*)]
            \item Write the matrix representation of $T$ with respect to the standard basis.
            \item Is $T$ diagonalizable? Justify your answer.
        \end{enumerate}
        \end{exercise}
        \newpage
        
        \begin{exercise}
        Let $T:\mathbf{R}^3\to \mathbf{R}^3$ be the linear transformation that expands radially by a factor of three around the line parameterized by $L(t)=\begin{bmatrix} 2 \\ 2 \\ -1\end{bmatrix} t$, leaving the line itself fixed.
        \begin{enumerate}[label=(\alph*)]
            \item Find an eigenbasis for $T$ and provide the matrix representation of $T$ with respect to that basis.
            \item Provide the matrix representation of $T$ with respect to the standard basis.
        \end{enumerate}
        \end{exercise}
        
        \begin{exercise}
        Let $a,b\in \mathbf{R}$ and $T:\mathbf{R}^3\to \mathbf{R}^3$ be the linear transformation which is reflection across the plane $z=ax+by$.
        \begin{enumerate}[label=(\alph*)]
            \item Find the eigenvalues of $T$ and for each find a basis for the corresponding eigenspace.
            \item Is $T$ diagonalizable? Justify.
            \item What is the characteristic polynomial of $T$?
            \item What is the minimal polynomial of $T$?
        \end{enumerate}
        \end{exercise}
        
    
        
            
            
        

    
        
    



\end{document}

