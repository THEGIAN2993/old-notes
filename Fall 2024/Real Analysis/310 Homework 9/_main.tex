\documentclass[11pt,twoside,openany]{memoir}
\usepackage{amsthm}
\usepackage{amssymb}
\usepackage{newpxtext,eulerpx,eucal}
\usepackage{amsmath}
\usepackage{datetime}
    \newdateformat{specialdate}{\THEYEAR\ \monthname\ \THEDAY}
\usepackage[margin=1in]{geometry}
\usepackage{fancyhdr}
    \fancyhf{}
    \cfoot{\footnotesize \thepage}
    \pagestyle{fancy}
    \renewcommand{\headrulewidth}{0pt}
\usepackage{thmtools}
    \declaretheoremstyle[
        spaceabove=15pt,
        headfont=\normalfont\bfseries,
        notefont=\mdseries, notebraces={(}{)},
        bodyfont=\normalfont,
        postheadspace=0.5em
        %qed=\qedsymbol
        ]{defs}

    \declaretheoremstyle[
        spaceabove=15pt, % space above the theorem
        headfont=\normalfont\bfseries,
        bodyfont=\normalfont\itshape,
        postheadspace=0.5em
        ]{thmstyle}
    
    \declaretheorem[
        style=thmstyle,
        numberwithin=section
    ]{theorem}

    \declaretheorem[
        style=thmstyle,
        sibling=theorem,
    ]{proposition}

    \declaretheorem[
        style=thmstyle,
        sibling=theorem,
    ]{lemma}

    \declaretheorem[
        style=thmstyle,
        sibling=theorem,
    ]{corollary}

    \declaretheorem[
        numberwithin=section,
        style=defs,
    ]{example}

    \declaretheorem[
        numberwithin=section,
        style=defs,
    ]{definition}

    \declaretheorem[
        numbered=unless unique,
        shaded={rulecolor=black,
    rulewidth=1pt, bgcolor={rgb}{1,1,1}}
    ]{axiom}

    \declaretheorem[numbered=unless unique,style=defs]{note}
    \declaretheorem[numbered=no,style=defs]{question}
    \declaretheorem[numbered=no,style=remark]{answer}
    \declaretheorem[numbered=no,style=remark]{remark}
    \declaretheorem[numbered=no,style=remark]{solution}
    \declaretheorem[numbered=unless unique,style=defs]{exercise}
\usepackage{enumitem}
\usepackage{titlesec}
    \titleformat{\chapter}[display]
    {\normalfont\fillast\large}
    {Chapter {\thechapter}}
    {1ex minus .1ex}
    {\large}
    \titlespacing{\chapter}
    {3pc}{*3}{*2}[3pc]

    \titleformat{\section}[runin]
    {\normalfont\bfseries\large}
    {\S\ \thesection.}{.5em}{}[]
    \titlespacing{\section}
    {\parindent}{1.5ex plus .1ex minus .2ex}{0pt}
\usepackage[utf8x]{inputenc}

\setlength{\parindent}{0pt}
%%%%%%%%%%%%%%%%%%%%%%%%%%%%%%%%%%%%%%%%%%%%%%%%%%%%%%%%%%%%%
%%%%%%%%%%%%%%%%%%%%%%%%%%%%%%%%%%%%%%%%%%%%%%%%%%%%%%%%%%%%%
%to make the correct symbol for Sha
%\newcommand\cyr{%
%\renewcommand\rmdefault{wncyr}%
%\renewcommand\sfdefault{wncyss}%
%\renewcommand\encodingdefault{OT2}%
%\normalfont \selectfont} \DeclareTextFontCommand{\textcyr}{\cyr}


\DeclareMathOperator{\ab}{ab}
\newcommand{\absgal}{\G_{\bbQ}}
\DeclareMathOperator{\ad}{ad}
\DeclareMathOperator{\adj}{adj}
\DeclareMathOperator{\alg}{alg}
\DeclareMathOperator{\Alt}{Alt}
\DeclareMathOperator{\Ann}{Ann}
\DeclareMathOperator{\arith}{arith}
\DeclareMathOperator{\Aut}{Aut}
\DeclareMathOperator{\Be}{B}
\DeclareMathOperator{\Bd}{Bd}
\DeclareMathOperator{\card}{card}
\DeclareMathOperator{\Char}{char}
\DeclareMathOperator{\csp}{csp}
\DeclareMathOperator{\codim}{codim}
\DeclareMathOperator{\coker}{coker}
\DeclareMathOperator{\coh}{H}
\DeclareMathOperator{\compl}{compl}
\DeclareMathOperator{\conj}{conj}
\DeclareMathOperator{\cont}{cont}
\DeclareMathOperator{\crys}{crys}
\DeclareMathOperator{\Crys}{Crys}
\DeclareMathOperator{\cusp}{cusp}
\DeclareMathOperator{\diag}{diag}
\DeclareMathOperator{\diam}{diam}
\DeclareMathOperator{\Dom}{Dom}
\DeclareMathOperator{\disc}{disc}
\DeclareMathOperator{\dist}{dist}
\DeclareMathOperator{\dR}{dR}
\DeclareMathOperator{\Eis}{Eis}
\DeclareMathOperator{\End}{End}
\DeclareMathOperator{\ev}{ev}
\DeclareMathOperator{\eval}{eval}
\DeclareMathOperator{\Eq}{Eq}
\DeclareMathOperator{\Ext}{Ext}
\DeclareMathOperator{\Fil}{Fil}
\DeclareMathOperator{\Fitt}{Fitt}
\DeclareMathOperator{\Frob}{Frob}
\DeclareMathOperator{\G}{G}
\DeclareMathOperator{\Gal}{Gal}
\DeclareMathOperator{\GL}{GL}
\DeclareMathOperator{\Gr}{Gr}
\DeclareMathOperator{\Graph}{Graph}
\DeclareMathOperator{\GSp}{GSp}
\DeclareMathOperator{\GUn}{GU}
\DeclareMathOperator{\Hom}{Hom}
\DeclareMathOperator{\id}{id}
\DeclareMathOperator{\Id}{Id}
\DeclareMathOperator{\Ik}{Ik}
\DeclareMathOperator{\IM}{Im}
\DeclareMathOperator{\Image}{im}
\DeclareMathOperator{\Ind}{Ind}
\DeclareMathOperator{\Inf}{inf}
\DeclareMathOperator{\Isom}{Isom}
\DeclareMathOperator{\J}{J}
\DeclareMathOperator{\Jac}{Jac}
\DeclareMathOperator{\lcm}{lcm}
\DeclareMathOperator{\length}{length}
\DeclareMathOperator*{\limit}{limit}
\DeclareMathOperator{\Log}{Log}
\DeclareMathOperator{\M}{M}
\DeclareMathOperator{\Mat}{Mat}
\DeclareMathOperator{\N}{N}
\DeclareMathOperator{\Nm}{Nm}
\DeclareMathOperator{\NIk}{N-Ik}
\DeclareMathOperator{\NSK}{N-SK}
\DeclareMathOperator{\new}{new}
\DeclareMathOperator{\obj}{obj}
\DeclareMathOperator{\old}{old}
\DeclareMathOperator{\ord}{ord}
\DeclareMathOperator{\Or}{O}
\DeclareMathOperator{\op}{op}
\DeclareMathOperator{\PGL}{PGL}
\DeclareMathOperator{\PGSp}{PGSp}
\DeclareMathOperator{\rank}{rank}
\DeclareMathOperator{\Ran}{Ran}
\DeclareMathOperator{\Rel}{Rel}
\DeclareMathOperator{\Real}{Re}
\DeclareMathOperator{\RES}{res}
\DeclareMathOperator{\Res}{Res}
%\DeclareMathOperator{\Sha}{\textcyr{Sh}}
\DeclareMathOperator{\Sel}{Sel}
\DeclareMathOperator{\semi}{ss}
\DeclareMathOperator{\sgn}{sign}
\DeclareMathOperator{\SK}{SK}
\DeclareMathOperator{\SL}{SL}
\DeclareMathOperator{\SO}{SO}
\DeclareMathOperator{\Sp}{Sp}
\DeclareMathOperator{\Span}{span}
\DeclareMathOperator{\Spec}{Spec}
\DeclareMathOperator{\spin}{spin}
\DeclareMathOperator{\st}{st}
\DeclareMathOperator{\St}{St}
\DeclareMathOperator{\SUn}{SU}
\DeclareMathOperator{\supp}{supp}
\DeclareMathOperator{\Sup}{sup}
\DeclareMathOperator{\Sym}{Sym}
\DeclareMathOperator{\Tam}{Tam}
\DeclareMathOperator{\tors}{tors}
\DeclareMathOperator{\tr}{tr}
\DeclareMathOperator{\Tr}{Tr}
\DeclareMathOperator{\un}{un}
\DeclareMathOperator{\Un}{U}
\DeclareMathOperator{\val}{val}
\DeclareMathOperator{\vol}{vol}

\DeclareMathOperator{\Sets}{S \mkern1.04mu e \mkern1.04mu t \mkern1.04mu s}
    \newcommand{\cSets}{\scalebox{1.02}{\contour{black}{$\Sets$}}}
    
\DeclareMathOperator{\Groups}{G \mkern1.04mu r \mkern1.04mu o \mkern1.04mu u \mkern1.04mu p \mkern1.04mu s}
    \newcommand{\cGroups}{\scalebox{1.02}{\contour{black}{$\Groups$}}}

\DeclareMathOperator{\TTop}{T \mkern1.04mu o \mkern1.04mu p}
    \newcommand{\cTop}{\scalebox{1.02}{\contour{black}{$\TTop$}}}

\DeclareMathOperator{\Htp}{H \mkern1.04mu t \mkern1.04mu p}
    \newcommand{\cHtp}{\scalebox{1.02}{\contour{black}{$\Htp$}}}

\DeclareMathOperator{\Mod}{M \mkern1.04mu o \mkern1.04mu d}
    \newcommand{\cMod}{\scalebox{1.02}{\contour{black}{$\Mod$}}}

\DeclareMathOperator{\Ab}{A \mkern1.04mu b}
    \newcommand{\cAb}{\scalebox{1.02}{\contour{black}{$\Ab$}}}

\DeclareMathOperator{\Rings}{R \mkern1.04mu i \mkern1.04mu n \mkern1.04mu g \mkern1.04mu s}
    \newcommand{\cRings}{\scalebox{1.02}{\contour{black}{$\Rings$}}}

\DeclareMathOperator{\ComRings}{C \mkern1.04mu o \mkern1.04mu m \mkern1.04mu R \mkern1.04mu i \mkern1.04mu n \mkern1.04mu g \mkern1.04mu s}
    \newcommand{\cComRings}{\scalebox{1.05}{\contour{black}{$\ComRings$}}}

\DeclareMathOperator{\hHom}{H \mkern1.04mu o \mkern1.04mu m}
    \newcommand{\cHom}{\scalebox{1.02}{\contour{black}{$\hHom$}}}

         %  \item $\cGroups$
          %  \item $\cTop$
          %  \item $\cHtp$
          %  \item $\cMod$




\renewcommand{\k}{\kappa}
\newcommand{\Ff}{F_{f}}
%\newcommand{\ts}{\,^{t}\!}


%Mathcal

\newcommand{\cA}{\mathcal{A}}
\newcommand{\cB}{\mathcal{B}}
\newcommand{\cC}{\mathcal{C}}
\newcommand{\cD}{\mathcal{D}}
\newcommand{\cE}{\mathcal{E}}
\newcommand{\cF}{\mathcal{F}}
\newcommand{\cG}{\mathcal{G}}
\newcommand{\cH}{\mathcal{H}}
\newcommand{\cI}{\mathcal{I}}
\newcommand{\cJ}{\mathcal{J}}
\newcommand{\cK}{\mathcal{K}}
\newcommand{\cL}{\mathcal{L}}
\newcommand{\cM}{\mathcal{M}}
\newcommand{\cN}{\mathcal{N}}
\newcommand{\cO}{\mathcal{O}}
\newcommand{\cP}{\mathcal{P}}
\newcommand{\cQ}{\mathcal{Q}}
\newcommand{\cR}{\mathcal{R}}
\newcommand{\cS}{\mathcal{S}}
\newcommand{\cT}{\mathcal{T}}
\newcommand{\cU}{\mathcal{U}}
\newcommand{\cV}{\mathcal{V}}
\newcommand{\cW}{\mathcal{W}}
\newcommand{\cX}{\mathcal{X}}
\newcommand{\cY}{\mathcal{Y}}
\newcommand{\cZ}{\mathcal{Z}}


%mathfrak (missing \fi)

\newcommand{\fa}{\mathfrak{a}}
\newcommand{\fA}{\mathfrak{A}}
\newcommand{\fb}{\mathfrak{b}}
\newcommand{\fB}{\mathfrak{B}}
\newcommand{\fc}{\mathfrak{c}}
\newcommand{\fC}{\mathfrak{C}}
\newcommand{\fd}{\mathfrak{d}}
\newcommand{\fD}{\mathfrak{D}}
\newcommand{\fe}{\mathfrak{e}}
\newcommand{\fE}{\mathfrak{E}}
\newcommand{\ff}{\mathfrak{f}}
\newcommand{\fF}{\mathfrak{F}}
\newcommand{\fg}{\mathfrak{g}}
\newcommand{\fG}{\mathfrak{G}}
\newcommand{\fh}{\mathfrak{h}}
\newcommand{\fH}{\mathfrak{H}}
\newcommand{\fI}{\mathfrak{I}}
\newcommand{\fj}{\mathfrak{j}}
\newcommand{\fJ}{\mathfrak{J}}
\newcommand{\fk}{\mathfrak{k}}
\newcommand{\fK}{\mathfrak{K}}
\newcommand{\fl}{\mathfrak{l}}
\newcommand{\fL}{\mathfrak{L}}
\newcommand{\fm}{\mathfrak{m}}
\newcommand{\fM}{\mathfrak{M}}
\newcommand{\fn}{\mathfrak{n}}
\newcommand{\fN}{\mathfrak{N}}
\newcommand{\fo}{\mathfrak{o}}
\newcommand{\fO}{\mathfrak{O}}
\newcommand{\fp}{\mathfrak{p}}
\newcommand{\fP}{\mathfrak{P}}
\newcommand{\fq}{\mathfrak{q}}
\newcommand{\fQ}{\mathfrak{Q}}
\newcommand{\fr}{\mathfrak{r}}
\newcommand{\fR}{\mathfrak{R}}
\newcommand{\fs}{\mathfrak{s}}
\newcommand{\fS}{\mathfrak{S}}
\newcommand{\ft}{\mathfrak{t}}
\newcommand{\fT}{\mathfrak{T}}
\newcommand{\fu}{\mathfrak{u}}
\newcommand{\fU}{\mathfrak{U}}
\newcommand{\fv}{\mathfrak{v}}
\newcommand{\fV}{\mathfrak{V}}
\newcommand{\fw}{\mathfrak{w}}
\newcommand{\fW}{\mathfrak{W}}
\newcommand{\fx}{\mathfrak{x}}
\newcommand{\fX}{\mathfrak{X}}
\newcommand{\fy}{\mathfrak{y}}
\newcommand{\fY}{\mathfrak{Y}}
\newcommand{\fz}{\mathfrak{z}}
\newcommand{\fZ}{\mathfrak{Z}}


%mathbf
\newcommand{\bfA}{\mathbf{A}}
\newcommand{\bfB}{\mathbf{B}}
\newcommand{\bfC}{\mathbf{C}}
\newcommand{\bfD}{\mathbf{D}}
\newcommand{\bfE}{\mathbf{E}}
\newcommand{\bfF}{\mathbf{F}}
\newcommand{\bfG}{\mathbf{G}}
\newcommand{\bfH}{\mathbf{H}}
\newcommand{\bfI}{\mathbf{I}}
\newcommand{\bfJ}{\mathbf{J}}
\newcommand{\bfK}{\mathbf{K}}
\newcommand{\bfL}{\mathbf{L}}
\newcommand{\bfM}{\mathbf{M}}
\newcommand{\bfN}{\mathbf{N}}
\newcommand{\bfO}{\mathbf{O}}
\newcommand{\bfP}{\mathbf{P}}
\newcommand{\bfQ}{\mathbf{Q}}
\newcommand{\bfR}{\mathbf{R}}
\newcommand{\bfS}{\mathbf{S}}
\newcommand{\bfT}{\mathbf{T}}
\newcommand{\bfU}{\mathbf{U}}
\newcommand{\bfV}{\mathbf{V}}
\newcommand{\bfW}{\mathbf{W}}
\newcommand{\bfX}{\mathbf{X}}
\newcommand{\bfY}{\mathbf{Y}}
\newcommand{\bfZ}{\mathbf{Z}}

\newcommand{\bfa}{\mathbf{a}}
\newcommand{\bfb}{\mathbf{b}}
\newcommand{\bfc}{\mathbf{c}}
\newcommand{\bfd}{\mathbf{d}}
\newcommand{\bfe}{\mathbf{e}}
\newcommand{\bff}{\mathbf{f}}
\newcommand{\bfg}{\mathbf{g}}
\newcommand{\bfh}{\mathbf{h}}
\newcommand{\bfi}{\mathbf{i}}
\newcommand{\bfj}{\mathbf{j}}
\newcommand{\bfk}{\mathbf{k}}
\newcommand{\bfl}{\mathbf{l}}
\newcommand{\bfm}{\mathbf{m}}
\newcommand{\bfn}{\mathbf{n}}
\newcommand{\bfo}{\mathbf{o}}
\newcommand{\bfp}{\mathbf{p}}
\newcommand{\bfq}{\mathbf{q}}
\newcommand{\bfr}{\mathbf{r}}
\newcommand{\bfs}{\mathbf{s}}
\newcommand{\bft}{\mathbf{t}}
\newcommand{\bfu}{\mathbf{u}}
\newcommand{\bfv}{\mathbf{v}}
\newcommand{\bfw}{\mathbf{w}}
\newcommand{\bfx}{\mathbf{x}}
\newcommand{\bfy}{\mathbf{y}}
\newcommand{\bfz}{\mathbf{z}}

%blackboard bold

\newcommand{\bbA}{\mathbb{A}}
\newcommand{\bbB}{\mathbb{B}}
\newcommand{\bbC}{\mathbb{C}}
\newcommand{\bbD}{\mathbb{D}}
\newcommand{\bbE}{\mathbb{E}}
\newcommand{\bbF}{\mathbb{F}}
\newcommand{\bbG}{\mathbb{G}}
\newcommand{\bbH}{\mathbb{H}}
\newcommand{\bbI}{\mathbb{I}}
\newcommand{\bbJ}{\mathbb{J}}
\newcommand{\bbK}{\mathbb{K}}
\newcommand{\bbL}{\mathbb{L}}
\newcommand{\bbM}{\mathbb{M}}
\newcommand{\bbN}{\mathbb{N}}
\newcommand{\bbO}{\mathbb{O}}
\newcommand{\bbP}{\mathbb{P}}
\newcommand{\bbQ}{\mathbb{Q}}
\newcommand{\bbR}{\mathbb{R}}
\newcommand{\bbS}{\mathbb{S}}
\newcommand{\bbT}{\mathbb{T}}
\newcommand{\bbU}{\mathbb{U}}
\newcommand{\bbV}{\mathbb{V}}
\newcommand{\bbW}{\mathbb{W}}
\newcommand{\bbX}{\mathbb{X}}
\newcommand{\bbY}{\mathbb{Y}}
\newcommand{\bbZ}{\mathbb{Z}}
\newcommand{\jota}{\jmath}

\newcommand{\bmat}{\left( \begin{matrix}}
\newcommand{\emat}{\end{matrix} \right)}

\newcommand{\pmat}{\left( \begin{smallmatrix}}
\newcommand{\epmat}{\end{smallmatrix} \right)}

\newcommand{\lat}{\mathscr{L}}
\newcommand{\mat}[4]{\begin{pmatrix}{#1}&{#2}\\{#3}&{#4}\end{pmatrix}}
\newcommand{\ov}[1]{\overline{#1}}
\newcommand{\res}[1]{\underset{#1}{\RES}\,}
\newcommand{\up}{\upsilon}

\newcommand{\tac}{\textasteriskcentered}

%mahesh macros
\newcommand{\tm}{\textrm}

%Comments
\newcommand{\com}[1]{\vspace{5 mm}\par \noindent
\marginpar{\textsc{Comment}} \framebox{\begin{minipage}[c]{0.95
\textwidth} \tt #1 \end{minipage}}\vspace{5 mm}\par}

\newcommand{\Bmu}{\mbox{$\raisebox{-0.59ex}
  {$l$}\hspace{-0.18em}\mu\hspace{-0.88em}\raisebox{-0.98ex}{\scalebox{2}
  {$\color{white}.$}}\hspace{-0.416em}\raisebox{+0.88ex}
  {$\color{white}.$}\hspace{0.46em}$}{}}  %need graphicx and xcolor. this produces blackboard bold mu 

\newcommand{\hooktwoheadrightarrow}{%
  \hookrightarrow\mathrel{\mspace{-15mu}}\rightarrow
}

\makeatletter
\newcommand{\xhooktwoheadrightarrow}[2][]{%
  \lhook\joinrel
  \ext@arrow 0359\rightarrowfill@ {#1}{#2}%
  \mathrel{\mspace{-15mu}}\rightarrow
}
\makeatother

\renewcommand{\geq}{\geqslant}
\renewcommand{\leq}{\leqslant}
\newcommand{\midd}{\hspace{4pt}\middle|\hspace{4pt}}
    
    \newcommand{\bone}{\mathbf{1}}
    \newcommand{\sign}{\mathrm{sign}}
    \newcommand{\eps}{\varepsilon}
    \newcommand{\textui}[1]{\uline{\textit{#1}}}
    
    %\newcommand{\ov}{\overline}
    %\newcommand{\un}{\underline}
    \newcommand{\fin}{\mathrm{fin}}
    
    \newcommand{\chnum}{\titleformat
    {\chapter} % command
    [display] % shape
    {\centering} % format
    {\Huge \color{black} \shadowbox{\thechapter}} % label
    {-0.5em} % sep (space between the number and title)
    {\LARGE \color{black} \underline} % before-code
    }
    
    \newcommand{\chunnum}{\titleformat
    {\chapter} % command
    [display] % shape
    {} % format
    {} % label
    {0em} % sep
    { \begin{flushright} \begin{tabular}{r}  \Huge \color{black}
    } % before-code
    [
    \end{tabular} \end{flushright} \normalsize
    ] % after-code
    }

\newcommand{\nl}{\newline \mbox{}}

\newcommand{\h}[1]{\hspace{#1pt}}

\newcommand{\littletaller}{\mathchoice{\vphantom{\big|}}{}{}{}}
\newcommand\restr[2]{{% we make the whole thing an ordinary symbol
  \left.\kern-\nulldelimiterspace % automatically resize the bar with \right
  #1 % the function
  \littletaller % pretend it's a little taller at normal size
  \right|_{#2} % this is the delimiter
  }}

\newcommand{\mtext}[1]{\hspace{6pt}\text{#1}\hspace{6pt}}

\newcommand{\lnorm}{\left\lVert}
\newcommand{\rnorm}{\right\rVert}

\newcommand{\ds}{\displaystyle}
\newcommand{\ts}{\textstyle}

%This adds a "front cover" page.
%{\thispagestyle{empty}
%\vspace*{\fill}
%\begin{tabular}{l}
%\begin{tabular}{l}
%\includegraphics[scale=0.24]{oxy-logo.png}
%\end{tabular} \\
%\begin{tabular}{l}
%\Large \color{black} Module Theory, Linear Algebra, and Homological Algebra \\ \Large \color{black} Gianluca Crescenzo
%\end{tabular}
%\end{tabular}
%\newpage

\newcommand{\sfrac}[2]{{}^{#1}\mskip -5mu/\mskip -3mu_{#2}}


\makeatletter
\newcommand*{\da@rightarrow}{\mathchar"0\hexnumber@\symAMSa 4B }
\newcommand*{\da@leftarrow}{\mathchar"0\hexnumber@\symAMSa 4C }
\newcommand*{\xdashrightarrow}[2][]{%
  \mathrel{%
    \mathpalette{\da@xarrow{#1}{#2}{}\da@rightarrow{\,}{}}{}%
  }%
}
\newcommand{\xdashleftarrow}[2][]{%
  \mathrel{%
    \mathpalette{\da@xarrow{#1}{#2}\da@leftarrow{}{}{\,}}{}%
  }%
}
\newcommand*{\da@xarrow}[7]{%
  % #1: below
  % #2: above
  % #3: arrow left
  % #4: arrow right
  % #5: space left 
  % #6: space right
  % #7: math style 
  \sbox0{$\ifx#7\scriptstyle\scriptscriptstyle\else\scriptstyle\fi#5#1#6\m@th$}%
  \sbox2{$\ifx#7\scriptstyle\scriptscriptstyle\else\scriptstyle\fi#5#2#6\m@th$}%
  \sbox4{$#7\dabar@\m@th$}%
  \dimen@=\wd0 %
  \ifdim\wd2 >\dimen@
    \dimen@=\wd2 %   
  \fi
  \count@=2 %
  \def\da@bars{\dabar@\dabar@}%
  \@whiledim\count@\wd4<\dimen@\do{%
    \advance\count@\@ne
    \expandafter\def\expandafter\da@bars\expandafter{%
      \da@bars
      \dabar@ 
    }%
  }%  
  \mathrel{#3}%
  \mathrel{%   
    \mathop{\da@bars}\limits
    \ifx\\#1\\%
    \else
      _{\copy0}%
    \fi
    \ifx\\#2\\%
    \else
      ^{\copy2}%
    \fi
  }%   
  \mathrel{#4}%
}
\makeatother


\begin{document}
\begin{center}
{\large Math 310 \\[0.1in]Homework 9 \\[0.1in]
Due: 10/9/2024}\\[.25in]
{Name:} {\underline{Gianluca Crescenzo\hspace*{2in}}}\\[0.15in]
\end{center}
\vspace{4pt}
%%%%%%%%%%%%%%%%%%%%%%%%%%%%%%%%%%%%%%%%%%%%%%%%%%%%%%%%%%%%%
    \begin{exercise}
        Suppose $f:[0,1] \rightarrow \bfR$ is a continuous function with $f(0) = f(1)$. Show that there is a $c \in \left[0, \frac{1}{2}\right]$ with $f(c) = f \left(c + \frac{1}{2}\right)$. Conclude that there are always antipodal points on the earth's equator with the same temperature. (Hint: consider $g(x) = f(x) - f \left(x+ \frac{1}{2}\right)$ on $\left[0, \frac{1}{2}\right]$).
    \end{exercise}
        \begin{proof}
            Let $g(x) = f(x) - f \left(x+ \frac{1}{2}\right)$ on $\left[0, \frac{1}{2}\right]$. Note that:
                \begin{equation*}
                \begin{split}
                    g(0) &= f(0) - f \left(\textstyle \frac{1}{2}\right) \\
                    g(\textstyle \frac{1}{2}) &= f(\textstyle \frac{1}{2}) - f(0) = -g(0).
                \end{split}
                \end{equation*}
            This gives that $g(0)g(\frac{1}{2}) < 0$. By location of roots, there exists $c \in (0,\frac{1}{2})$ with $g(c) = 0$. Equivalently, $f(c) - f(c+\frac{1}{2}) = 0$. Hence $f(c) = f(c + \frac{1}{2})$.
        \end{proof}
%%%%%%%%%%%%%%%%%%%%%%%%%%%%%%%%%%%%%%%%%%%%%%%%%%%%%%%%%%%%%
    \begin{exercise}
        Show that the function $f(x) = \frac{1}{x^2}$ is uniformly continuous on $[1,\infty)$ but not on $(0,\infty)$.
    \end{exercise}
        \begin{proof}
            Let $u,v \in [1,\infty)$. We have:
                \begin{equation*}
                \begin{split}
                    |f(u) - f(v)| & = \left|\frac{1}{u^2} - \frac{1}{v^2}\right| \\
                    & = \left|\frac{(u+v)(u-v)}{(uv)^2}\right| \\
                    & \leq \frac{u+v}{(uv)^2}|u-v| \\
                    & \leq 2 |u-v|.
                \end{split}
                \end{equation*}
            Thus $f$ is Lipschitz, giving that $f$ is uniformly continuous on $[1\infty)$. \nl
            
            Now consider $(u_n)_n,(v_n)_n \in (0,\infty)^\bfN$ defined by $u_n = \frac{1}{n}$ and $v_n = \frac{1}{n+1}$. Clearly $(u_n - v_n)_n \rightarrow 0$. Moreover, we have:
                \begin{equation*}
                \begin{split}
                    |f(u_n) - f(v_n)|
                    & = |n^2 - (n+1)^2| \\
                    & = |n^2 - n^2 - 2n - 1| \\
                    & = |-2n - 1| \\
                    & \geq 3.
                \end{split}
                \end{equation*}
            Let $\epsilon_0 = 3$. By the work above, we've shown there exists sequences $(u_n)_n,(v_n)_n$ with $(u_n - v_n)_n \rightarrow 0$ and $|f(u_n) - f(v_n)| \geq \epsilon_0$. Thus $f$ is not uniformly continuous on $(0,\infty)$.
        \end{proof}
%%%%%%%%%%%%%%%%%%%%%%%%%%%%%%%%%%%%%%%%%%%%%%%%%%%%%%%%%%%%%
    \begin{exercise}
        Suppose $f:\bfR \rightarrow \bfR$ is continuous and vanishes at infinity, that is $\limit_{x \rightarrow \pm\infty}f = 0$. Prove that $f$ is uniformly continuous.
    \end{exercise}
        \begin{proof}
            Let $\epsilon > 0$ be given. \nl
            
            Since $f$ vanishes at infinity, there exists $M > 0$ such that $|x| > M$ implies $|f(x)| < \frac{\epsilon}{2}$. \nl
            
            Moreover, since $f$ is continuous on $[-M-1,M+1]$, it is uniformly continuous. In particular, for $u,v \in [-M-1,M +1]$, there exists $\delta > 0$ such that $|u-v| < \delta$ implies $|f(u) - f(v)| < \epsilon$. \nl
            
            Let $\delta_1 = \min\{\delta,1\}$. Let $u,v \in \bfR$ and $|u-v| < \delta_1$. We proceed by cases. \nl
            
            Case 1: $u,v \in [-M,M]$. Then $f$ is uniformly continuous by compactness. \nl
            
            Case 2: $u,v \not\in [-M,M]$. Then $|f(u)-f(v)| \leq |f(u)| + |f(v)| < \frac{\epsilon}{2} + \frac{\epsilon}{2} = \epsilon$. \nl
            
            Case 3: Without loss of generality, suppose $u \in [-M,M]$, $v \not\in [-M,M]$. Since $|u-v| < \delta_1$, it must be that $v \in [-M-1,M+1]$, furthermore $u\in[-M-1,M+1]$ by inclusion. Then by compactness $f$ is uniformly continuous. \nl
            
            Thus $f$ is uniformly continuous on $\bfR$.
        \end{proof}
%%%%%%%%%%%%%%%%%%%%%%%%%%%%%%%%%%%%%%%%%%%%%%%%%%%%%%%%%%%%%
    \begin{exercise}
        Show that $f(x) = x$ and $g(x) = \sin(x)$ are both uniformly continuous on $\bfR$, but the product:
            \begin{equation*}
            \begin{split}
                h(x) = x \sin(x)
            \end{split}
            \end{equation*}
        is not uniformly continuous on $\bfR$.
    \end{exercise}
        \begin{proof}
            Let $u,v \in \bfR$. Observe that:
                \begin{equation*}
                \begin{split}
                    |f(u) - f(v)| = |u-v|.
                \end{split}
                \end{equation*}
            Since $f$ is Lipschitz, $f$ is uniformly continuous. Without loss of generality, suppose $u  <v$. Apply the Mean Value Theorem to $g$ on $[u,v]$. Then there exists $c \in (u,v)$ with:
                \begin{equation*}
                \begin{split}
                    \frac{\sin(v) - \sin(u)}{v-u} = \sin'(c) = \cos(c).
                \end{split}
                \end{equation*}
            Taking the absolute value of both sides gives:
                \begin{equation*}
                \begin{split}
                    \left|\frac{\sin(v) - \sin(u)}{v-u}\right| = |cos(c)| \leq 1.
                \end{split}
                \end{equation*}
            Whence
                \begin{equation*}
                \begin{split}
                    |\sin(v) - \sin(u)| \leq |v-u|.
                \end{split}
                \end{equation*}
            Thus $g$ is Lipschitz, implying that it is uniformly continuous. \nl
            
            Let $(u_n)_n,(v_n)_n \in \bfR^\bfN$ defined by $u_n = n\pi$ and $v_n = n\pi + \frac{1}{n}$. Clearly $(u_n - v_n)_n \rightarrow 0$. Moreover:
                \begin{equation*}
                \begin{split}
                    |f(u_n) - f(v_n)|
                    & = \left|n\pi\sin(n\pi) - \left(n\pi + \frac{1}{n}\sin \left(n\pi + \frac{1}{n}\right)\right)\right| \\
                    & = \left|n\pi\cos(n\pi)\sin \left(\frac{1}{n}\right) + \frac{1}{n}\cos(n\pi)\sin \left(\frac{1}{n}\right)\right| \\
                    & = \left|n\pi(-1)^n \sin\left(\frac{1}{n}\right)+ \frac{1}{n}(-1)^n\sin \left(\frac{1}{n}\right)\right| \\
                    & = \left|n\pi (-1)^n \frac{1}{n} + \frac{1}{n^2}\right|. \quad\quad \text{\tiny (For large $n$)}
                \end{split}
                \end{equation*}
            So for $n$ large, $|f(u_n) - f(v_n)| \geq \frac{\pi}{2}$. Take $\epsilon_0 = \frac{\pi}{2}$. Then by the work above, $h(x) = x\sin(x)$ is not uniformly convergent.
        \end{proof}
%%%%%%%%%%%%%%%%%%%%%%%%%%%%%%%%%%%%%%%%%%%%%%%%%%%%%%%%%%%%%
    \begin{exercise}
        If $f:D \rightarrow \bfR$ is uniformly continuous and $|f(x)| \geq k > 0$ for some $k$, show that $\frac{1}{f}$ is uniformly continuous on $D$.
    \end{exercise}  
        \begin{proof}
            Let $\epsilon$ be given and $u,v \in D$. Since $f$ is uniformly continuous, there exists $\delta$ such that $|u-v| < \delta$ implies $|f(u)-f(v)| < k^2 \epsilon$. Moreover, $|u-v| < \delta$ implies:
                \begin{equation*}
                \begin{split}
                    \left|\frac{1}{f(u)} - \frac{1}{f(v)}\right|
                    & = \left|\frac{f(u) - f(v)}{f(u)f(v)}\right| \\
                    & = \frac{1}{f(u)f(v)}\left|f(u) - f(v)\right| \\
                    & < \frac{1}{k^2} k^2 \epsilon \\
                    & < \epsilon.
                \end{split}
                \end{equation*}
            Thus $\frac{1}{f}$ is uniformly continuous.
        \end{proof}
%%%%%%%%%%%%%%%%%%%%%%%%%%%%%%%%%%%%%%%%%%%%%%%%%%%%%%%%%%%%%
    \begin{exercise}
        If $D \subseteq \bfR$ is a bounded set and $f:D \rightarrow \bfR$ is uniformly continuous, show that $f$ is bounded (This gives another proof that $f(x) = \frac{1}{x}$ is not uniformly continuous on $(0,1)$).
    \end{exercise}
        \begin{proof}
            Suppose towards contradiction $f$ is unbounded. Then for all $n\geq 1$, there exists $x_n$ such that $|f(x_n)| \geq n$. Since $(x_n)_n \in D^\bfN$, by Bolzano-Weierstrass there exists a convergent subsequence $(x_{n_k})_k \rightarrow c$. Since $f$ is continuous, we have that $(f(x_{n_k}))_k \rightarrow f(c)$. However, this is a contradiction, as $|f(x_{n_k})| \geq n_k$. Thus $f$ is bounded.
        \end{proof}
%%%%%%%%%%%%%%%%%%%%%%%%%%%%%%%%%%%%%%%%%%%%%%%%%%%%%%%%%%%%%
    \begin{exercise}
        Prove that there does not exist a continuous function $f:\bfR \rightarrow \bfR$ with:
            \begin{equation*}
            \begin{split}
                f(\bfQ) \subseteq \bfR \setminus \bfQ; \hspace{8pt} f(\bfR \setminus \bfQ) \subseteq \bfQ.
            \end{split}
            \end{equation*}
    \end{exercise}
        \begin{proof}
            Note that $f(\bfR) = f(\bfR \setminus \bfQ) \cup f(\bfQ)$. Since $f(\bfR \setminus \bfQ)$ and $f(\bfQ)$ are countable, it must be that $f(\bfR)$ is countable. Moreover, since $\bfR$ is an interval, it must be that $f(\bfR)$ is an interval. Hence $f(\bfR) = \{a\}$ for some $a \in \bfR$. \nl
            
            But this gives that $f(\bfR\setminus\bfQ) = \{a\}$ and $f(\bfQ) = \{a\}$. Hence $a \in \bfR\setminus\bfQ$ and $a \in \bfQ$, which is a contradiction.
        \end{proof}
%%%%%%%%%%%%%%%%%%%%%%%%%%%%%%%%%%%%%%%%%%%%%%%%%%%%%%%%%%%%%
    \begin{exercise}
        Let $n \in \bfN$ and consider the function:
            \begin{equation*}
            \begin{split}
                f(x) =
                    \begin{cases}
                        x^n, & x > 0 \\
                        0, & x \leq 0 .
                    \end{cases}
            \end{split}
            \end{equation*}
        For which values of $n$ is $f$ differentiable at $x = 0$?
    \end{exercise}
        \begin{proof}
            Observe that:
                \begin{equation*}
                \begin{split}
                    \limit_{x \rightarrow 0^+}\frac{f(x) - f(0)}{x-0} = \limit_{x \rightarrow 0^+}\frac{x^n}{x} = \limit_{x \rightarrow 0^+}x^{n-1} = \begin{cases} 0, & n > 1 \\ 1, & n = 1.\end{cases}
                \end{split}
                \end{equation*}
                \begin{equation*}
                \begin{split}
                    \limit_{x \rightarrow 0^-}\frac{f(x) - f(0)}{x-0} = \limit_{x \rightarrow 0^-}\frac{0}{x} = 0.
                \end{split}
                \end{equation*}
            Thus $f$ is differentiable at $x=0$ for $n > 1$.
        \end{proof}
%%%%%%%%%%%%%%%%%%%%%%%%%%%%%%%%%%%%%%%%%%%%%%%%%%%%%%%%%%%%%
    \begin{exercise}
        Consider the function:
            \begin{equation*}
            \begin{split}
                f(x) =
                    \begin{cases}
                        x^2, & x \in \bfQ \\
                        0, & x \not\in \bfQ .
                    \end{cases}
            \end{split}
            \end{equation*}
        Show that $f$ is differentiable at $x = 0$ and find $f'(0)$.
    \end{exercise}
        \begin{proof}
            Observe that:
                \begin{equation*}
                \begin{split}
                    \limit_{x \rightarrow 0 }\frac{f(x) - f(0)}{x- 0} = \limit_{x \rightarrow 0}\frac{ x^2 \mathbf{1}_\bfQ}{x} = \limit_{x \rightarrow 0}x \mathbf{1}_\bfQ = 0.
                \end{split}
                \end{equation*}
        \end{proof}
%%%%%%%%%%%%%%%%%%%%%%%%%%%%%%%%%%%%%%%%%%%%%%%%%%%%%%%%%%%%%
    \begin{exercise}
        Determine the values of $x$ where $f(x) = x|x|$ is differentiable.
    \end{exercise}
        \begin{proof}
            Note that:
                \begin{equation*}
                \begin{split}
                    f(x) = x|x| = \begin{cases} x^2, & x \geq 0 \\ -x^2, & x < 0.\end{cases}
                \end{split}
                \end{equation*}
            Clearly $f$ is differentiable at $c \neq 0$. So observe that:
                \begin{equation*}
                \begin{split}
                    \limit_{x \rightarrow 0^+} \frac{f(x) - f(0)}{x- 0} = \limit_{x \rightarrow 0^+}\frac{x^2}{x} = \limit_{x \rightarrow 0^+}x = 0.
                \end{split}
                \end{equation*}
                \begin{equation*}
                \begin{split}
                    \limit_{x \rightarrow 0^-} \frac{f(x) - f(0)}{x- 0} = \limit_{x \rightarrow 0^-}\frac{-x^2}{x} = \limit_{x \rightarrow 0^-}-x = 0.
                \end{split}
                \end{equation*}
            Thus $f$ is differentiable everywhere.
        \end{proof}
%%%%%%%%%%%%%%%%%%%%%%%%%%%%%%%%%%%%%%%%%%%%%%%%%%%%%%%%%%%%%
    \begin{exercise}
        Let $I$ be an interval and suppose $f:I\rightarrow \bfR$ is differentiable with $f'(x) < 0$ for all $x \in I$. Show that $f$ is strictly decreasing on $I$.
    \end{exercise}
       \begin{proof}
            Let $x_1,x_2 \in I$ with $x_1 < x_2$. Apply the Mean Value Theorem to $f$ on $[x_1,x_2]$. Then there exists $c \in (x_1,x_2)$ with:
                \begin{equation*}
                \begin{split}
                    f'(c) = \frac{f(x_2) -f(x_1)}{x_2 - x_1} < 0.
                \end{split}
                \end{equation*}
            Since $x_2 - x_1 > 0$, it must be that $f(x_2) - f(x_1) < 0$. Thus $f(x_2) < f(x_1)$, establishing $f$ to be strictly decreasing on $I$.
        \end{proof}
%%%%%%%%%%%%%%%%%%%%%%%%%%%%%%%%%%%%%%%%%%%%%%%%%%%%%%%%%%%%%
    \begin{exercise}
        Prove that the function $f(x) = x^3 + e^x$ has a unique real root. 
    \end{exercise}
        \begin{proof}
            Note that $f(1) = 1 + e > 0$ and $f(-1) = \frac{1-e}{e} < 0$. By the Intermediate Value Theorem, there exists $c \in (-1,1)$ with $f(c) = 0$. Since $f'(x) > 0$ on $\bfR$, we have that $f$ is strictly increasing. Hence $c$ is unique.
        \end{proof}
%%%%%%%%%%%%%%%%%%%%%%%%%%%%%%%%%%%%%%%%%%%%%%%%%%%%%%%%%%%%%
    \begin{exercise}
        Show that $\log(x) \leq x-1$ for all $x > 0$.
    \end{exercise}
        \begin{proof}
            We proceed by cases. \nl
            
            Case 1: $0 < x < 1$. Apply the Mean Value Theorem to $\log(x)$ on $[x,1]$. Then there exists $c \in (x,1)$ such that:
                \begin{equation*}
                \begin{split}
                    \frac{\log(1) - \log(x)}{1 - x} = \frac{1}{c} \geq 1.
                \end{split}
                \end{equation*}
            Whence $-\log(x) \geq 1-x$; i.e., $\log(x) \leq x-1$. \nl
            
            Case 2: $x = 1$. Then clearly $\log(x) = x-1$. \nl
            
            Case 3: $x > 1$. Apply the Mean Value Theorem to $\log(x)$ on $[1,x]$. Then there exists $c \in (1,x)$ such that:
                \begin{equation*}
                \begin{split}
                    \frac{\log(x) - \log(1)}{x-1} = \frac{1}{c} \leq 1.
                \end{split}
                \end{equation*}
            Whence $\log(x) \leq x-1$ for all $x > 0$.
        \end{proof}
%%%%%%%%%%%%%%%%%%%%%%%%%%%%%%%%%%%%%%%%%%%%%%%%%%%%%%%%%%%%%
    \newpage
    \begin{exercise}
        Suppose $f:[0,2] \rightarrow \bfR$ is continuous on $[0,2]$ and differentiable on $(0,2)$ and satisfies $f(0) = 0$, $f(1) = 1$, and $f(2) = 1$.
            \begin{enumerate}[label = (\roman*)]
                \item Show that there is a $c_1 \in (0,1)$ with $f'(c_1) = 1$.
                    \begin{proof}
                        Apply the Mean Value Theorem to $f$ on $[0,1]$. Then there exists $c_1 \in (0,1)$ so that $f'(c_1) = \frac{f(1) - f(0)}{1-0} = 1$.
                    \end{proof}
                \item Show that there is a $c_2 \in (1,2)$ with $f'(c_2) = 0$.
                    \begin{proof}
                        Apply the Mean Value Theorem to $f$ on $[1,2]$. Then there exists $c_2 \in (1,2)$ so that $f'(c_2) = \frac{f(2) - f(1)}{2-1} = 0$.
                    \end{proof}
                \item Show that there is a $c_3 \in (0,2)$ with $f'(c_3) = \frac{1}{3}$.
                    \begin{proof}
                        Apply Darboux's Theorem to $f$ on $[c_1,c_2]$. Note that $f'(c_2) = 0 < \frac{1}{3} < 1 = f'(c_2)$. So there exists $c_3 \in [c_1,c_2]$ so that $f'(c_3) = \frac{1}{3}$.
                    \end{proof}
            \end{enumerate}
    \end{exercise}
%%%%%%%%%%%%%%%%%%%%%%%%%%%%%%%%%%%%%%%%%%%%%%%%%%%%%%%%%%%%%
    \begin{exercise}
        Suppose $f,g:\bfR \rightarrow (0,\infty)$ are everywhere differentiable with $f' = f$ and $g' = g$. Prove that $f = \alpha g$ for some constant $\alpha > 0$.
    \end{exercise}
        \begin{proof}
            Observe that:
                \begin{equation*}
                \begin{split}
                    \left(\frac{f}{g}\right)' = \frac{fg'-f'g}{g^2} = \frac{fg-fg}{g^2} = 0.
                \end{split}
                \end{equation*}
            Thus $\frac{f}{g} = \alpha$ for some $\alpha > 0$. Whence $f = \alpha g$.
        \end{proof}
%%%%%%%%%%%%%%%%%%%%%%%%%%%%%%%%%%%%%%%%%%%%%%%%%%%%%%%%%%%%%
    \begin{exercise}
        Let $h = \mathbf{1}_{[0,\infty)}$. Prove that there does not exist a function $f:\bfR \rightarrow \bfR$ for which $f' = h$ on $\bfR$.
    \end{exercise}
        \begin{proof}
            The converse of Darboux's Theorem says:
                \begin{equation*}
                \begin{split}
                    (\exists k \in (f'(a),f'(b)))(\forall c \in (a,b)) : f'(c) \neq k \implies \mtext{\hspace{-6pt}$f$ is not differentiable.}
                \end{split}
                \end{equation*}
            Let $k = \frac{1}{2}$ Notice that for all $c \in \bfR$, we have that $f'(c) \neq \frac{1}{2}$. Whence $f$ is not differentiable.
        \end{proof}
%%%%%%%%%%%%%%%%%%%%%%%%%%%%%%%%%%%%%%%%%%%%%%%%%%%%%%%%%%%%%
\end{document}
