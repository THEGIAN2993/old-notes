\chapter{Sequences}\label{chapter:sequences}
\vspace{12pt}

\section{Basic Definitions and Examples}
    \begin{definition}
        A \textui{sequence} in a metric space $X$ is a map $x: \bfN \rightarrow X$. We often write $x = (x_n)_{n \geq 1} = (x_1,x_2,x_3,...)$, where $x_n = x(n)$. If $X = \bfR$, we call $x$ a \textui{real sequence}.
    \end{definition}

    \begin{example}[Sequences Defined Explictly]
        \phantom{a}
        \begin{enumerate}[label = (\arabic*)]
            \item A constant sequence: $x_n = t$, $ (x_n)_{n \geq 1} = (t,t,t,t,...)$.
            \item Sequences defined by a function: $d_n = (1+\frac{1}{n})^n$.
            \item Geometric sequences\footnote{These are called geometric because the ratio between each $x_n$ is constant: $x_{n+1}/x_n = b^{n+1}/b^n = b$.}: fix $b \in \bfR$, $x_n = b^n$. Then $(x_n)_{n \geq 1} = (1,b,b^2,b^3,...)$.
        \end{enumerate}
    \end{example}

    \begin{example}[Sequences Defined Recursively]
        \phantom{a}
        \begin{enumerate}[label = (\arabic*)]
            \item Let $a_1 = 1$, $a_{n+1} = 2a_n + 1$. Then $(a_n)_{n \geq 1} = (1,3,7,15,...)$.
            \item Let $f_1 = 1$, $f_2 = 1$, $f_{n+1} = f_n + f_{n-1}$. Then $(f_n)_{n=1}^\infty = (1,1,2,3,5,8,...)$. This is the \textit{Fibonacci sequence}.
            \item Let $X$ be a metric space and $f:X \rightarrow X$ be an endomorphism. Fix $x_0 \in X$. Then define:
                \begin{equation*}
                \begin{split}
                    x_1 &= f(x_0) \\
                    x_2 &= f(x_1) \\
                    &\vdots \\
                    x_n &= f(x_{n-1}).
                \end{split}
                \end{equation*}
        \end{enumerate}
    \end{example}

    \begin{example}[New Sequences from Old]
        \phantom{a}
        \begin{enumerate}[label = (\arabic*)]
            \item Let $(a_n)_{n\geq 1}$ and $(b_n)_{n \geq 1}$ be sequences. Then define:
                \begin{equation*}
                \begin{split}
                    (a_n)_{n\geq 1} \pm (b_n)_{n\geq 1} &= (a_n+b_n)_{n\geq 1}, \\
                    t(a_n)_{n\geq 1} &= (ta_n)_{n\geq 1}, \\
                    (a_n)_{n\geq 1}\cdot (b_n)_{n\geq 1} &= (a_n\cdot b_n)_{n\geq 1}. \\
                \end{split}
                \end{equation*}
            If $(b_n)_{n\geq 1} \neq 0$ for all $n$, then:
                \begin{equation*}
                \begin{split}
                    \frac{(a_n)_{n\geq 1}}{(b_n)_{n\geq 1}} & = \left(\frac{a_b}{b_n}\right)_{n\geq 1}.
                \end{split}
                \end{equation*}

            \item Given $(x_n)_{n \geq 1}$ and $k \in \bfN$, consider $(x_{n+k})_{n=0}^\infty = (x_k,x_{k+1},x_{k+1},...)$. This is called a \textit{shift} or the \textit{$k^{\text{th}}$ tail} of $(x_n)_{n \geq 1}$.
            
            \item If $(a_n)_{n \geq 1}$ is a sequence, $a_n \neq 0$ for all $n$, consider:
                \begin{equation*}
                \begin{split}
                    r_n = \frac{a_{n+1}}{a_n}.
                \end{split}
                \end{equation*}
            So $(r_n)_{n \geq 1} = \left(\frac{a_2}{a_1},\frac{a_3}{a_2},\frac{a_4}{a_3},...\right)$. These are called sequences of \textit{ratios}.

            \item Given a real sequence $(x_k)_{k=1}^\infty$, consider the sequence $(s_n)_{n=1}^\infty$ where:
                \begin{equation*}
                \begin{split}
                    s_1 &= x_1 \\
                    s_2 &= x_1 + x_2 = s_1 + x_2 \\
                    s_3 &= x_1 + x_2 + x_3 = s_2 + x_3 \\
                    &\vdots \\
                    s_n &= \sum_{k = 1}^n x_k = s_{n-1} + x_k.
                \end{split}
                \end{equation*}
            We call these \textit{$n^{\text{th}}$ partial sums}. An example of these are geometric sequences and telescoping sequences.
        \end{enumerate}
    \end{example}

\section{Convergence}
    \begin{definition}
        Let $(x_n)_{n\geq 1}$ be a sequence.
        \begin{enumerate}[label = (\arabic*)]
            \item $x_n$ is \textui{increasing} if $x_1 \leq x_2 \leq x_3 \leq ...$
            \item $x_n$ is \textui{decreasing} if $x_1 \geq x_2 \geq x_3 \geq ...$
            \item $x_n$ is \textui{strictly increasing} if $x_1 < x_2 < x_3 < ...$
            \item $x_n$ is \textui{strictly decreasing} if $x_1 > x_2 > x_3 > ...$
        \end{enumerate}
    \end{definition}

    \begin{note}
        A sequence is said to \textit{eventually} have a certain property, if it does not have the said property across all its ordered instances, but will after some instances have passed.
    \end{note}

    \begin{note}
        $x_n$ is \textui{monotone} if it is either increasing or decreasing, strictly increasing, or strictly decreasing.
    \end{note}

    \begin{definition}\label{def:convergence}
        A sequence $(x_n)_n$ in a metric space $X$ \textui{converges} to $x \in X$ if:
            \begin{equation*}
            \begin{split}
                (\forall \epsilon > 0)(\exists N_\epsilon \in \bfN) \mtext{s.t.} n \geq N_\epsilon \implies d(x_n,x) < \epsilon.\footnotemark
            \end{split}
            \end{equation*}
        If no such $x$ exists, the sequence is \textui{divergent}. If $(x_n)_n$ converges to $x$, we write $(x_n)_n \xrightarrow{n \rightarrow \infty} x$ or $\lim_{n \rightarrow \infty}x_n = x$.
        \footnotetext{I try not to use first-order logic symbols but this will be one of the few exceptions.}
    \end{definition}

    \begin{example}
        Let $X = \bfR$. Then from the above definition, write $d(x_n,x) = |x_n - x|$. Recall that this is equivalent to $x_n \in V_\epsilon(x)$. We can visually represent convergence as follows:

        \begin{center}
            \begin{tikzpicture}
                \begin{axis}[
                    axis lines = middle,
                    xmin=0, xmax=11, % x-axis starts at 0 and extends to 10
                    ymin=-1, ymax=10, % y-axis range
                    xtick={1,2,3,4,5,6,7,8,9,10}, % add ticks 1 to 8 on the x-axis
                    xticklabels={1,2,3,4,5,6,7,8,9,10}, % labels for the x-axis ticks
                    ytick={4, 5, 6}, % positions for the y-axis ticks
                    yticklabels={$x-\epsilon$, $x$, $x+\epsilon$}, % labels for the y-axis ticks
                ]

                
                % Add dashed horizontal lines at x+\epsilon and x-\epsilon
                \addplot[dashed, color=red] coordinates {(0, 4) (11, 4)}; % dashed line at x+\epsilon
                \addplot[dashed, color=red] coordinates {(0, 6) (11, 6)}; % dashed line at x-\epsilon

                \addplot[only marks,mark=*,mark options={fill=green},color=green]coordinates {(1, 7)};
                \addplot[only marks,mark=*,mark options={fill=green},color=green]coordinates {(2, 2.5)};
                \addplot[only marks,mark=*,mark options={fill=green},color=green]coordinates {(3, 6.5)};
                \addplot[only marks,mark=*,mark options={fill=green},color=green]coordinates {(4, 3)};
                \addplot[only marks,mark=*,mark options={fill=green},color=green]coordinates {(5, 6.2)};
                \addplot[only marks,mark=*,mark options={fill=green},color=green]coordinates {(6, 4)};
                \addplot[only marks,mark=*,mark options={fill=green},color=green]coordinates {(7, 5.8)};
                \addplot[only marks,mark=*,mark options={fill=green},color=green]coordinates {(8, 4.4)};
                \addplot[only marks,mark=*,mark options={fill=green},color=green]coordinates {(9, 5.3)};
                \addplot[only marks,mark=*,mark options={fill=green},color=green]coordinates {(10, 5)};
        
                \end{axis}
            \end{tikzpicture}
        \end{center}
        If the sequence is convergent it will eventually be contained between the two dashed lines.
    \end{example}

    \begin{example}
        Prove $\left(\frac{1}{n}\right)_{n\geq 1} \rightarrow 0$.
    \end{example}
    \begin{solution}
        Let $\epsilon > 0$ be given. Find $N_\epsilon$ large so $\frac{1}{N_\epsilon} < \epsilon$ (\nameref{prop:arch-2}). So if $n \geq N_\epsilon$, then $\frac{1}{n} \leq \frac{1}{N_\epsilon}$, implying that:
            \begin{equation*}
            \begin{split}
                \left|\frac{1}{n} - 0\right| = \frac{1}{n} \leq \frac{1}{N_\epsilon} < \epsilon.
            \end{split}
            \end{equation*}
    \end{solution}

    \begin{example}
        Prove $\left(\frac{5n-1}{3-n}\right)_{n = 4}^\infty \rightarrow -5$. 
    \end{example}
    \begin{solution}
        Note that:
            \begin{equation*}
            \begin{split}
                |x_n - x|
                & = \left|\frac{5n-1}{3-n} - (-5)\right| \\
                & = \frac{14}{\left|3-n\right|} \\
                & = \frac{14}{n-3}.
            \end{split}
            \end{equation*}
        So given $\epsilon > 0$, we want $\frac{14}{n-3} < \epsilon$, provided $n$ is big enough. This means $\frac{14}{\epsilon} + 3 < n$. We can now start the proof.

        Given $\epsilon > 0$, find $N_\epsilon$ such that $N_\epsilon > \frac{14}{\epsilon} + 3$ (\nameref{prop:arch-1}). Now, if $n \geq N_\epsilon$, then $n > \frac{14}{\epsilon} + 3$ implies $n-3 > \frac{14}{\epsilon}$. Hence:
            \begin{equation*}
            \begin{split}
                \frac{14}{n-3} = \left|x_n - x\right| < \epsilon.
            \end{split}
            \end{equation*}
    \end{solution}
    
    \begin{lemma}
        Let $(X,d)$ be a metric space. Then $(x_n)_n \rightarrow x$ if and only if $(d(x_n,x))_n \rightarrow 0$.
    \end{lemma}
        \begin{proof}
            Suppose $(x_n)_n \rightarrow x$. Let $\epsilon > 0$. Find $N_\epsilon \in \bfN$ such that $n \geq N_\epsilon$ implies $d(x_n , x) \leq \epsilon$. This is equivalent to $|d(x_n, x) - 0 | \leq \epsilon$. The converse follows identically.
        \end{proof}

    \begin{lemma}
        If $(t_n)_n$ is a real sequence, then $(t_n)_n \rightarrow 0$ if and only if $(|t_n|)_n \rightarrow 0$.
    \end{lemma}
        \begin{proof}
            {\color{red} need to do}
        \end{proof}

    \begin{lemma}\label{lemma:thething}
        Let $(X,d)$ be a metric space and $(x_n)_n$ a sequence in $(X,d)$. If $d(x_n,x) \leq c\epsilon_n$, where $c$ is a constant and $(\epsilon_n)_n \rightarrow 0$ with $\epsilon_n > 0$ for all $n$, then $(x_n)_n \rightarrow x$.
    \end{lemma}
        \begin{proof}
        {\color{red} need to do}
        \end{proof}

    \begin{example}
        Prove $\left(\frac{\sin(n^2 - 1)}{n^2 + 3}\right)_n \rightarrow 0$.
    \end{example}
        \begin{solution}
            Note that:
                \begin{equation*}
                \begin{split}
                \left|\frac{\sin(n^2 - 1)}{n^2 + 3} - 0 \right|
                 = \frac{|\sin(n^2 - 1)|}{n^2 + 3} 
                 \leq \frac{1}{n^2 + 3} 
                 \leq \frac{1}{n^2} 
                 \leq \frac{1}{n}.
                \end{split}
                \end{equation*}
            By Lemma~\ref{lemma:thething}, take $c=1$ and $\epsilon_n = \frac{1}{n}$.
        \end{solution}

    \begin{example}
        Prove $\left(\frac{1}{2^n}\right)_n \rightarrow 0$.
    \end{example}
        \begin{solution}
            Note that:
                \begin{equation*}
                \begin{split}
                    \left|\frac{1}{2^n}\right| = \frac{1}{2^n} \leq \frac{1}{n}.
                \end{split}
                \end{equation*}
        \end{solution}
    
    \begin{example}
        Prove $\left(\frac{1}{n} - \frac{1}{n+1}\right)_n \rightarrow 0$.
    \end{example}
        \begin{solution}
            Note that:
                \begin{equation*}
                \begin{split}
                    \left|\frac{1}{n} - \frac{1}{n+1}\right| = \frac{1}{n} - \frac{1}{n+1} \leq \frac{1}{n}.
                \end{split}
                \end{equation*}
        \end{solution}

    \begin{lemma}
        Let $k \geq 1$ be fixed. Given a sequence $(x_n)_n$ in a metric space $(X,d)$, $(x_n)_n \rightarrow x$ if and only if $(x_{k+n})_n \rightarrow x$.
    \end{lemma}
        \begin{proof}
            Let $(x_n)_n \rightarrow x$. Let $\epsilon >0$. We know there exists $N_\epsilon \in \bfN$ with $n \geq N_\epsilon$ implying $d(x_n , x) < \epsilon$. But if $n\geq N_\epsilon$, then $n+k \geq N_\epsilon$. Hence $d(x_{n+k}, x) < \epsilon$.

            Conversely, assume that $(x_{n+k})_n \rightarrow 0$. Let $\epsilon > 0$. We know there exists $N_\epsilon \in \bfN$ such that $n \geq N_\epsilon$ implies $d(x_{n+k}, x) < \epsilon$. Consider $M = N_\epsilon + k$. Then $n \geq M$ implies $n \geq N_\epsilon + k$; i.e., $n-k \geq N_\epsilon$. Hence $d(x_{(n-k)+k}, x)= d(x_n , x) < \epsilon$.
        \end{proof}
    
    \begin{proposition}
        Suppose $(x_n)_n$ is a real sequence with $\left(\left|\frac{x_{n+1}}{x_n}\right|\right)_n \rightarrow L < 1$. Then $(x_n)_n \rightarrow 0$.
    \end{proposition}
        \begin{proof}
            Since $L<1$, let $\rho$ be any number satisfying $L < \rho < 1$. Set $\epsilon = \rho - L$. Since $\left(\left|\frac{x_{n+1}}{x_n}\right|\right)_n \rightarrow L$, we know there exists $N_\epsilon \in \bfN$ such that $n \geq N_\epsilon$ implies $\left|\frac{x_{n+1}}{x_n}\right| < \rho$, or equivalently $|x_{n+1}| < \rho|x_n|$. Now observe that:
                \begin{equation*}
                \begin{split}
                    |x_{N+1}| &< \rho|x_N| \\
                    |x_{N+2}| &< \rho|x_{N+1}| < \rho \cdot \rho |x_N| = \rho^2 |x_N|\\
                    &\vdots
                \end{split}
                \end{equation*}
            Inductively, $|x_{N+n}| < \rho^n|x_N|$ for $n \in \bfN$. But note that $|x_{N+n}| = |x_{N+n} - 0|$ is a tail of $(x_n)_n$. So by taking $\epsilon_n = \rho^n$ and $c = |x_N|$, Lemma~\ref{lemma:thething} gives $(x_n)_n \rightarrow 0$.
        \end{proof}

    \begin{note}
        The negation of Definition~\ref{def:convergence} is:
            \begin{equation*}
            \begin{split}
                (\exists \epsilon_0 > 0)(\forall N_\epsilon \in \bfN), \exists n \in \bfN \mtext{s.t.} n \geq N_\epsilon \mtext{and} d(x_n,x) \geq \epsilon_0.
            \end{split}
            \end{equation*}
    \end{note}

    \begin{example}
        Prove $((-1)^n)_n$ is divergent.
    \end{example}
        \begin{solution}
            Suppose $((-1)^n)_n \rightarrow x$. Let $\epsilon_0 = \max \left\{|x-1|,|x+1|\right\} > 0$. Let $N \in \bfN$. Set $n = 2N$. Then:
                \begin{equation*}
                \begin{split}
                    (-1)^{2N} = 1 \\
                    (-1)^{2N+1} = -1
                \end{split}
                \end{equation*}
            Hence $d((-1)^{2N},x) = |x-1| \geq \epsilon_0$ or $d((-1)^{2N+1},x) = |x+1| \geq \epsilon_0$.
        \end{solution}

    \begin{exercise}
        Prove $(\sin(n))_n$ is divergent.
    \end{exercise}

    \begin{proposition}
        Let $(X,d)$ be a metric space. A sequence $(x_n)_n$ can have at most one limit.
    \end{proposition}
        \begin{proof}
            Suppose $(x_n)_n \rightarrow L_1$ and $(x_n)_n \rightarrow L_2$ with $L_1 \neq L_2$. Set $\delta = \frac{|L_1 - L_2|}{2}$. Then $V_\delta(L_1) \cap V_\delta(L_2) = \emptyset$. Since $(x_n)_n \rightarrow L_1$, there exists $N_1 \in \bfN$ such that $n \geq N_1$ implies $x_n \in V_\delta(L_1)$. Similarly, since $(x_n)_n \rightarrow L_2$, there exists $N_2 \in \bfN$ such that $n \geq N_2$ implies $x_n \in V_\delta(L_2)$. Pick $N = \max\{N_1,N_2\}$. Then $x_N \in V_\delta(L_1) \cap V_\delta(L_2)$, which is a contradiction.
        \end{proof}

    \begin{lemma}\label{lemma:abs-convergence}
        If $(x_n)_n \rightarrow x$, then $(|x_n|)_n \rightarrow |x|$.
    \end{lemma}
        \begin{proof}
            Note that $(|x_n|)_n$ if and only if $(|x_n -x|)_n \rightarrow 0$. Since $||x_n| - |x|| \leq |x_n - x|$, taking $\epsilon_n = |x_n - x|$ and $c = 0$, applying Lemma~\ref{lemma:thething} yields the desired result.
        \end{proof}

    \begin{question}
        Does the converse hold in general?
    \end{question}
    \begin{answer}
        No. Consider $(x_n)_n = ((-1)^n)_n$. Then $(|x_n|)_n \rightarrow 1$, but $((-1)^n)_n$ diverges.
    \end{answer}

    \begin{lemma}\label{lemma:abs-convergence}
        $(a_n)_n \rightarrow 0$ if and only if $(|a_n|)_n \rightarrow 0$.
    \end{lemma}
        \begin{proof}
            The forward direction follows directly from Lemma~\ref{lemma:abs-convergence}. If $(|a_n|)_n \rightarrow 0$, we have then that $||a_n| - 0| \leq |a_n - 0|$. Taking $\epsilon_n = |a_n - 0|$ and $c = 1$, applying Lemma~\ref{lemma:thething} establishes the proof.
        \end{proof}

    \begin{definition}
        \phantom{a}
        \begin{enumerate}[label = (\arabic*)]
            \item A sequence $(x_n)_n$ \textui{diverges properly} to $+\infty$ if:
                \begin{equation*}
                \begin{split}
                    (\forall M > 0)(\exists N_M \in \bfN)\mtext{s.t.}n \geq N_M \implies x_n \geq M.
                \end{split}
                \end{equation*}
            We write $(x_n)_n \rightarrow +\infty$.

            \item A sequence $(x_n)_n$ \textui{diverges properly} to $-\infty$ if:
                \begin{equation*}
                \begin{split}
                    (\forall M<0)(\exists N_M \in \bfN)\mtext{s.t.}n \geq N_M \implies x_n \leq M.
                \end{split}
                \end{equation*}
                We write $(x_n)_n \rightarrow -\infty$.
        \end{enumerate}
    \end{definition}

    \begin{example}
        Prove $(n-\sqrt{n})_n \rightarrow +\infty$.
    \end{example}
        \begin{solution}
            Write $(n-\sqrt{n})_n = \left(n\left(1-\frac{1}{\sqrt{n}}\right)\right)_n$ {\color{red} help}.
        \end{solution}

    \begin{example}
        Prove:
            \begin{equation*}
            \begin{split}
                (b^n)_{n=0}^\infty \rightarrow \begin{cases} 0, & |b| < 1 \\ 1,& b = 1 \\ +\infty, & b > 1 \\ \text{DNE},& b \leq -1\end{cases}
            \end{split}
            \end{equation*}
    \end{example}
        \begin{solution}
            We have proven cases $b = 0$, $1$, and $-1$ previously.

            Case 1: $0 < b < 1$. Then $b<1$ implies $\frac{1}{b} > 1$. Then $\frac{1}{b} = 1+a$ for some $a>0$. We have:
                \begin{equation*}
                \begin{split}
                    \left(\frac{1}{b}\right)^n = (1+a)^n \geq 1+na. \quad\mtext{\tiny (\nameref{prop:bernoulli})}
                \end{split}
                \end{equation*}
            Hence:
                \begin{equation*}
                \begin{split}
                    |b^n - 0| \leq \frac{1}{1+na} \leq \frac{1}{na} = \frac{1}{a}\left(\frac{1}{n}\right).
                \end{split}
                \end{equation*}
            Take $\epsilon_n = \frac{1}{n}$ and $c = \frac{1}{a}$, then Lemma~\ref{lemma:thething} establishes our claim.

            Case 2: $-1 < b < 0$. Since $(|b^n|)_n = (|b|^n)_n \rightarrow 0$ by above, Lemma~\ref{lemma:abs-convergence} gives that $(b^n)_n \rightarrow 0$.

            Case 3: $b > 1$. Then $b = 1+a$ for some $a > 0$. Then:
                \begin{equation*}
                \begin{split}
                    b^n = (a+1)^n \geq 1+na \geq na. \quad \mtext{\tiny (\nameref{prop:bernoulli})}
                \end{split}
                \end{equation*}
            Given $M>0$, let $N_M = \frac{\lceil M \rceil}{a}$. This implies that $N_M \geq \frac{M}{a}$. Now if $n\geq N_M$, then $n \geq \frac{M}{a}$, which is equivalent to $na \geq M$. Hence $b^n \geq M$, establishing $(b^n)_n \rightarrow +\infty$.

            Case 4: $b < 1$. If $(b^n)_n \rightarrow L$ for some $L \in \bfR$, then $(|b^n|)_n \rightarrow |L|$. So $(|b|^n)_n \rightarrow |L|$, which contradicts the $b>1$ case. Now if $(b^n)_n \rightarrow +\infty$, then there exists $N_1 \in \bfN$ such that $n \geq N_1$ implies $b^n \geq 1$. However, for $n$ odd, $b^n < 0$, which is a contradiction. Assuming $(b^n)_n \rightarrow -\infty$ leads to a similar contradiction.
        \end{solution}

    \begin{example}
        Prove that if $c>0$, then $(c^{\frac{1}{n}})_n \rightarrow 1$.
    \end{example}
        \begin{solution}
            If $c=1$ then {\color{red} idk}. Let $c > 1$. Then $c^{\frac{1}{n}} > 1$. Write $c^{\frac{1}{n}} = 1 + a_n$ where $a_n > 0$ for all $n \in \bfN$. So:
                \begin{equation*}
                \begin{split}
                    c = (c^\frac{1}{n})^n = (1+a_n)^n \geq 1+na_n > na_n. \quad \mtext{\tiny (\nameref{prop:bernoulli})}
                \end{split}
                \end{equation*}
            Thus $0 < a_n < \frac{c}{n}$. We have:
                \begin{equation*}
                \begin{split}
                    \left|c^{\frac{1}{n}} - 1\right| = a_n < \frac{c}{n}.
                \end{split}
                \end{equation*}
            Take $\epsilon_n = \frac{1}{n}$. Since $c$ is constant, Lemma~\ref{lemma:thething} gives $(c^\frac{1}{n})_n \rightarrow 1$.

            Now let $0 < c < 1$ {\color{red} dont know!}.
        \end{solution}

    \begin{lemma}
        If $(x_n)_n \rightarrow x \in \bfR$ with $x_n \geq 0$, then $x\geq 0$ and $(\sqrt{x_n})_n \rightarrow \sqrt{x}$.
    \end{lemma}
        \begin{proof}
            If $x<0$, set $\epsilon = \frac{-x}{2} > 0$. For $n$ sufficiently large, $x_n \in V_\epsilon(x) \subseteq (-\infty,0)$, which is contradiction. Hence $x \geq 0$. Now observe that:
                \begin{equation*}
                \begin{split}
                    |\sqrt{x_n} - \sqrt{x}| \leq |\sqrt{x_n} - \sqrt{x}||\sqrt{x_n} + \sqrt{x}| = |x_n - x| < \epsilon.
                \end{split}
                \end{equation*}
        \end{proof}

    \begin{example}
        Prove $(n^\frac{1}{n})_n \rightarrow 1$.
    \end{example}
        \begin{solution}
            Recall that:
                \begin{equation*}
                \begin{split}
                    (x+y)^n = \sum_{k=0}^n {n\choose k}x^{n-k}y^k.
                \end{split}
                \end{equation*}
            Note that $n^\frac{1}{n} > 1$ for all $n> 1$. Write $n^\frac{1}{n} = 1 + a_n$. Then:
                \begin{equation*}
                \begin{split}
                    n = (1+a_n)^n = \sum_{k=0}^n {n \choose k}a_n^k \geq {n \choose 0} + {n \choose 2}a_n^2 = 1 + \frac{n(n-1)}{2}a_n^2.
                \end{split}
                \end{equation*}
            We have:
                \begin{equation*}
                \begin{split}
                    n-1 \geq \frac{n(n-1)}{2}a_n^2,
                \end{split}
                \end{equation*}
            which simplifies to:
                \begin{equation*}
                \begin{split}
                    \frac{2}{n} \geq a_n^2.
                \end{split}
                \end{equation*}
            Hence $a_n \leq \sqrt{2} \hspace{3pt} \displaystyle{\frac{1}{n}}$, thus by our lemma $(a_n)_n^\infty \rightarrow 0$. Therefore:
                \begin{equation*}
                \begin{split}
                    |n^\frac{1}{n} - 1| = d_n,
                \end{split}
                \end{equation*}
            establishing that $(n^\frac{1}{n})_n \rightarrow 1$.
        \end{solution}

    \begin{proposition}
        A convergent sequence is bounded.
    \end{proposition}
        \begin{proof}
            Suppose $(x_n)_n \rightarrow x$. Since $(x_n)_n$ is convergent, we know for all $\epsilon > 0$ that $|x_n - x| < \epsilon$. Pick $\epsilon = 1$ and consider the following diagram:
                \begin{center}
                \begin{tikzpicture}
                % Draw the number line
                \draw[<->] (-3,0) -- (3,0); 
                
                % Tick for x
                \draw (0,0.1) -- (0,-0.1) node[below] {$x$}; 
                
                % Parenthesis "(" at x-1
                \node at (-1.5,0) {$($};
                \node at (-1.5,-0.3)  {$x-1$};
                
                % Parenthesis ")" at x+1
                \node at (1.5,0)  {$)$};
                \node at (1.5,-0.3) {$x+1$};
                \end{tikzpicture}
                \end{center}
            Eventually the entire sequence will be contained in $V_1(x)$. More formally, there exists $N_1 \in \bfN$ such that $n \geq N_1$ implies $x_n \in V_1(x)$. Define:
                \begin{equation*}
                \begin{split}
                    c = \max \left\{|x_1|,|x_n|,...,|x_{N_1}|,|x-1|,|x+1|\right\}.
                \end{split}
                \end{equation*}
            If $n\leq N_1$, then $|x_n| \leq c$. If $n \geq N_1$, then $x-1 \leq x_n \leq x+1$; i.e., $|x_n| \leq c$. Thus $|x_n| \leq c$ for all $n$, establishing this sequence as bounded.
        \end{proof}

    \begin{theorem}
        Let $x_n$, $y_n$, $z_n$ be convergent sequences with $(x_n)_n \rightarrow x$, $(y_n)_n \rightarrow y$, and $(z_n)_n \rightarrow z$ and $t \in \bfR$. Moreover, let $z_n \neq 0$ for all $n$ and $z \neq 0$. We have:
            \begin{enumerate}[label = (\arabic*)]
                \item $(x_n \pm y_n)_n \rightarrow x \pm y$. 
                \item $(tx_n)_n \rightarrow tx$.
                \item $(x_n y_n)_n \rightarrow xy$.
                \item $\left(\frac{1}{z_n}\right)_n \rightarrow \frac{1}{z}$.
                \item $\left(\frac{x_n}{z_n}\right)_n \rightarrow \frac{x}{z}$.
            \end{enumerate}
    \end{theorem}
        \begin{proof}
            (3) We have:
                \begin{equation*}
                \begin{split}
                    |x_n y_n - xy|
                    & = |x_n y_n - xy_n + x y_n - xy| \\
                    & = \left|(x_n - x)y_n + x(y_n - y)\right|\\
                    & \leq |(x_n - x)y_n |+| x(y_n - y)| \\
                    & = |x_n - x||y_n| + |x||y_n - y|.
                \end{split}
                \end{equation*}
            Since $y_n$ is convergent, it is bounded. So there exists a $c > 0$ with $|y_n| \leq c$ for all $n \geq 1$. Hence:
                \begin{equation*}
                \begin{split}
                    |x_n - x||y_n| + |x||y_n - y|
                    & \leq \stackrel{\rightarrow 0}{\cancel{|x_n -x|}}c + |x|\stackrel{\rightarrow 0}{\cancel{|y_n - y|}.}
                \end{split}
                \end{equation*}
            Thus $(|x_n y_n - xy|)_n \rightarrow 0$, which implies $(x_n y_n)_n \rightarrow xy$.

            (4) We have:
                \begin{equation*}
                \begin{split}
                    \left|\frac{1}{z_n} - \frac{1}{z}\right| = \frac{|z- z_n|}{|z||z_n|}.
                \end{split}
                \end{equation*}

            Since $z \neq 0$, it won't be "near" zero. We have the following picture:
                \begin{center}
                \begin{tikzpicture}
                % Draw the number line
                \draw[<->] (-3,0) -- (3,0); 
                
                % Tick for x
                \draw (0,0.1) -- (0,-0.1) node[below] {$0$}; 
                \draw (1,0.1) -- (1,-0.1) node[below] {$z$}; 
                
                % Parenthesis ")" at x+1
                \node at (1.5,0)  {$)$};
                \node at (0.5,0)  {$($};
                \end{tikzpicture}
                \end{center}
            Let $\delta = \frac{|z|}{2} > 0$. There exists $N \in \bfN$ such that $n \geq N$ implies $z_n \in V_\delta(z)$. We have:
                \begin{equation*}
                \begin{split}
                    &\phantom{\implies} z - \delta < z_n < z + \delta \\
                    &\implies z - \frac{|z|}{2} < z_n \\
                    & \implies \frac{|z|}{2} < |z_n|.
                \end{split}
                \end{equation*}
            Since $|z_n| \geq \frac{|z|}{2}$, we have $\frac{1}{|z_n|} < \frac{2}{|z|}$. So for $n \geq N$, 
                \begin{equation*}
                \begin{split}
                    \left|\frac{1}{z_n} - \frac{1}{z}\right| = \frac{|z- z_n|}{|z||z_n|} \leq \frac{2}{|z|^2}|z - z_n|.
                \end{split}
                \end{equation*}
            By Lemma~\ref{lemma:thething}, $\left(\frac{1}{z_n}\right)_n \rightarrow \frac{1}{z}$.
        \end{proof}

    \begin{theorem}\label{thm:ordering-of-sequences}
        Suppose $(x_n) \rightarrow x$ and $(y_n)_n \rightarrow y$ with $x_n \leq y_n$ for all $n$. Then $x \leq y$.
    \end{theorem}
        \begin{proof}
            We have that $(y_n - x_n)_n \rightarrow y-x$, and $y_n - x_n \geq 0$ for all $n$. Thus $y - x \geq 0$.
        \end{proof}

    \begin{corollary}
        If $(x_n)_n \rightarrow x$ and $a \leq x_n \leq b$, then $a \leq x \leq b$.
    \end{corollary}
        \begin{proof}
            From Theorem~\ref{thm:ordering-of-sequences}, taking $(y_n)_n = (a,a,a,...)$ and $(y_n)_n = (b,b,b,...)$ gives the desired result.
        \end{proof}

    \begin{theorem}[Squeeze Theorem]
        Let $(x_n)_n$, $(y_n)_n$, and $(z_n)_n$ be sequences with $(x_n)_n \leq (y_n)_n \leq (z_n)_n$ for all $n \geq 1$. If $\lim x_n = \lim z_n = L$, then $(y_n)_n \rightarrow L$.
    \end{theorem}
        \begin{proof}
            Let $\epsilon > 0$. There exists $N_1 \in \bfN$ such that $n \geq N_1 $ implies $x_n \in V_\epsilon(L)$. Likewise, there exists $N_2 \in \bfN$ such that $n \geq N_2$ implies $z_n \in V_\epsilon(L)$. So for $n \geq \max \left\{N_1,N_2\right\} := N$, both $x_n,z_n \in V_\epsilon(L)$. We have:
                \begin{equation*}
                \begin{split}
                    L -\epsilon < x_n \leq y_n < z_n \leq L + \epsilon.
                \end{split}
                \end{equation*}
            Thus $y_n \in V_\epsilon(L)$ for $n \geq N$.
        \end{proof}

    

