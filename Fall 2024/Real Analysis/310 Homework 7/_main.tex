\documentclass[11pt,twoside,openany]{memoir}
\usepackage{amsthm}
\usepackage{amssymb}
\usepackage{newpxtext,eulerpx,eucal}
\usepackage{amsmath}
\usepackage{datetime}
    \newdateformat{specialdate}{\THEYEAR\ \monthname\ \THEDAY}
\usepackage[margin=1in]{geometry}
\usepackage{fancyhdr}
    \fancyhf{}
    \cfoot{\footnotesize \thepage}
    \pagestyle{fancy}
    \renewcommand{\headrulewidth}{0pt}
\usepackage{thmtools}
    \declaretheoremstyle[
        spaceabove=15pt,
        headfont=\normalfont\bfseries,
        notefont=\mdseries, notebraces={(}{)},
        bodyfont=\normalfont,
        postheadspace=0.5em
        %qed=\qedsymbol
        ]{defs}

    \declaretheoremstyle[
        spaceabove=15pt, % space above the theorem
        headfont=\normalfont\bfseries,
        bodyfont=\normalfont\itshape,
        postheadspace=0.5em
        ]{thmstyle}
    
    \declaretheorem[
        style=thmstyle,
        numberwithin=section
    ]{theorem}

    \declaretheorem[
        style=thmstyle,
        sibling=theorem,
    ]{proposition}

    \declaretheorem[
        style=thmstyle,
        sibling=theorem,
    ]{lemma}

    \declaretheorem[
        style=thmstyle,
        sibling=theorem,
    ]{corollary}

    \declaretheorem[
        numberwithin=section,
        style=defs,
    ]{example}

    \declaretheorem[
        numberwithin=section,
        style=defs,
    ]{definition}

    \declaretheorem[
        numbered=unless unique,
        shaded={rulecolor=black,
    rulewidth=1pt, bgcolor={rgb}{1,1,1}}
    ]{axiom}

    \declaretheorem[numbered=unless unique,style=defs]{note}
    \declaretheorem[numbered=no,style=defs]{question}
    \declaretheorem[numbered=no,style=remark]{answer}
    \declaretheorem[numbered=no,style=remark]{remark}
    \declaretheorem[numbered=no,style=remark]{solution}
    \declaretheorem[numbered=unless unique,style=defs]{exercise}
\usepackage{enumitem}
\usepackage{titlesec}
    \titleformat{\chapter}[display]
    {\normalfont\fillast\large}
    {Chapter {\thechapter}}
    {1ex minus .1ex}
    {\large}
    \titlespacing{\chapter}
    {3pc}{*3}{*2}[3pc]

    \titleformat{\section}[runin]
    {\normalfont\bfseries\large}
    {\S\ \thesection.}{.5em}{}[]
    \titlespacing{\section}
    {\parindent}{1.5ex plus .1ex minus .2ex}{0pt}
\usepackage[utf8x]{inputenc}

\setlength{\parindent}{0pt}
%%%%%%%%%%%%%%%%%%%%%%%%%%%%%%%%%%%%%%%%%%%%%%%%%%%%%%%%%%%%%
%%%%%%%%%%%%%%%%%%%%%%%%%%%%%%%%%%%%%%%%%%%%%%%%%%%%%%%%%%%%%
%to make the correct symbol for Sha
%\newcommand\cyr{%
%\renewcommand\rmdefault{wncyr}%
%\renewcommand\sfdefault{wncyss}%
%\renewcommand\encodingdefault{OT2}%
%\normalfont \selectfont} \DeclareTextFontCommand{\textcyr}{\cyr}


\DeclareMathOperator{\ab}{ab}
\newcommand{\absgal}{\G_{\bbQ}}
\DeclareMathOperator{\ad}{ad}
\DeclareMathOperator{\adj}{adj}
\DeclareMathOperator{\alg}{alg}
\DeclareMathOperator{\Alt}{Alt}
\DeclareMathOperator{\Ann}{Ann}
\DeclareMathOperator{\arith}{arith}
\DeclareMathOperator{\Aut}{Aut}
\DeclareMathOperator{\Be}{B}
\DeclareMathOperator{\Bd}{Bd}
\DeclareMathOperator{\card}{card}
\DeclareMathOperator{\Char}{char}
\DeclareMathOperator{\csp}{csp}
\DeclareMathOperator{\codim}{codim}
\DeclareMathOperator{\coker}{coker}
\DeclareMathOperator{\coh}{H}
\DeclareMathOperator{\compl}{compl}
\DeclareMathOperator{\conj}{conj}
\DeclareMathOperator{\cont}{cont}
\DeclareMathOperator{\crys}{crys}
\DeclareMathOperator{\Crys}{Crys}
\DeclareMathOperator{\cusp}{cusp}
\DeclareMathOperator{\diag}{diag}
\DeclareMathOperator{\diam}{diam}
\DeclareMathOperator{\Dom}{Dom}
\DeclareMathOperator{\disc}{disc}
\DeclareMathOperator{\dist}{dist}
\DeclareMathOperator{\dR}{dR}
\DeclareMathOperator{\Eis}{Eis}
\DeclareMathOperator{\End}{End}
\DeclareMathOperator{\ev}{ev}
\DeclareMathOperator{\eval}{eval}
\DeclareMathOperator{\Eq}{Eq}
\DeclareMathOperator{\Ext}{Ext}
\DeclareMathOperator{\Fil}{Fil}
\DeclareMathOperator{\Fitt}{Fitt}
\DeclareMathOperator{\Frob}{Frob}
\DeclareMathOperator{\G}{G}
\DeclareMathOperator{\Gal}{Gal}
\DeclareMathOperator{\GL}{GL}
\DeclareMathOperator{\Gr}{Gr}
\DeclareMathOperator{\Graph}{Graph}
\DeclareMathOperator{\GSp}{GSp}
\DeclareMathOperator{\GUn}{GU}
\DeclareMathOperator{\Hom}{Hom}
\DeclareMathOperator{\id}{id}
\DeclareMathOperator{\Id}{Id}
\DeclareMathOperator{\Ik}{Ik}
\DeclareMathOperator{\IM}{Im}
\DeclareMathOperator{\Image}{im}
\DeclareMathOperator{\Ind}{Ind}
\DeclareMathOperator{\Inf}{inf}
\DeclareMathOperator{\Isom}{Isom}
\DeclareMathOperator{\J}{J}
\DeclareMathOperator{\Jac}{Jac}
\DeclareMathOperator{\lcm}{lcm}
\DeclareMathOperator{\length}{length}
\DeclareMathOperator*{\limit}{limit}
\DeclareMathOperator{\Log}{Log}
\DeclareMathOperator{\M}{M}
\DeclareMathOperator{\Mat}{Mat}
\DeclareMathOperator{\N}{N}
\DeclareMathOperator{\Nm}{Nm}
\DeclareMathOperator{\NIk}{N-Ik}
\DeclareMathOperator{\NSK}{N-SK}
\DeclareMathOperator{\new}{new}
\DeclareMathOperator{\obj}{obj}
\DeclareMathOperator{\old}{old}
\DeclareMathOperator{\ord}{ord}
\DeclareMathOperator{\Or}{O}
\DeclareMathOperator{\op}{op}
\DeclareMathOperator{\PGL}{PGL}
\DeclareMathOperator{\PGSp}{PGSp}
\DeclareMathOperator{\rank}{rank}
\DeclareMathOperator{\Ran}{Ran}
\DeclareMathOperator{\Rel}{Rel}
\DeclareMathOperator{\Real}{Re}
\DeclareMathOperator{\RES}{res}
\DeclareMathOperator{\Res}{Res}
%\DeclareMathOperator{\Sha}{\textcyr{Sh}}
\DeclareMathOperator{\Sel}{Sel}
\DeclareMathOperator{\semi}{ss}
\DeclareMathOperator{\sgn}{sign}
\DeclareMathOperator{\SK}{SK}
\DeclareMathOperator{\SL}{SL}
\DeclareMathOperator{\SO}{SO}
\DeclareMathOperator{\Sp}{Sp}
\DeclareMathOperator{\Span}{span}
\DeclareMathOperator{\Spec}{Spec}
\DeclareMathOperator{\spin}{spin}
\DeclareMathOperator{\st}{st}
\DeclareMathOperator{\St}{St}
\DeclareMathOperator{\SUn}{SU}
\DeclareMathOperator{\supp}{supp}
\DeclareMathOperator{\Sup}{sup}
\DeclareMathOperator{\Sym}{Sym}
\DeclareMathOperator{\Tam}{Tam}
\DeclareMathOperator{\tors}{tors}
\DeclareMathOperator{\tr}{tr}
\DeclareMathOperator{\Tr}{Tr}
\DeclareMathOperator{\un}{un}
\DeclareMathOperator{\Un}{U}
\DeclareMathOperator{\val}{val}
\DeclareMathOperator{\vol}{vol}

\DeclareMathOperator{\Sets}{S \mkern1.04mu e \mkern1.04mu t \mkern1.04mu s}
    \newcommand{\cSets}{\scalebox{1.02}{\contour{black}{$\Sets$}}}
    
\DeclareMathOperator{\Groups}{G \mkern1.04mu r \mkern1.04mu o \mkern1.04mu u \mkern1.04mu p \mkern1.04mu s}
    \newcommand{\cGroups}{\scalebox{1.02}{\contour{black}{$\Groups$}}}

\DeclareMathOperator{\TTop}{T \mkern1.04mu o \mkern1.04mu p}
    \newcommand{\cTop}{\scalebox{1.02}{\contour{black}{$\TTop$}}}

\DeclareMathOperator{\Htp}{H \mkern1.04mu t \mkern1.04mu p}
    \newcommand{\cHtp}{\scalebox{1.02}{\contour{black}{$\Htp$}}}

\DeclareMathOperator{\Mod}{M \mkern1.04mu o \mkern1.04mu d}
    \newcommand{\cMod}{\scalebox{1.02}{\contour{black}{$\Mod$}}}

\DeclareMathOperator{\Ab}{A \mkern1.04mu b}
    \newcommand{\cAb}{\scalebox{1.02}{\contour{black}{$\Ab$}}}

\DeclareMathOperator{\Rings}{R \mkern1.04mu i \mkern1.04mu n \mkern1.04mu g \mkern1.04mu s}
    \newcommand{\cRings}{\scalebox{1.02}{\contour{black}{$\Rings$}}}

\DeclareMathOperator{\ComRings}{C \mkern1.04mu o \mkern1.04mu m \mkern1.04mu R \mkern1.04mu i \mkern1.04mu n \mkern1.04mu g \mkern1.04mu s}
    \newcommand{\cComRings}{\scalebox{1.05}{\contour{black}{$\ComRings$}}}

\DeclareMathOperator{\hHom}{H \mkern1.04mu o \mkern1.04mu m}
    \newcommand{\cHom}{\scalebox{1.02}{\contour{black}{$\hHom$}}}

         %  \item $\cGroups$
          %  \item $\cTop$
          %  \item $\cHtp$
          %  \item $\cMod$




\renewcommand{\k}{\kappa}
\newcommand{\Ff}{F_{f}}
%\newcommand{\ts}{\,^{t}\!}


%Mathcal

\newcommand{\cA}{\mathcal{A}}
\newcommand{\cB}{\mathcal{B}}
\newcommand{\cC}{\mathcal{C}}
\newcommand{\cD}{\mathcal{D}}
\newcommand{\cE}{\mathcal{E}}
\newcommand{\cF}{\mathcal{F}}
\newcommand{\cG}{\mathcal{G}}
\newcommand{\cH}{\mathcal{H}}
\newcommand{\cI}{\mathcal{I}}
\newcommand{\cJ}{\mathcal{J}}
\newcommand{\cK}{\mathcal{K}}
\newcommand{\cL}{\mathcal{L}}
\newcommand{\cM}{\mathcal{M}}
\newcommand{\cN}{\mathcal{N}}
\newcommand{\cO}{\mathcal{O}}
\newcommand{\cP}{\mathcal{P}}
\newcommand{\cQ}{\mathcal{Q}}
\newcommand{\cR}{\mathcal{R}}
\newcommand{\cS}{\mathcal{S}}
\newcommand{\cT}{\mathcal{T}}
\newcommand{\cU}{\mathcal{U}}
\newcommand{\cV}{\mathcal{V}}
\newcommand{\cW}{\mathcal{W}}
\newcommand{\cX}{\mathcal{X}}
\newcommand{\cY}{\mathcal{Y}}
\newcommand{\cZ}{\mathcal{Z}}


%mathfrak (missing \fi)

\newcommand{\fa}{\mathfrak{a}}
\newcommand{\fA}{\mathfrak{A}}
\newcommand{\fb}{\mathfrak{b}}
\newcommand{\fB}{\mathfrak{B}}
\newcommand{\fc}{\mathfrak{c}}
\newcommand{\fC}{\mathfrak{C}}
\newcommand{\fd}{\mathfrak{d}}
\newcommand{\fD}{\mathfrak{D}}
\newcommand{\fe}{\mathfrak{e}}
\newcommand{\fE}{\mathfrak{E}}
\newcommand{\ff}{\mathfrak{f}}
\newcommand{\fF}{\mathfrak{F}}
\newcommand{\fg}{\mathfrak{g}}
\newcommand{\fG}{\mathfrak{G}}
\newcommand{\fh}{\mathfrak{h}}
\newcommand{\fH}{\mathfrak{H}}
\newcommand{\fI}{\mathfrak{I}}
\newcommand{\fj}{\mathfrak{j}}
\newcommand{\fJ}{\mathfrak{J}}
\newcommand{\fk}{\mathfrak{k}}
\newcommand{\fK}{\mathfrak{K}}
\newcommand{\fl}{\mathfrak{l}}
\newcommand{\fL}{\mathfrak{L}}
\newcommand{\fm}{\mathfrak{m}}
\newcommand{\fM}{\mathfrak{M}}
\newcommand{\fn}{\mathfrak{n}}
\newcommand{\fN}{\mathfrak{N}}
\newcommand{\fo}{\mathfrak{o}}
\newcommand{\fO}{\mathfrak{O}}
\newcommand{\fp}{\mathfrak{p}}
\newcommand{\fP}{\mathfrak{P}}
\newcommand{\fq}{\mathfrak{q}}
\newcommand{\fQ}{\mathfrak{Q}}
\newcommand{\fr}{\mathfrak{r}}
\newcommand{\fR}{\mathfrak{R}}
\newcommand{\fs}{\mathfrak{s}}
\newcommand{\fS}{\mathfrak{S}}
\newcommand{\ft}{\mathfrak{t}}
\newcommand{\fT}{\mathfrak{T}}
\newcommand{\fu}{\mathfrak{u}}
\newcommand{\fU}{\mathfrak{U}}
\newcommand{\fv}{\mathfrak{v}}
\newcommand{\fV}{\mathfrak{V}}
\newcommand{\fw}{\mathfrak{w}}
\newcommand{\fW}{\mathfrak{W}}
\newcommand{\fx}{\mathfrak{x}}
\newcommand{\fX}{\mathfrak{X}}
\newcommand{\fy}{\mathfrak{y}}
\newcommand{\fY}{\mathfrak{Y}}
\newcommand{\fz}{\mathfrak{z}}
\newcommand{\fZ}{\mathfrak{Z}}


%mathbf
\newcommand{\bfA}{\mathbf{A}}
\newcommand{\bfB}{\mathbf{B}}
\newcommand{\bfC}{\mathbf{C}}
\newcommand{\bfD}{\mathbf{D}}
\newcommand{\bfE}{\mathbf{E}}
\newcommand{\bfF}{\mathbf{F}}
\newcommand{\bfG}{\mathbf{G}}
\newcommand{\bfH}{\mathbf{H}}
\newcommand{\bfI}{\mathbf{I}}
\newcommand{\bfJ}{\mathbf{J}}
\newcommand{\bfK}{\mathbf{K}}
\newcommand{\bfL}{\mathbf{L}}
\newcommand{\bfM}{\mathbf{M}}
\newcommand{\bfN}{\mathbf{N}}
\newcommand{\bfO}{\mathbf{O}}
\newcommand{\bfP}{\mathbf{P}}
\newcommand{\bfQ}{\mathbf{Q}}
\newcommand{\bfR}{\mathbf{R}}
\newcommand{\bfS}{\mathbf{S}}
\newcommand{\bfT}{\mathbf{T}}
\newcommand{\bfU}{\mathbf{U}}
\newcommand{\bfV}{\mathbf{V}}
\newcommand{\bfW}{\mathbf{W}}
\newcommand{\bfX}{\mathbf{X}}
\newcommand{\bfY}{\mathbf{Y}}
\newcommand{\bfZ}{\mathbf{Z}}

\newcommand{\bfa}{\mathbf{a}}
\newcommand{\bfb}{\mathbf{b}}
\newcommand{\bfc}{\mathbf{c}}
\newcommand{\bfd}{\mathbf{d}}
\newcommand{\bfe}{\mathbf{e}}
\newcommand{\bff}{\mathbf{f}}
\newcommand{\bfg}{\mathbf{g}}
\newcommand{\bfh}{\mathbf{h}}
\newcommand{\bfi}{\mathbf{i}}
\newcommand{\bfj}{\mathbf{j}}
\newcommand{\bfk}{\mathbf{k}}
\newcommand{\bfl}{\mathbf{l}}
\newcommand{\bfm}{\mathbf{m}}
\newcommand{\bfn}{\mathbf{n}}
\newcommand{\bfo}{\mathbf{o}}
\newcommand{\bfp}{\mathbf{p}}
\newcommand{\bfq}{\mathbf{q}}
\newcommand{\bfr}{\mathbf{r}}
\newcommand{\bfs}{\mathbf{s}}
\newcommand{\bft}{\mathbf{t}}
\newcommand{\bfu}{\mathbf{u}}
\newcommand{\bfv}{\mathbf{v}}
\newcommand{\bfw}{\mathbf{w}}
\newcommand{\bfx}{\mathbf{x}}
\newcommand{\bfy}{\mathbf{y}}
\newcommand{\bfz}{\mathbf{z}}

%blackboard bold

\newcommand{\bbA}{\mathbb{A}}
\newcommand{\bbB}{\mathbb{B}}
\newcommand{\bbC}{\mathbb{C}}
\newcommand{\bbD}{\mathbb{D}}
\newcommand{\bbE}{\mathbb{E}}
\newcommand{\bbF}{\mathbb{F}}
\newcommand{\bbG}{\mathbb{G}}
\newcommand{\bbH}{\mathbb{H}}
\newcommand{\bbI}{\mathbb{I}}
\newcommand{\bbJ}{\mathbb{J}}
\newcommand{\bbK}{\mathbb{K}}
\newcommand{\bbL}{\mathbb{L}}
\newcommand{\bbM}{\mathbb{M}}
\newcommand{\bbN}{\mathbb{N}}
\newcommand{\bbO}{\mathbb{O}}
\newcommand{\bbP}{\mathbb{P}}
\newcommand{\bbQ}{\mathbb{Q}}
\newcommand{\bbR}{\mathbb{R}}
\newcommand{\bbS}{\mathbb{S}}
\newcommand{\bbT}{\mathbb{T}}
\newcommand{\bbU}{\mathbb{U}}
\newcommand{\bbV}{\mathbb{V}}
\newcommand{\bbW}{\mathbb{W}}
\newcommand{\bbX}{\mathbb{X}}
\newcommand{\bbY}{\mathbb{Y}}
\newcommand{\bbZ}{\mathbb{Z}}
\newcommand{\jota}{\jmath}

\newcommand{\bmat}{\left( \begin{matrix}}
\newcommand{\emat}{\end{matrix} \right)}

\newcommand{\pmat}{\left( \begin{smallmatrix}}
\newcommand{\epmat}{\end{smallmatrix} \right)}

\newcommand{\lat}{\mathscr{L}}
\newcommand{\mat}[4]{\begin{pmatrix}{#1}&{#2}\\{#3}&{#4}\end{pmatrix}}
\newcommand{\ov}[1]{\overline{#1}}
\newcommand{\res}[1]{\underset{#1}{\RES}\,}
\newcommand{\up}{\upsilon}

\newcommand{\tac}{\textasteriskcentered}

%mahesh macros
\newcommand{\tm}{\textrm}

%Comments
\newcommand{\com}[1]{\vspace{5 mm}\par \noindent
\marginpar{\textsc{Comment}} \framebox{\begin{minipage}[c]{0.95
\textwidth} \tt #1 \end{minipage}}\vspace{5 mm}\par}

\newcommand{\Bmu}{\mbox{$\raisebox{-0.59ex}
  {$l$}\hspace{-0.18em}\mu\hspace{-0.88em}\raisebox{-0.98ex}{\scalebox{2}
  {$\color{white}.$}}\hspace{-0.416em}\raisebox{+0.88ex}
  {$\color{white}.$}\hspace{0.46em}$}{}}  %need graphicx and xcolor. this produces blackboard bold mu 

\newcommand{\hooktwoheadrightarrow}{%
  \hookrightarrow\mathrel{\mspace{-15mu}}\rightarrow
}

\makeatletter
\newcommand{\xhooktwoheadrightarrow}[2][]{%
  \lhook\joinrel
  \ext@arrow 0359\rightarrowfill@ {#1}{#2}%
  \mathrel{\mspace{-15mu}}\rightarrow
}
\makeatother

\renewcommand{\geq}{\geqslant}
\renewcommand{\leq}{\leqslant}
\newcommand{\midd}{\hspace{4pt}\middle|\hspace{4pt}}
    
    \newcommand{\bone}{\mathbf{1}}
    \newcommand{\sign}{\mathrm{sign}}
    \newcommand{\eps}{\varepsilon}
    \newcommand{\textui}[1]{\uline{\textit{#1}}}
    
    %\newcommand{\ov}{\overline}
    %\newcommand{\un}{\underline}
    \newcommand{\fin}{\mathrm{fin}}
    
    \newcommand{\chnum}{\titleformat
    {\chapter} % command
    [display] % shape
    {\centering} % format
    {\Huge \color{black} \shadowbox{\thechapter}} % label
    {-0.5em} % sep (space between the number and title)
    {\LARGE \color{black} \underline} % before-code
    }
    
    \newcommand{\chunnum}{\titleformat
    {\chapter} % command
    [display] % shape
    {} % format
    {} % label
    {0em} % sep
    { \begin{flushright} \begin{tabular}{r}  \Huge \color{black}
    } % before-code
    [
    \end{tabular} \end{flushright} \normalsize
    ] % after-code
    }

\newcommand{\nl}{\newline \mbox{}}

\newcommand{\h}[1]{\hspace{#1pt}}

\newcommand{\littletaller}{\mathchoice{\vphantom{\big|}}{}{}{}}
\newcommand\restr[2]{{% we make the whole thing an ordinary symbol
  \left.\kern-\nulldelimiterspace % automatically resize the bar with \right
  #1 % the function
  \littletaller % pretend it's a little taller at normal size
  \right|_{#2} % this is the delimiter
  }}

\newcommand{\mtext}[1]{\hspace{6pt}\text{#1}\hspace{6pt}}

\newcommand{\lnorm}{\left\lVert}
\newcommand{\rnorm}{\right\rVert}

\newcommand{\ds}{\displaystyle}
\newcommand{\ts}{\textstyle}

%This adds a "front cover" page.
%{\thispagestyle{empty}
%\vspace*{\fill}
%\begin{tabular}{l}
%\begin{tabular}{l}
%\includegraphics[scale=0.24]{oxy-logo.png}
%\end{tabular} \\
%\begin{tabular}{l}
%\Large \color{black} Module Theory, Linear Algebra, and Homological Algebra \\ \Large \color{black} Gianluca Crescenzo
%\end{tabular}
%\end{tabular}
%\newpage

\newcommand{\sfrac}[2]{{}^{#1}\mskip -5mu/\mskip -3mu_{#2}}


\makeatletter
\newcommand*{\da@rightarrow}{\mathchar"0\hexnumber@\symAMSa 4B }
\newcommand*{\da@leftarrow}{\mathchar"0\hexnumber@\symAMSa 4C }
\newcommand*{\xdashrightarrow}[2][]{%
  \mathrel{%
    \mathpalette{\da@xarrow{#1}{#2}{}\da@rightarrow{\,}{}}{}%
  }%
}
\newcommand{\xdashleftarrow}[2][]{%
  \mathrel{%
    \mathpalette{\da@xarrow{#1}{#2}\da@leftarrow{}{}{\,}}{}%
  }%
}
\newcommand*{\da@xarrow}[7]{%
  % #1: below
  % #2: above
  % #3: arrow left
  % #4: arrow right
  % #5: space left 
  % #6: space right
  % #7: math style 
  \sbox0{$\ifx#7\scriptstyle\scriptscriptstyle\else\scriptstyle\fi#5#1#6\m@th$}%
  \sbox2{$\ifx#7\scriptstyle\scriptscriptstyle\else\scriptstyle\fi#5#2#6\m@th$}%
  \sbox4{$#7\dabar@\m@th$}%
  \dimen@=\wd0 %
  \ifdim\wd2 >\dimen@
    \dimen@=\wd2 %   
  \fi
  \count@=2 %
  \def\da@bars{\dabar@\dabar@}%
  \@whiledim\count@\wd4<\dimen@\do{%
    \advance\count@\@ne
    \expandafter\def\expandafter\da@bars\expandafter{%
      \da@bars
      \dabar@ 
    }%
  }%  
  \mathrel{#3}%
  \mathrel{%   
    \mathop{\da@bars}\limits
    \ifx\\#1\\%
    \else
      _{\copy0}%
    \fi
    \ifx\\#2\\%
    \else
      ^{\copy2}%
    \fi
  }%   
  \mathrel{#4}%
}
\makeatother


\begin{document}
\begin{center}
{\large Math 310 \\[0.1in]Homework 7 \\[0.1in]
Due: 10/9/2024}\\[.25in]
{Name:} {\underline{Gianluca Crescenzo\hspace*{2in}}}\\[0.15in]
\end{center}
\vspace{4pt}
%%%%%%%%%%%%%%%%%%%%%%%%%%%%%%%%%%%%%%%%%%%%%%%%%%%%%%%%%%%%%
    \begin{exercise}
        Let $D \subseteq \bfR$ and suppose $c \in \bfR$. Show that the following are equivalent:
            \begin{enumerate}[label = (\roman*)]
                \item $c$ is a cluster point of $D$;
                \item There is a sequence $(x_n)_n$ in $D\setminus \{c\}$ with $(x_n)_n \rightarrow c$.
            \end{enumerate}
    \end{exercise}
        \begin{proof}
            $(\Rightarrow)$ Let $c$ be a cluster point of $D$. By definition:
                \begin{equation*}
                \begin{split}
                    (\forall \delta > 0)(\dot{V}_\delta(c) \cap D \neq \emptyset).
                \end{split}
                \end{equation*}
            By induction, we have that $x_n \in \dot{V}_{\frac{1}{n}}(c) \cap D$. Hence $x_n \neq c$, $x_n \in D$, and $0 < |x_n - c|  < \frac{1}{n}$, giving $(x_n)_n \rightarrow c$. \nl
            
            $(\Leftarrow)$ Let $(x_n)_n \in (D\setminus\{c\})^\bfN$ with $(x_n)_n \rightarrow c$. Let $\delta > 0$ be given. Then for $N$ large, $n \geq N$ implies $0 < |x_n - c| <\epsilon$. Whence $x_N \in \dot{V}_\delta(c) \cap D$. Thus $c$ is a cluster point.
        \end{proof}
%%%%%%%%%%%%%%%%%%%%%%%%%%%%%%%%%%%%%%%%%%%%%%%%%%%%%%%%%%%%%
    \begin{exercise}
        Show that $f$ can have at most one limit at $c$.
    \end{exercise}
        \begin{proof}
            Suppose towards contradiction that $f$ has more than one limit, that is,
                \begin{equation*}
                \begin{split}
                    \limit_{x \rightarrow c}f &= L_1, \mtext{and} \\
                    \limit_{x \rightarrow c}f &= L_2.
                \end{split}
                \end{equation*}
            where $L_1 \neq L_2$. Then for all sequences $(x_n)_n$ in $D$, $(x_n)_n \rightarrow c$ implies $(f(x_n))_n \rightarrow L_1$ and $(f(x_n))_n \rightarrow L_2$. This is a contradiction, as sequences can only have at most one limit. Thus $f$ must have at most one limit.
        \end{proof}
%%%%%%%%%%%%%%%%%%%%%%%%%%%%%%%%%%%%%%%%%%%%%%%%%%%%%%%%%%%%%
    \begin{exercise}
        Show that the following are equivalent:
            \begin{enumerate}[label = (\roman*)]
                \item $\ds \limit_{x \rightarrow c}f = L$;
                \item For every sequence $(x_n)_n$ in $D \setminus \{c\}$ satisfying $(x_n)_n \rightarrow c$, we have $(f(x_n))_n \rightarrow L$.
            \end{enumerate}
    \end{exercise}
        \begin{proof}
            $(\Rightarrow)$ Suppose $\limit_{x \rightarrow c}f(x) = L$. Let $(x_n)_n$ be in $D$ with $x_n \neq c$ and $(x_n)_n \rightarrow c$. Given $\epsilon > 0$, we know there exists $\delta > 0$ such that $x \in D$ and $0 < |x-c| < \delta$ implies $|f(x)-L| < \epsilon$. We know there exists some $N \in \bfN$ with $n\geq N$ implying $|x_n - c| < \delta$. Whence $|f(x_n) - L| < \epsilon$; i.e., $(f(x_n))_n \rightarrow L$. \nl

            $(\Leftarrow)$ Towards a contradiction, suppose that for every sequence $(x_n)_n$ in $D$ such that $x_n \neq c$ and $(x_n)_n \rightarrow c$, it holds that $(f(x_n))_n \rightarrow L$, \textit{yet} $\limit_{x \rightarrow c}f(x) \neq L$. Then by definition:
                \begin{equation*}
                \begin{split}
                    (\exists \epsilon_0 > 0)(\forall \delta >0) \ni (x \in \dot{V}_\delta(c)\cap D \hspace{4pt}\land\hspace{4pt} f(x) \not\in V_{\epsilon_0}(L)).
                \end{split}
                \end{equation*}
            So for each $\delta = \frac{1}{n}$, we can find $x_n \in \dot{V}_{\frac{1}{n}}(c) \cap D$ and $f(x_n) \not\in V_{\epsilon_0}(L)$, or equivalently $(x_n)_n \rightarrow c$ and $(f(x_n))_n \not\rightarrow L$. This is a contradiction, since $(x_n)_n \rightarrow c$ implies $(f(x_n))_n \rightarrow L$. This establishes that $\limit_{x \rightarrow c}f(x) = L$.
        \end{proof}
%%%%%%%%%%%%%%%%%%%%%%%%%%%%%%%%%%%%%%%%%%%%%%%%%%%%%%%%%%%%%
    \begin{exercise}
        If $\limit_{x \rightarrow c}f = L$ exists, show that there is a $\delta > 0$ such that:
            \begin{equation*}
            \begin{split}
                \sup_{0 < |x-c| < \delta}|f(x)| < \infty,
            \end{split}
            \end{equation*}
        that is, $f$ is bounded on a deleted neighborhood of $c$.
    \end{exercise}
        \begin{proof}
            Let $\epsilon = 1$. Then for all $x \in \dot{V}_\delta(c) \cap D$, $0 < |x-c| < \delta$ implies:
                \begin{equation*}
                \begin{split}
                    |f(x)|
                    & = |f(x) - L + L| \\
                    & \leq |f(x) - L| + |L| \\
                    & < 1 + |L| \\
                    & < \infty.
                \end{split}
                \end{equation*}
            Whence $\sup_{0 < |x-c| < \delta}|f(x)| = 1 + |L| < \infty$.
        \end{proof}
%%%%%%%%%%%%%%%%%%%%%%%%%%%%%%%%%%%%%%%%%%%%%%%%%%%%%%%%%%%%%
    \begin{exercise}
        Establish the following limits.
            \begin{enumerate}[label = (\alph*)]
                \item $\ds \limit_{x \rightarrow 1} \frac{3x}{1+x} = \frac{3}{2}$.
                    \begin{proof}
                        Observe that:
                            \begin{equation*}
                            \begin{split}
                                |f(x) - L|
                                & = \left|\frac{3x}{1+x} - \frac{3}{2}\right| \\
                                & = \frac{3}{2} \left|\frac{x-1}{x+1}\right|.
                            \end{split}
                            \end{equation*}
                        If $|x-1| < \frac{1}{2}$, then $\frac{1}{2} < x < \frac{3}{2}$ implies $\frac{2}{3} > \frac{1}{x+1} > \frac{2}{5}$. Thus:
                            \begin{equation*}
                            \begin{split}
                                \frac{3}{2} \left|\frac{x-1}{x+1}\right| 
                                & < \frac{3}{2} \cdot \frac{2}{3} |x-1| \\
                                & = |x-1|.
                            \end{split}
                            \end{equation*}
                        Formally, given $\epsilon > 0$, let $\delta = \min \left\{\frac{1}{2}, \epsilon\right\}$. If $0 < |x-1| < \delta$, then by the work above $|f(x) - L| < \epsilon$.
                    \end{proof}
                \item $\ds \limit_{x \rightarrow 6} \frac{x^2 -3x}{x+3} = 2$.
                    \begin{proof}
                        Observe that:
                            \begin{equation*}
                            \begin{split}
                                |f(x) - L|
                                & = \left|\frac{x^2 - 3x}{x+3} - 2\right| \\
                                & = \left|\frac{x^2 - 5x-6}{x+3}\right| \\
                                & = \left|\frac{x+1}{x+3}\right|\cdot |x-6|.
                            \end{split}
                            \end{equation*}
                        If $|x-6| < 1$, then $6 < x+1 < 8$ and $\frac{1}{8} > \frac{1}{x+1} > \frac{1}{11}$. Thus:
                            \begin{equation*}
                            \begin{split}
                                \left|\frac{x+1}{x+3}\right|\cdot |x-6| 
                                & < \frac{8}{8}|x-6| \\
                                & = |x-6|.
                            \end{split}
                            \end{equation*}
                        Formally, given $\epsilon > 0$, let $\delta = \min\{1,\epsilon\}$. If $0 < |x-6|< \delta$, then by the work above $|f(x) - L| < \epsilon$.
                    \end{proof}
                \item $\ds \limit_{x \rightarrow 0}x\mathbf{1}_\bfQ(x) = 0$.
                    \begin{proof}
                        Let $\epsilon > 0$ be given. Let $\delta = \epsilon$. If $0 < |x - 0| < \delta$, then $0 < |x| < \epsilon$. Whence $|f(x) = 0| |f(x)| \leq |x| < \epsilon$.
                    \end{proof}
                \item $\ds \limit_{x \rightarrow 0} \frac{x^2}{|x|} = 0$.
                    \begin{proof}
                        Observe that:
                            \begin{equation*}
                            \begin{split}
                                \left|\frac{x^2}{|x|} - 0\right| 
                                & = \frac{|x|^2}{|x|} \\
                                & = |x|.
                            \end{split}
                            \end{equation*}
                        Given $\epsilon > 0$, set $\delta = \epsilon$. If $0 < |x-0| < \delta$, then $\left|\frac{x^2}{|x|} - 0\right| = |x| <\delta =  \epsilon$.
                    \end{proof}
            \end{enumerate}
    \end{exercise}
%%%%%%%%%%%%%%%%%%%%%%%%%%%%%%%%%%%%%%%%%%%%%%%%%%%%%%%%%%%%%
    \begin{exercise}
        For which values of $k \in \bfN_0$ does:
            \begin{equation*}
            \begin{split}
                \limit_{x \rightarrow 0}x^k \sin \left(\frac{1}{x}\right)
            \end{split}
            \end{equation*}
        exist?
    \end{exercise}
        \begin{proof}
            For $k \geq 1$, observe that:
                \begin{equation*}
                    -x^k \leq x^k \sin \left(\frac{1}{x}\right) \leq x^k
                \end{equation*}
                \begin{equation*}
                    \iff
                \end{equation*}
                \begin{equation*}
                    \limit_{x \rightarrow 0}-x^k \leq \limit_{x \rightarrow 0}x^k \sin \left(\frac{1}{x}\right) \leq \limit_{x \rightarrow 0}x^k
                \end{equation*}
                \begin{equation*}
                    \iff
                \end{equation*}
                \begin{equation*}
                    0 \leq \limit_{x \rightarrow 0}x^k \sin \left(\frac{1}{x}\right) \leq 0.
                \end{equation*}
            So by the Squeeze Theorem, $\limit_{x \rightarrow 0}x^k \sin \left(\frac{1}{x}\right) = 0$. \nl
            
            If $k=0$, then $\limit_{x \rightarrow 0}\sin \left(\frac{1}{x}\right)$ does not exist.
        \end{proof}
%%%%%%%%%%%%%%%%%%%%%%%%%%%%%%%%%%%%%%%%%%%%%%%%%%%%%%%%%%%%%
    \begin{exercise}
        Assume $f(x) \geq 0$ for all $x \in D$ and suppose $\limit_{x \rightarrow c}f := L$ exists. Show that $L \geq 0$ and that:
            \begin{equation*}
            \begin{split}
                \limit_{x \rightarrow c}\sqrt{f} = \sqrt{L}.
            \end{split}
            \end{equation*}
    \end{exercise}
        \begin{proof}
            Since $\limit_{x \rightarrow c}f = L$ exists:
                \begin{equation*}
                \begin{split}
                    (\forall (x_n)_n \in (D \setminus\{c\})^\bfN)((x_n)_n \rightarrow c \implies (f(x_n))_n \rightarrow L).
                \end{split}
                \end{equation*}
            Since $f(x) \geq 0$ for all $x$, $L \geq 0$. Moreover, $(f(x_n))_n \rightarrow L$ implies $\left(\sqrt{f(x_n)}\right)_n \rightarrow \sqrt{L}$. Thus $\limit_{x \rightarrow c}\sqrt{f} = \sqrt{L}$.
        \end{proof}
%%%%%%%%%%%%%%%%%%%%%%%%%%%%%%%%%%%%%%%%%%%%%%%%%%%%%%%%%%%%%
    \begin{exercise}
        Assume $f:\bfR \rightarrow \bfR$ is such that $f(x+y) = f(x) + f(y)$ for all $x,y \in \bfR$. If $\limit_{x \rightarrow 0}f := L$ exists, show that $L=0$ and show that $\limit_{x \rightarrow c}f$ exists for all $c \in \bfR$.
    \end{exercise}
        \begin{proof}
            Since $f(x+y) = f(x) + f(y)$, observe that:
                \begin{equation*}
                \begin{split}
                    f(1)
                    & = f\left(n \cdot \frac{1}{n}\right) \\
                    & = n\cdot f \left(\frac{1}{n}\right).
                \end{split}
                \end{equation*}
            So $f \left(\frac{1}{n}\right) = \frac{1}{n}f(1)$. Since $\limit_{x \rightarrow 0}f = L$ exists, $\left(\frac{1}{n}\right)_n \rightarrow 0$ implies $\left( f \left(\frac{1}{n}\right)\right)_n \rightarrow L$. But $\left(f \left(\frac{1}{n}\right)\right)_n = \left(\frac{1}{n}f(1)\right)_n \rightarrow 0$. Thus $L = 0$. \nl
            
            Let $(x_n)_n \rightarrow c$, $x_n \neq c$. Then $(x_n - c)_n \rightarrow 0$. Observe that:
                \begin{equation*}
                \begin{split}
                    (f(x_n))_n 
                    & = (f(x_n - c + c))_n \\
                    & = (f(x_n - c) + f(c))_n \\
                    & = (f(x_n -c))_n + (f(c))_n \\
                     \xrightarrow{n \rightarrow \infty} &\phantom{=} 0 + f(c) \\
                     & = f(c).
                \end{split}
                \end{equation*}
            Thus $\limit_{x \rightarrow c}f = f(c)$ exists.
        \end{proof}
%%%%%%%%%%%%%%%%%%%%%%%%%%%%%%%%%%%%%%%%%%%%%%%%%%%%%%%%%%%%%
    \addtocounter{exercise}{1}
%%%%%%%%%%%%%%%%%%%%%%%%%%%%%%%%%%%%%%%%%%%%%%%%%%%%%%%%%%%%%
    \begin{exercise}
        Suppose $f(0,\infty) \rightarrow \bfR$. Show that the following are equivalent:
            \begin{enumerate}[label = (\roman*)]
                \item $\ds \limit_{x \rightarrow \infty}f = L$ (where $L$ can be $\infty$);
                \item For every sequence $(x_n)_n$ in $(0,\infty)$ with $(x_n)_n \rightarrow \infty$ we have $(f(x_n))_n \rightarrow L$.
            \end{enumerate}
    \end{exercise}
        \begin{proof}
            $(\Rightarrow)$ We proceed by cases. \nl
            
            Case 1: $L < \infty$. Let $\epsilon > 0$ be given. Since $\limit_{x \rightarrow \infty}f = L$ exists, there exists $\alpha > 0$ such that $x \geq \alpha$ implies $|f(x) - L| < \epsilon$. Since $(x_n)_n \rightarrow \infty$, there exists $N \in \bfN$ such that $n \geq N$ implies $x_n \geq \alpha$. Whence $|f(x_n)- L|<\epsilon$, giving $(f(x_n))_n \rightarrow L$. \nl
            
            Case 2: $L = \infty$. Let $M > 0$ be given. Since $\limit_{x \rightarrow \infty}f = \infty$ exists, there exists $\alpha > 0$ such that $x \geq \alpha$ implies $f(x) \geq M$. Since $(x_n)_n \rightarrow \infty$, there exists $N \in \bfN$ such that $n \geq N$ implies $x_n \geq \alpha$. Whence $f(x_n) \geq M$, giving $(f(x_n))_n \rightarrow \infty$. \nl
            
            $(\Leftarrow)$ Suppose towards contradiction $\limit_{x \rightarrow \infty}f \neq L $. We proceed by cases. \nl
            
            Case 1: $L < \infty$. Then by the sequential definition of limits:
                \begin{equation*}
                \begin{split}
                    (\exists (x_n)_n \in (0,\infty)^\bfN)((x_n)_n \rightarrow \infty \hspace{2pt}\land\hspace{2pt}(f(x_n))_n \not\rightarrow L).
                \end{split}
                \end{equation*}
            But this contradicts our assumption that for all $(x_n)_n$, $(x_n)_n \rightarrow \infty$ implies $(f(x_n))_n \rightarrow L$. \nl
            
            Case 2: $L = \infty$. Then by the sequential definition of limits:
                \begin{equation*}
                \begin{split}
                    (\exists (x_n)_n \in (0,\infty)^\bfN)((x_n)_n \rightarrow \infty \hspace{2pt}\land\hspace{2pt} f(x_n) \leq M )
                \end{split}
                \end{equation*}
            But this contradicts our assumption that for all $(x_n)_n$, $(x_n)_n \rightarrow \infty$ implies $(f(x_n))_n \rightarrow \infty$.
        \end{proof}
%%%%%%%%%%%%%%%%%%%%%%%%%%%%%%%%%%%%%%%%%%%%%%%%%%%%%%%%%%%%%
    \begin{exercise}
         If $f:(a,\infty)\rightarrow \bfR$ is such that $\limit_{x \rightarrow \infty}xf(x) := L$ exists, show that:
            \begin{equation*}
            \begin{split}
                \limit_{x \rightarrow \infty}f(x) = 0.
            \end{split}
            \end{equation*}
    \end{exercise}
        \begin{proof}
            Let $(x_n)_n \rightarrow \infty$. We want to show that $(f(x_n))_n \rightarrow 0$. \nl
            
            We know that $(x_n f(x_n))_n \rightarrow L$, so it is bounded. That is, there exists $c > 0$ such that $|x_n f(x_n)| \leq c$ for all $n$. Observe that:
                \begin{equation*}
                \begin{split}
                    |f(x_n)| 
                    & = \left|\frac{x_n f(x_n)}{x_n}\right| \\
                    & \leq \frac{c}{|x_n|}. 
                \end{split}
                \end{equation*}
            Since $\left(\frac{c}{|x_n|}\right)_n \rightarrow 0$, we have $(f(x_n))_n \rightarrow 0$. Thus, by Exercise 10, $\limit_{x \rightarrow \infty}f(x) = 0$.
        \end{proof}
%%%%%%%%%%%%%%%%%%%%%%%%%%%%%%%%%%%%%%%%%%%%%%%%%%%%%%%%%%%%%
    \begin{exercise}
        Suppose $f,g:(0,\infty) \rightarrow \bfR$ are such that $\limit_{x \rightarrow \infty} f := L > 0$, and $\limit_{x \rightarrow \infty}g = \infty$. Show that $\limit_{x \rightarrow \infty}fg = \infty$. Does this hold if $L = 0$ as well?
    \end{exercise}
        \begin{proof}
            Let $(x_n)_n \rightarrow \infty$. Let $M >  0$ be given. There exists $N_1$ such that $n \geq N_1$ implies $f(x) \geq \frac{L}{2}$. There exists $N_2$ such that $n \geq N_2$ implies $g(x) \geq \frac{2M}{L}$. So for $n \geq \max\{N_1,N_2\}$, $f(x_n)g(x_n) \geq \frac{L}{2} \frac{2M}{L} = M$. \nl
            
            Note this does not hold for $L=0$. Take $f(x) = \frac{1}{x}$ and $g(x) = x$. Then $\limit_{x \rightarrow \infty}fg = 1$.
            
        \end{proof}
%%%%%%%%%%%%%%%%%%%%%%%%%%%%%%%%%%%%%%%%%%%%%%%%%%%%%%%%%%%%%
\end{document}