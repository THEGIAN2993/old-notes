\chapter*{Monotone Convergence Theorem}
\pagenumbering{arabic}
\vspace{12pt}

\section*{Definitions}
   \begin{enumerate}[label = (\arabic*)]
    \item A sequence $(x_n)_n$ is \textui{monotone} if it is either increasing, decreasing, strictly increasing, or strictly decreasing.
   \end{enumerate}

\section*{Theorems/Propositions/Lemmas}
   \begin{enumerate}[label = (\arabic*)]
        \item A convergent sequence is bounded.
            {\color{red} \begin{proof}
                Suppose $(x_n)_n \rightarrow x$. Since $(x_n)_n$ is convergent, we know:
                    \begin{equation*}
                    \begin{split}
                        (\forall \epsilon > 0)(\exists N \in \bfN)\ni n \geq N \implies |x_n - x| < \epsilon.
                    \end{split}
                    \end{equation*}
                Pick $\epsilon = 1$. Then there exists $N_1 \in \bfN$ such that $n \geq N_1$ implies $x_n \in V_1(x)$. Define:
                    \begin{equation*}
                    \begin{split}
                        c = \max\{|x_1|,|x_2|,...,|x_N|,|x-1|,|x+1|\}.
                    \end{split}
                    \end{equation*}
                If $n \leq N$, then $|x_n| \leq c$. If $n \geq N_1$, then $|x_n| \leq c$.
            \end{proof}}

        \item (Monotone Convergence Theorem) Let $(x_n)_n$ be a monotone sequence. $(x_n)_n$ is convergent if and only if $(x_n)_n$ is bounded. Moreover, If $(x_n)_n$ is increasing and bounded above, then $\lim x_n = \sup\{x_n \mid n \in \bfN\}$ \textit{or} if $(x_n)_n$ is decreasing and bounded below, then $\lim x_n = \inf\{x_n \mid n \in \bfN\}$
            {\color{red} \begin{proof}
                $(\Rightarrow)$ This direction was showed in (1). $(\Leftarrow)$ Suppose $(x_n)_n$ is bounded above and increasing. Let $u = \sup\{x_n \mid n \in \bfN\}$. Supremum property says given $\epsilon > 0$, there exists $N \in \bfN$ with $u - \epsilon < x_N$.
                    \begin{center}
                        \begin{tikzpicture}
                            % Draw the horizontal line
                            \draw[<->] (-1,0) -- (3,0) node[right] {};
                            
                            % Place the ticks and labels
                            \draw (2,0.1) -- (2,-0.1) node[below] {$u$};
                            \draw (0,0.1) -- (0,-0.1) node[below] {$u - \epsilon$};

                            % Draw the downward arrow labeled x_N
                            \draw[->] (1,1) -- (1,0.5) node[below] {$x_N$};
                        \end{tikzpicture}
                    \end{center}
                But for $n \geq N$:
                    \begin{equation*}
                    \begin{split}
                        u-\epsilon < x_N \leq x_n \leq u < u + \epsilon.
                    \end{split}
                    \end{equation*}
                Hence $|x_n-u| < \epsilon$, establishing that $(x_n)_n \rightarrow u$. Now let $y_n = -x_n$. Then $y_n$ is increasing and bounded above. We get:
                    \begin{equation*}
                    \begin{split}
                        \lim y_n = \sup\{y_n \mid n \in \bfN\}
                        & \implies -\lim x_n = \sup\{-x_n \mid n \in \bfN\} \\
                        & \implies -\lim x_n = -\inf\{x_n \mid n \in \bfN\} \\
                        & \implies \phantom{-}\lim x_n = \inf\{x_n \mid n \in \bfN\}. \qedhere
                    \end{split}
                    \end{equation*}
            \end{proof}}

        \item If $(x_n)_n$ is increasing and unbounded, then $(x_n)_n$ diverges properly to $+\infty$.
            {\color{red} \begin{proof}
                Pick $M$ large. Since $(x_n)_n$ is unbounded, there exists $N \in \bfN$ with $x_N > M$. Hence if $n \geq N$, then $x_n \geq x_N > M$, establishing $(x_n)_n \rightarrow +\infty$.
            \end{proof}}
   \end{enumerate}

\section*{Examples}
   \begin{enumerate}[label = (\arabic*)]
        \item Let $x_1 = 8$ and inductively set $x_{n+1} = \frac{1}{2}x_n + 2$. Show that $(x_n)_n$ converges and find its limit.
            {\color{red} \begin{solution}
                Note that $(x_n)_n = (8,6,5,\frac{9}{2},...)$. We will show this sequence is bounded below by $4$ and decreasing. Clearly $x_1 = 8 \geq 4$. Now assume $x_n \geq 4$. Then:
                    \begin{equation*}
                    \begin{split}
                        x_{n+1} &= \frac{1}{2}x_n + 2 \\
                        &\geq \frac{1}{2} (4) + 2 \\
                        & = 4.
                    \end{split}
                    \end{equation*}
                Moreover,
                    \begin{equation*}
                    \begin{split}
                        x_{n+1} \leq x_n 
                        & \iff \frac{1}{2}x_n + 2 \leq x_n \\
                        & \iff 4 \leq x_n.
                    \end{split}
                    \end{equation*}
                Thus $(x_n)_n$ is bounded below by $4$ and decreasing. By MCT $(x_n)_n \rightarrow L$. Observe that:
                    \begin{equation*}
                    \begin{split}
                        x_{n+1} = \frac{1}{2}x_n + 2
                        & \stackrel{n \rightarrow \infty}{\iff} L = \frac{1}{2}L + 2 \\
                        & \iff L = 4.
                    \end{split}
                    \end{equation*}
            \end{solution}}

        \item Let $x_n = \sum_{k = 1}^n \frac{1}{k^2}$. Show that $(x_n)_n$ converges.
            {\color{red} \begin{solution}
                Clearly $x_n \leq x_{n+1}$. We have:
                    \begin{equation*}
                    \begin{split}
                        x_n 
                        & = \sum_{k=1}^n \frac{1}{k^2} \\
                        & = 1 + \sum_{k=2}^n \frac{1}{k^2} \\
                        & \leq 1 + \sum_{k=2}^n \frac{1}{k(k-1)} \quad\quad \text{\tiny Since $k^2 \geq k(k-1)$}\\
                        & = 1 + \sum_{k=2}^n \left(\frac{1}{k-1} - \frac{1}{k}\right) \quad\quad \text{\tiny Partial fractions} \\
                        & = 1 + \left[ \left(1 - \frac{1}{2}\right) + \left(\frac{1}{2} - \frac{1}{3}\right) + ... + \left(\frac{1}{n-1} - \frac{1}{n}\right)\right] \\
                        & = 1 + 1 - \frac{1}{n} \\
                        & = 2 - \frac{1}{n} \\
                        & \leq 2.
                    \end{split}
                    \end{equation*}
                Since $(x_n)_n$ is increasing and bounded above, by MCT $(x_n)_n \rightarrow L$.
            \end{solution}}
        
        \item Given $a > 0$, construct a sequence $(x_n)_n$ which converges to $\sqrt{a}$.
            {\color{red} \begin{solution}
                Let $x_1 = 1$ and inductively set $x_{n+1} = \frac{1}{2}\left(x_n + \frac{a}{x_n}\right)$. Observe that:
                    \begin{equation*}
                    \begin{split}
                        2x_{n+1} = x_n + \frac{a}{x_n}
                        & \implies 2x_{n+1}x_n = x_n^2 + a \\
                        & \implies x_n^2 - 2x_{n+1}x_n +a = 0.
                    \end{split}
                    \end{equation*}
                By assumption $(x_n)_n$ converges, hence this polynomial has a real root. So:
                    \begin{equation*}
                    \begin{split}
                        \Delta \geq 0 
                        &\implies  4x_{n+1}^2 - 4a \geq 0 \\
                        & \implies x_{n+1}^2 \geq a.
                    \end{split}
                    \end{equation*}
                Whence $(x_n)_n$ bounded below. It remains to show that $(x_n)_n$ is decreasing. Observe that:
                    \begin{equation*}
                    \begin{split}
                        x_n \geq x_{n+1}
                        & \iff x_n \geq \frac{1}{2}\left(x_n + \frac{a}{x_n}\right) \\
                        & \iff 2x_n \geq x_n + \frac{a}{x_n} \\
                        & \iff x_n \geq \frac{a}{x_n} \quad\quad \text{\tiny Since $x_n + x_n \geq x_n + \frac{a}{x_n}$}\\
                        & \iff x_n^2 \geq a \\
                        & \iff x_{n+1}^2 \geq a. \quad\quad \text{\tiny Since $a$ is a lowerbound}
                    \end{split}
                    \end{equation*}
                Hence by MCT, $(x_n)_n \rightarrow L$. This gives:
                    \begin{equation*}
                    \begin{split}
                        x_{n+1} = \frac{1}{2}\left(x_n + \frac{a}{x_n}\right)
                        & \stackrel{n \rightarrow \infty}{\implies} L = \frac{1}{2}\left(L + \frac{a}{L}\right) \\
                        & \implies L^2 = a \\
                        & \implies L = \sqrt{a}.
                    \end{split}
                    \end{equation*}
            \end{solution}}

        \item Let $h_n = \sum_{k = 1}^n \frac{1}{k}$. Show that $(h_n)_n \rightarrow +\infty$.
            {\color{red} \begin{solution}
                Clearly $(h_n)_n$ is increasing. Observe that:
                    \begin{equation*}
                    \begin{split}
                        h_2 &= 1 + \frac{1}{2} \geq 1 + \frac{1}{2} \\
                        h_{2^2} &= 1 + \frac{1}{2} + \frac{1}{3} + \frac{1}{4} \geq 1 + \frac{1}{2} + \frac{1}{4} + \frac{1}{4} = 1 + 2 \left(\frac{1}{2}\right) \\
                        h_{2^3} & = ... = 1 + 3 \left(\frac{1}{2}\right)
                    \end{split}
                    \end{equation*}
                Inductively, $h_{2^n} \geq 1 + \frac{n}{2}$. Since $(1 + \frac{1}{n})_n \rightarrow +\infty$, $(h_n)_n \rightarrow +\infty$.
            \end{solution}}
   \end{enumerate}
