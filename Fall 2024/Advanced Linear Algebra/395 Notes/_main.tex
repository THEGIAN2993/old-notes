%%%%%%%PACKAGES%%%%%%%
\documentclass[11pt,twoside,openany]{memoir}

\usepackage[p,osf]{scholax}
% T1 and textcomp are loaded by package. Change that here, if you want
% load sans and typewriter packages here, if needed
\usepackage{amsmath,amsthm}% must be loaded before newtxmath
% amssymb should not be loaded
\usepackage[scaled=1.075,ncf,vvarbb]{newtxmath}% need to scale up math package
% vvarbb selects the STIX version of blackboard bold.

\usepackage{titlesec}
        \titlespacing*{\chapter}
        {0pt} % Left margin
        {*1} % Space before the chapter number (increase this value for more space)
        {0pt} % Space after the chapter number and before the title
\usepackage{anyfontsize}
\usepackage{fancybox}
\usepackage[dvipsnames,svgnames,x11names,hyperref]{xcolor}
\usepackage{enumerate}
\usepackage{comment}
\usepackage{amsfonts}
\usepackage{mathrsfs}
\usepackage{fullpage}
\usepackage{bm}
\usepackage{cprotect}
\usepackage{calligra}
\usepackage{emptypage}
\usepackage{titleps}
\usepackage{microtype}
\usepackage{float}
\usepackage{ocgx}
\usepackage{appendix}
\usepackage{graphicx}
\usepackage{pdfcomment}
\usepackage{enumitem}
\usepackage{mathtools}
\usepackage{tikz-cd}
\usepackage{relsize}
\usepackage[font=footnotesize,labelfont=bf]{caption}
\usepackage{changepage}
\usepackage{xcolor}
\usepackage{ulem}
\usepackage{pgfplots}
\usepackage{marginnote}
        \newcommand*{\mnote}[1]{ % <----------
        \checkoddpage
        \ifoddpage
            \marginparmargin{left}
        \else
            \marginparmargin{right}
        \fi
            \marginnote{\tiny \textcolor{oorange}{#1}}
        }

\usepackage{datetime}
        \newdateformat{specialdate}{\THEYEAR\ \monthname\ \THEDAY}
\usepackage[margin=0.9in]{geometry}
        \setlength{\voffset}{-0.4in}
        \setlength{\headsep}{30pt}
\usepackage{fancyhdr}
        \fancyhf{}
        \pagestyle{fancy}
        \cfoot{\footnotesize \thepage}
        \fancyhead[R]{\tiny \rightmark}
        \fancyhead[L]{\tiny \leftmark}
\usepackage[T1]{fontenc}% http://ctan.org/pkg/fontenc
\usepackage[outline]{contour}% http://ctan.org/pkg/contour
        \renewcommand{\arraystretch}{1.5}
        \contourlength{0.4pt}
        \contournumber{10}%
\usepackage{letterspace}
        \linespread{1.1}
\usepackage{thmtools}
        \declaretheoremstyle[
            spaceabove=15pt,
            headfont=\normalfont\bfseries,
            notefont=\mdseries, notebraces={(}{)},
            bodyfont=\normalfont,
            postheadspace=0.5em
            %qed=\qedsymbol
            ]{defs}

        \declaretheoremstyle[
            spaceabove=15pt, % space above the theorem
            headfont=\normalfont\bfseries,
            bodyfont=\normalfont\itshape,
            postheadspace=0.5em
            ]{thmstyle}
        
        \declaretheorem[
            style=thmstyle,
            numberwithin=section
        ]{theorem}

        \declaretheorem[
            style=thmstyle,
            sibling=theorem,
        ]{proposition}

        \declaretheorem[
            style=thmstyle,
            sibling=theorem,
        ]{lemma}

        \declaretheorem[
            style=thmstyle,
            sibling=theorem,
        ]{corollary}

        \declaretheorem[
            numberwithin=section,
            style=defs,
        ]{example}

        \declaretheorem[
            numberwithin=section,
            style=defs,
        ]{definition}

        \declaretheorem[
            style=thmstyle,
            sibling=theorem,
            numberwithin=section,
        ]{exercise}

        \declaretheorem[
            numbered=unless unique,
            shaded={rulecolor=black,
        rulewidth=1pt, bgcolor={rgb}{1,1,1}}
        ]{axiom}

    \declaretheorem[numbered=unless unique,style=defs]{note}
    \declaretheorem[numbered=unless unique,style=defs]{question}
    \declaretheorem[numbered=no,style=remark]{answer}
    \declaretheorem[numbered=no,style=remark]{remark}
    \declaretheorem[numbered=no,style=remark]{solution}
    \declaretheorem[numbered=no,style=defs]{recall}
\usepackage{hyperref}
\usepackage{romanbar}


%%%%%%%%%%%%%%%%%%%%%%%%%%%%%%%%%%%%%%%%%%%%%%%%%%%%%%%%%%%
%%%%%%%%%%%%%%%%%%%%%%%%%%%%%%%%%%%%%%%%%%%%%%%%%%%%%%%%%%%
%%%%%%%%%%%%%%%%%%%%%%%%%%%%%%%%%%%%%%%%%%%%%%%%%%%%%%%%%%%
%%%%%%%%%%%%%%%%%%%%%%%%%%%%%%%%%%%%%%%%%%%%%%%%%%%%%%%%%%%
%%%%%%%%%%%%%%%%%%%%%%%%%%%%%%%%%%%%%%%%%%%%%%%%%%%%%%%%%%%
%%%%%%%%%%%%%%%%%%%%%%%%%%%%%%%%%%%%%%%%%%%%%%%%%%%%%%%%%%%
%%%%%%%%%%%%%%%%%%%%%%%%%%%%%%%%%%%%%%%%%%%%%%%%%%%%%%%%%%%
%%%%%%%%%%%%%%%%%%%%%%%%%%%%%%%%%%%%%%%%%%%%%%%%%%%%%%%%%%%
%%%%%%%%%%%%%%%%%%%%%%%%%%%%%%%%%%%%%%%%%%%%%%%%%%%%%%%%%%%



%to make the correct symbol for Sha
%\newcommand\cyr{%
%\renewcommand\rmdefault{wncyr}%
%\renewcommand\sfdefault{wncyss}%
%\renewcommand\encodingdefault{OT2}%
%\normalfont \selectfont} \DeclareTextFontCommand{\textcyr}{\cyr}


\DeclareMathOperator{\ab}{ab}
\newcommand{\absgal}{\G_{\bbQ}}
\DeclareMathOperator{\ad}{ad}
\DeclareMathOperator{\adj}{adj}
\DeclareMathOperator{\alg}{alg}
\DeclareMathOperator{\Alt}{Alt}
\DeclareMathOperator{\Ann}{Ann}
\DeclareMathOperator{\arith}{arith}
\DeclareMathOperator{\Aut}{Aut}
\DeclareMathOperator{\Be}{B}
\DeclareMathOperator{\Bd}{Bd}
\DeclareMathOperator{\card}{card}
\DeclareMathOperator{\Char}{char}
\DeclareMathOperator{\csp}{csp}
\DeclareMathOperator{\codim}{codim}
\DeclareMathOperator{\coker}{coker}
\DeclareMathOperator{\coh}{H}
\DeclareMathOperator{\compl}{compl}
\DeclareMathOperator{\conj}{conj}
\DeclareMathOperator{\cont}{cont}
\DeclareMathOperator{\crys}{crys}
\DeclareMathOperator{\Crys}{Crys}
\DeclareMathOperator{\cusp}{cusp}
\DeclareMathOperator{\diag}{diag}
\DeclareMathOperator{\diam}{diam}
\DeclareMathOperator{\Dom}{Dom}
\DeclareMathOperator{\disc}{disc}
\DeclareMathOperator{\dist}{dist}
\DeclareMathOperator{\dR}{dR}
\DeclareMathOperator{\Eis}{Eis}
\DeclareMathOperator{\End}{End}
\DeclareMathOperator{\ev}{ev}
\DeclareMathOperator{\eval}{eval}
\DeclareMathOperator{\Eq}{Eq}
\DeclareMathOperator{\Ext}{Ext}
\DeclareMathOperator{\Fil}{Fil}
\DeclareMathOperator{\Fitt}{Fitt}
\DeclareMathOperator{\Frob}{Frob}
\DeclareMathOperator{\G}{G}
\DeclareMathOperator{\Gal}{Gal}
\DeclareMathOperator{\GL}{GL}
\DeclareMathOperator{\Gr}{Gr}
\DeclareMathOperator{\Graph}{Graph}
\DeclareMathOperator{\GSp}{GSp}
\DeclareMathOperator{\GUn}{GU}
\DeclareMathOperator{\Hom}{Hom}
\DeclareMathOperator{\id}{id}
\DeclareMathOperator{\Id}{Id}
\DeclareMathOperator{\Ik}{Ik}
\DeclareMathOperator{\IM}{Im}
\DeclareMathOperator{\Image}{im}
\DeclareMathOperator{\Ind}{Ind}
\DeclareMathOperator{\Inf}{inf}
\DeclareMathOperator{\Isom}{Isom}
\DeclareMathOperator{\J}{J}
\DeclareMathOperator{\Jac}{Jac}
\DeclareMathOperator{\lcm}{lcm}
\DeclareMathOperator{\length}{length}
\DeclareMathOperator*{\limit}{limit}
\DeclareMathOperator{\Log}{Log}
\DeclareMathOperator{\M}{M}
\DeclareMathOperator{\Mat}{Mat}
\DeclareMathOperator{\N}{N}
\DeclareMathOperator{\Nm}{Nm}
\DeclareMathOperator{\NIk}{N-Ik}
\DeclareMathOperator{\NSK}{N-SK}
\DeclareMathOperator{\new}{new}
\DeclareMathOperator{\obj}{obj}
\DeclareMathOperator{\old}{old}
\DeclareMathOperator{\ord}{ord}
\DeclareMathOperator{\Or}{O}
\DeclareMathOperator{\op}{op}
\DeclareMathOperator{\PGL}{PGL}
\DeclareMathOperator{\PGSp}{PGSp}
\DeclareMathOperator{\rank}{rank}
\DeclareMathOperator{\Ran}{Ran}
\DeclareMathOperator{\Rel}{Rel}
\DeclareMathOperator{\Real}{Re}
\DeclareMathOperator{\RES}{res}
\DeclareMathOperator{\Res}{Res}
%\DeclareMathOperator{\Sha}{\textcyr{Sh}}
\DeclareMathOperator{\Sel}{Sel}
\DeclareMathOperator{\semi}{ss}
\DeclareMathOperator{\sgn}{sign}
\DeclareMathOperator{\SK}{SK}
\DeclareMathOperator{\SL}{SL}
\DeclareMathOperator{\SO}{SO}
\DeclareMathOperator{\Sp}{Sp}
\DeclareMathOperator{\Span}{span}
\DeclareMathOperator{\Spec}{Spec}
\DeclareMathOperator{\spin}{spin}
\DeclareMathOperator{\st}{st}
\DeclareMathOperator{\St}{St}
\DeclareMathOperator{\SUn}{SU}
\DeclareMathOperator{\supp}{supp}
\DeclareMathOperator{\Sup}{sup}
\DeclareMathOperator{\Sym}{Sym}
\DeclareMathOperator{\Tam}{Tam}
\DeclareMathOperator{\tors}{tors}
\DeclareMathOperator{\tr}{tr}
\DeclareMathOperator{\Tr}{Tr}
\DeclareMathOperator{\un}{un}
\DeclareMathOperator{\Un}{U}
\DeclareMathOperator{\val}{val}
\DeclareMathOperator{\vol}{vol}

\DeclareMathOperator{\Sets}{S \mkern1.04mu e \mkern1.04mu t \mkern1.04mu s}
    \newcommand{\cSets}{\scalebox{1.02}{\contour{black}{$\Sets$}}}
    
\DeclareMathOperator{\Groups}{G \mkern1.04mu r \mkern1.04mu o \mkern1.04mu u \mkern1.04mu p \mkern1.04mu s}
    \newcommand{\cGroups}{\scalebox{1.02}{\contour{black}{$\Groups$}}}

\DeclareMathOperator{\TTop}{T \mkern1.04mu o \mkern1.04mu p}
    \newcommand{\cTop}{\scalebox{1.02}{\contour{black}{$\TTop$}}}

\DeclareMathOperator{\Htp}{H \mkern1.04mu t \mkern1.04mu p}
    \newcommand{\cHtp}{\scalebox{1.02}{\contour{black}{$\Htp$}}}

\DeclareMathOperator{\Mod}{M \mkern1.04mu o \mkern1.04mu d}
    \newcommand{\cMod}{\scalebox{1.02}{\contour{black}{$\Mod$}}}

\DeclareMathOperator{\Ab}{A \mkern1.04mu b}
    \newcommand{\cAb}{\scalebox{1.02}{\contour{black}{$\Ab$}}}

\DeclareMathOperator{\Rings}{R \mkern1.04mu i \mkern1.04mu n \mkern1.04mu g \mkern1.04mu s}
    \newcommand{\cRings}{\scalebox{1.02}{\contour{black}{$\Rings$}}}

\DeclareMathOperator{\ComRings}{C \mkern1.04mu o \mkern1.04mu m \mkern1.04mu R \mkern1.04mu i \mkern1.04mu n \mkern1.04mu g \mkern1.04mu s}
    \newcommand{\cComRings}{\scalebox{1.05}{\contour{black}{$\ComRings$}}}

\DeclareMathOperator{\hHom}{H \mkern1.04mu o \mkern1.04mu m}
    \newcommand{\cHom}{\scalebox{1.02}{\contour{black}{$\hHom$}}}

         %  \item $\cGroups$
          %  \item $\cTop$
          %  \item $\cHtp$
          %  \item $\cMod$




\renewcommand{\k}{\kappa}
\newcommand{\Ff}{F_{f}}
%\newcommand{\ts}{\,^{t}\!}


%Mathcal

\newcommand{\cA}{\mathcal{A}}
\newcommand{\cB}{\mathcal{B}}
\newcommand{\cC}{\mathcal{C}}
\newcommand{\cD}{\mathcal{D}}
\newcommand{\cE}{\mathcal{E}}
\newcommand{\cF}{\mathcal{F}}
\newcommand{\cG}{\mathcal{G}}
\newcommand{\cH}{\mathcal{H}}
\newcommand{\cI}{\mathcal{I}}
\newcommand{\cJ}{\mathcal{J}}
\newcommand{\cK}{\mathcal{K}}
\newcommand{\cL}{\mathcal{L}}
\newcommand{\cM}{\mathcal{M}}
\newcommand{\cN}{\mathcal{N}}
\newcommand{\cO}{\mathcal{O}}
\newcommand{\cP}{\mathcal{P}}
\newcommand{\cQ}{\mathcal{Q}}
\newcommand{\cR}{\mathcal{R}}
\newcommand{\cS}{\mathcal{S}}
\newcommand{\cT}{\mathcal{T}}
\newcommand{\cU}{\mathcal{U}}
\newcommand{\cV}{\mathcal{V}}
\newcommand{\cW}{\mathcal{W}}
\newcommand{\cX}{\mathcal{X}}
\newcommand{\cY}{\mathcal{Y}}
\newcommand{\cZ}{\mathcal{Z}}


%mathfrak (missing \fi)

\newcommand{\fa}{\mathfrak{a}}
\newcommand{\fA}{\mathfrak{A}}
\newcommand{\fb}{\mathfrak{b}}
\newcommand{\fB}{\mathfrak{B}}
\newcommand{\fc}{\mathfrak{c}}
\newcommand{\fC}{\mathfrak{C}}
\newcommand{\fd}{\mathfrak{d}}
\newcommand{\fD}{\mathfrak{D}}
\newcommand{\fe}{\mathfrak{e}}
\newcommand{\fE}{\mathfrak{E}}
\newcommand{\ff}{\mathfrak{f}}
\newcommand{\fF}{\mathfrak{F}}
\newcommand{\fg}{\mathfrak{g}}
\newcommand{\fG}{\mathfrak{G}}
\newcommand{\fh}{\mathfrak{h}}
\newcommand{\fH}{\mathfrak{H}}
\newcommand{\fI}{\mathfrak{I}}
\newcommand{\fj}{\mathfrak{j}}
\newcommand{\fJ}{\mathfrak{J}}
\newcommand{\fk}{\mathfrak{k}}
\newcommand{\fK}{\mathfrak{K}}
\newcommand{\fl}{\mathfrak{l}}
\newcommand{\fL}{\mathfrak{L}}
\newcommand{\fm}{\mathfrak{m}}
\newcommand{\fM}{\mathfrak{M}}
\newcommand{\fn}{\mathfrak{n}}
\newcommand{\fN}{\mathfrak{N}}
\newcommand{\fo}{\mathfrak{o}}
\newcommand{\fO}{\mathfrak{O}}
\newcommand{\fp}{\mathfrak{p}}
\newcommand{\fP}{\mathfrak{P}}
\newcommand{\fq}{\mathfrak{q}}
\newcommand{\fQ}{\mathfrak{Q}}
\newcommand{\fr}{\mathfrak{r}}
\newcommand{\fR}{\mathfrak{R}}
\newcommand{\fs}{\mathfrak{s}}
\newcommand{\fS}{\mathfrak{S}}
\newcommand{\ft}{\mathfrak{t}}
\newcommand{\fT}{\mathfrak{T}}
\newcommand{\fu}{\mathfrak{u}}
\newcommand{\fU}{\mathfrak{U}}
\newcommand{\fv}{\mathfrak{v}}
\newcommand{\fV}{\mathfrak{V}}
\newcommand{\fw}{\mathfrak{w}}
\newcommand{\fW}{\mathfrak{W}}
\newcommand{\fx}{\mathfrak{x}}
\newcommand{\fX}{\mathfrak{X}}
\newcommand{\fy}{\mathfrak{y}}
\newcommand{\fY}{\mathfrak{Y}}
\newcommand{\fz}{\mathfrak{z}}
\newcommand{\fZ}{\mathfrak{Z}}


%mathbf
\newcommand{\bfA}{\mathbf{A}}
\newcommand{\bfB}{\mathbf{B}}
\newcommand{\bfC}{\mathbf{C}}
\newcommand{\bfD}{\mathbf{D}}
\newcommand{\bfE}{\mathbf{E}}
\newcommand{\bfF}{\mathbf{F}}
\newcommand{\bfG}{\mathbf{G}}
\newcommand{\bfH}{\mathbf{H}}
\newcommand{\bfI}{\mathbf{I}}
\newcommand{\bfJ}{\mathbf{J}}
\newcommand{\bfK}{\mathbf{K}}
\newcommand{\bfL}{\mathbf{L}}
\newcommand{\bfM}{\mathbf{M}}
\newcommand{\bfN}{\mathbf{N}}
\newcommand{\bfO}{\mathbf{O}}
\newcommand{\bfP}{\mathbf{P}}
\newcommand{\bfQ}{\mathbf{Q}}
\newcommand{\bfR}{\mathbf{R}}
\newcommand{\bfS}{\mathbf{S}}
\newcommand{\bfT}{\mathbf{T}}
\newcommand{\bfU}{\mathbf{U}}
\newcommand{\bfV}{\mathbf{V}}
\newcommand{\bfW}{\mathbf{W}}
\newcommand{\bfX}{\mathbf{X}}
\newcommand{\bfY}{\mathbf{Y}}
\newcommand{\bfZ}{\mathbf{Z}}

\newcommand{\bfa}{\mathbf{a}}
\newcommand{\bfb}{\mathbf{b}}
\newcommand{\bfc}{\mathbf{c}}
\newcommand{\bfd}{\mathbf{d}}
\newcommand{\bfe}{\mathbf{e}}
\newcommand{\bff}{\mathbf{f}}
\newcommand{\bfg}{\mathbf{g}}
\newcommand{\bfh}{\mathbf{h}}
\newcommand{\bfi}{\mathbf{i}}
\newcommand{\bfj}{\mathbf{j}}
\newcommand{\bfk}{\mathbf{k}}
\newcommand{\bfl}{\mathbf{l}}
\newcommand{\bfm}{\mathbf{m}}
\newcommand{\bfn}{\mathbf{n}}
\newcommand{\bfo}{\mathbf{o}}
\newcommand{\bfp}{\mathbf{p}}
\newcommand{\bfq}{\mathbf{q}}
\newcommand{\bfr}{\mathbf{r}}
\newcommand{\bfs}{\mathbf{s}}
\newcommand{\bft}{\mathbf{t}}
\newcommand{\bfu}{\mathbf{u}}
\newcommand{\bfv}{\mathbf{v}}
\newcommand{\bfw}{\mathbf{w}}
\newcommand{\bfx}{\mathbf{x}}
\newcommand{\bfy}{\mathbf{y}}
\newcommand{\bfz}{\mathbf{z}}

%blackboard bold

\newcommand{\bbA}{\mathbb{A}}
\newcommand{\bbB}{\mathbb{B}}
\newcommand{\bbC}{\mathbb{C}}
\newcommand{\bbD}{\mathbb{D}}
\newcommand{\bbE}{\mathbb{E}}
\newcommand{\bbF}{\mathbb{F}}
\newcommand{\bbG}{\mathbb{G}}
\newcommand{\bbH}{\mathbb{H}}
\newcommand{\bbI}{\mathbb{I}}
\newcommand{\bbJ}{\mathbb{J}}
\newcommand{\bbK}{\mathbb{K}}
\newcommand{\bbL}{\mathbb{L}}
\newcommand{\bbM}{\mathbb{M}}
\newcommand{\bbN}{\mathbb{N}}
\newcommand{\bbO}{\mathbb{O}}
\newcommand{\bbP}{\mathbb{P}}
\newcommand{\bbQ}{\mathbb{Q}}
\newcommand{\bbR}{\mathbb{R}}
\newcommand{\bbS}{\mathbb{S}}
\newcommand{\bbT}{\mathbb{T}}
\newcommand{\bbU}{\mathbb{U}}
\newcommand{\bbV}{\mathbb{V}}
\newcommand{\bbW}{\mathbb{W}}
\newcommand{\bbX}{\mathbb{X}}
\newcommand{\bbY}{\mathbb{Y}}
\newcommand{\bbZ}{\mathbb{Z}}
\newcommand{\jota}{\jmath}

\newcommand{\bmat}{\left( \begin{matrix}}
\newcommand{\emat}{\end{matrix} \right)}

\newcommand{\pmat}{\left( \begin{smallmatrix}}
\newcommand{\epmat}{\end{smallmatrix} \right)}

\newcommand{\lat}{\mathscr{L}}
\newcommand{\mat}[4]{\begin{pmatrix}{#1}&{#2}\\{#3}&{#4}\end{pmatrix}}
\newcommand{\ov}[1]{\overline{#1}}
\newcommand{\res}[1]{\underset{#1}{\RES}\,}
\newcommand{\up}{\upsilon}

\newcommand{\tac}{\textasteriskcentered}

%mahesh macros
\newcommand{\tm}{\textrm}

%Comments
\newcommand{\com}[1]{\vspace{5 mm}\par \noindent
\marginpar{\textsc{Comment}} \framebox{\begin{minipage}[c]{0.95
\textwidth} \tt #1 \end{minipage}}\vspace{5 mm}\par}

\newcommand{\Bmu}{\mbox{$\raisebox{-0.59ex}
  {$l$}\hspace{-0.18em}\mu\hspace{-0.88em}\raisebox{-0.98ex}{\scalebox{2}
  {$\color{white}.$}}\hspace{-0.416em}\raisebox{+0.88ex}
  {$\color{white}.$}\hspace{0.46em}$}{}}  %need graphicx and xcolor. this produces blackboard bold mu 

\newcommand{\hooktwoheadrightarrow}{%
  \hookrightarrow\mathrel{\mspace{-15mu}}\rightarrow
}

\makeatletter
\newcommand{\xhooktwoheadrightarrow}[2][]{%
  \lhook\joinrel
  \ext@arrow 0359\rightarrowfill@ {#1}{#2}%
  \mathrel{\mspace{-15mu}}\rightarrow
}
\makeatother

\renewcommand{\geq}{\geqslant}
\renewcommand{\leq}{\leqslant}
\newcommand{\midd}{\hspace{4pt}\middle|\hspace{4pt}}
    
    \newcommand{\bone}{\mathbf{1}}
    \newcommand{\sign}{\mathrm{sign}}
    \newcommand{\eps}{\varepsilon}
    \newcommand{\textui}[1]{\uline{\textit{#1}}}
    
    %\newcommand{\ov}{\overline}
    %\newcommand{\un}{\underline}
    \newcommand{\fin}{\mathrm{fin}}
    
    \newcommand{\chnum}{\titleformat
    {\chapter} % command
    [display] % shape
    {\centering} % format
    {\Huge \color{black} \shadowbox{\thechapter}} % label
    {-0.5em} % sep (space between the number and title)
    {\LARGE \color{black} \underline} % before-code
    }
    
    \newcommand{\chunnum}{\titleformat
    {\chapter} % command
    [display] % shape
    {} % format
    {} % label
    {0em} % sep
    { \begin{flushright} \begin{tabular}{r}  \Huge \color{black}
    } % before-code
    [
    \end{tabular} \end{flushright} \normalsize
    ] % after-code
    }

\newcommand{\nl}{\newline \mbox{}}

\newcommand{\h}[1]{\hspace{#1pt}}

\newcommand{\littletaller}{\mathchoice{\vphantom{\big|}}{}{}{}}
\newcommand\restr[2]{{% we make the whole thing an ordinary symbol
  \left.\kern-\nulldelimiterspace % automatically resize the bar with \right
  #1 % the function
  \littletaller % pretend it's a little taller at normal size
  \right|_{#2} % this is the delimiter
  }}

\newcommand{\mtext}[1]{\hspace{6pt}\text{#1}\hspace{6pt}}

\newcommand{\lnorm}{\left\lVert}
\newcommand{\rnorm}{\right\rVert}

\newcommand{\ds}{\displaystyle}
\newcommand{\ts}{\textstyle}

%This adds a "front cover" page.
%{\thispagestyle{empty}
%\vspace*{\fill}
%\begin{tabular}{l}
%\begin{tabular}{l}
%\includegraphics[scale=0.24]{oxy-logo.png}
%\end{tabular} \\
%\begin{tabular}{l}
%\Large \color{black} Module Theory, Linear Algebra, and Homological Algebra \\ \Large \color{black} Gianluca Crescenzo
%\end{tabular}
%\end{tabular}
%\newpage

\newcommand{\sfrac}[2]{{}^{#1}\mskip -5mu/\mskip -3mu_{#2}}


\makeatletter
\newcommand*{\da@rightarrow}{\mathchar"0\hexnumber@\symAMSa 4B }
\newcommand*{\da@leftarrow}{\mathchar"0\hexnumber@\symAMSa 4C }
\newcommand*{\xdashrightarrow}[2][]{%
  \mathrel{%
    \mathpalette{\da@xarrow{#1}{#2}{}\da@rightarrow{\,}{}}{}%
  }%
}
\newcommand{\xdashleftarrow}[2][]{%
  \mathrel{%
    \mathpalette{\da@xarrow{#1}{#2}\da@leftarrow{}{}{\,}}{}%
  }%
}
\newcommand*{\da@xarrow}[7]{%
  % #1: below
  % #2: above
  % #3: arrow left
  % #4: arrow right
  % #5: space left 
  % #6: space right
  % #7: math style 
  \sbox0{$\ifx#7\scriptstyle\scriptscriptstyle\else\scriptstyle\fi#5#1#6\m@th$}%
  \sbox2{$\ifx#7\scriptstyle\scriptscriptstyle\else\scriptstyle\fi#5#2#6\m@th$}%
  \sbox4{$#7\dabar@\m@th$}%
  \dimen@=\wd0 %
  \ifdim\wd2 >\dimen@
    \dimen@=\wd2 %   
  \fi
  \count@=2 %
  \def\da@bars{\dabar@\dabar@}%
  \@whiledim\count@\wd4<\dimen@\do{%
    \advance\count@\@ne
    \expandafter\def\expandafter\da@bars\expandafter{%
      \da@bars
      \dabar@ 
    }%
  }%  
  \mathrel{#3}%
  \mathrel{%   
    \mathop{\da@bars}\limits
    \ifx\\#1\\%
    \else
      _{\copy0}%
    \fi
    \ifx\\#2\\%
    \else
      ^{\copy2}%
    \fi
  }%   
  \mathrel{#4}%
}
\makeatother



\setsecnumdepth{subsection}
\definecolor{darkgreen}{rgb}{0, 0.5976, 0}
\hypersetup{pdfauthor=Gianluca Crescenzo, pdftitle=Advanced Linear Algebra, pdfstartview=FitH, colorlinks=true, linkcolor=darkgreen, citecolor=darkgreen}

\begin{document}

\pagenumbering{roman}

\tableofcontents
    
\chunnum
\vfill
\specialdate
Last update: \today
\chnum

\chapter{Statistics for the Working Economist}
\pagenumbering{arabic}

\section{Measurements}
    \begin{definition}
        \phantom{a}
        \begin{enumerate}[label = (\arabic*),itemsep=1pt,topsep=3pt]
            \item The large body of data that is the target of our interest is called the \textit{population}.
            \item A subset selected from a given population is called a \textit{sample}.
        \end{enumerate}
    \end{definition}
    
    It is important to note that we cannot make any measurements based off of a given population \textemdash our only resource is making inferences based off of data gathered from a sample. For example, suppose we make $N$ observations $Y_1,...,Y_N$ from a given population and compute its mean:
        \begin{equation*}
        \begin{split}
            \overline{Y} = \frac{1}{n} \sum_{i = 1}^N Y_i.
        \end{split}
        \end{equation*}
    
    The value $\overline{Y}$ is merely an \textit{approximation} or \textit{estimation} of what the true value of the population mean is. The population mean, typically denoted $\mu$, is an unknown constant which we can only estimate using information from a given sample. This leaves us with the following definition:
    
    \begin{definition}
        An \textit{estimator} is a formula that tells how to calculate the value of an estimate based on the measurements contained in a sample.
    \end{definition}

    Typically, if $\theta$ is a fixed parameter from a population, we denote its estimator as $\widehat{\theta}$.

\section{Linear Models}
    In this chapter, we undertake a study of inferential procedures that can be used
    when a random variable $Y$, called the \textit{dependent variable}, has a mean that is a function of one or more non-random variables $X_1,...,X_k$ called \textit{independent variables}.

    \begin{definition}
        A \textit{linear statistical model} relating a random response $Y$ to a set of independent variables $X_1,...,X_k$ is of the form:
            \begin{equation*}
            \begin{split}
                Y = \beta_0 + \beta_1 X_1 + ... + \beta_k X_k + \epsilon,
            \end{split}
            \end{equation*}
        where $\beta_0,...,\beta_k$ are unknown parameters, $\epsilon$ is a random variable, and the variables $X_1,...,X_k$ assume known values. We assume that $E[\epsilon] = 0$ and hence that:
            \begin{equation*}
            \begin{split}
                E[Y] = \beta_0 + \beta_1 X_1 + ... + \beta_k X_k.
            \end{split}
            \end{equation*}
    \end{definition}

    The notation of our linear statistical model needs to be extended to include reference to the number of observations. Suppose that from our random response $Y$ we make $n$ independent observations $Y_1,...,Y_n$. We can write the observation $Y_i$ as:
        \begin{equation*}
        \begin{split}
            Y_i = \beta_0 + \beta_1 X_{i,1} + ... + \beta_N X_{i,n} + \epsilon_i,
        \end{split}
        \end{equation*}
    where $X_{i,j}$ is the $i^\text{th}$ observation of the $j^\text{th}$ independent variable and $\epsilon_i$ is the $i^\text{th}$ observation of the random variable. This is essentially a system of $n$ linear equations. Let:
        \begin{equation*}
        \begin{split}
            \bfY = \bmat Y_1 \\ \vdots \\\ Y_n \emat, \h9 \bfX = \bmat X_1 \\ \vdots \\\ X_n \emat, \h9 \boldsymbol{\epsilon} = \bmat \epsilon_1 \\ \vdots \\\ \epsilon_n \emat, \h9 \boldsymbol{\beta} =\bmat \beta_1 \\ \vdots \\\ \beta_n \emat.
        \end{split}
        \end{equation*}
    We can now express our linear statistical model as:
        \begin{equation*}
        \begin{split}
            \bfY = \boldsymbol{\beta}\bfX + \boldsymbol{\epsilon}.
        \end{split}
        \end{equation*}
    Note that if $\widehat{\boldsymbol{\beta}}$ is an estimator of $\boldsymbol{\beta}$, then $\widehat{\bfY} = \widehat{\boldsymbol{\beta}}\bfX + \boldsymbol{\epsilon}$ is an estimator for $E[\bfY]$. 

    \begin{definition}
        The \textit{sum of squares for errors} is:
            \begin{equation*}
            \begin{split}
                \text{SSE} = \bfY^t \bfY - \boldsymbol{\beta}^t \bfX^t \bfY,
            \end{split}
            \end{equation*}
    \end{definition}

    \begin{exercise}
        Show that $\bfY^t \bfY - \boldsymbol{\beta}^t \bfX^t \bfY = \sum_{i = 1}^n (y_i - \widehat{y_i})^2$.
    \end{exercise}

\section{Method of Least Squares}
    The least-squares procedure for fitting a line through a set of $n$ data points is similar to the method that we might use if we fit a line by eye; that is, we want the differences between the observed values and corresponding points on the fitted line to be “small” in some overall sense. A convenient way to accomplish this, and one that yields estimators with good properties, is to minimize the sum of squares of the vertical deviations from the fitted line:
        \begin{equation*}
        \begin{split}
            \frac{\partial \text{SSE}}{\partial \widehat{\boldsymbol{\beta}}}
            & = \bmat \frac{\partial \text{SSE}}{\partial \widehat{\beta_1}} \\ \vdots \\ \frac{\partial \text{SSE}}{\partial \widehat{\beta_n}} \emat = \mathbf{0}.
        \end{split}
        \end{equation*}
    We will not show the full derivation of finding $\widehat{\boldsymbol{\beta}}$, as this is not used in practice often. Typically, one will use computational software such as Stata to compute linear regressions using the method of least-squares. Regardless, consider the following proposition for \textit{simple linear regression models}, which has merely one independent variable $X$.

    \begin{proposition}
        The least-squares estimators for simple linear regression models is given by:
            \begin{equation*}
            \begin{split}
                \widehat{\beta_1} &= \frac{\sum_{i = 1}^n (X_i - \overline{X})(Y_i - \overline{Y})}{\sum_{i = 1}^n (X_i - \overline{X})^2} \\
                \widehat{\beta_0} & = \overline{Y} - \widehat{\beta_1}\h2\overline{X}.
            \end{split}
            \end{equation*}
    \end{proposition}
    
    \begin{exercise}
        For a simple linear regression model, show that:
        \begin{equation*}
        \begin{split}
            \widehat{\boldsymbol{\beta}} 
             = (\bfX^t \bfX)^{-1}\bfX^{-1}\bfY = \bmat \frac{\sum_{i = 1}^n (X_i - \overline{X})(Y_i - \overline{Y})}{\sum_{i = 1}^n (X_i - \overline{X})^2} \\ \\ \overline{Y} - \widehat{\beta_1}\h2\overline{X} \emat.
        \end{split}
        \end{equation*}
    \end{exercise}
    


\chapter{Bases and Dimension}\label{chapter:bases-and-dimension}

\vspace{12pt}
\section{Basic Definitions}\label{sec:basic-definitions}
    Unless otherwise stated assume $V$ to be an $F$-vector space.
    \begin{definition}
        Let $\cB = \{v_i\}_{i \in I}$ be a subset of $V$ where $I$ is an indexing set (possibly infinite). We say $v \in V$ is an \textui{$F$-linear combination of $\cB$} (or just \textui{linear combination}) if there is a set $\{a_i\}_{i \in I}$ with $a_i = 0$ for all but finitely many $i$ such that $v = \sum_{i \in I}a_i v_i$. The collection of $F$-linear combinations is denoted $\Span_F{(\cB)}$.
    \end{definition}

    \begin{example}
        Let $V = P_2(F)$.
        \begin{enumerate}[label = (\arabic*)]
            \item Set $\cB = \{1,x,x^2\}$. We have $\Span_F{(\cB)} = P_2(F)$.
            \item Set $\cC = \{1,(x-1),(x-1)^2\}$. We have $\Span_F{(\cC)} = P_2(F)$. 
        \end{enumerate}
    \end{example}

    \begin{definition}
        Let $\cB = \{v_i\}_{i \in I}$ be a subset of $V$. We say $\cB$ is \textui{$F$-linearly independent} (or just \textui{linearly independent}) if whenever $\sum_{i \in I}a_i v_i = 0$ then $a_i = 0$ for all $i \in I$.
    \end{definition}

    \begin{definition}
        Let $\cB = \{v_i\}_{i \in I}$ be a subset of $V$. We say $\cB$ is an \textui{$F$-basis} (or just \textui{basis}) of $V$ if:
            \begin{itemize}
                \item $\Span_F{(\cB)} = V$, and
                \item $\cB$ is linearly independent.
            \end{itemize}
    \end{definition}

    \begin{example}
        Let $V = F^n$. Let $\cE_n = \{e_1,...,e_n\}$ with
            \begin{equation*}
            \begin{split}
                e_1 &= (1,0,0,...,0) \\
                e_2 &= (0,1,0,...,0) \\
                &\vdots \\
                e_n &= (0,0,0,...,1).
            \end{split}
            \end{equation*}
        We have that $\cE_n$ is a basis of $F^n$ and is referred to as the \textui{standard basis}.
    \end{example}

\section{Every Vector Space Admits a Basis}
    \begin{definition}
        A \textui{relation} from $A$ to $B$ is a subset $R \subseteq A \times B$. Typically, when one says "a relation on $A$" that means a relation from $A$ to $A$; i.e., $R \subseteq A \times A$.
    \end{definition}

    \begin{definition}
        Let $A$ be a set. An \textui{ordering} of $A$ is a relation $R$ on $A$ that is 
            \begin{enumerate}[label = (\arabic*)]
                \item \textui{reflexive}: $(a,a) \in R$ for all $a \in A$,
                \item \textui{transitive}: $(a,b),(b,c) \in R$ implies $(a,c) \in R$, and 
                \item \textui{antisymmetric}: $(a,b),(b,a) \in R$ implies $a = b$.
            \end{enumerate}
        If this is the case, we write $(a,b) \in R$ as $a \leq_R b$. If $A$ is an ordered set we write it as the ordered pair $(A,\leq_R)$ (or just $A$ if the ordering is obvious by context).
    \end{definition}

    \begin{definition}
        An ordered set $(X, \leq_R)$ is \textui{total} if for all $a,b \in X$ we have that $a \leq_R b$ or $b \leq_R a$.
    \end{definition}

    \begin{definition}
        Let $(X,\leq)$ be an ordered set and $A \subseteq X$ nonempty.
        \begin{enumerate}[label = (\arabic*)]
            \item $A$ is called a \textui{chain} if $(A,\leq)$ is a total ordering.
            \item $A$ is called \textui{bounded above} if there exists an element $u \in X$ with $a \leq u$ for all $a \in A$. Such a $u$ is called an \textui{upperbound} for $A$.
            \item A \textui{maximal element of $A$} is an element $m \in A$ such that if $a \geq m$, then $a = m$.
        \end{enumerate}
    \end{definition}

    \begin{lemma}[Zorn's Lemma]\label{lemma:zorns}
        Let $X$ be an ordered set with the property that every chain has an upperbound. Then $X$ contains a maximal element.
    \end{lemma}

    \begin{theorem}
        Let $\cA$ and $\cC$ be subsets of $V$ with $\cA \subseteq \cC$. Assume $\cA$ is linearly independent and $\Span_F{(\cC)} = V$. Then there exists a basis $\cB$ of $V$ with $\cA \subseteq \cB \subseteq \cC$\footnote{Given any linearly-independent set $\cA$, we can constructing a basis $\cB$ by adding elements. Given any spanning set $\cC$, we can construct a basis $\cB$ by removing elements.}.
    \end{theorem}
        \begin{proof}
            Let $X = \{\cB' \subseteq V \mid \cA \subseteq \cB' \subseteq \cC, \hspace{4pt} \text{$\cB'$ is linearly independent}\}$. We have $\cA \in X$, so $X \neq \emptyset$. $X$ is ordered with respect to inclusion, and has an upperbound of $\cC$. By \nameref{lemma:zorns} we have a maximal element in $X$, call it $\cB$. 
            
            Claim: $\Span_F{(\cB)} = V$. Suppose towards contradiction it's not, then there exists a $v \in \cC$ with $v \not\in \Span_F{(\cB)}$. But then $\cB \cup \{v\}$ is still linearly independent, and $\cB \cup \{v\} \subseteq \cC$. This gives $\cB \subseteq \cB \cup \{v\}$, which is a contradiction because $\cB$ is maximal in $X$. Thus $\Span_F{(\cB)} = V$.
        \end{proof}

\section{Cardinality and Dimension}
    \begin{lemma}\label{lemma:homogenous-system}
        A homogenous system of $m$ linear equations in $n$ unknowns with $m<n$ has a nonzero solution.
    \end{lemma}
        \begin{proof}
            \color{red} do this
        \end{proof}
    
    \begin{corollary}
        Let $\cB \subseteq V$ with $\Span_F{(\cB)} = V$ and $|\cB |  = m$. Any set with more than $m$ elements cannot be linearly independent.
    \end{corollary}
        \begin{proof}
            Let $\cC = \{w_1,...,w_n\}$ with $n > m$. We will show $\cC$ cannot be linearly independent. Write $\cB = \{v_1,...,v_m\}$ with $\Span_F{(\cB)} = V$. For each $i$, write
                \begin{equation*}
                \begin{split}
                    w_i  = \sum_{j = 1}^m a_{ji}v_j \hspace{4pt} \text{for some $a_{ji} \in F$}.
                \end{split}
                \end{equation*}
            Consider the equations
                \begin{equation*}
                \begin{split}
                    \sum_{i = 1}^{n}a_{ji}x_i = 0.
                \end{split}
                \end{equation*}
            By Lemma~\ref{lemma:homogenous-system} there exists nonzero solutions $(x_1,...,x_n) = (c_1,...,c_n) \neq (0,...,0)$. We have
                \begin{equation*}
                \begin{split}
                    0 & = \sum_{j = 1}^m \left(\sum_{i = 1}^n a_{ji}c_i\right)v_j \\
                    & = \sum_{i=1}^n c_i \left(\sum_{j=1}^m a_{ji} v_j\right) \\
                    & = \sum_{i = 1}^n c_i w_i.
                \end{split}
                \end{equation*}
            Thus $\cC = \{w_1,...,w_n\}$ is not linearly independent.
        \end{proof}

    \begin{corollary}
        If $\cB$ and $\cC$ are both finite bases of $V$, then $|\cB| = |\cC|$.
    \end{corollary}
        \begin{proof}
            Let $|\cB| = m$ and $|\cC| = n$. Because $\Span_F{(\cB)} = V$ and $\cC$ is linearly independent, it must be the case that $n \leq m$. But since $\Span_F{(\cC)} = V$ and $\cB$ is also linearly independent, it must be the case that $m \leq n$. By antisymmetry, $n = m$.
        \end{proof}

    \begin{definition}
        Let $\cB$ be a basis of $V$. The \textui{dimension} of $V$, written $\dim_F{(V)}$, is the cardinality of $\cB$; i.e., $\dim_F{(V)} = |\cB|$.
    \end{definition}

    \begin{theorem}
        Let $V$ be a finite dimensional vector space with $\dim_F{(V)} = n$. Let $\cC \subseteq V$ with $|\cC| = m$.
            \begin{enumerate}[label = (\arabic*)]
                \item If $m > n$, then $\cC$ is not linearly independent.
                \item If $m < n$, then $\Span_F{(\cC)} \neq V$.
                \item If $m=n$, then the following are equivalent:
                    \begin{itemize}
                        \item $\cC$ is a basis;
                        \item $\cC$ is linearly independent;
                        \item $\Span_F{(\cC)} = V$.
                    \end{itemize}
            \end{enumerate}
    \end{theorem}

    \begin{corollary}
        Let $W \subseteq V$ be a subspace. We have $\dim_F{(W)} \leq \dim_F{(V)}$. If $\dim_F{(V)} < \infty$, then $V = W$ if and only if $\dim_F{(V)} = \dim_F{(W)}$.
    \end{corollary}

    \begin{example}
        Let $V = \bfC$.
            \begin{enumerate}[label = (\arabic*)]
                \item If $F = \bfC$, then $\cB = \{1\}$ is a basis and $\dim_{\bfC}{(\bfC)} = 1$.
                \item If $F = \bfR$, then $\cB = \{1,i\}$ is a basis and $\dim_{\bfR}{(\bfC)} = 2$.
                \item If $F = \bfQ$, then $|\cB| = \fc$ and $\dim_{\bfQ}{(\bfC)} = \fc$ (the \textit{continuum}).
            \end{enumerate}
    \end{example}

    \begin{example}
        Let $V = F[x]$ and let $f(x) \in F[x]$. We can use this polynomial to split $F[x]$ into equivalence classes analogous to how one creates the field $\bfF_p$. Define $g(x) ~ h(x)$ if $f(x) \mid (g(x) - h(x))$. This is an equivalence relation. We let $[g(x)]$ denote the equivalence class containing $g(x) \in F[x]$. Let $F[x]/(f(x)) = \{[g(x)] \mid g(x) \in F[x]\}$ denote the collection of equivalence classes. Define $[g(x)] + [h(x)] = [g(x) + h(x)]$ and $\alpha[g(x)] = [\alpha g(x)]$, this makes $F[x]/(f(x))$ into a vector space.

        Set $n = \deg{(f(x))}$. Let $\cB = \{[1],[x],...,[x^{n-1}]\}$. We will show this is a basis for $F[x]/(f(x))$. Suppose there exists $a_0,...,a_{n-1} \in F$ with $a_0[1] + a_1[x] + ... +a_{n-1}[x^{n-1}] = [0]$. So $[a_0 + a_1 x + ... + a_{n-1}x^{n-1}] = [0]$, hence $f(x) \mid (a_0 + a_1 x + ... + a_{n-1}x^{n-1})$. But $\deg{(f(x))} = n$, so we must have $a_0 = a_1 = ... = 0$ (linear independence).

        Let $[g(x)] \in F[x]/(f(x))$. The Euclidean algorithm of polynomials gives $g(x) = f(x)q(x) + r(x)$ for some $q(x),r(x) \in F[x]/(f(x))$ with $r(x) = 0$ or $\deg{(r(x))} \leq \deg{(g(x))}$. Observe that $[g(x)] = [f(x)q(x) + r(x)] = [f(x)q(x)] + [r(x)] = [r(x)]$. Since $[r(x)]$ can be written as a linear combination of basis elements from $\cB$, we have $[g(x)] \in \Span_F{(\cB)}$. Note that any element of $\Span_F{(\cB)}$ is clearly contained in $F[x]/(f(x))$, establishing $\Span_F{(\cB)} = F[x]/(f(x))$.
    \end{example}

    \begin{lemma}
        Let $V$ be an $F$-vector space and $\cC = \{v_i\}_{i \in I}$ be a subset of $V$. Then $\cC$ is a basis if and only if each $v \in V$ can be written uniquely as a linear combination of elements of $\cC$.
    \end{lemma}
        \begin{proof}
            Suppose $\cC$ is a basis. Let $v \in V$ and suppose 
                \begin{equation*}
                \begin{split}
                    v = \sum_{i 
                    \in I}a_i v_i = \sum_{i 
                    \in I}b_i v_i,
                \end{split}
                \end{equation*}
            for some $a_i ,b_i \in F$. Observe that:
                \begin{equation*}
                \begin{split}
                    0_v = \sum_{i \in I}(a_i - b_i)v_i.
                \end{split}
                \end{equation*}
            Since $\cC$ is a basis, it is linearly independent, so $a_i - b_i = 0$ for all $i$. Thus $a_i = b_i$ for all $i$ establishing that the expansion is unique.

            Conversely, suppose every vector $v \in V$ is a unique linear combination of $\cC$. Certainly we have $\Span_F{(\cC)} = V$. Suppose $0_v = \sum_{i \in I}a_i v_i$ for some $a_i \in F$. We also have that $0_v = \sum_{i \in I}0 \cdot v_i$. Uniqueness gives $a_i = 0$ for all $i \in I$; i.e., $\cC$ is linearly independent.
        \end{proof}
    
    \begin{proposition}\label{prop:basis-sent}
        Let $V,W$ be $F$-vector spaces.
            \begin{enumerate}[label=(\arabic*)]
                \item Let $T \in \Hom_F{(V,W)}$. We have that $T$ is determined by what it does to a basis (where it maps it).
                \item Let $\cB = \{v_i\}_{i \in I}$ be a basis of $V$ and $\cC = \{w_i\}_{i \in I}$ be a subset of $V$. If $|\cB| = |\cC|$, there is a $T \in \Hom_F{(V,W)}$ such that $T(v_i) = w_i$ for all $i \in I$.
            \end{enumerate}
    \end{proposition}
        \begin{proof}
            (1) Let $v \in V$. Let $\cB = \{v_i\}_{i \in I}$ be a basis of $V$ and write $v = \sum_{i \in I}a_i v_i$. We have $T(v) = T(\sum_{i \in I}a_i v_i) = \sum_{i \in I}a_i T(v_i)$.

            (2) Define $T:V \rightarrow W$ by $v \mapsto \sum_{i \in I}a_i w_i$. If $v = \sum_{i \in I}a_i v_i$ this map is linear {\color{red} (show this)}.
        \end{proof}
    
    \begin{corollary}
        Let $T \in \Hom_F{(V,W)}$ with $\cB = \{v_i\}_{i \in I}$ a basis of $V$ and $\cC = \{w_i = T(v_i)\}_{i \in I}$ a subset of $W$. We have $\cC$ is a basis of $W$ if and only if $T$ is an isomorphism.
    \end{corollary}
        \begin{proof}
            Suppose $\cC$ is a basis of $W$. Using the result from Proposition~\ref{prop:basis-sent}, define $S \in \Hom_F{(W,V)}$ with $S(w_i) = v_i$. {\color{red} Check $T\circ S = \id_W$ and $S \circ T = \id_V$}. Thus $T$ is an isomorphism.

            Conversely, let $T$ be an isomorphism. Let $w \in W$. As $T$ is surjective, there exists a $v \in V$ such that $T(v) = w$. Using $\cB$ as a basis of $V$, write $v = \sum_{i \in I}a_i v_i$. So observe that:
                \begin{equation*}
                    w = T(v) = T\left(\sum_{i \in I}a_i v_i\right) = \sum_{i \in I}a_i T(v_i) \in \Span_F{(\cC)}, 
                \end{equation*} 
            hence $W = \Span_F{(\cC)}$ (note the other direction is trivial \textemdash you never need to show that). Now suppose there exists a collection of elements $a_i \in F$ with $\sum_{i \in I}a_i T(v_i) = 0_W$. Since $T$ is linear, this is equivalent to $T(\sum_{i \in I}a_i v_i) = 0_W$, and since $T$ is injective it must be the case that $\sum_{i \in I}a_i v_i = 0_V$. Since $\cB$ is a basis we get $a_i = 0$ for all $i \in I$, establishing that $\cC$ is linearly independent.
        \end{proof}

    \begin{theorem}[Rank-Nullity Theorem]
        Let $V$ be an $F$-vector space with $\dim_F{(V)} < \infty$. Then:
            \begin{equation*}
            \begin{split}
                \dim_F{(V)} = \dim_F{(\ker{(T)})} + \dim_F{(\Image{(T)})}.
            \end{split}
            \end{equation*}
    \end{theorem}
        \begin{proof}
            Let $\dim_F{(\ker{(T)})} = k$ and $\dim_F{(V)} = n$. Let $\cA = \{v_1,...,v_k\}$ be a basis of $\ker{(T)}$. Extend this to a basis $\cB = \{v_1,...,v_n\}$ of $V$. We'd like to show that $\cC = \{T(v_{k+1}),...,T(v_n)\}$ is a basis of $\Image{(T)}$.

            Let $w \in \Image{(T)}$. So there exists a $v \in V$ with $T(v) = w$. Write $v = \sum_{i = 1}^n a_i v_i$. We have:
                \begin{equation*}
                \begin{split}
                    w 
                    & = T(v) \\
                    & = T\left(\sum_{i = 1}^n a_i v_i\right) \\
                    & = \sum_{i = 1}^n a_i T(v_i) \\
                    & = \sum_{i = k+1}^n a_i T(v_i) \in \Span_F{(\cC)}.  \quad\quad \text{\tiny b/c $v_1,...,v_k \in \ker{(T)}$}\\
                \end{split}
                \end{equation*}
            Thus $\Span_F{(\cC)} = \Image{(T)}$. Now suppose we have $\sum_{i = k+1}^n a_i T(v_i) = 0_W$. Since $T$ is linear we have $T(\sum_{i = 1}^n a_i v_i) = 0_w$, which gives $\sum_{i = 1}^n a_i v_i \in \ker{(T)}$. Thus we can write it in terms of the basis $\cA$ of $\ker{(T)}$: there exists $a_1,...,a_k$ such that 
                \begin{equation*}
                \begin{split}
                    \sum_{i = k+1}^n a_i v_i = \sum_{i=1}^k a_i v_i,
                \end{split}
                \end{equation*}
            which is equivalent to $\sum_{i = 1}^k a_i v_i + \sum_{i = k+1}^n a_i v_i = 0_V$. However, $\cB$ is a basis of $V$ so $a_1 = ... = a_n = 0$.
        \end{proof}
    
    \begin{corollary}
        Let $V,W$ be $F$-vector spaces with $\dim_F{(V)} = n$. Let $V_1 \subseteq V$ be a subspace with $\dim_F{(V_1)} = k$ and $W_1 \subseteq W$ a subspace with $\dim_F{(W_1)} = n-k$. Then there exists a $T \in \Hom_F{(V,W)}$ such that $\ker{(T)} = V_1$ and $\Image{(T)} = W_1$.
    \end{corollary}
        \begin{proof}
            \color{red} do it
        \end{proof}

    \begin{corollary}
        Let $T \in \Hom_F{(V,W)}$ with $\dim_F{(V)} = \dim_F{(W)} < \infty$. The following are equivalent:
            \begin{enumerate}[label = (\arabic*)]
                \item $T$ is an isomorphism.
                \item $T$ is injective.
                \item $T$ is surjective.
            \end{enumerate}
    \end{corollary}
        \begin{proof}
            \color{red} do it
        \end{proof}
    
    \begin{corollary}
        Let $A = \Mat_n{(F)}$. The following are equivalent:
            \begin{enumerate}[label = (\arabic*)]
                \item $A$ is invertible.
                \item There exists an element $B \in \Mat_n{(F)}$ such that $BA = 1_n$.
                \item There exists an element $B \in \Mat_n{(F)}$ such that $AB = 1_n$.
            \end{enumerate}
    \end{corollary}
        \begin{proof}
            \color{red} do it
        \end{proof}

    \begin{corollary}
        Let $\dim_F{(V)} = m$ and $\dim_F{(W)} = n$.
            \begin{enumerate}[label = (\arabic*)]
                \item If $m <n$ and $T \in \Hom_F{(V,W)}$, then $T$ is not surjective.
                \item If $m > n$ and $T \in \Hom_F{(V,W)}$, then $T$ is not injective.
                \item If $m=n$ then $V \cong W$.
            \end{enumerate}
    \end{corollary}

    \begin{example}
        {\color{red} This follows shortly after corollary 2.2.30 (write it down later)}
    \end{example}

\section{Direct Sums and Quotient Spaces}
    \begin{definition}
        Let $V$ be an $F$-vector space and $V_1,...,V_k$ be subspaces. The \textui{sum} of $V_1,...,V_k$ is 
            \begin{equation*}
            \begin{split}
                V_1 + ... + V_k = \{v_1 + ... + v_k \mid v_i \in V_i \}.
            \end{split}
            \end{equation*}
    \end{definition}

    \begin{proposition}
        Let $V$ be an $F$-vector space and $V_1,...,V_k$ be subspaces. Then $V_1 + ... + V_k$ is also a subspace of $V$.
    \end{proposition}
        \begin{proof}
            \color{red} do this
        \end{proof}
    
    \begin{definition}
        Let $V_1,...,V_k$ be subspaces of $V$. We say $V_1,...,V_k$ are \textui{independent} if whenever $v_1 + ... + v_k = 0_V$ then $v_i = 0_V$.
    \end{definition}
    
    \begin{definition}
        Let $V_1,...,V_k$ be subspaces of $V$. We say $V$ is the \textui{direct sum} of $V_1,...,V_k$ and write $V = V_1 \oplus ... \oplus V_k$ if:
            \begin{enumerate}[label = (\arabic*)]
                \item $V = V_1 + ... + V_k$, and
                \item $V_1,...,V_k$ are independent.
            \end{enumerate}
        \phantom{a}
    \end{definition}

    \begin{example}
        \phantom{a}
        \begin{enumerate}[label = (\arabic*)]
            \item Let $V = F^2$ with $V_1 = \{(x,0) \mid x \in F$\} and $V_2 = \{(0,y) \mid y \in F$\}. Then
                \begin{equation*}
                \begin{split}
                    V_1 + V_2 &= \{(x,0) + (0,y) \mid x,y \in F\} \\
                    &= \{(x,y) \mid x,y \in F\} \\
                    & = V
                \end{split}
                \end{equation*}
            If $(x,0) + (y,0) = (0,0)$, then $x=y=0$ which means $V_1$ and $V_2$ are independent. Hence $F^2 = V_1 \oplus V_2$.

            \item Let $V = F[x]$ and $V_1 = F$, $V_2 = Fx = \{\alpha x \mid \alpha \in F \}$, and $V_3 = P_1(F)$. Note that $P_1(F) = V_1 \oplus V_2$. But $V_1,V_3$ are not independent because $1_F \in V_1$ and $-1_F \in V_3$ and $(-1_F) + 1_F = 0$.
            \item Let $\cB = \{v_1,...,v_n\}$ be a basis of $V$ and $\Span_F{(v_i)} = V_i$. Then $V = V_1 \oplus ... \oplus V_n$.
        \end{enumerate}
    \end{example}

    \begin{lemma}
        Let $V$ be an $F$-vector space with $V_1,...,V_k$ as subspaces. We have $V = V_1 \oplus ... \oplus V_k$ if and only if every $v \in V$ can be written uniquely in the form $v = v_1 + ... + v_k$ for all $v_i \in V_i$.
    \end{lemma}
        \begin{proof}
            Suppose $V = V_1 \oplus ... \oplus V_k$. Let $v \in V$. Suppose $v = v_1 + ... + v_k = \tilde{v_1} + ... + \tilde{v_k}$ for $v_i,\tilde{v_i} \in V_i$. Then $0_V = (v_1 - \tilde{v_1}) + ... + (v_k  - \tilde{v_k})$. Since $V_1,...,V_k$ are independent and $v_i - \tilde{v_i} \in V$, this gives $v_i - \tilde{v_i} = 0_V$ for all $i$. Thus the expansion for $V$ is unique.

            Conversely, suppose every $v \in V$ can be written uniquely in the form $v = v_1 + ... + v_k$ with $v_i \in V_i$. Then $V = V_1 + ... + V_k$ by definition of sums of subspaces. If $0_V = v_1 + ... + v_k$ for some $v_i \in V_i$, and $0_v = 0_v + ... + 0_v$, then (by uniqueness) it must be the case that $v_i = 0_V$ for all $i$.
        \end{proof}

    \begin{exercise}
        Let $V_1,...,V_k$ be subspaces of $V$. For each $1\leq i \leq k$, let $\cB_i$ be a basis of $V_i$. Let $\cB = \bigcup_{i = 1}^k \cB_i$. Show that:
            \begin{enumerate}[label = (\arabic*)]
                \item $\cB$ spans $V$ if and only if $V = V_1 + ... + V_k$.
                \item $\cB$ is linearly independent if and only if $V_1,...,V_k$ are independent.
                \item $\cB$ is a basis if and only if $V = V_1 \oplus ... \oplus V_k$.
            \end{enumerate}
    \end{exercise}
        \begin{proof}
            \color{red} do this shit
        \end{proof}

    \begin{lemma}
        Let $U \subseteq V$ be a subspace. Then $U$ has a complement.
    \end{lemma}
        \begin{proof}
            \color{red} do this shit
        \end{proof}

    \begin{definition}
        Let $W \subseteq V$ be a subsapce. Define $v_1 ~ v_2$ if $v_1 - v_2 \in W$ for some $v_1,v_2 \in V$. This forms an equivalence relation. Denote the equivalence class containing $v$ as $[v]_W = v + W = \{\tilde{v} \in V \mid v ~ \tilde{v}\} = \{v+w \mid w \in W\}$. The set containing all equivalence classes over $W$ is denoted $V/W = \{v + W \mid v \in V\}$.
    \end{definition}

    \begin{proposition}
        Let $v_1 + W, v_2 + W \in V/W$ and $\alpha \in F$. With addition and scalar multiplication defined as follows:
            \begin{equation*}
            \begin{split}
                (v_1 + W) + (v_2 + W) &= (v_1 + v_2) + W\\
                \alpha(v_1 + W) &= \alpha v_1 + W,
            \end{split}
            \end{equation*}
        it's operations are well-defined and $V/W$ forms an $F$-vector space.
    \end{proposition}
        \begin{proof}
            Let $v_1 + W = \tilde{v_1} + W$ and $v_2 + W = \tilde{v_2} + W$. Then $v_1 = \tilde{v_1} + w_1$ and $v_2 = \tilde{v_2} + w_2$ for some $w_1,w_2 \in W$. Observe that:
                \begin{equation*}
                \begin{split}
                    (v_1 + W) + (v_2 + W)
                    & = (v_1 + v_2 + W) \\
                    & = (\tilde{v_1} + w_2 + \tilde{v_2} + w_2) + W \\
                    & = (\tilde{v_1} + \tilde{v_2}) + W \\
                    & = (\tilde{v_1} + W) + (\tilde{v_2} + W).
                \end{split}
                \end{equation*} 

                \begin{equation*}
                \begin{split}
                    c(v_1 + W)
                    & = c v_1 + W \\
                    & = c(\tilde{v_1} + w) + W \\
                    & = c\tilde{v_1} + W \\
                    & = c(\tilde{v_1} + W).
                \end{split}
                \end{equation*}
            Hence addition and scalar multiplication are well-defined. {\color{red} show the vector space axioms here}.
        \end{proof}
    
    \begin{example}
        Let $V = \bfR^2$ and $W = \{(x,0) \mid x \in \bfR \}$. Let $(x_0 ,y_0) \in V$. We have that $(x_0,y_0) \sim (x,y)$ if $(x_0,y_0) - (x,y) = (x_0 - x, y_0 - y) \in W$. So $(x_0,y_0) + W = \{(x,y_0) \mid x \in \bfR\}$. Then $V/W$ is a vector space only when $y = 0$.

        Define $\tau : \bfR \rightarrow V/W$ by $y \mapsto (0,y) + W$. This is an isomorphism. Let $y_1,y_2,c \in \bfR$. Observe that:
            \begin{equation*}
            \begin{split}
                \tau(y_1 + c y_2)
                & = (0,y_1 + c y_2) + W \\
                & = ((0,y_1) + (0,cy_2)) + W \\
                & = ((0,y_1) + c(0,y_2)) + W \\
                & = ((0,y_1) + W) + c((0,y_2) + W)\\
                & = \tau(y_1) + c\tau(y_2).
            \end{split}
            \end{equation*}
        Hence $\tau \in \Hom_F{(\bfR, V/W)}$. Let $(x,y) + W \in V/W$. Then $(x,y) + W = (0,y) + W$. So $\tau$ is surjective because $\tau(y) = (0,y) + W$. Now let $y \in \ker{(\tau)}$. Then $\tau(y) = (0,y)+ W = (0,0) + W$. This implies $y=0$, meaning the kernel is trivial and so $\tau$ is injective.
        
        Alternatively, it is routine to show that $\tau^{-1} \in \Hom_F{(V/W,\bfR)}$ with $\tau^{-1} \circ \tau = \id_\bfR$ and $\tau \circ \tau^{-1} = \id_{V/W}$.
    \end{example}

    \begin{definition}
        Let $W \subseteq V$ be a subspace. The \textui{canonical projection map} $\pi_W:V \rightarrow V/W$ is defined by $v \mapsto v+W$. Note that $\pi_W \in \Hom_F{(V,V/W)}$.
    \end{definition}

    \begin{note}
    To define a map $T:V/W \rightarrow V'$, you always have to check it is well-defined.
    \end{note}

    \begin{theorem}[First Isomorphism Theorem]
        Let $T \in \Hom_F{(V,W)}$. Define $\overline{T}:V/\ker{(T)} \rightarrow W$ by $v + \ker{(T)} \mapsto T(v)$. Then $\overline{T}$ is a linear map. Moreover, $V/\ker{(T)} \cong \Image{(T)}$.
    \end{theorem}
        \begin{proof}
            \color{red} finish this
        \end{proof}
    
\section{Dual Spaces}
    Note that when one refers to something as \textit{"canonical"}, it means the object in question does not depend on a basis.
    \begin{definition}
        Let $V$ be an $F$-vector space. The \textui{dual space}, denoted $V^\vee$, is defined to be $V^\vee = \Hom_F{(V,F)}$.
    \end{definition}

    \begin{theorem}
        We have $V$ is isomorphic to a subspace of $V^\vee$. If $\dim_F{(V)} < \infty$, then $V \cong V^\vee$.
    \end{theorem}
        \begin{proof}
            Let $\cB = \{v_i\}_{i \in I}$ be a basis (hence this theorem is not canonical). For each $i \in I$, define:
                \begin{equation*}
                    v_i^\vee (v_j) =
                \begin{cases}
                    1,& i=j \\
                    0,&\text{otherwise}.
                \end{cases}
                \end{equation*}
            We get $\{v_i ^\vee\}_{i \in}$ are elements of $V^\vee$. We obtain $T \in \Hom_F{(V,V^\vee)}$ by $T(v_i) = v_i^\vee$. To show that $V$ is isomorphic to a subspace of $V^\vee$, it is enough to show $T$ is injective, then by the first isomorphism theorem $V \cong \Image{(T)}$ (a subspace of $V^\vee$).

            Let $v \in \ker{(T)}$, then $T(v) = 0_{V^\vee}$. Write $v = \sum_{i \in I}a_i v_i$. So:
                \begin{equation*}
                \begin{split}
                    0_{V^\vee}
                    & = T(v) \\
                    & = T\left(\sum_{i \in I}a_i v_i\right) \\
                    & = \sum_{i \in I}a_i T(v_i) \\
                    & = \sum_{i \in I}a_i v_i^\vee.
                \end{split}
                \end{equation*}
            Towards contradiction, pick some $j$ with $a_j \neq 0$. Note that $0_{V^\vee} = \sum_{i \in I}a_i v_i^\vee(v_j) = a_j$ (every term except for $a_jv_j ^\vee (v_j)$ equals 0). This is a contradiction, hence $T$ is injective. 

            Now assume $\dim_F{(V)} = n$ and write $\cB = \{v_1,...,v_n\}$. Let $v^\vee \in V^\vee$. Define $a_i = v^\vee(v_i)$. Set $v = \sum_{i=1}^n a_i v_i$ and define $S:V^\vee \rightarrow V$ by $S(v^\vee) = v = \sum_{i=1}^n v^\vee (v_i) v_i$. We'd like to show that $S \in \Hom_F{(V^\vee,V)}$ and is the inverse of $T$. Let $v^\vee,w^\vee \in V^\vee$ and $c \in F$. Set $a_i = v^\vee(v_i)$ and $b_i = w^\vee(v_i)$. Then:
                \begin{equation*}
                \begin{split}
                    S(v^\vee + cw^\vee)
                    & = \sum_{i=1}^n \left[(v^\vee + c w^\vee)(v_i) \right]v_i \\
                    & = \sum_{i=1}^n\left[ v^\vee(v_i) + c w^\vee(v_i)\right]v_i \\
                    & = \sum_{i=1}^n v^\vee(v_i)v_i + c \sum_{i=1}^n w^\vee(v_i)v_i \\
                    & = S(v^\vee) + cS(w^\vee).
                \end{split}
                \end{equation*}
            Hence $S$ is linear. Now observe that:
                \begin{equation*}
                \begin{split}
                    (S \circ T)(v_j)
                    & = S(T(v_j)) \\
                    & = S(v_j^\vee) \\
                    & = \sum_{i=1}^n v_j^\vee(v_i)v_i \\
                    & = v_j
                \end{split}
                \end{equation*}
            Let $v^\vee \in V^\vee$. Note that $(T\circ S)(v^\vee)$ is a function, so it will require an input. Observe that
                \begin{equation*}
                \begin{split}
                    (T\circ S)(v^\vee)(v_j)
                    & = T(S(v^\vee))(v_j) \\
                    & = T(\sum_{i=1}^n v^\vee(v_i)v_i)(v_j) \\
                    & = \left[\sum_{i=1}^n v^\vee(v_i)T(v_i)\right](v_j) \\
                    & = \sum_{i=1}^n v^\vee(v_i)(v^\vee_i(v_j)) \\
                    & = v^\vee(v_j).
                \end{split}
                \end{equation*}
        \end{proof}

    \begin{definition}
        Let $\cB = \{v_1,...,v_n\}$ be a basis of $V$. The \textui{dual basis} for $V^\vee$ is $\cB^\vee = \{v_1^\vee,...,v^\vee_n\}$.
    \end{definition}

    \begin{proposition}
        There is a canonical injective linear map from $V$ to $(V^\vee)^\vee$. If $\dim_F{(V)} < \infty$, this is an isomorphism.
    \end{proposition}
        \begin{proof}
            Let $v \in V$. Define $\hat{v}:V^\vee \rightarrow F$ by $\varphi \mapsto \varphi(v)\footnote{\text{This can be notated as $\text{eval}_v$, but $\hat{v}$ appears more often in literature}}$. We can easily verify that $\hat{v}$ is linear. Therefore, we have $\hat{v} \in \Hom_F{(V^\vee,F)} = (V^\vee)^\vee$. We have a map:
                \begin{equation*}
                \begin{split}
                    \Phi:V \rightarrow (V^\vee)^\vee \mtext{defined by} v \mapsto \hat{v}.
                \end{split}
                \end{equation*}
            We want to verify that $\Phi$ is an injective linear map. Let $v_1,v_2 \in V$ and $c \in F$. Let $\varphi \in V^\vee$, then:
                \begin{equation*}
                \begin{split}
                    \Phi(v_1 + c v_2)(\varphi)
                    & = \widehat{v_1 + cv_2}(\varphi) \\
                    & = \varphi(v_1 + cv_2) \\
                    & = \varphi(v_1) + c \varphi(v_2) \\
                    & = \hat{v_1}(\varphi) + c \hat{v_2}(\varphi) \\
                    & = \Phi(v_1)(\varphi) + c \Phi(v_2)(\varphi).
                \end{split}
                \end{equation*}
            We will now show that $\Phi$ is injective. Let $v \in V$ and assume $v \neq 0_V$. We will form a basis $\cB$ of $V$ that contains v ({\color{red} why is this still canonical?}). Let $v^\vee \in V^\vee$, then $v^\vee(v) = 1$ and $v^\vee(w) = 0$ for all $w \in \cB$, $w \neq v$. Now assume $v \in \ker{(\Phi)}$. Then $\Phi(v)(\varphi) = \varphi(v) = 0$ for all $\varphi \in V^\vee$. But picking $\varphi = v^\vee$ gives:
                \begin{equation*}
                \begin{split}
                    0
                    & = \Phi(v)(v^\vee) \\
                    & = v^\vee(v) \\
                    & = 1.
                \end{split}
                \end{equation*}
            This is a contradiction, hence $\Phi$ is injective.
        \end{proof}

    \begin{definition}\label{def:induced-dual}
        Let $T \in \Hom_F{(V,W)}$. We get an induced map $T^\vee : W^\vee \rightarrow V^\vee$ with $T^\vee(\varphi) = \varphi \circ T$. The following diagram commutes:
            \begin{center}
                \begin{tikzcd}
                    V \arrow[r, "T"] \arrow[rd, "T^\vee(\varphi)"', dashed] & W \arrow[d, "\varphi"] \\
                                                                            & F .                     
                    \end{tikzcd}
            \end{center}
    \end{definition}
\chapter{Linear Transformations and Matrices}\label{chapter:Linear Transformations and Matrices}

\vspace{12pt}
\section{Choosing Coordinates}\label{sec:Choosing Coordinates}
    \begin{example}[Choosing Coordinates]
        Let $V$ be an $F$-vector space with $\dim_F(V) <\infty$. Let $\cB = \{v_1,...,v_n\}$ be a basis for $V$. This basis fixes an isomorphism $V \cong F^n$. We can see this as follows: let $v \in V$, then $v = \sum_{i = 1}^n a_i v_i$. Define $T_\cB :V \rightarrow F^n$ by:
            \begin{equation*}
            \begin{split}
                T_\cB(v) = \bmat a_1 \\ \vdots \\ a_n \emat \in F^n.
            \end{split}
            \end{equation*}
        This is an isomorphism. Given $v \in V$, we define $[v]_\cB := T_\cB(v)$. We refer to this as \textit{choosing coordinates} on $V$.
    \end{example}

    
    \begin{example}
        \phantom{a}
        \begin{enumerate}[label = (\arabic*)]
            \item Let $V = \bfQ^2$ and $\cB = \{\pmat 1 \\ 1 \epmat, \pmat 1 \\ -1 \epmat \}$. This forms a basis of $V$. Let $v \in V$ with $v = \pmat a \\ b \epmat$. We have:
            \begin{equation*}
            \begin{split}
                v = \frac{a+b}{2} \bmat 1 \\ 1 \emat + \frac{a-b}{2} \bmat 1 \\ -1 \emat, \hspace{4pt} \text{hence} \hspace{4pt} [v]_\cB = \bmat \frac{a+b}{2} \\ \frac{a-b}{2} \emat.
            \end{split}
            \end{equation*}
            Had we considered the standard basis $\cE_2 = \{\pmat 1 \\ 0 \epmat , \pmat 0 \\ 1 \epmat \}$, then $[v]_{\cE_2} = \pmat a \\ b \epmat$.

            \item Let $V = P_2(\bfR)$. Let $\cC = \{1, (x-1), (x-1)^2 \}$. This forms a basis of $V$. Let $f(x) = a + bx + cx^2 \in P_2(\bfR)$. Written in terms of $\cC$, we have $f(x) = (a + b + c) + (b+2c)(x-1) + c(x-1)^2$. Thus:
                \begin{equation*}
                \begin{split}
                    [f(x)]_\cC = \bmat a+b+c \\ b+2c \\ c \emat
                \end{split}
                \end{equation*}
        \end{enumerate}
    \end{example}

    \begin{example}[Linear Transformations as Matrices]
        Recall that given a matrix $A \in \Mat_{m,n}(F)$, we obtain a linear map $T_A \in \Hom_F{(F^n, F^m)}$ by $T_A(v) = Av$. We aim to show that this process "works in reverse" \textemdash given a linear transformation $T \in \Hom_F{(F^n,F^m)}$, we want to find a matrix $A$ so that $T = T_A$.

        Let $\cE_n = \{e_1,...,e_n\}$ be the standard basis of $F^n$ and $\cF_n = \{f_1,...,f_n\}$ be the standard basis of $F^m$. We have that $T(e_j) \in F^m$ for each $j$. Hence $T(e_j) = v$ for some $v \in F^m$. Representing $v$ in terms of $\cF_n$, let $a_{1j},a_{2j},...,a_{mj}$ be the unique vectors such that $T(e_j) = \sum_{i = 1}^m a_{ij}f_i$. We obtain a matrix $(a_{ij}) := A \in \Mat_{m,n}(F)$, and furthermore:
            \begin{equation*}
            \begin{split}
                T_A(e_j) = A e_j = \sum_{i = 1}^m a_{ij}f_i = a_{1j}f_1 + ... + a_{mj}f_m.
            \end{split}
            \end{equation*}
            \begin{equation*}
            \begin{split}
                \bmat 
                a_{11} & a_{12} & ... & ... &a_{1n} \\
                \vdots & \ddots & & \\
                \vdots & & \ddots & \\
                \vdots & & & \ddots \\
                a_{m1} & a_{m2} & ... & ... &a_{mn} \\
                \emat
                \bmat
                0 \\ \vdots \\ 1_j \\ \vdots \\ 0 
                \emat \hspace{4pt}=\hspace{4pt}
                \bmat
                a_{1j} \\ a_{2j} \\ a_{3j} \\ \vdots \\  a_{mj}
                \emat
            \end{split}
            \end{equation*}
        
    \end{example}

    \begin{example}[Linear Transformations as Matrices]\label{example:linear-transformations-as-matrices}
        Recall that given a matrix $A \in \Mat_{m,n}(F)$, we obtain a linear map $T_A \in \Hom_F{(F^n, F^m)}$ by $T_A(v) = Av$. This process "works in reverse" \textemdash given a linear transformation $T \in \Hom_F{(F^n,F^m)}$, there is a matrix $A$ so that $T = T_A$.

        Let $\cE_n  = \{e_1,...,e_n\}$ be the standard basis of $F^n$ and $\cF_m = \{f_1,...,f_m\}$ be the standard basis of $F^m$. We have that $T(e_j) \in F^m$ for each $j$, meaning we have elements $a_{ij} \in F$ with $T(e_j) = \sum_{i=1}^m a_{ij}f_i$. Define $A = (a_{ij}) \in Mat_{m,n}(F)$. Observe that:
            \begin{equation*}
            \begin{split}
                T_A(e_j) = A e_j = \sum_{i = 1}^m a_{ij}f_i = a_{1j}f_1 + ... + a_{mj}f_m.
            \end{split}
            \end{equation*}
            \begin{equation*}
            \begin{split}
                \bmat 
                a_{11} & a_{12} & ... & ... &a_{1n} \\
                \vdots & \ddots & & \\
                \vdots & & \ddots & \\
                \vdots & & & \ddots \\
                a_{m1} & a_{m2} & ... & ... &a_{mn} \\
                \emat
                \bmat
                0 \\ \vdots \\ 1_j \\ \vdots \\ 0 
                \emat \hspace{4pt}=\hspace{4pt}
                \bmat
                a_{1j} \\ a_{2j} \\ a_{3j} \\ \vdots \\  a_{mj}
                \emat
            \end{split}
            \end{equation*}
        Working "in reverse", let $T \in \Hom_F{(V,W)}$ with $\cB = \{v_1,...,v_n\}$ a basis for $V$ and $\cC = \{w_1,...,w_m\}$ a basis for W. Define:
            \begin{equation*}
            \begin{split}
                P &= T_\cB:V \rightarrow F^n \hspace{4pt}\text{by}\hspace{4pt} v \mapsto [v]_\cB \\
                Q & = T_\cC: W \rightarrow F^m \hspace{4pt}\text{by}\hspace{4pt} w \mapsto [w]_\cC 
            \end{split}
            \end{equation*}
        From the following diagram:
            \begin{center}
                \begin{tikzcd}
                    V \arrow[r, "T"] \arrow[d, "P"']                 & W \arrow[d, "Q"] \\
                    F^n \arrow[r, "Q \circ T \circ P^{-1}"', dashed] & F^m             
                    \end{tikzcd}
            \end{center}
        we have that $Q\circ T \circ P^{-1}$ corresponds to a matrix $A \in \Mat_{m,n}(F)$. Write $[T]_\cB ^\cC = A$, this is the unique matrix that satisfies $[T]_\cB ^\cC [v]_\cB = [T(v)]_\cC$. Given $T(v_j) = \sum_{i = 1}^m a_{ij}w_i$, observe that:
            \begin{equation*}
            \begin{split}
                \left[T\right]_\cB ^\cC v_j = [T(v_j)]_\cC = \left[\sum_{i=1}^m a_{ij}w_i\right]_\cC = \bmat a_{1j} \\ \vdots \\ a_{mj} \emat.
            \end{split}
            \end{equation*}
        So $\left[T\right]_\cB ^\cC v_j$ corresponds to the $j^\text{th}$ column of the matrix $\left[T\right]_\cB ^\cC$ Thus we have:
            \begin{equation*}
            \begin{split}
                \left[ T \right]_\cB ^\cC = 
                \bmat \left[ T(v_1) \right]_\cC &\mid & ... & \mid & \left[ T(v_n) \right]_\cC \emat
            \end{split}
            \end{equation*}
    \end{example}

    \begin{example}
        \phantom{a}
        \begin{enumerate}[label = (\arabic*)]
            \item Let $V = P_3(\bfR)$ with $\cB = \left\{ 1,x,x^2,x^3 \right\}$. Define $T \in \Hom_\bfR{(V,V)}$ by $T(f(x)) = f'(x)$. Following Example~\ref{example:linear-transformations-as-matrices} gives:
            \begin{equation*}
            \begin{split}
                T(1) &=  0 =0\cdot 1 + 0 \cdots x + 0 \cdot x^2 + 0 \cdot x^3 \\
                T(x) & = 1 = 1 \cdot 1 + 0 \cdot x + 0 \cdot x^2 + 0 \cdot x^3 \\
                T(x^2) & = 2x = 0 \cdot 1 + 2 \cdot x + 0 \cdot x^2 + 0 \cdot x^3 \\
                T(x^3) & = 3x^2 = 0 \cdot 1 + 0 \cdot x + 3 \cdot x^2 + 0 \cdot x^3
            \end{split}
            \end{equation*}

            \begin{equation*}
            \begin{split}
                \left[ T(1) \right]_\cB = \pmat 0 \\ 0 \\ 0 \\ 0 \epmat\\
                \left[ T(x) \right]_\cB = \pmat 1 \\ 0 \\ 0 \\ 0 \epmat\\
                \left[ T(x^2) \right]_\cB = \pmat 0 \\ 2 \\ 0 \\ 0\epmat\\
                \left[ T(x^3) \right]_\cB = \pmat 0 \\ 0 \\ 3 \\ 0\epmat\\
            \end{split}
            \end{equation*}

            \begin{equation*}
            \begin{split}
                \left[ T \right]_\cB^\cB =  \bmat 0 & 1 & 0 & 0 \\ 0& 0&2&0\\0&0&0&3\\0&0&0&0\emat.
            \end{split}
            \end{equation*}

            \item Let $V = P_3(\bfR)$ with $\cB = \left\{ 1,x,x^2,x^3 \right\}$ with $\cC = \left\{1,(1-x), (1-x)^2, (1-x^3)  \right\}$. Then
                \begin{equation*}
                \begin{split}
                    T(1) &= 0 \\
                    T(x) &= 1 \\
                    T(x^2) &= 2 + 2(x-1) \\
                    T(x^3) &= -9 - 6(x-1) + 3(x-1)^2
                \end{split}
                \end{equation*}

                \begin{equation*}
                \begin{split}
                    \left[ T(1) \right]_\cC & = \pmat 0 \\ 0 \\ 0 \\ 0\epmat \\
                    \left[ T(x) \right]_\cC & = \pmat 1 \\ 0 \\ 0 \\ 0\epmat\\
                    \left[ T(x^2) \right]_\cC & = \pmat 2 \\ 2 \\ 0 \\ 0\epmat\\
                    \left[ T(x^3) \right]_\cC & = \pmat -9 \\ -6 \\ 3 \\ 0\epmat
                \end{split}
                \end{equation*}

                \begin{equation*}
                \begin{split}
                    \left[ T \right]_\cB ^\cC = 
                    \bmat 
                    0 & 1 & 2 & -9 \\
                    0 & 0 & 2 & -6 \\
                    0 & 0 & 0 & 3 \\
                    0 & 0 & 0 & 0
                    \emat.
                \end{split}
                \end{equation*}
        \end{enumerate}
    \end{example}

    \begin{exercise}
        \phantom{a}
        \begin{enumerate}[label = (\arabic*)]
            \item Let $\cA$ be a basis of $U$, $\cB$ a basis of $V$ and $\cC$ a basis of $W$. Let $S \in \Hom_F{(U,V)}$ and $T \in \Hom_F{(V,W)}$. Show
                \begin{equation*}
                \begin{split}
                    \left[ T \circ S \right]_\cA ^ \cC = \left[ T \right]_\cB ^\cC \left[ S \right]_\cA ^ \cB.
                \end{split}
                \end{equation*}
            \item Given $A \in \Mat_{m,k}(F)$ and $B \in \Mat_{n,m}(F)$, we have corresponding linear maps $T_A$ and $T_B$. Show that you can recover the definition of matrix multiplication by using part (1).
        \end{enumerate}
    \end{exercise}

    \begin{note}
        Instead of $\left[ T \right]_\cB ^ \cB$ we will write $\left[ T \right]_\cB$.
    \end{note}

    \begin{example}[Change of Basis]
        Let $V$ be an $F$-vector space and $\cB,\cB'$ bases of $V$. Given $V$ expressed in terms of $\cB$, we'd like to express it in terms of $\cB'$ (or vice versa).

        Let $\cB = \{v_1,...,v_n\}$ and $\cB' = \{v_1 ' ,...,v_n '\}$. Define:
            \begin{equation*}
            \begin{split}
                T:V \rightarrow F^n \hspace{4pt}\text{by}\hspace{4pt}v \mapsto [v]_\cB\\
                S:V \rightarrow F^n \hspace{4pt}\text{by}\hspace{4pt}v \mapsto [v]_{\cB'}.\\
            \end{split}
            \end{equation*}
        We obtain a diagram similar to Example~\ref{example:linear-transformations-as-matrices}:
            \begin{center}
                \begin{tikzcd}
                    V \arrow[d, "T"'] \arrow[r, "\id_V"]         & V \arrow[d, "S"] \\
                    F^n \arrow[r, "S \circ \id_V \circ T^{-1}"'] & F^n             
                    \end{tikzcd}
            \end{center}
        Hence the change of basis matrix is $\left[ \id_V \right]_\cB^{\cB'}$
    \end{example}

    \begin{exercise}
        Let $\cB = \{v_1,...,v_n\}$. Show that $\left[ \id_V \right]_\cB ^ {\cB'} = \left( \left[ v_1 \right]_{\cB'} \mid ... \mid \left[ v_n \right]_{\cB'} \right)$.
    \end{exercise}

    \begin{example}
        \phantom{a}
        \begin{enumerate}[label = (\arabic*)]
            \item Let $V = \bfQ^2$ with $\cB = \{ e_1 = \pmat 1\\0 \epmat , e_2 = \pmat 0 \\ 1 \epmat\}$ and $\cB' = \{ v_1 = \pmat 1\\-1 \epmat, v_2 = \pmat 1\\1\epmat \}$. Observe that:
                \begin{equation*}
                \begin{split}
                    e_1 &= \frac{1}{2}v_1 + \frac{1}{2}v_2 \\
                    e_2 &= -\frac{1}{2}v_1 + \frac{1}{2}v_2
                \end{split}
                \end{equation*}

                \begin{equation*}
                \begin{split}
                    [e_1]_\cB &= \bmat \frac{1}{2} \\ \frac{1}{2} \emat \\
                    [e_2]_\cB &= \bmat -\frac{1}{2} \\ \frac{1}{2} \emat \\
                \end{split}
                \end{equation*}

                \begin{equation*}
                \begin{split}
                    \left[ \id_V \right]_{\cE_2}^{\cB'} = \bmat \frac{1}{2} & -\frac{1}{2} \\ \frac{1}{2} & \frac{1}{2} \emat.
                \end{split}
                \end{equation*}
            Consider $v = \pmat 2 \\ 3 \epmat \in \bfQ^2$. We can express $v$ in terms of $\cB'$ by doing the following calculation:
                \begin{equation*}
                \begin{split}
                    \left[ \id_V \right]_{\cE_2}^{\cB'}[v_2]_{\cE_2}
                    & = \bmat \frac{1}{2} & -\frac{1}{2} \\ \frac{1}{2} & \frac{1}{2} \emat \bmat 2 \\ 3 \emat \\
                    & = \bmat -\frac{1}{2} \\ \frac{5}{2} \emat \\
                    & = [v]_{\cB'}.
                \end{split}
                \end{equation*}
            
            \item Let $V = P_2(\bfR)$ with $\cB = \{1,x,x^2\}$ and $\cB' = \{1,(x-2),(x-2)^2\}$. Then:
                \begin{equation*}
                \begin{split}
                    1 &= 1 \cdot 1 + 0 \cdot (x-2) + 0 \cdot (x-2)^2 \\
                    x &= 2 \cdot 1 + 1 \cdot (x-2) + 0 \cdot (x-2)^2 \\
                    x^2 &= 4 \cdot 1 + 4 \cdot (x-2) + 1 \cdot (x-2)^2 \\
                \end{split}
                \end{equation*}

                \begin{equation*}
                \begin{split}
                    [1]_{\cB'} =  \pmat 1 \\ 0 \\ 0\epmat\\
                    [x]_{\cB'} = \pmat 2 \\ 1 \\ 0\epmat\\
                    [x^2]_{\cB'} = \pmat 4 \\ 4 \\ 1\epmat
                \end{split}
                \end{equation*}

                \begin{equation*}
                \begin{split}
                    \left[ \id_V \right]_\cB^{\cB'} = 
                    \bmat
                    1 & 2 & 4 \\
                    0 & 1 & 4 \\
                    0 & 0 & 1 
                    \emat.
                \end{split}
                \end{equation*}
        \end{enumerate}
    \end{example}

    \begin{example}[Similar Matrices]\label{example:similar-matrices}
        Let $A,B \in \Mat_n(F)$. Let $\cE_n$ be the standard basis for $F^n$ and $T_A \in \Hom_F{(F^n,F^n)}$ such that $A = [T_A]_{\cE_n}$. We can relate $A$ in terms of an arbitrary basis $\cB$ as follows:
            \begin{center}
                \begin{tikzcd}
                    F^n \arrow[d, "T_\cB"'] \arrow[r, "T_A"] & F^n \arrow[d, "T_\cB"] \\
                    F^n \arrow[r, "{[T_A]_\cB}"']            & F^n .                  
                    \end{tikzcd}
            \end{center}
        But by extending our diagram using our change of basis algorithm, we obtain the following:
            \begin{center}
                \begin{tikzcd}
                    F^n \arrow[d, "T_\cB"'] \arrow[r, "\id_{F^n}"] & F^n \arrow[d, "T_{\cE_n}"'] \arrow[r, "T_A"] & F^n \arrow[d, "T_{\cE_n}"] \arrow[r, "\id_{F^n}"] & F^n \arrow[d, "T_\cB"] \\
                    F^n \arrow[r, "{[\id_{F^n}]_\cB^{\cE_n}}"'] & F^n \arrow[r, "{[T_A]_{\cE_n}}"'] & F^n \arrow[r, "{[\id_{F^n}]^\cB_{\cE_n}}"'] & F^n
                \end{tikzcd}
            \end{center}
        So $[T_A]_\cB = \left[ \id_{F^n} \right]_\cB^{\cE_n} [T_A]_{\cE_n} \left[ \id_{F^n} \right]^\cB_{\cE_n}$. Assigning $P^{-1} = \left[ \id_{F^n} \right]_\cB^{\cE_n}$ and $P = \left[ \id_{F^n} \right]^\cB_{\cE_n}$ yields the familiar equation $[T_A]_\cB = P^{-1}A P$; i.e., $A = P [T_A]_\cB P^{-1}$. In particular, the matrix $A = [T_A]_{\cE_n}$ is similar to $[T_A]_\cB$ for any basis $\cB$.
    \end{example}

    \begin{example}
        Let $A = \pmat 1 & 3 & -5 \\ -2 & -1 & 6 \\ 3 & 2 & 1 \epmat$. Let $\cE_3 = \{e_1,e_2,e_3\}$ be the standard basis of $F^3$. We have:
            \begin{equation*}
            \begin{split}
                T_A(e_1) &= e_1 - 2e_2 + 3e_3 \\
                T_A(e_2) &= 3e_1 - e_2 + 2 e_3 \\
                T_A(e_3) &= 3e_1 + 2e_2 + e_3.
            \end{split}
            \end{equation*}
        Now consider $\cB = \{v_1 = \pmat 1 \\ 1 \\ 0 \epmat, v_2 = \pmat -1 \\ 0 \\ 1 \epmat, v_3 = \pmat 0 \\ 2 \\ 3 \epmat \}$. One can check this is indeed a basis. Observe that:
            \begin{equation*}
            \begin{split}
                e_1 &= -2v_1 + -3v_2 + v_3\\
                e_2 &= 3v_1 + 3v_2 -v_3 \\
                e_3 &= -2v_1 - 2v_2 + v_3.
            \end{split}
            \end{equation*}
        So the change of basis matrix from $\cE_3$ to $\cB$ is given by $P = \left[ \id_{F^3} \right]_{\cE_3}^\cB = \pmat -2 & 3 & -2 \\ -3 & 3 & -2 \\ 1 & -1 & 1 \epmat$. We have $P^{-1} = \pmat 1 & -1 & 0 \\ 1 & 0 & 2 \\ 0 & 1 & 3 \epmat$. Thus $A$ is similar to the matrix $B = P^{-1}AP = \pmat -29 & 32 & -25 \\ -38 & 45 & -31 \\ -20 & 27 & -15 \epmat$.
    \end{example}

\section{Row Operations}
    \begin{definition}
        Let $A = (a_{ij}) \in \Mat_{m,n}(F)$. We say $a_{kl}$ is a \textui{pivot} of $A$ if $a_{kl} \neq 0$ and $a_{ij} = 0$ if $i > k$ or $j < l$.
    \end{definition}

    \begin{example}
        Let $A = \pmat 2 & 1 & 4 & 5 \\ 0 & 0 & 1 & 7 \\ 0 & 0 & 0 & 5 \\ 0 & 0 & 0 & 0 \epmat$. Then 2, 1, and 5 are pivots.
    \end{example}

    \begin{definition}
        Let $A \in \Mat_{m,n}(F)$. We say $A$ is in \textui{row echelon form} if all its nonzero rows have a pivot and all its zero rows are located below nonzero rows. We say it is \textui{reduced row echelon form} if it is in row echelon form and all of its pivots are $1$ and the only nonzero elements in the columns containing pivots.
    \end{definition}

    \begin{example}
        From the previous example, expressing $A = \pmat 2 & 1 & 4 & 5 \\ 0 & 0 & 1 & 7 \\ 0 & 0 & 0 & 5 \\ 0 & 0 & 0 & 0 \epmat$ in reduced row echelon form yields $A' = \pmat 2 & 1 & 0 & 0 \\ 0 & 0 & 1 & 0 \\ 0 & 0 & 0 & 1 \\ 0 & 0 & 0 & 0 \epmat$.
    \end{example}

    \begin{example}
        Let $A = \pmat 3 & 4 & 5 & 6 \\ 1 & 2 & 3 & 4 \\ 1 & 1 & 2 & 3 \epmat$. Then $T_A: F^4 \rightarrow F^4$. Let $\cB_4 = \{e_1,e_2,e_3,e_4\}$ and $\cF_3 = \{f_1,f_2,f_3\}$. So $A = \left[ T_A \right]_{\cB_3}^{\cF_3}$. We have the following set of equations:
            \begin{equation*}
            \begin{split}
                T_A(e_1) &= 3f_1 + f_2 + f_3\\
                T_A(e_2) &= 4f_1 + 2f_2 + f_3\\
                T_A(e_3) &= 5f_1 + 3f_2 + 2f_3\\
                T_A(e_4) &= 6f_1 + 4f_2 + 3f_3.
            \end{split}
            \end{equation*}
        We are going to perform row operations of $A$ by making substitutions to its basis elements. Consider the operation $R_1 \leftrightarrow R_3$.
            \begin{equation*}
            \begin{split}
                \cF_3^{(2)} = \{f_1^{(2)} = f_3, f_2^{(2)} = f_2, f_3^{(2)} = f_1\}.
            \end{split}
            \end{equation*}

            \begin{equation*}
            \begin{split}
                T_A(e_1) &= f_1^{(2)} + f_2^{(2)} + 3f_3^{(2)}\\
                T_A(e_2) &= f_1^{(2)} + 2f_2^{(2)} + 4f_3^{(2)}\\
                T_A(e_3) &= 2f_1^{(2)} + 3f_2^{(2)} + 5f_3^{(2)}\\
                T_A(e_4) &= 3f_1^{(2)} + 4f_2^{(2)} + 6f_3^{(2)}.
            \end{split}
            \end{equation*}
        So $\left[ T_A \right]_{\cB_3}^{\cF_3^{(2)}} =\pmat 1 & 1 & 2 & 3 \\ 1 & 2 & 3 & 4 \\ 3 & 4 & 5 & 6 \epmat $. Now consider the row operation $-R_1 +R_2 \leftrightarrow R_2$.
            \begin{equation*}
            \begin{split}
                \cF_3^{(3)} = \{f_1^{(3)} = f_1^{(2)} + f_2^{(2)}, f_2^{(3)} = f_2^{(2)}, f_3^{(3)} = f_3^{(2)}\}.
            \end{split}
            \end{equation*}

            \begin{equation*}
            \begin{split}
                T_A(e_1) &= f_1^{(2)} + f_2^{(2)} + 3f_3^{(2)} \\
                &= f_1^{(3)} + 3f_3^{(3)}.\\
                \\
                T_A(e_2) &= 
                f_1^{(2)} + 2f_2^{(2)} + 4f_3^{(2)}\\
                & = f_1^{(2)} + f_2^{(2)} + f_2^{(2)} + 4f_3^{(2)} \\
                & = f_1^{(3)} + f_2^{(3)} + 4f_3^{(3)}.\\
                \\
                T_A(e_3) &= ... \\
                T_A(e_4) &= ... 
            \end{split}
            \end{equation*}
        So $\left[ T_A \right]_{\cB_3}^{\cF_3^{(3)}} =\pmat 1 & 1 & 2 & 3 \\ 0 & 1 & 1 & 1 \\ 3 & 4 & 5 & 6 \epmat $. Now consider the row operation $-3R_1 + R_3 \leftrightarrow R_3$.
            \begin{equation*}
            \begin{split}
                \cF_3^{(4)} = \{f_1^{(4)} = f_1^{(3)} + 3f_3^{(3)}, f_2^{(4)} = f_2^{(3)}, f_3^{(4)} = f_3^{(3)} \}.
            \end{split}
            \end{equation*}

            \begin{equation*}
            \begin{split}
                T_A(e_1) &= f_1^{(3)} + 3f_3^{(3)} \\
                &= f_1^{(4)} \\
                \\
                T_A(e_2) &= ... \\
                T_A(e_3) &= ... \\
                T_A(e_4) &= ... 
            \end{split}
            \end{equation*}
        The rest of the steps to convert $A$ to reduced row echelon form follow similarly.
    \end{example}

    \begin{theorem}
        Let $A \in \Mat_{m,n}(F)$. The matrix $A$ can be put in row echelon form through a series of row operations of the form:
            \begin{enumerate}[label = (\arabic*)]
                \item $R_i \leftrightarrow R_j$
                \item $R_i \leftrightarrow cR_i$
                \item $cR_i + R_J \leftrightarrow R_j$.
            \end{enumerate}
    \end{theorem}

    \begin{example}
        Instead of directly changing the basis of a matrix, we can use linear maps to perform row operations. Let $\cC = \{w_1,...,w_n\}$ be a basis of $W$.
            \begin{enumerate}[label = (\arabic*)]
                \item Define $T_{i,j}: W \rightarrow W$ by
                    \begin{equation*}
                    \begin{split}
                        T_{i,j}(w_k) &= w_k \mtext{if} k \neq i,j, \\
                        T_{i,j}(w_i) &= w_j,\\
                        T_{i,j}(w_j) &= w_i.
                    \end{split}
                    \end{equation*}
                Then $E_{i,j} = \left[ T_{i,j} \right]_{\cC}^\cC$ corresponds to the identity matrix except the $i^{\text{th}}$ and $j^{\text{th}}$ rows are switched.

                \item Let $c \in F$, $c \neq 0$. Define $T_i^{(c)}: W \rightarrow W$ by:
                    \begin{equation*}
                    \begin{split}
                        T_i^{(c)}(w_j) &= w_j \mtext{if} j \neq i,\\
                        T_i^{(c)}(w_i) &= c w_i
                    \end{split}
                    \end{equation*}
                Then $E_i^{(c)} = \left[ T_i^{(c)} \right]_\cC ^ \cC$ corresponds to the identity matrix with the $i^\text{th}$ row multiplied by $c$.

                \item Define $T_{i,j}^{(c)}:W \rightarrow W$ by:
                    \begin{equation*}
                    \begin{split}
                        T_{i,j}^{(c)}(w_k) &= w_k \mtext{if} k \neq j,\\
                        T_{i,j}^{(c)}(w_j) &= w_j + cw_i
                    \end{split}
                    \end{equation*}
                Then $E_{i,j}^{(c)} = \left[ T_{i,j}^{(c)} \right]_\cC ^ \cC$ corresponds to the identity matrix with the {\color{red} what does this mean?}
            \end{enumerate}

        Now let $T_A:F^4 \rightarrow F^3$ with $A = \pmat 3 & 4 & 5 & 6 \\ 1 & 2 & 3 & 4 \\ 1 & 1 & 2 & 3 \epmat$ and $\cE_4$ and $\cF_3$ their respective standard bases. Performing the row operation $R_1 \leftrightarrow R_3$ using the above method yields:
            \begin{equation*}
            \begin{split}
                (T_{1,3} \circ T_A)(e_1) &= T_{1,3}(3f_1 + f_2 + f_3)\\
                & = 3T_{1,3}(f_1) + T_{1,3}(f_2) + T_{1,3}(f_3) \\
                & = 3f_3 + f_2 + f1
            \end{split}
            \end{equation*}

            \begin{equation*}
            \begin{split}
                \left[ T_{1,3} \circ {T_A}_{\cE_4}^{\cF_3} \right] 
                &= \left[ T_{1,3} \right]_{\cF_3}^{\cF_3}\left[ T_A \right]_{\cE_4}^{\cF_3} \\
                &\phantom{a} \\
                &= E_{1,3}A \\
                &\phantom{a} \\
                &= \bmat 1 & 1 & 2 & 3 \\ 1 & 2 & 3 & 4 \\ 3 & 4 & 5 & 6 \emat.
            \end{split}
            \end{equation*}
        
        The rest of the row operations follow similarly. The reduced-row echelon form of $A$ can then be expressed as:
            \begin{equation*}
            \begin{split}
                \left[ T_{1,3}^{(-1)} \circ T_{2,3}^{(-1)} \circ T_{(3)}^{(\frac{1}{2})} \circ T_{3,2}^{(-1)} \circ T_{3,1}^{(-3)} \circ T_{1,2}^{(-1)} \circ T_{1,3} \circ T_A \right]_{\cE_4}^{\cF_3}.
            \end{split}
            \end{equation*}
            
    \end{example}

\section{Column-space and Null-space}
    \begin{definition}
        Let $A \in \Mat_{m,n}(F)$.
            \begin{enumerate}[label = (\arabic*)]
                \item The \textui{column-space} of $A$ is the $F$-span of the column vectors, denoted as $CS(A)$.
                \item The \textui{null-space} of $A$ is the $F$-span of vectors $v \in F^n$ such that $Av = 0_V$, denoted as $NS(A)$.
                \item The \textui{rank} of $A$ is $\rank{A} = \dim_F{CS(A)}$.
            \end{enumerate}
    \end{definition}

    \begin{example}
        Let $T_A \in \Hom_F{(F^n,F^m)}$ where $\cE_n = \{e_1,...,e_n\}$ is the standard basis of $F^n$ and $\cF_n = \{f_1,...,f_m\}$ is the standard basis of $F^m$. Since
            \begin{equation*}
            \begin{split}
                \left[T_A\right]_{\cE_n}^{\cF_m} = A = \bmat T_A(e_1) \mid & ... & \mid T_A(e_n) \emat,
            \end{split}
            \end{equation*}
        we have that $CS(A) = \Image{(T_A)}$, so $\rank{A} = \dim_F{\Image{(T_A)}}$. Recall from an introductory linear algebra course that the column space is calculated by:
            \begin{enumerate}[label = (\alph*)]
                \item Put $A$ into row echelon form,
                \item Look at which columns have pivots,
                \item The same columns in $A$ are then a basis of $CS(A)$.
            \end{enumerate}
        Why does this work? There exists an isomorphism $E:F^n \rightarrow F^m$ so that $\left[E \circ T_A\right]_{\cE_n}^{\cF_m} = \left[E\right]_{\cE_n}^{\cF_m} A$   is in row echelon form. The column space of $\left[E \circ T_A\right]_{\cE_n}^{\cF_m}$ has as its basis the columns containing pivots (denoted ${e_i}_1,...,{e_i}_k$):
            \begin{equation*}
            \begin{split}
                \underbrace{\left[E \circ T_A({e_i}_1)\right]_{\cF_m}, \hspace{4pt}... \hspace{5pt},\left[E \circ T_A({e_i}_k)\right]_{\cF_m}}_{\text{this is a basis of $CS(\left[E \circ T_A\right]_{\cE_n}^{\cF_m})$}}
            \end{split}
            \end{equation*}
        Since $E$ is an isomorphism, there is an inverse $E^{-1}:F^m \rightarrow F^m$ with:
            \begin{equation*}
            \begin{split}
                E^{-1}(w_1) &= \left[E \circ T_A({e_i}_1)\right]_{\cF_m} \\
                &\vdots \\
                E^{-1}(w_k) & = \left[E \circ T_A({e_i}_k)\right]_{\cF_m}.
            \end{split}
            \end{equation*}
        These are linearly independent since $E^{-1}$ is an isomorphism. If there is a vector $v \in CS(A)$ with \newline$v \not\in \Span_F{\left(\left[E \circ T_A({e_i}_1)\right]_{\cF_m}, ... ,\left[E \circ T_A({e_i}_k)\right]_{\cF_m}\right)}$, then $E(v)$ cannot be in $\Span_F{(w_1,...,w_k)}$. So the columns \newline $\left[E \circ T_A({e_i}_1)\right]_{\cF_m}, ... ,\left[E \circ T_A({e_i}_k)\right]_{\cF_m}$ give a basis for the column space of $A$.
    \end{example}

    \begin{example}
        Let $A = \pmat 3 & 4 & 5 & 6 \\ 1 & 2 & 3 & 4 \\ 1 & 1 & 2 & 3 \epmat$. Rewritten in row echelon form is $A' = \pmat 1 & 1 & 2 & 3 \\ 0 & 1 & 1 & 1 \\ 0 & 0 & -2 & -4 \epmat$. Thus:
            \begin{equation*}
            \begin{split}
                CS(B) &= \Span_F{\left(\pmat 1\\ 0 \\ 0 \epmat , \pmat 1 \\ 1 \\ 0 \epmat, \pmat 2 \\ 1 \\ -1 \epmat \right)}\\
                CS(A) & = \Span_F{\left(\pmat 3\\ 1 \\ 1 \epmat , \pmat 4 \\ 2 \\ 1 \epmat, \pmat 5 \\3 \\ 2 \epmat \right)}.e\\
            \end{split}
            \end{equation*}
    \end{example}

    \begin{example}
        We have $v \in NS(A)$ if and only if $Av = 0_{F^m} = T_A(v)$. Note that $T_A(v) = 0_{F^m}$ if and only if $v \in \ker{(T_A)}$, hence $NS(A) = \ker{(T_A)}$. In an introductory algebra class, the null space of a matrix $A$ is calculated by:
            \begin{enumerate}[label = (\arabic*)]
                \item Putting $A$ into reduced row echelon form,
                \item Solving the equation $A'x = 0_{F^n}$.
            \end{enumerate}
        This works because given a map $T_A:F^n \rightarrow F^m$, row operations change the basis of the codomain, not the domain. So $NS(A) = NS(A')$.
    \end{example}

    \begin{example}
        Let $A = \pmat 4 & -4 & 2 \\ -4 & 4 & -2 \\ 2 & -1 & 1 \epmat$. The reduce row echelon form of $A$ is $A' = \pmat 1 & 0 & \frac{1}{2} \\ 0 & 1 & 0 \\ 0 & 0 & 0 \epmat$. Solving the equation:
            \begin{equation*}
            \begin{split}
                \bmat 1 & 0 & \frac{1}{2} \\ 0 & 1 & 0 \\ 0 & 0 & 0 \emat \bmat x_1 \\ x_2 \\ x_3 \emat = \bmat 0 \\ 0 \\ 0 \emat
            \end{split}
            \end{equation*}
        gives $x_2 = 0$ and $x_1 = -\frac{1}{2}x_3$. Hence $NS(A) = \Span_F{\pmat -\frac{1}{2} \\ 0 \\ 1 \epmat}$.
    \end{example}

\section{The Transpose of a Matrix}
    \begin{definition}\label{def:transpose}
        Let $A \in \Mat_{m,n}(F)$ with $\cE_n = \{e_1,...,e_n\}$ and $\cF_m = \{f_1,...,f_m\}$ as standard bases. Then $A = \left[T_A\right]_{\cE^n}^{\cF_m}$, and furthermore $T_A \in \Hom_F{(F^n, F^m)}$ induces a dual map $T_A^\vee \in \Hom_F{({F^m}^\vee, {F^n}^\vee)}$. The \textui{transpose} of $A$ is defined as:
            \begin{equation*}
            \begin{split}
                A^t = \left[T_A^\vee\right]_{\cF_m^\vee}^{\cE_n^\vee}.
            \end{split}
            \end{equation*}
    \end{definition}

    \begin{lemma}
        Let $A = (a_{ij}) \in \Mat_{m,n}(F)$. Then $A^t = (b_{ij}) \in \Mat_n,m(F)$ with $b_{ij} = a_{ji}$.
    \end{lemma}
        \begin{proof}
            We use the same setup as Definition~\ref{def:transpose}. We have:
                \begin{equation*}
                \begin{split}
                    T_A(e_i)&= \sum_{k=1}^m a_{ki}f_k \\
                    T_A^\vee(f_j^\vee) &= \sum_{k=1}^n b_{kj}e_k^\vee.
                \end{split}
                \end{equation*}
            Applying $f_j^\vee$ to $T_A(e_i)$ yields\footnote{I was really confused about this. In short, given a $T \in \Hom_F{(V,V)}$ and basis $\cB$ we have a matrix representation $[T]_\cB$. It is natural to wonder what, $[T^\vee]_{\cB^\vee}$ looks like, and it turns out to be the "transpose" we were familiar with from 214. Basically, applying $f_j^\vee$ to $T_A(e_i)$ gives us coefficients (by definition of dual basis elements) which correspond to a particular column vector of $[T_A]_\cB$. Likewise, since we have that fancy property from Definition~\ref{def:induced-dual}, naturally we should evaluate $T_A^\vee(f_j^\vee)$ at $e_i$, which gives us coefficients which correspond to column vectors of $[T_A^\vee]_{\cB^\vee}$. The rest is self-explanatory.}:
                \begin{equation*}
                \begin{split}
                    (f_j^\vee \circ T_A)(e_i) &= f_j^\vee \left(\sum_{k=1}^m a_{ki}f_k\right)\\
                    & = \sum_{k=1}^m a_{ki}f_j^\vee(f_k) \\
                    &= a_{ji}.
                \end{split}
                \end{equation*}
            Evaluating the $T_A^\vee(f_j^\vee)$ at $e_i$ gives:
                \begin{equation*}
                \begin{split}
                    T_A^\vee(f_j^\vee)(e_i)
                    & = \sum_{k=1}^n b_{kj}e_k^\vee(e_i) \\
                    & = b_{ij}.
                \end{split}
                \end{equation*}
            By Definition~\ref{def:induced-dual}, we have $(f_j^\vee \circ T_A)(e_i) = T_A^\vee(f_j^\vee)(e_i)$. Hence $a_{ji} = b_{ij}$
        \end{proof}

    \begin{exercise}
        Let $A_1,A_2 \in \Mat_{m,n}(F)$ and $c \in F$. Show that:
            \begin{equation*}
            \begin{split}
                (A_1 + A_2)^t & = A_1^t + A_2^t \\
                (cA_1)^t &= c A_1^t.
            \end{split}
            \end{equation*}
    \end{exercise}

    \begin{lemma}
        Let $A \in \Mat_{m,n}(F)$ and $B \in \Mat_{p,m}(F)$. Then $(BA)^t = A^t B^t$.
    \end{lemma}
        \begin{proof}
            Let $\cE_m$, $\cE_n$, and $\cE_p$ be standard bases with $\left[T_A\right]_{\cE_n}^{\cE_m} = A$ and $\left[T_B\right]_{\cE_m}^{\cE_p} = B$. Then $BA = \left[T_B \circ T_A\right]_{\cE_n}^{\cE_p}$. Thus:
                \begin{equation*}
                \begin{split}
                    (BA)^t
                    & = \left[(T_B \circ T_A)^\vee\right]_{\cE_p^\vee}^{\cE_n^\vee} \\
                    & = \left[T_A^\vee \circ T_B^\vee\right]_{\cE_p^\vee}^{\cE_n^\vee} \\
                    & = \left[T_A^\vee\right]_{\cE_m^\vee}^{\cE_n^\vee} \left[T_B^\vee\right]_{\cE_p^\vee}^{\cE_m^\vee}\\
                    & = A^t B^t.
                \end{split}
                \end{equation*}
        \end{proof}

    \begin{lemma}
        Let $A \in \GL_n(F)$. Then $(A^{-1})^t = (A^t)^{-1}$.
    \end{lemma}
        \begin{proof}
            Let $A = \left[T_A\right]_{\cE_n}^{\cE_n}$. Then $A^{-1} = \left[T_A^{-1}\right]_{\cE_n}^{\cE_n}$. We have:
                \begin{equation*}
                \begin{split}
                    1_n 
                    & = \left[\id_{F^n}^\vee\right]_{\cE_n^\vee}^{\cE_n^\vee} \\
                    & = \left[(T_A^{-1} \circ T_A)^\vee\right]_{\cE_n^\vee}^{\cE_n^\vee} \\
                    & = \left[T_A^\vee \circ (T_A^{-1})^\vee\right]_{\cE_n^\vee}^{\cE_n^\vee} \\
                    & = \left[T_A^\vee\right]_{\cE_n^\vee}^{\cE_n^\vee} \left[(T_A^{-1})^\vee\right]_{\cE_n^\vee}^{\cE_n^\vee} \\
                    & = A^t(A^{-1})^t.
                \end{split}
                \end{equation*}
            By the uniqueness of inverses, we must have that $(A^{-1})^t = (A^t)^{-1}$ Showing left invertibility follows identically.
        \end{proof}
\chapter{Generalized Eigenvectors and Jordan Canonical Form}
\vspace{12pt}

\section{Diagonalization}
    \begin{recall}
        We say $A \sim B$ if and only if $A = PBP^{-1}$ for some $P \in \GL_n(F)$. In particular, this means $A = [T]_\cA$ and $B = [T]_\cB$ for some bases $\cA$ and $\cB$ (Example~\ref{example:similar-matrices}).
    \end{recall}

    \begin{definition}
        We say $A$ is \textui{diagonalizable} if $A \sim D$ for some diagonal matrix $D$. In terms of linear transformations, $A = [T]_\cA$ is diagonalizable if there is a basis $\cB$ such that $[T]_\cB = D$.
    \end{definition}

    \begin{example}
        If $A \sim B$ then $A$ is diagonalizable if and only if $B$ is diagonalizable. If $A$ and $B$ are diagonalizable, they must be similar to the same diagonal matrix up to reordering the diagonals. 
    \end{example}

    \begin{example}
        Let $V = F^2$ and $T \in \Hom_F{(V,V)}$. Let $T(e_1) = 3e_1$ and $T(e_2) = -2e_2$. We have that:
            \begin{equation*}
            \begin{split}
                \left[T\right]_{\cE_2} = \bmat 3 & 0 \\ 0 & -2 \emat.
            \end{split}
            \end{equation*}
        It follows that $V = V_1 \oplus V_2$, where $V_1 = \Span_F{(e_1)}$ and $V_2 = \Span_F{(e_2)}$. In this case, we have that $T(V_1) \subseteq V_1$ and $T(V_2) \subseteq V_2$, allowing us to write $T$ as a diagonal matrix.
    \end{example}

    \begin{example}
        Let $V = F^2$ and $T \in \Hom_F{(V,V)}$. Consider $T(e_1) = 3e_1$ and $T(e_2) = e_1 + 3e_2$. Then:
            \begin{equation*}
            \begin{split}
                \left[T\right]_{\cE_2} = \bmat 3 & 1 \\ 0 & 3 \emat.
            \end{split}
            \end{equation*}
        Then $V = V_1 \oplus V_2$ with $V_1 = \Span_F(e_1)$ and $V_2 = \Span_F{(e_2)}$. But while we have $T(V_1) \subseteq V_1$, we do not have $T(V_2) \subseteq V_2$.

        Suppose towards contradiction we have $W_1,W_2 \neq \{0\}$ with $T(W_1) \subseteq W_1$ and $T(W_2) \subseteq W_2$. Write $W_i = \Span_F{(w_i)}$. In particular, this means we can write $T(w_1) = \alpha w_1$ and $T(w_2) = \beta w_2$. For $\cB = \{w_1,w_2\}$, we have:
            \begin{equation*}
            \begin{split}
                \left[T\right]_\cB = \bmat \alpha & 0 \\ 0 & \beta \emat.
            \end{split}
            \end{equation*}
        Write $w_1 = ae_1 + be_2$ and $w_2 = ce_1 + de_2$. Then:
            \begin{equation*}
            \begin{split}
                \alpha w_1
                & = T(w_1) \\
                & = aT(e_1) + bT(e_2) \\
                & = a(3e_1) + b (e_1 + 3e_2) \\
                & = (3a+b)e_1 + (3b)e_2.
            \end{split}
            \end{equation*}
        Thus, $\alpha(ae_1 + be_2) = (3a+b)e_1 + (3b) e_2$, meaning $\alpha a = 3b+b$ and $\alpha b = 3b$. Either $b = 0$ or $\alpha = 3$. It must be the case that $\alpha = 3$, hence $T(w_1) = 3w_1$. A similar argument for $w_1$ gives:
            \begin{equation*}
            \begin{split}
                \beta w_2
                & = T(w_2) \\
                & = ... \\
                & = (3c+d)e_1 + (3d)e_2.
            \end{split}
            \end{equation*}
        This implies $\beta c = ec + d$ and $\beta d = 3d$. If $\beta = 3$, then this contradicts the first equation. If $w_2 = ce_1$, this contradicts $w_1,w_2$ being a basis.
    \end{example}

    \begin{example}\label{example:field-extension-invertible}
        Let $A = \pmat 1 & 2 \\ 3 & 4 \epmat$. Let $F = \bfQ$. Let $P \in GL_2(\bfQ)$, where $P = \pmat a & b \\ c & d \epmat$. We have:
            \begin{equation*}
            \begin{split}
                P^{-1}AP = \frac{1}{ad-bc}\bmat ad - 2ab + 2cd - 4bc & -3bd - 3b^2 + 2d^2 \\ 3ac + 3a^2 - 2c^2 & -bc + 3ab - 2cd + 4ad \emat.
            \end{split}
            \end{equation*}
        We must have that $3a^2 + 4ac - 2c^2 = 0$. If $c= 0$, then $a = 0$, which contradicts $P$ being invertible. So $c\neq 0$, meaning we can divide by $c^2$ and set $x = \frac{a}{c}$. Then the roots of $3x^2 + 3x - 2 = 0$ are:
            \begin{equation*}
            \begin{split}
                x = \frac{-3 \pm \sqrt{33}}{6},
            \end{split}
            \end{equation*}
        which gives:
            \begin{equation*}
            \begin{split}
                a = \frac{-3 \pm \sqrt{33}}{6} c.
            \end{split}
            \end{equation*}
        Since $c \neq 0$, $a \not\in \bfQ$. Thus we cannot diagonalize $A$ over $\bfQ$. But if we were to take $F = \bfQ(\sqrt{33})$, then we have that:
            \begin{equation*}
            \begin{split}
                \cB = \{v_1 = \pmat 1 \\ \frac{3 + \sqrt{33}}{4} \epmat, v_2 = \pmat 1 \\ \frac{3 - \sqrt{33}}{4} \epmat \},
            \end{split}
            \end{equation*}
            \begin{equation*}
            \begin{split}
                \left[T\right]_\cB = \bmat \frac{5 + \sqrt{33}}{2} & 0 \\ 0 & \frac{5 - \sqrt{33}}{2} \emat .
            \end{split}
            \end{equation*}
    \end{example}

    \begin{definition}
        Let $V$ be an $F$-vector space and $T \in \Hom_F{(V,V)}$. A subspace $W \subseteq V$ is said to be \textui{$T$-invariant} or \textui{$T$-stable} if $T(W) \subseteq W$.
    \end{definition}

    \begin{theorem}\label{thm:stable-block-diagonal}
        Let $\dim_F{(V)} = n$ and $W \subseteq V$ a $k$-dimensional subspace. Let $\cB_W = \{v_1,...,v_k\}$ be a basis of $W$ and extend to a basis $\cB = \{v_1,...,v_n\}$ of $V$. Let $T \in \Hom_F{(V,V)}$. We have $W$ is $T$-stable if and only if $\left[T\right]_\cB$ is block upper-triangular of the form
            \begin{equation*}
            \begin{split}
                \bmat A & B \\ 0& D \emat
            \end{split}
            \end{equation*}
        where $A = \left[\restr{T}{W}\right]_{\cB_W}$.
    \end{theorem}

    \begin{example}
        Let $V = \bfQ^4$ with basis $\cE_4 = \{e_1,e_2,...,e_4\}$ and define $T$ by:
            \begin{equation*}
            \begin{split}
                T(e_1) &= 2e_1 + 3e_3 \\
                T(e_2) &= e_1 + e_4 \\
                T(e_3) &= e_1 - e_3 \\
                T(e_4) &= 2e_1 - 2e_2 + 5e_3 - 4e_4.
            \end{split}
            \end{equation*}
        Set $W = \Span_\bfQ(e_1,e_3)$, then $W$ is $T$-stable. Since $\cB_W = \{e_1,e_3\}$ and $\cB = \{e_1,e_2,e_3,e_4\}$, we have:
            \begin{equation*}
            \begin{split}
                \left[T\right]_\cB
                &= \bmat 2 & 1 & 1 & 2 \\ 3 & -1 & 0 & 5 \\ 0 & 0 & 0 & -2 \\ 0 & 0 & 1 & -4 \emat
                \begin{tikzpicture}[overlay, remember picture]
                    \draw[gray, thick] (-1.85,-1.32) rectangle (-2.9,-0.1); % Adjust the coordinates for placement
                \end{tikzpicture}
            \end{split}
            \end{equation*}
    \end{example}

    \begin{example}
        A special case is when $\dim_F{W} = 1$. If $W = \Span_F{(w_1)}$ and $W$ is $T$-stable, then $T(w_1) \in W_1$; i.e., $T(w_1) = \lambda w_1$ for some $\lambda \in F$ Equivalently, this can be written as $(T-\lambda \id_V)(w_1) = 0_V$, meaning $w_1 \in \ker{(T-\lambda \id_V)}$.
    \end{example}

\section{Eigenvalues and Eigenvectors}

    \begin{definition}
        Let $T \in \Hom_F{(V,V)}$ and $\lambda \in F$. If $\ker{(T-\lambda \id_V)} \neq \{0_V\}$, we say $\lambda$ is an \textui{eigenvalue} of $T$. Any nonzero vector in $\ker{(T-\lambda \id_V)}$ is called a \textui{$\lambda$-eigenvector}. The set $E_\lambda^1 = \ker{(T - \lambda \id_V)}$ is called the \textui{eigenspace} associated with $\lambda$.
    \end{definition}

    \begin{exercise}
        Show that $E_\lambda^1$ is a subspace.
    \end{exercise}
    
    \begin{exercise}
        Let $T \in \Hom_F{(V,V)}$. If $\lambda_1,\lambda_2 \in F$ with $\lambda_1 \neq \lambda_2$, then $E_{\lambda_1}^1 \cap E_{\lambda_2}^1 = \{0_V\}$.
    \end{exercise}

    \begin{example}
        Let $A = \pmat 12 & 35 \\ -6 & 17 \epmat \in \Mat_2{(\bfQ)}$ and $T_A \in \Hom_\bfQ{(\bfQ^2,\bfQ^2)}$. We have:
            \begin{equation*}
            \begin{split}
                \phantom{a}\\
                \bmat -12 & 35 \\ -6 & 17 \emat \bmat 1 \\ \frac{2}{5} \emat &= 2 \bmat 1 \\ \frac{2}{5} \emat \\
                \phantom{a}\\
                \bmat -12 & 35 \\ -6 & 17 \emat \bmat 1 \\ \frac{3}{7} \emat &= 3 \bmat 1 \\ \frac{3}{7} \emat \\
            \end{split}
            \end{equation*}
        So $T_A$ has eigenvalues of $2$ and $3$. Then 
            \begin{equation*}
            \begin{split}
                E_2^1 &= \Span_\bfQ{ \left(v_1 =\pmat 1\\ 2/5 \epmat\right)}\\
                E_3^1 &= \Span_\bfQ{ \left(v_2 =\pmat 1\\ 3/7 \epmat\right)}\\
            \end{split}
            \end{equation*}
        gives:
            \begin{equation*}
            \begin{split}
                \left[T_A\right]_{\{v_1,v_2\}} = \bmat 2 & 0 \\ 0 & 3 \emat.
            \end{split}
            \end{equation*}
    \end{example}

    \begin{example}[$F \mathrlap{\lfloor}\lceil x \mathrlap{\rfloor}\rceil$-Modules]
        Let $T \in \Hom_F{(V,V)}$. Note that $V$ is by definition an $F$-module, but we are able to view $V$ as an $F[x]$-module given some linear transformation $T$. The action $ F[x] \times V \rightarrow V$ is defined by $(f(x),v) \mapsto f(T)(v)$.

        Write $T^m = \underbrace{T \circ T \circ ... \circ T}_{m-\text{times}}$. Write $f(x) \in F[x]$ as $f(x) = a_mx^m + ... + a_1x + a_0$. Then
            \begin{equation*}
            \begin{split}
                f(T) = a_m T^m + ... + a_1 T + a_0 \id_V \in \Hom_F{(V,V)}.
            \end{split}
            \end{equation*}
        For example, let $g(x) = 2x^2 + 3 \in \bfR[x]$. Then $g(T) = 2T^2 + 3\id_V$ and $g(T)(v) = 2T(T(v)) + 3v$. If $f(x) = g(x)h(x)$ for some $g(x),h(x) \in F[x]$, then $f(T) = g(T) \circ h(T)$. Instead of writing $f(T)(v) = g(T)(h(T)(v))$, we will abuse notation and write $g(T)h(T)(v)$. Normally function composition does not commute, but these do {\color{red} for some reason}.
    \end{example}

    \begin{theorem}
        Let $\dim_F{(V)} = n$ and $T \in \Hom_F{(V,V)}$. There is a unique monic polynomial $m_T(x) \in F[x]$ of lowest degree so that $m_T(T)(v) = 0_V$ for all $v \in V$. Moreover, $\deg_{m_T}{(T)}\leq n^2$.
    \end{theorem}
        \begin{proof}
            Recall that $\Hom_F{(V,V)}$ is an $F$-vector space. We have $\Hom_F{(V,V)} \cong \Mat_n{(F)}$, hence $\dim_F{(\Hom_F{(V,V)})} = n^2$.

            Given $T \in \Hom_F{(V,V)}$, consider the set $\{\id_V,T,T^2,...,T^{n^2}\} \subseteq \Hom_F{(V,V)}$. This has $n^2 + 1$ elements, so it must be linearly dependent (meaning a linear combination of some subset can equal $0$). Let $m$ be the smallest integer so that
                \begin{equation*}
                \begin{split}
                    a_mT^m + ... + a_1T + a_0 \id_V\footnotemark = 0_{\Hom_F{(V,V)}}.
                \end{split}
                \end{equation*}
            \footnotetext{This seems kind of out of nowhere, so think of it like this: Let $I_T =\{p \in F[x] \mid p(T)(v) = 0_V \mtext{for all}v \in V\}$. $F[x]$ is a P.I.D., so every ideal is generated by a single element. The minimal polynomial $m_T(x)$ is the generator of this ideal.}We obtain a set $\{\id_V,T,T^2,...,T^m\}$. Since $m$ is minimal, $a_m \neq 0$. Define:
                \begin{equation*}
                \begin{split}
                    m_T(x) = x^m + b_{m-1}x^{m-1} + ... + b_1x + b_0 \in F[x], \mtext{where} b_i = \frac{a_i}{a_m}.
                \end{split}
                \end{equation*}
            This gives $m_T(T) = 0_{\Hom_F{(V,V)}}$; i.e., $m_T(T)(v) = 0_V$ for all $v \in V$. 
                It remains to that $m_T(x)$ is unique. Suppose there exists an $f(x) \in F[x]$ which satisfies $f(T)(v) = 0_V$ for all $v \in V$. Write:
                    \begin{equation*}
                    \begin{split}
                        f(x) = m_T(x)q(x) + r(x)
                    \end{split}
                    \end{equation*}
                for some $q(x), r(x) \in F[x]$ with $r(x) = 0$ or $\deg{(r(x))} < \deg{(m_T(x))}$. We have for all $v \in V$:
                    \begin{equation*}
                    \begin{split}
                        0_V
                        & = f(T)(v) \\
                        & = q(T)m_T(T)(v) + r(T)(v) \\
                        & = q(T)(0_V) + r(T)(v) \\
                        & = r(T)(v)
                    \end{split}
                    \end{equation*}
                It must be the case that $r(x) = 0$, otherwise we have a polynomial of lower degree than $m_T(x)$ which kills all vectors. So $f(x) = m_T(x)q(x)$; i.e., $m_T(x) \mid f(x)$. But if $m_T(x)$ and $f(x)$ are both monic and of minimal degree, it must be the case that they are the same degree. This gives $m_T(x) = f(x)$.
        \end{proof}

    \begin{definition}
        The unique monic polynomial $m_T(x)$ is called the \textui{minimal polynomial} of $T$.
    \end{definition}

    \begin{corollary}
        If $f(x) \in F[x]$ satisfies $f(T)(v) = 0_V$ for all $v \in V$, then $m_T(x) \mid f(x)$.
    \end{corollary}
        \begin{proof}
            {\color{red} wat}
        \end{proof}
    
    \begin{example}
        Let $F = \bfQ$ and $A = \pmat 1 & 2  \\ 3 & 4 \epmat$. We can see that:
            \begin{equation*}
            \begin{split}
                A - a_0 1_2 \neq 0_2 \mtext{for any} a_0 \in F.
            \end{split}
            \end{equation*}
        But $A^2 = \pmat 7 & 10 \\ 15 & 22 \epmat$ gives $A^2 -5A -2\cdot1_2 = 0_2$. Hence $m_A(x) = x^2 - 5x - 2$. Note the relationship between this example and Example~\ref{example:field-extension-invertible}.
    \end{example}

    \begin{example}
        Let $V = \bfQ^3$, $\cE_3 = \{e_1,e_2,e_3\}$, and 
            \begin{equation*}
            \begin{split}
                \left[T_A\right]_{\cE_3} = A = \bmat 1 & 2 & 3 \\  0 & 1 & 4 \\ 0 & 0 & -1 \emat.
            \end{split}
            \end{equation*}
        Let $W = \Span_\bfQ(e_1)$ Then $T(W) = T(\alpha e_1) = \alpha e_1 \in W$. Hence $T(W) \subseteq W$, meaning $W$ is $T$-stable. This gives $1$ as an eigenvalue. On a completely unrelated note, $m_{T_A}(x) = (x-1)^2(x+1)$.
    \end{example}

    \begin{theorem}
        Let $V$ be an $F$-vector space and $T \in \Hom_F{(V,V)}$. We have $\lambda$ is an eigenvalue if and only if $\lambda$ is the root of $m_T(x)$. In particular, if $(x-\lambda) \mid m_T(x)$, then $E_\lambda^1 \neq \{0_V\}$ (i.e., there is a nonzero $v \in V$ such that $T(v) = \lambda v$).
    \end{theorem}
        \begin{proof}
            Let $\lambda$ be an eigenvalue with eigenvector $v$ and write $m_T(x) = x^m + ... + a_1 x + a_0$. We have:
                \begin{equation*}
                \begin{split}
                    0_V
                    & = m_T(T)(v) \\
                    & = (T^m + a_{m-1}T^{m-1} + ... + a_1 T + a_0 \id_V)(v) \\
                    & = T^m(v) + a_{m-1}T^{m-1}(v) + ... + a_1T(v) + a_0v \\
                    & = \lambda^m v + a_{m-1}\lambda^{m-1}v + ... + a_1 \lambda v + a_0 v \\
                    & = (\lambda^m  + a_{m-1}\lambda^{m-1} + ... + a_1 \lambda  + a_0)v \\
                    & = m_T(\lambda)\cdot v.
                \end{split}
                \end{equation*}
            Since $v \neq 0$ and $m_T(\lambda) \in F$, it must be the case that $m_T(\lambda) = 0$. Hence $\lambda$ is a root.

            Now suppose $m_T(\lambda) = 0$. This gives $m_T(x) = (x-\lambda)f(x)$ for some $f(x) \in F[x]$. Since $\deg{f(x)} < \deg{m_T(x)}$, this gives a nonzero vector $v \in V$ so that $f(T)(v) \neq 0$ (since $m_T(x)$ is the smallest polynomial that satisfies $m_T(T)(v) = 0_V$, it must be the case that there is a nonzero $v \in V$ that satisfies $f(T)(v) \neq 0$). Set $w = f(T)(v)$, then:
                \begin{equation*}
                \begin{split}
                    0_V
                    & = (T-\lambda \id_V)f(T) \\
                    & = (T - \lambda \id_V)w,
                \end{split}
                \end{equation*}
            which simplifies to $T(w) = \lambda w$. Thus $\lambda$ is an eigenvalue.
        \end{proof}

    \begin{corollary}
        Let $\lambda_1,...,\lambda_n \in F$ be distinct eigenvalues of $T$. For each $i$, let $v_i$ be an eigenvector with eigenvalue $\lambda_i$. The set $\{v_1,...,v_m\}$ is linearly independent.
    \end{corollary}
        \begin{proof}
            We have $m_T(x) = (x-\lambda_1)(x-\lambda_2)...(x-\lambda_m)f(x)$ for some $f(x) \in F[x]$. Suppose $a_1v_1 + ... + a_mv_m = 0_V$ for $a_i \in F$. Define $g_1(x) = (x-\lambda_2)...(x-\lambda_m)f(x)$. Note that $g_1(T)(v_i) = 0_V$ for $2 \leq i \leq m$. Then:
                \begin{equation*}
                \begin{split}
                    0_V 
                    & = g_1(T)(0_V) \\
                    & = \sum_{j=1}^m a_j g_1(T)(v_j) \\
                    & = a_1 g_1(T)(v_1) \\
                    & = a_1 g_1 (\lambda_1) v_1
                \end{split}
                \end{equation*}
            But $g_1(\lambda_1) \neq 0$ and $v\neq 0$,  so it must be that case that $a_1 =0$. Inductively, it follows for $2,...,m$.
        \end{proof}

    \begin{corollary}
        If $\deg{(m_T(x))} = \dim_F(V)$ and $m_T(x)$ has distinct roots, all of which are in $F$, then we can find a basis $\cB$ so that $[T]_\cB$ is diagonal.
    \end{corollary}

    \begin{example}
        Let $A = \pmat 1 & 0 & 0 \\ 0 & 2 & 0 \\ 0 & 0 & 2 \epmat$ and $B = \pmat 1 & 0 & 0 \\ 0 & 1 & 0 \\ 0 & 0 & 2 \epmat$. These matrices are not similar, however $m_A(x) = m_B(x) = (x-1)(x-2)$. The minimal polynomial is not enough information on the similarity of matrices.
    \end{example}

    \begin{example}\label{example:gen-eigen-vec}
        Let:
            \begin{equation*}
            \begin{split}
                A = \bmat 1 & 2 & 3 \\ 0 & 1 & 4 \\ 0 & 0 & -1 \emat.
            \end{split}
            \end{equation*}
        We have that $m_A(x) = (x-1)^2(x+1)$. Note that $Ae_1 = e_1$, so $E_1^1 \supseteq \Span_F{(e_1)}$ (or, more simply, $e_1 \in E_1^1$). Note that $Ae_2 = \pmat 2 \\ 1 \\ 0 \epmat$. So $e_2 \not\in E_1^1$ (another way of saying this is $(A-1_3)e_2 \neq \pmat 0 \\ 0 \\ 0 \epmat$). But now consider:
            \begin{equation*}
            \begin{split}
                 (A - 1_3)^2 = \bmat 0 & 0 & 2 \\ 0 & 0 & -8 \\ 0 & 0 & 4 \emat.
            \end{split}
            \end{equation*}
        We have $(A-1_3)^2 e_2 = \pmat 0 \\ 0\\ 0 \epmat$. Thus $e_1,e_2 \in \ker{(A-\id_{F^3})^2}$.
    \end{example}

    \begin{definition}
        Let $T \in \Hom_F{(V,V)}$. For $k \geq 1$, the \textui{$k^\text{th}$ generalized eigenspace} of $T$ associated to $\lambda$ is $E_\lambda^k = \ker(T - \lambda \id_V)^k = \{v \in V \mid (T-\lambda \id_V)^k v = 0_V \}$. Elements of $E_\lambda^k$ are called \textui{generalized eigenvectors}. Set $E_\lambda^\infty = \bigcup_{k \geq 1} E_\lambda^k$.
    \end{definition}

    \begin{example}
        Continuing Example~\ref{example:gen-eigen-vec}, let $\alpha e_1 + \beta e_2 \in \Span_F{(e_1,e_2)}$. Then:
            \begin{equation*}
            \begin{split}
                (A - 1_3)^2 (\alpha e_1 + \beta e_2)
                & = \alpha (A - 1_3)^2  e_1 + \beta(A - 1_3)^2  e_2
                 = \pmat 0 \\ 0 \\ 0 \epmat.
            \end{split}
            \end{equation*}
        So $\Span_F{(e_1,e_2)} \subseteq E_1^2$. We also have $-1$ as an eigenvalue with eigenvector $v_3 = \pmat \frac{1}{2} \\ -2 \\ 1 \epmat$. Check that $v_3 \not\in E_1^2$. So $\dim_F{(E_1^2)} \leq 2$; i.e., $E_1^2 = \Span_F{(e_1,e_2)}$. {\color{red} why does $v_3 \not\in E_1^2$ imply the dimension which implies containment in the other direction}.
    \end{example}
    
    \begin{lemma}\label{lemma:eventual-kernel}
        Let $V$ be a finite dimensional $F$-vector space, $\dim_F{(V)} = n$, and $T \in \Hom_F{(V,V)}$. There exists $m$ with $1 \leq m \leq n$ such that $\ker(T^,) = \ker(T^{m+1})$. Moreover, for such an $m$, $\ker(T^m) = \ker(T^{m+j})$ for all $j \geq 0$.
    \end{lemma}
        \begin{proof}
            We have $\ker(T^1) \subseteq \ker(T^2) \subseteq...$ If these containments are always strict, then the dimension increases indefinitely, which contradicts $\dim_F(V) = n$. Hence we have an $m$ with $1 \leq m \leq n$ and $\ker(T^m) = \ker(T^{m+1})$.

            Let $m$ be the smallest value where $\ker(T^m) = \ker(T^{m+1})$. We use induction on $j$. Base case of $j=1$ is what defines $m$. Assume $\ker(T^m) = \ker(T^{m+j})$ for all $1 \leq j \leq N$. Let $v \in \ker(T^{m+N+1})$. This gives:
                \begin{equation*}
                \begin{split}
                    0_V 
                    & = T^{m+N+1}(v) \\
                    & = T^{m+1}(T^N(v)).
                \end{split}
                \end{equation*}
            So $T^N(v) \in \ker(T^{m+1})$. However $\ker(T^{m+1}) = \ker(T^{m})$, so $T^N(v) \in \ker(T^{m})$. Hence:
                \begin{equation*}
                \begin{split}
                    0_V 
                    & = T^m(T^n(v)) \\
                    & = T^{m+N}(v), 
                \end{split}
                \end{equation*}
            so $v \in \ker(T^{m+N})$. Induction hypothesis gives $\ker(T^{m+N}) = \ker(T^m)$, giving $v \in \ker{(T^m)}$. Thus $\ker(T^{m+N+1}) \subseteq \ker(T^m)$. The other direction of containment is trivial.
        \end{proof}

    \begin{example}
        Let $\cB = \{v_1,...,v_n\}$ be a basis of $V$ and $T \in \Hom_F{(V,V)}$, $\lambda \in F$ such that:
            \begin{equation*}
            \begin{split}
                [T]_\cB = 
                \bmat
                \lambda & 1 & 0 & ... & 0 \\
                0 & \lambda  & 1 & ... & 0 \\
                0 & 0 & \lambda & ... & 0 \\
                \vdots & \vdots & \vdots & \ddots & 1 \\
                0 & 0 & 0 & 0 & \lambda
                \emat.
            \end{split}
            \end{equation*}
        In other words, $[T]_\cB$ contains $\lambda$ along the diagonal and $1$ along the super-diagonal. Let $A = [T]_\cB$. Consider:
            \begin{equation*}
            \begin{split}
                (A-\lambda 1_n) =
                \bmat
                0 & 1 & 0 & ... & 0 \\
                0 & 0  & 1 & ... & 0 \\
                0 & 0 & 0 & ... & 0 \\
                \vdots & \vdots & \vdots & \ddots & 1 \\
                0 & 0 & 0 & 0 & 0
                \emat.
            \end{split}
            \end{equation*}
        We get:
            \begin{equation*}
            \begin{split}
                (A - \lambda 1_n)v_1 &= 0_V \\
                (A - \lambda 1_n)v_2 &= v_1 \\
                &\vdots \\
                (A - \lambda 1_n)v_n &= v_{n-1}.
            \end{split}
            \end{equation*}
        This gives $E_\lambda^1 = \Span_F{(v_1)}$ (by the first equation). Now observe:
            \begin{equation*}
            \begin{split}
                (A - \lambda 1_n)^2 v_1 &= 0_V \\
                \phantom{a} \\
                (A - \lambda 1_n)^2 v_2 &= (A - \lambda 1_n)(A - \lambda 1_n)v_2 \\
                & = (A - \lambda 1_n)v_1 \\
                & = 0_V \\
                \phantom{a} \\
                (A - \lambda 1_n)^2 v_3 &= v_1 \\
                &\vdots \\
                (A - \lambda 1_n)^2 v_n &= v_{n-2}. 
            \end{split}
            \end{equation*}
        So $E_\lambda^2 = \Span_F{(v_1,v_2)}$. In general, we have that $E_\lambda^k = \Span_F{(v_1,...,v_k)}$. Moreover,  Lemma~\ref{lemma:eventual-kernel} gives $E_\lambda^1 \subseteq E_\lambda^2 \subseteq ... \subseteq E_\lambda^k$.
    \end{example}

    \begin{corollary}
        If $\dim_F(V) = n$ and $T \in \Hom_F{(V,V)}$, there exists an $m$ with $1 \leq m \leq n$ so that for any $\lambda \in F$, $E_\lambda^\infty = E_\lambda^m$.
    \end{corollary}

    \begin{theorem}
        Let $T \in \Hom_F{(V,V)}$, and $\lambda \in F$ with $(x-\lambda)^k \mid m_T(x)$. We have:
            \begin{equation*}
            \begin{split}
                \dim_F(E_\lambda^k) \geq k.
            \end{split}
            \end{equation*}
    \end{theorem}
        \begin{proof}
            Write $m_T(x) = (x-\lambda)^k f(x)$ where $f(x) \in F[x]$, $f(\lambda) \neq 0$. Define $g_k(x) = (x-\lambda)^k$. We have that $(x-\lambda)^{k-1}f(x) = g_{k-1}(x)f(x)$ is \textit{not} the minimal polynomial. So there is a $v \in V$ with $v \neq 0_V$ such that:
                \begin{equation*}
                \begin{split}
                    g_{k-1}(T)f(T)(v) \neq 0_V.
                \end{split}
                \end{equation*}
            Set $v_k = f(T)(v)$. Observe that:
                \begin{equation*}
                \begin{split}
                    \left(T - \lambda \id_V\right)^k (v_k)
                    & = (T-\lambda \id_V)^k f(T)(v)\\
                    & = m_T(T)(v) \\
                    & = 0_V.
                \end{split}
                \end{equation*}
            So $v_k \in E_\lambda^k$. Moreover, by our construction:
                \begin{equation*}
                \begin{split}
                    (T-\lambda \id_V)^{k-1}(v_k)
                    & = g_{k-1}(T)(v_k) \\
                    & = g_{k-1}(T)f(T)(v) \\
                    & \neq 0_V.
                \end{split}
                \end{equation*}
            Hence $v_k \in E_\lambda^{k} \setminus E_\lambda^{k-1}$. Now set $v_{k-1} = (T - \lambda \id_V)v_k = (T - \lambda \id_V)f(T)(v)$. Note:
                \begin{equation*}
                \begin{split}
                    (T - \lambda \id_V)^{k-1}(v_{k-1}) 
                    & = (T - \lambda \id_V)^{k-1} (T - \lambda \id_V)(v_k) \\
                    & = (T - \lambda \id_V)^k (v_k)\\
                    & = (T - \lambda \id_V)^k f(T)(v) \\
                    & = m_T(T)(v) \\
                    & = 0_V.
                \end{split}
                \end{equation*}
            So $v_{k-1} \in E_\lambda^{k-1}$. Again, by our construction:
                \begin{equation*}
                \begin{split}
                    (T - \lambda \id_V)^{k-2}(v_{k-1})
                    & = (T - \lambda \id_V)^{k-2}(T - \lambda \id_V)(v_k) \\
                    & =(T - \lambda \id_V)^{k-1}(v_k) \\
                    & \neq 0_V.
                \end{split}
                \end{equation*}
            So $v_{k-1} \in E_\lambda^{k-1} \setminus E_\lambda^{k-2}$. Setting $v_{k-2} = (T - \lambda \id_V)^2v_k$ gives a similar result. By this construction, we obtain a set $\{v_k,v_{k-1},...,v_2,v_1\}$. Claim: this set is linearly independent. Suppose towards contradiction it's not, that is, $a_1v_1 + ... + a_k v_k = 0_V$ does not imply $a_1 = ... = a_k = 0$. This gives $v_k = \frac{-1}{a_k}(a_1v_1 + ... + a_{k-1}v_{k-1}) \in E_\lambda^{k-1}$, which is a contradiction. It follows that $a_1 = ... = a_k = 0$, hence $\{v_k,v_{k-1},...,v_2,v_1\}$ is linearly independent (linear independent set $\subseteq$ a basis, so thats why the theorem is established).
        \end{proof}

    \begin{example}
        Let $T_A \in \Hom_F{(F^3,F^3)}$ be defined by:
            \begin{equation*}
            \begin{split}
                A = \bmat 2 & 1 & 3  \\ 0 & 2 & 4 \\ 0 & 0 & 2 \emat.
            \end{split}
            \end{equation*}
        We have that $m_T(x) = (x-2)^3$. Now observe:
            \begin{equation*}
            \begin{split}
                (A - 2\cdot 1_3)^2 = \bmat 0 & 0 & 4 \\ 0 & 0 & 0 \\ 0 & 0 & 0 \emat.
            \end{split}
            \end{equation*}
        Note $(A - 2\cdot 1_3)^2 e_3 = 4e_3 \neq 0_F^3$, but $(A - 2 \cdot 1_3)^3 e_3 = 0_{F^3}$. Set $v_3 = e_3$, we have $v_3 \in E_2^3$. Now observe:
            \begin{equation*}
            \begin{split}
                v_2 &= (A - 2 \cdot 1_3)(v_3) \\
                & = \bmat 0 & 1 & 3 \\ 0 & 0 & 4 \\ 0 & 0 & 0 \emat \bmat 0 \\ 0 \\ 1 \emat \\
                & = \bmat 3 \\ 4\\ 0 \emat.
            \end{split}
            \end{equation*}
        Similarly:
            \begin{equation*}
            \begin{split}
                v_1 &= (A - 2 \cdot 1_3)(v_2) \\
                & = ... \\
                & = \bmat 4 \\ 0\\ 0 \emat.
            \end{split}
            \end{equation*}
        Hence:
            \begin{equation*}
            \begin{split}
                E_2^3 &= \Span_F{(\pmat 0 \\ 0 \\ 1\epmat ,\pmat 3 \\ 4 \\ 0\epmat , \pmat 4 \\ 0 \\ 0\epmat)} \\
                E_2^2 & = \Span_F{(\pmat 3 \\ 4 \\ 0\epmat , \pmat 4 \\ 0 \\ 0\epmat)} \\
                E_2^1 & = \Span_F{(\pmat 4 \\ 0 \\ 0\epmat)}.
            \end{split}
            \end{equation*}
        Setting $\cB = \{v_1,v_2,v_3\}$, we have:
            \begin{equation*}
            \begin{split}
                \left[T_A\right]_\cB = \bmat 2& 1 & 0 \\ 0 & 2 & 1 \\ 0 & 0 & 2 \emat.
            \end{split}
            \end{equation*}
    \end{example}

    \iffalse
    \begin{definition}
        Let $A$ be a defective matrix, that is to say, it contains repeated eigenvalues. Then $A$ is similar to a matrix $J$, where $J$ is of the form:
            \begin{equation*}
            \begin{split}
                J = 
                \bmat
                \lambda_1 & 1 & \\
                 & \lambda_1 & 1 \\
                 & & \lambda_1 \\
                 & & & \lambda_2 & 1 \\
                 & & & & \lambda_2 \\
                 & & & & & \lambda_3 \\
                 & & & & & & \ddots \\
                 & & & & & & & \lambda_n & 1\\
                 & & & & & & & & \lambda_n  \\ 
                \emat.
                \begin{tikzpicture}[overlay, remember picture]
                    \draw[gray, thick] (-1.7,-3.35) rectangle (-0.4,-2); % Adjust the coordinates for placement
                \end{tikzpicture}
                \begin{tikzpicture}[overlay, remember picture]
                    \draw[gray, thick] (-3.18,-1.04) rectangle (-2.6,-0.5); % Adjust the coordinates for placement
                \end{tikzpicture}
                \begin{tikzpicture}[overlay, remember picture]
                    \draw[gray, thick] (-4.64,-0.3) rectangle (-3.4,1.1); % Adjust the coordinates for placement
                \end{tikzpicture}
                \begin{tikzpicture}[overlay, remember picture]
                    \draw[gray, thick] (-6.8,1.3) rectangle (-4.8,3.5); % Adjust the coordinates for placement
                \end{tikzpicture}
            \end{split}
            \end{equation*}
        We say $J$ is the \textui{Jordan normal form} of $A$. The outlined squares are known as \textui{Jordan blocks}.
    \end{definition}
    \fi

\section{Characteristic Polynomials}
    \begin{definition}
        Let $A \in \Mat_n(F)$. The \textui{characteristic polynomial} is $c_A(x) = \det(x 1_n - A)$.
    \end{definition}

    \begin{definition}
        Let $f(x) = x^n + a_{n-1}x^{n-1} + ... + a_1 x + a_0 \in F[x]$. The \textui{companion matrix} of $f(x)$ is given by:
            \begin{equation*}
            \begin{split}
                C(f(x)) = 
                \bmat
                -a_0 & 0 & 0 & ... & 0\\
                -a_1 & 1 & 0 & ... & 0\\
                -a_2 & 0 & 1 & ... & 0\\
                \vdots & \vdots & \vdots & \ddots & \vdots \\
                -a_{n-1} & 0 & 0 & ... & 1
                \emat
            \end{split}
            \end{equation*}
        The companion matrix shows that any polynomial $f(x) \in F[x]$ can be realized as the characteristic polynomial of a matrix.
    \end{definition}

    \begin{lemma}
        If $A = C(f(x))$, then $c_A(x) = f(x)$.
    \end{lemma}

    \begin{lemma}
        Let $A,B \in \Mat_n(F)$ be similar matrices. Then $c_A(x) = c_B(x)$.
    \end{lemma}
        \begin{proof}
            Let $A = PBP^{-1}$ for some $P \in GL_n(F)$. We have:
                \begin{equation*}
                \begin{split}
                    c_A(x)
                    & = \det(x1_n - A) \\
                    & = \det(x1_n - PBP^{-1}) \\
                    & = \det(P(x1_n)P^{-1} - PBP^{-1}) \\
                    & = \det(P(x1_n - B)P^{-1}) \\
                    & = \det(P)\det(x1_n - B)\det(P^{-1}) \\
                    & = \det(x1_n - B) \\
                    & = c_B(x).
                \end{split}
                \end{equation*}
        \end{proof}

    \begin{definition}
        For $T \in \Hom_F(V,V)$, let $\cB$ be a basis of $V$ and set $c_T(x) = c_{\left[T\right]_\cB}(x)$.
    \end{definition}

    \begin{theorem}
        Let $v \in V$, $v\neq 0_V$. Let $\dim_F(V) = n$. Then there is a unique monic polynomial $m_{T,v}(x) \in F[x]$ so that $m_{T,v}(T)(v) = 0_V$. Moreover, if $f(x) \in F[x]$ with $f(T)(v) = 0_V$, then $m_{T,v}(x) \mid f(x)$.
    \end{theorem}
        \begin{proof}
            Consider the set $\{v,T(v),T^2(v),...,T^n(v)\}$. Since this set contains $n+1$ elements and the dimension of $V$ is $n$, the set must be linearly dependent. Write:
                \begin{equation*}
                \begin{split}
                    a_mT^m(v) + ... + a_1T(v) + a_0 = 0_V
                \end{split}
                \end{equation*}
            for some $m \leq n$ of minimal order and $a_i \neq 0$ for all $i$. Set:
                \begin{equation*}
                \begin{split}
                    p(x) = x^m + \frac{a_{m-1}}{a_m}x^{m-1} + ... + \frac{a_{1}}{a_m}x + \frac{a_{0}}{a_m} \in F[x].
                \end{split}
                \end{equation*}
            By construction $p(T)(v) = 0_V$. Set $I_v = \{g(x) \in F[x] \mid g(T)(v) = 0_V \}$. We have that $p(x)$ is a monic nonzero polynomial in $I_v$ of minimal degree. Set $m_{T,v}(x) = p(x)$.

            Let $f(x) \in I_v$. We'd like to show that $m_{T,v}(x) \mid f(x)$. Write:
                \begin{equation*}
                \begin{split}
                    f(x) = q(x)m_{T,v}(x) + r(x),
                \end{split}
                \end{equation*}
            with $q(x),r(x) \in F[x]$ and $\deg(r(x)) = 0$ or $deg(r) < \deg(m_{T,v}(x))$. Observe that:
                \begin{equation*}
                \begin{split}
                    r(T)(v) &= f(T)(v) - q(T)m_{T,v}(T)(v) \\
                    & = 0_V - q(T)0_V \\
                    & = 0_V.
                \end{split}
                \end{equation*}
            So $r(x) \in I_v$. But $m_{T,v}(x)$ had minimal degree, so it must be the case that $r(x) = 0$. Thus $f(x) = q(x)m_{T,v}(x)$, implying $m_{T,v}(x) \mid f(x)$\footnote{The proof of $F[x]$ being a P.I.D. follows identically. Instead of considering $I_v$ we would consider an arbitrary polynomial in $F[x]$.}. Now suppose $h(x) \in I_v$ with $\deg(h(x)) = \deg(m_{T,v}(x))$. Since both polynomials are monic and of equal degree, if $m_{T,v}(x) \mid h(x)$ then $m_{T,v}(x) = h(x)$.
        \end{proof}

    \begin{definition}
        We refer to $m_{T,v}(x)$ as the \textui{$T$-annihilator} of $v$.
    \end{definition}

    \begin{example}
        Let $V = F^n$ and $\cE_n = \{e_1,...,e_n\}$. Define $T \in \Hom_F{(V,V)}$ by:
            \begin{equation*}
            \begin{split}
                T(e_1) &= 0_v \\ 
                T(e_j) &= e_{j-1} \mtext{for} 2 \leq j \leq n.
            \end{split}
            \end{equation*}
        Consider $f(x) = x$. Then $f(T)(e_1) = T(e_1) = 0_V$. Hence $m_{T,e_1}(x) \mid x$. So either $m_{T,e_1}(x) = 1$ or $m_{T,e_1}(x) = x$. But $\id_V(e_1) = e_1 \neq 0_V$, hence it must be the case that $m_{T,e_1}(x) = x$.

        Now consider $g(x) = x^2$. Then $g(T)(e_2) = T^2(e_2) = T(T(e_2)) = T(e_1) = 0_V$. Hence $m_{T,e_2}(x) \mid x^2$. So $m_{T,e_2}(x)= 1$ or $x$ or $x^2$. If $m_{T,e_2}(x) = 1$, then $\id_V(e_2) = e_2 \neq 0_V$. If $m_{T,e_2}(x) = x$, then $T(e_2) = e_1 \neq 0$. So $m_{T,e_2}(x) = x^2$. It follows for $i \leq j \leq n$, $m_{T,e_j}(x)=x^j$.
    \end{example}

    \begin{example}
        Let $V = \bfQ^2$. Define $T \in \Hom_\bfQ{(\bfQ^2,\bfQ^2)}$ by:
            \begin{equation*}
            \begin{split}
                T(e_1) &= e_1 + 3e_2 \\
                T(e_2) &= 2e_1 + 4e_2.
            \end{split}
            \end{equation*}
        We are trying to find $m_{T,e_1}(x)$. Since $V$ is two-dimensional, $\deg(m_{T,e_1}(x)) = 1$ or 2. Write $m_{T,e_1}(x) = x +a$. Then:
            \begin{equation*}
            \begin{split}
                m_{T,e_1}(T)(e_1)
                & = T(e_1) + ae_1 \\
                & = e_1 + 3e_2 + ae_1 \\
                & \neq 0_V.
            \end{split}
            \end{equation*}
        So it must be that $\deg(m_{T,e_1}(x)) = 2$. Note that:
            \begin{equation*}
            \begin{split}
                T^2(e_1)
                & = T(e_1 + 3e_2) \\
                & = T(e_1) + 3T(e_2) \\
                & = 7e_1 + 15e_2.
            \end{split}
            \end{equation*}
        Now let:
            \begin{equation*}
            \begin{split}
                T^2(e_1) + bT(e_1) + ce_1 = 0_V,
            \end{split}
            \end{equation*}
        for some $b,c \in \bfQ$. This will yield a system of equations, and solving for it gives:
            \begin{equation*}
            \begin{split}
                b & = -5 \\
                c & = -2.
            \end{split}
            \end{equation*}
        Hence $m_{T,e_1}(x) = x^2 - 5x - 2$.
    \end{example}

    \begin{exercise}
        \phantom{a}
        \begin{enumerate}
            \item Show $m_{T,e_2}(x) = x^2 - 5x -2$.
            \item Calculate $m_{T,e_1}(x)$ and $m_{T,e_2}(x)$ of $F = \bfF_3$.
        \end{enumerate}
    \end{exercise}

    \begin{theorem}
        Let $\dim_F{(V)} = n$ and $\cB = \{v_1,...,v_n\}$ be a basis of $V$. Let $T \in \Hom_F{(V,V)}$. We have:
            \begin{equation*}
            \begin{split}
                m_T(x) = \underset{1 \leq i \leq n}\lcm m_{T,v_i}(x).
            \end{split}
            \end{equation*}
    \end{theorem}
        \begin{proof}
            Let $f(x) = \underset{1 \leq i \leq n}\lcm m_{T,v_i}(x)$. Note that $m_T(T)(v_i) = 0_V$, so $m_{T,v_i}(x) \mid m_T(x)$ for each $i$. Hence $f(x) \mid m_T(x)$.

            Now let $v \in V$. Write $v = \sum_{i = 1}^n a_i v_i$. We have:
                \begin{equation*}
                \begin{split}
                    f(T)(v)
                    & = f(T)(\sum_{i = 1}^n a_i v_i) \\
                    & = \sum_{i = 1}^n a_i f(T)(v_i) \\
                    & = 0_V,
                \end{split}
                \end{equation*}
            because $m_{T,v_i}(x) \mid f(x)$ for all $i$. Hence $m_T(x)\mid f(x)$. {\color{red} i dont quite get this number theory stuff}
        \end{proof}
    
    \begin{lemma}
        Let $T \in \Hom_F{(V,V)}$. Let $v_1,...,v_k \in V$, and set $p_i(x) = m_{T,v_i}(x)$. Suppose $p_i(x)$ are pairwise relatively prime. Set $v = v_1+...+v_k$. Then:
            \begin{equation*}
            \begin{split}
                m_{T,v}(x) = p_1(x)...p_k(x).
            \end{split}
            \end{equation*}
    \end{lemma}
        \begin{proof}
            We prove this for $k \geq 2$; i.e., $m_{T,v_1 + v_2}(x) = m_{T,v_1}(x)m_{T,v_2}(x)$. Since $p_1(x)$ and $p_2(x)$ are relatively prime, there exists $q_1(x),q_2(x) \in F[x]$ so that $1 = p_1(x)q_1(x) + p_2(x)q_2(x)$. In particular, $\id_V = p_1(T)q_1(T) + p_2(T)q_2(T)$. Set $v = v_1 + v_2$. We have:
                \begin{equation*}
                \begin{split}
                    v
                    & = \id_V(v) \\
                    & = (p_1(T)q_1(T) + p_2(T)q_2(T))(v) \\
                    & = p_1(T)q_1(T)(v) + p_2(T)q_2(T)(v) \\
                    & = p_1(T)q_1(T)(v_1 + v_2) + p_2(T)q_2(T)(v_1 + v_2) \\
                    & = p_1(T)q_1(T)(v_2) + p_2(T)q_2(T)(v_2).
                \end{split}
                \end{equation*}
            Write $w_1 = p_1(T)q_1(T)(v_2)$ and $w_2 = p_2(T)q_2(T)(v_1)$. This means $v = w_1 + w_2$. Note:
                \begin{equation*}
                \begin{split}
                    p_1(T)(w_1)
                    & = p_1(T)p_2(T)q_2(T)(v_1) \\
                    & = q_2(T)p_2(T)\underbrace{p_1(T)(v_1)} \\
                    & = 0_V.
                \end{split}
                \end{equation*}
            Hence $w_1 \in \ker(p_1(T))$. It follows similarly that $w_1 \in \ker(p_2(T))$. Let $r(x) \in F[x]$ with $r(T)(v) = 0_V$. We have $v = w_1 + w_2$ and $w_2 \in \ker(p_2(T))$, so:
                \begin{equation*}
                \begin{split}
                    p_2(T)(v)
                    &  = p_2(T)(w_1 + w_2) \\
                    & = p_2(T)(w_1).
                \end{split}
                \end{equation*}
            Thus:
                \begin{equation*}
                \begin{split}
                    0_V 
                    & = p_2(T)q_2(T)(0_V) \\
                    & = p_2(T)q_2(T)r(T)(v) \\
                    & = r(T)p_2(T)q_2(T)(v) \\
                    & = r(T)p_2(T)q_2(T)(w_1).
                \end{split}
                \end{equation*}
            We also know $r(T)q_1(T)p_1(T)(w_1) = 0_V$ because $w_1 \in \ker(p_1(T))$. Hence:
                \begin{equation*}
                \begin{split}
                    0_V 
                    & = r(T)p_2(T)q_2(T)(w_1) + r(T)p_1(T)q_1(T)(w_1) \\
                    & = r(T)\underbrace{\left(p_2(T)q_2(T) + p_1(T)q_1(T)\right)}_{\id_V} (w_1) \\
                    & = r(T)(w_1).
                \end{split}
                \end{equation*}
            This gives:
                \begin{equation*}
                \begin{split}
                    0_V 
                    & = r(T)(w_1) \\
                    & = r(T)p_2(T)q_2(T)(v_1).
                \end{split}
                \end{equation*}
            So $r(T)p_2(T)q_2(T)(v_1) = 0_V$. Thus $p_1(x)\mid r(x)p_2(x)q_2(x)$. Now note that:
                \begin{equation*}
                \begin{split}
                    \gcd{(p_1(x),p_2(x)q_2(x))} = 1,
                \end{split}
                \end{equation*}
            which means $p_1(x)\mid r(x)$. A similar argument shows $p_2(x) \mid r(x)$. And since $\gcd(p_1(x),p_2(x)) = 1$, this gives $\lcm(p_1(x),p_2(x)) = p_1(x)p_2(x)$. So $p_1(x)p_2(x) \mid r(x)$. Since $r(x)$ was arbitrary, take $r(x) = m_{T,v}(x)$. Then $p_1(x)p_2(x) \mid m_{T,v}(x)$. Finally, since $p_1(x)p_2(x)(v) = 0_V$, $m_{T,v}(x) \mid p_1(x)p_2(x)$, establishing the lemma.
        \end{proof}

    \begin{exercise}
        Show inductively that $m_{T,v} = p_1(x)p_2(x)...p_k(x)$\footnote{Pairwise coprime is a stronger statement than just coprime. It means that $\gcd(p_i,p_j) = 1$ for all $1\leq i,j \leq k$}.
    \end{exercise}

    \begin{theorem}
        Let $T \in \Hom_F(V,V)$. There exists $v \in V$ such that $m_{T,v}(x) = m_T(x)$. In particular, $\deg(m_T(x)) \leq n$.
    \end{theorem}
        \begin{proof}
            Let $\cB = \{v_1,...,v_n\}$ be a basis. We know:
            \begin{equation*}
                \begin{split}
                    m_T(x) = \underset{1 \leq i \leq n}\lcm m_{T,v_i}(x).
                \end{split}
                \end{equation*}
            Factor $m_T(x) = p_1(x)^{e_1}...p_k(x)^{e_k}$, with each $p_i(x)$ relatively prime and $e_1 \geq 1$. For $1 \leq j \leq k$, there exists $i_j \in \{1,...,n\}$ and $q_{i_j}(x) \in F[x]$ with:
                \begin{equation*}
                \begin{split}
                    m_{T,v_{i_j}}(x) = p_j(x)^{e_j}q_{i_j}(x).
                \end{split}
                \end{equation*}
            Set $w_j = q_{i_j}(T)(v_{i_j})$. This gives:
                \begin{equation*}
                \begin{split}
                    m_{T,w_j}(x) = p_j(x)^{e_j}.
                \end{split}
                \end{equation*}
            Now set $w = w_1 + ... + w_k$. The previous result gives $m_{T,w}(x) = p_1(x)^{e_1}...p_k(x)^{e_k} = m_T(x)${\color{red} ????}.
        \end{proof}
    
    \begin{lemma}
        Let $W \subseteq V$ be a $T$-invariant subspace. Then there is an induced map $\overline{T} \in \Hom_F(V/W,V/W)$ defined by $\overline{T}(v+W) = T(v)+W$.
    \end{lemma}

    \begin{lemma}
        Let $v \in V$. Then $m_{\overline{T},[v]}(x) \mid m_{T,v}(x)$. Similarly, $m_{\overline{T}}(x) \mid m_T(x)$.
    \end{lemma}
        \begin{proof}
            We have:
                \begin{equation*}
                \begin{split}
                    m_{T,v}(\overline{T})([v])
                    & = m_{T,v}(\overline{T})(v+W) \\
                    & = m_{T,v}(T)(v) + W \\
                    & = 0_V + W \\
                    & = 0_{V/W}.
                \end{split}
                \end{equation*}
            Then by definition of (in this case, $m_{\overline{T},[v]}(x)$) annihilator polynomials, $m_{\overline{T},[v]}(x) \mid m_{T,v}(x)$.
        \end{proof}

    \begin{definition}
        Let $T \in \Hom_F(V,V)$ and $\cA = \{v_1,...,v_k\}$ a set of vectors in $V$. The \textui{$T$-span} of $\cA$ is the subspace:
            \begin{equation*}
            \begin{split}
                W = \left\{\sum_{i=1}^k p_i(T)(v_i) \mid v_i \in \cA, \hspace{3pt}p_i(x) \in F[x]\right\}.
            \end{split}
            \end{equation*}
        We say the subset $W$ is \textui{$T$-generated} by $\cA$.
    \end{definition}

    \begin{exercise}\label{exercise:t-span}
        Show $W$ is a $T$-invariant subspace of $V$. Moreover, show it is the smallest $T$-invariant subspace with respect to inclusion of $V$ that contains $\cA$.
    \end{exercise}

    \begin{example}
        Let $V = \bfQ^4$. Define $T \in \Hom_\bfQ{(\bfQ^4,\bfQ^4)}$ by:
            \begin{equation*}
            \begin{split}
                T(e_1) &= 2e_1 + 3e_3 \\
                T(e_2) &= e_1 + e_2 \\
                T(e_3) &= e_1 - e_3 \\
                T(e_4) &= 2e_1 - 2e_2 + 5e_3 - 4e_4.\\
            \end{split}
            \end{equation*}
        Let $\cA = \{e_1\}$. Our goal is to find $T\text{-}\Span_{\bfQ}(\cA)$. Set $p(x) = 1$, then $p(T)(e_1) = \id_V(e_1) =e_1$. Hence $e_1 \in T\text{-}\Span_{\bfQ}(\cA)$. Now set $q(x) = \frac{1}{3}(x-2)$. Then:
            \begin{equation*}
            \begin{split}
                q(T)(e_1)
                & = \frac{1}{3}(T-2\id_V)(e_1)\\
                & = \frac{1}{3}(T(e_1) - 2e_1) \\
                & = e_3.
            \end{split}
            \end{equation*}
        Hence $e_3 \in T\text{-}\Span_{\bfQ}(\cA)$. So $\Span_\bfQ{(e_1,e_3)} \subseteq T\text{-}\Span_{\bfQ}(\cA)$ (basically $\alpha p(x) + \beta q(x) \in \Span_T,F(\cA)$, so plugging in a linear combination of $e_1$ and $e_3$ will give you back a linear combination of $e_1$ and $e_3$). Note that $\Span_F(e_1,e_3)$ is $T$-invariant. By Exercise~\ref{exercise:t-span}, since $T\text{-}\Span_{\bfQ}(\cA)$ is the smallest $T$-invariant subspace by inclusion, it must be the case that $T\text{-}\Span_{\bfQ}(\cA) \subseteq \Span_F{(e_1,e_3)}$. Hence $T\text{-}\Span_{\bfQ}(\cA) = \Span_F{(e_1,e_3)}$.
    \end{example}

    \begin{lemma}\label{lemma:4.3.9}
        Let $T \in \Hom_F{(V,V)}$, $w \in V$, and $W$ the subspace of $V$ that is $T$-generated by $\{w\}$. Then $\dim_F{(W)} = \deg(m_{T,w}(x))$.
    \end{lemma}
        \begin{proof}
            Let $\deg(m_{T,w}(x)) = k$. Consider the set $\{w,T(w),...,T^{k-1}(w)\}$. This is a basis of the $T$-span of $\{w\}$.
        \end{proof}

    \begin{theorem}
        Let $\dim_F(V) = n$.
        \begin{enumerate}[label = (\arabic*)]
            \item We have $m_T(x) \mid c_T(x)$.
            \item Every irreducible factor of $c_T(x)$ is a factor of $m_T(x)$.
        \end{enumerate}
    \end{theorem}
        \begin{proof}
            (1) Let $\deg(m_T(x)) = k \leq n$. Let $v \in V$ with $m_T(x) = m_{T,v}(x)$. Let $W_1$ be the $T$-span of $\{v\}$. By Lemma~\ref{lemma:4.3.9}, $\dim_F(W_1) = k$. Write:
                \begin{equation*}
                \begin{split}
                    v &= v_k \\
                    T(v) & = v_{k-1} \\
                    T^2(v) & = v_{k-2} \\
                    &\vdots \\
                    T^i(v) & = v_{k-i}.
                \end{split}
                \end{equation*}
            We have $\cB_1 = \{v_1,...,v_k\}$ is a basis of $W_1$ (see proof of previous lemma). Since:
                \begin{equation*}
                \begin{split}
                    0_V &= m_T(T)(v) \\
                    & = T^k(v) + a_{k-1}T^{k-1}(v) + ... + a_1T(v) = a_0 v,
                \end{split}
                \end{equation*}
            we have that $T^k(v) = -a_{k-1}T^{k-1}(v) - ... - a_1T(v) - a_0v$. Thus:
                \begin{equation*}
                \begin{split}
                    T(v_1) &= T(T^{k-1}(v)) = T^k(v) = -a_{k-1}T^{k-1}(v) - ... - a_1T(v) - a_0v.\\
                    T(v_2) & = T(T^{k-2}(v)) = T^{k-1}(v) \\
                    T(v_3) &= T(T^{k-3}(v)) = T^{k-2}(v) \\
                    &\vdots 
                \end{split}
                \end{equation*}
            So $\left[\restr{T}{W_1}\right]_{\cB_1} = C(m_T(x))$. We proceed with cases:
            
            Case 1: $k = n$. Then $W_1 = V$, and $[T]_{\cB_1} = C(m_T(x))$, which has characteristic polynomial $m_T(x)$, meaning $m_T(x) = c_T(x)$.

            Case 2: $k < n$. Expand $\cB_1$ to a full basis of $V$ as follows: Let $\cB_2 = \{v_{k+1},...,v_n\}$ and write:
                \begin{equation*}
                \begin{split}
                    \cB = \cB_1 \cup \cB_2.
                \end{split}
                \end{equation*}
            Since $W_1$ is $T$-invariant,  by Theorem~\ref{thm:stable-block-diagonal} $[T]_\cB$ will be block diagonal. So we have:
                \begin{equation*}
                \begin{split}
                    \left[T\right]_\cB = \bmat A & B \\ 0 & D \emat, \hspace{4pt} A = \left[\restr{T}{W_1}\right]_\cB = C(m_T(x)).
                \end{split}
                \end{equation*}
            Hence:
                \begin{equation*}
                \begin{split}
                    c_T(x)
                    & = \det(x1_n - \left[T\right]_\cB) \\
                    & = \det \bmat x1_k - A & -B \\ 0 & x1_{n-k}-D \emat \\
                    & = \det(x1_k - A)\det(x1_{n-k})-D \\
                    & = c_A(x)\det(x1_{n-k}-D) \\
                    & = m_T(x)\det(x1_{n-k}-D).
                \end{split}
                \end{equation*}
            Thus $m_T(x) \mid c_T(x)$.

            For (2), we induct on $\dim_F(V) = n$. If $n = 1$, then both the characteristic polynomial and minimal polynomial are monic and of degree $1$, hence they are equal. If $\deg{(m_T(x))} = n$, then $m_T(x) \mid c_T(x)$. Since both are degree $n$ and monic, they must be equal. Now suppose $\deg{m_T(x)} = k < n$ {\color{red}  The rest of this proof is hard}.
        \end{proof}

    \begin{example}
        Consider:
            \begin{equation*}
            \begin{split}
                A = 
                \bmat
                1& 2 \\
                3 & 4 \\
                & & 3 & 7 \\
                & & -1 & 2 \\
                & & & & -5 & 6 \\
                & & & & 2 & -3 \\
                \emat \in \Mat_9(\bfQ).
            \end{split}
            \end{equation*}
        We can verify that $c_A(x) = (x^2-5x-2)(x^2-x-1)(x^2+8x+3)$. Since every irreducible factor of $c_T(x)$ is a factor of $m_T(x)$, we have that $m_T(x) = (x^2-5x-2)(x^2-x-1)(x^2+8x+3)$.
    \end{example}

    \begin{theorem}[Cayley-Hamilton]\label{thm:cayley-hamilton}
        \phantom{a}
        \begin{enumerate}[label = (\arabic*)]
            \item Let $T \in \Hom_F(V,V)$ and $\dim_F(V) < \infty$. Then $c_T(T) = 0_{\Hom_F{(V,V)}}$.
            \item Let $A \in \Mat_n(F)$. Then $c_A(A) = 0_n$.
        \end{enumerate}
    \end{theorem}
        \begin{proof}
            Write $c_T(x) = f(x)m_T(x)$. Then for any $v \in V$:
                \begin{equation*}
                \begin{split}
                    c_T(T)(v)
                    & = f(T)m_T(T)(v) \\
                    & = f(T)(0_V) \\
                    & = 0_V.\qedhere
                \end{split}
                \end{equation*}
        \end{proof}

\section{Jordan Canonical Form}
For this section $V$ is always finite-dimensional.

    \begin{definition}
        Let $T \in \Hom_F(V,V)$. A \textui{Jordan basis} for $V$ with respect to $T$ is a basis $\cB$ so that:
            \begin{equation*}
            \begin{split}
                \left[T\right]_\cB = 
                        \bmat
                        \lambda_1 & 1 & \\
                        & \lambda_1 & 1 \\
                        & & \lambda_1 \\
                        & & & \lambda_2 & 1 \\
                        & & & & \lambda_2 \\
                        & & & & & \lambda_3 \\
                        & & & & & & \ddots \\
                        & & & & & & & \lambda_n & 1\\
                        & & & & & & & & \lambda_n  \\ 
                        \emat.
                        \begin{tikzpicture}[overlay, remember picture]
                            \draw[gray, thick] (-1.7,-3.35) rectangle (-0.4,-2); % Adjust the coordinates for placement
                        \end{tikzpicture}
                        \begin{tikzpicture}[overlay, remember picture]
                            \draw[gray, thick] (-3.27,-1.04) rectangle (-2.8,-0.5); % Adjust the coordinates for placement
                        \end{tikzpicture}
                        \begin{tikzpicture}[overlay, remember picture]
                            \draw[gray, thick] (-4.9,-0.3) rectangle (-3.5,1.1); % Adjust the coordinates for placement
                        \end{tikzpicture}
                        \begin{tikzpicture}[overlay, remember picture]
                            \draw[gray, thick] (-7.14,1.3) rectangle (-5.1,3.5); % Adjust the coordinates for placement
                        \end{tikzpicture}
            \end{split}
            \end{equation*}
        for some $\lambda_1,...,\lambda_n \in F$. More generally, if $V = V_1 \oplus ... \oplus V_k$ is a decomposition into $T$-invariant subspaces, and each $V_i$ has a Jordan basis $\cB_i$, we say $\cB = \bigcup_{i= 1 }^k\cB_i$ is a Jordan basis for $V$.
    \end{definition}

    \begin{definition}
        A matrix of the form:
            \begin{equation*}
            \begin{split}
                J_i = 
                \begin{bmatrix}
                    \lambda_i & 1            &      &   \\
                            & \lambda_i    & \ddots &   \\
                            &            & \ddots & 1   \\
                            &            &      & \lambda_i       
                \end{bmatrix}
            \end{split}
            \end{equation*}
        is called a \textui{Jordan block} associated to $\lambda_i$. We say a matrix $J$ is in \textui{Jordan canonical form} if it is a block diagonal matrix with each block representing a Jordan block.
            \begin{equation*}
            \begin{split}
                J = 
                \begin{bmatrix}
                    J_1 & \;     & \; \\
                    \;  & \ddots & \; \\ 
                    \;  & \;     & J_p
                \end{bmatrix}.
            \end{split}
            \end{equation*}
    \end{definition}



    

    



\chapter{Tensor Products, Exterior Algebras, and Determinants}
\vspace{12pt}

\section{Complexification}
    Recall that if $V$ is a $\bfC$-vector space, then $V$ is also an $\bfR$-vector space by restricting the scalars of $\bfC$. A natural question to ask is if $V$ is an $\bfR$-vector space, can we "extend" $V$ to be a $\bfC$-vector space?

    \begin{example}[Complexification of $\bfR$]
        Let $V = \bfR$. We cannot make $\bfR$ into a $\bfC$-vector space. However, we do have $\bfR \hookrightarrow \bfC$ by $x \mapsto x + 0i$, with $\bfC$ as a $\bfC$-vector space. But note that $z \in \bfC$ can we written as $z = x+yi$. There is an isomorphism $\bfR \oplus \bfR \cong \bfC$ as $\bfR$-vector spaces by:
            \begin{equation*}
            \begin{split}
                x+yi \mapsto (x,y)
            \end{split}
            \end{equation*}
        If we take $z = x+yi \in \bfC$ to be a vector, and $a+bi \in \bfC$ to be a scalar, we have:
            \begin{equation*}
            \begin{split}
                (a+bi)(x+yi) = (ax-by)+(ay+bx)i,
            \end{split}
            \end{equation*}
        meaning in $\bfR \oplus \bfR$ we define:
            \begin{equation*}
            \begin{split}
                (a+bi)(x,y) = (ax-by,ay+bx)
            \end{split}
            \end{equation*}
        With scalar multiplication defined as above, then $\bfR \oplus \bfR$ is a $\bfC$-vector space. Furthermore, we have $\bfR \oplus \bfR \cong \bfC$ as \textit{$\bfC$-vector spaces}!
    \end{example}

    \begin{definition}
        Let $V$ be a real vector space. The \textui{complexification} of $V$ is denoted $V_\bfC = V \oplus V$, where complex scalar multiplication is defined by:
            \begin{equation*}
            \begin{split}
                (a+bi)(v_1,v_2) = (av_1 - bv_2, av_2 + bv_1).
            \end{split}
            \end{equation*}
        Upon investigation one can see:
            \begin{equation*}
            \begin{split}
                i(v_1,v_2) = (-v_2,v_1).
            \end{split}
            \end{equation*}
    \end{definition}

    \begin{exercise}
        Prove that $V_\bfC$ is a $\bfC$-vector space.
    \end{exercise}

    \begin{proposition}
        Let $\cB = \{v_i\}_{i \in I}$ be an $\bfR$-basis of $V$. The set $\cB_\bfC = \{(v_j,0_V)\}_{j \in I}$ is a $\bfC$-basis of $V_\bfC$.
    \end{proposition}
        \begin{proof}
            Let $(w_1,w_2) \in V_\bfC$. We can write:
                \begin{equation*}
                \begin{split}
                    w_1 &= \sum_{j \in I}a_j v_j \\
                    w_2 &= \sum_{j \in I}b_j v_j
                \end{split}
                \end{equation*}
            for some $a_j,b_j \in \bfR$. We have:
                \begin{equation*}
                \begin{split}
                     (w_1,w_2)
                     & = \left(\sum_{j \in I}a_jv_j, \sum_{j \in I}b_j v_j\right) \\
                     & = \left(\sum_{j \in I}a_jv_j, 0_V\right) + \left( 0_V, \sum_{j \in I}b_j v_j\right) \\
                     & = \sum_{j \in I}a_j(v_j,0_V) + \sum_{j \in I}b_j(0_V,v_j) \\
                     & = \sum_{j \in I}a_j(v_j,0_V) + \sum_{j \in I}ib_j(v_j,0_V) \\
                     &\in \Span_\bfC \left\{(v_j,0_V)\right\}_{i \in I}.
                \end{split}
                \end{equation*}
            Now suppose we have $(0_V,0_V) = \sum_{j \in I}(a_j + ib_j)(v_j,0_V)$. Then:
                \begin{equation*}
                \begin{split}
                    (0_V,0_V) 
                    &= \sum_{j \in I}(a_j + ib_j)(v_j,0_V) \\
                    & = \sum_{j \in I}a_j(v_j,0_V) + \sum_{j \in I}ib_j(v_j,0_V) \\
                    & = \left(\sum_{j \in I}a_jv_j, 0_V\right) + i \left(\sum_{j \in I}b_jv_j,0_V\right) \\
                    & = \left(\sum_{j \in I}a_jv_j, 0_V\right) + \left(\sum_{j \in I}0_V,b_jv_j\right) \\
                    & = \left(\sum_{j \in I}a_jv_j, \sum_{j \in I}b_jv_j\right),
                \end{split}
                \end{equation*}
            meaning:
                \begin{equation*}
                \begin{split}
                    \sum_{j \in I}a_jv_j &= 0_V \\
                    \sum_{j \in I}b_jv_j &= 0_V.
                \end{split}
                \end{equation*}
            So $a_j = 0$ for all $j$ and $b_j = 0$ for all $j$. Thus $\left\{(v_j,0_V)\right\}_{j \in I}$ are linearly independent.
        \end{proof}

    \begin{proposition}
        Let $V,W$ be $\bfR$-vector spaces, and let $T \in \Hom_\bfR(V,W)$. There is a unique $T_\bfC \in \Hom_\bfC(V_\bfC,W_\bfC)$ that makes the following diagram commute:
            \begin{center}
                \begin{tikzcd}
                    V \arrow[d, "\iota_V"', hook] \arrow[r, "T"] & W \arrow[d, "\iota_W", hook] \\
                    V_\bfC \arrow[r, "T_\bfC"']                  & W_\bfC                      
                \end{tikzcd}
            \end{center}
    \end{proposition}
        \begin{proof}
            Define 
                \begin{equation*}
                \begin{split}
                    T_\bfC(v_1,v_2) = (T(v_1),T(v_2)).
                \end{split}
                \end{equation*}
            Let $v \in V$. We have $\iota_V(v) = (v,0_V)$, meaning:
                \begin{equation*}
                \begin{split}
                    T_\bfC(\iota_V(v)) 
                    &= T_\bfC((v,0_V)) \\
                    & = (T(v),T(0_V)) \\
                    & = (T(v),0_W),
                \end{split}
                \end{equation*}
            and:
                \begin{equation*}
                \begin{split}
                    \iota_W(T(v)) = (T(v),0_W).
                \end{split}
                \end{equation*}
            Hence the diagram commutes. We claim that $T_\bfC$ is $\bfC$-linear. Let $x+iy \in \bfC$ and $(v_1,v_2),(v_1',v_2
            ) \in V_\bfC$. Then:
                \begin{equation*}
                \begin{aligned}
                T_{\mathbf{C}}\left(\left(v_1, v_2\right)+(x+\mathrm{iy})\left(v_1^{\prime}, v_2^{\prime}\right)\right) & =T_{\mathbf{C}}\left(\left(v_1, v_2\right)+\left(x v_1^{\prime}-y v_2^{\prime}, x v_2^{\prime}+y v_1^{\prime}\right)\right) \\
                & =T_{\mathbf{C}}\left(\left(v_1+x v_1^{\prime}-y v_2^{\prime}, v_2+x v_2^{\prime}+y v_1^{\prime}\right)\right) \\
                & =\left(T\left(v_1+x v_1^{\prime}-y v_2^{\prime}\right), T\left(v_2+x v_2^{\prime}+y v_1^{\prime}\right)\right) \\
                & =\left(T\left(v_1\right), T\left(v_2\right)\right)+x\left(\mathrm{~T}\left(v_1^{\prime}\right), T\left(v_2^{\prime}\right)\right)+y\left(-T\left(v_2^{\prime}\right), T\left(v_1^{\prime}\right)\right) \\
                & =\left(T\left(v_1\right), T\left(v_2\right)\right)+(x+i y)\left(T\left(v_1^{\prime}\right), T\left(v_2^{\prime}\right)\right) \\
                & =T_{\mathbf{C}}\left(v_1, v_2\right)+(x+\mathrm{iy}) T_{\mathbf{C}}\left(v_1^{\prime}, v_2^{\prime}\right) .
                \end{aligned}
                \end{equation*}
            Hence $T_\bfC$ is linear. Now suppose there is an $S \in \Hom_\bfC(V_\bfC,W_\bfC)$ making the following diagram commute:
                \begin{center}
                    \begin{tikzcd}
                        V \arrow[d, "\iota_V"', hook] \arrow[r, "T"] & W \arrow[d, "\iota_W", hook] \\
                        V_\bfC \arrow[r, "S"']                  & W_\bfC                      
                    \end{tikzcd}
                \end{center}
            Let $v_1,v_2 \in V_\bfC$. Then:
                \begin{equation*}
                \begin{split}
                    S((v_1,v_2))
                    & = S((v_1,0_V) + (0_V,v_2)) \\
                    & = S((v_1,0_V) + i(v_2,0_V)) \\
                    & = S((v_1,0_V)) + iS((v_2,0_V)) \\
                    & = S(\iota_V(v_1)) + iS(\iota_V(v_2)) \\
                    & = \iota_W(T(v_1)) + i \iota_W(T(v_2)) \\
                    & = (T(v_1),0_W) + i(T(v_2),0_W) \\
                    & = (T(v_1,0_W)) + (0_W,T(v_2)) \\
                    & = (T(v_1),T(v_2)) \\
                    & = T_\bfC((v_1,v_2)).
                \end{split}
                \end{equation*}
            Thus $T_\bfC$ is unique.
        \end{proof}

\section{Free Vector Spaces}
    We showed in Section~\ref{section:vector-bases} that every vector space has a basis. In this section we show that given a set $X$, we can build a vector space that "has" $X$ as a basis. We will give a few basic definitions before investigating the properties of these spaces.

    \begin{definition}
        Let $f:\Omega \rightarrow F$ be a map whose domain is an arbitrary set $\Omega$. The \textui{support} if $f$, denoted $\supp(f)$ is the set of points in $\Omega$ where $f$ is nonzero:
            \begin{equation*}
            \begin{split}
                \supp(f) = \{x \in \Omega \mid f(x) \neq 0\}.
            \end{split}
            \end{equation*}
    \end{definition}

    \begin{definition}
        Let $F$ be a field. The set of all functions from $\Omega$ to $F$ is denoted:
            \begin{equation*}
            \begin{split}
                \cF(\Omega,F) = \{f \mid f:\Omega \rightarrow F\}.
            \end{split}
            \end{equation*}
    \end{definition}

    \begin{exercise}
        Show that $\cF(\Omega,F)$ is an $F$-vector space.
    \end{exercise}

    \begin{example}
        Fix $t \in \Gamma$. Recall that $\delta_t:\Gamma \rightarrow F$ is defined by:
            \begin{equation*}
            \begin{split}
                \delta_t (s) = \begin{cases} 1, & s = t \\ 0, & s \neq t \end{cases}.
            \end{split}
            \end{equation*}
        We have that $\delta_t \in \cF(\Gamma,F)$, and furthermore $\supp(\delta_t) = \{t \}$. If $f \in \cF(\Gamma,F)$ has finite support, then $\supp(f) = \{t_1,...,t_n\}$ for some $t_i \in F$. If:
            \begin{equation*}
            \begin{split}
                f(t_1) &= \alpha_1 \neq 0 \\
                f(t_2) &= \alpha_2 \neq 0 \\
                &\vdots \\
                f(t_n) &= \alpha_n \neq 0, \\
            \end{split}
            \end{equation*}
        then we can write $f = \sum_{j = 1}^n \alpha_j \delta_{t_j}$.
    \end{example}

    \begin{theorem}[Existence of Free Vector Spaces]\label{thm:free-vectorspace}
        Let $F$ be a field and $\Gamma$ a set. There is an $F$-vector space $\bbF(\Gamma)$ that has a basis isomorphic to $\Gamma$ as sets. Moreover, $\bbF(\Gamma)$ has the following universal property: if $W$ is any $F$-vector space and $t:\Gamma \rightarrow W$ is a map of sets, there is a unique $T \in \Hom_F(\bbF(\Gamma),W)$ such that $T(x) = t(x)$ for every $x \in \Gamma$; i.e., the following diagram commutes:
            \begin{center}
            \begin{tikzcd}
                \Gamma \arrow[r, "\iota", hook] \arrow[rd, "t"'] & \bbF(\Gamma) \arrow[d, "T"] \\
                                                            & W                  
            \end{tikzcd}
            \end{center}
    \end{theorem}
        \begin{proof}
            If $\Gamma$ is the empty set, take $\bbF(\Gamma) = \{0\}$. Let $\Gamma \neq \emptyset$. Define:
                \begin{equation*}
                \begin{split}
                    \bbF(\Gamma) = \{f:\Gamma \rightarrow F \mid \supp(f) < \infty\}.
                \end{split}
                \end{equation*}
            Since $\bbF(\Gamma) \subseteq \cF(\Gamma,F)$, this space inherits a natural vector space structure. In particular, if $f,g$ are finitely supported functions and $c \in F$, then $(f+g)(x) = f(x) + f(x)$ and $(cf)(x) = c f(x)$ will be finitely supported. Moreover, the zero element of this set if $f(x) = 0_{\bbF(\Gamma)}$. The rest of the vector space axioms are left as an exercise.

            We obtain an inclusion $\iota: \Gamma \hookrightarrow \bbF(\Gamma)$ by $x \mapsto \delta_x$. Let $\cX = \{\delta_x \mid x \in \Gamma\}$. This a subset of $\bbF(\Gamma)$ and furthermore we have a bijection $\Gamma \hooktwoheadrightarrow \cX$.

            Let $f \in \bbF(\Gamma)$. We can write $f = \sum_{x \in \Gamma}f(x)\delta_x \in \Span_F(\cX)$. Hence $\Span_F(\cX) = \bbF(\Gamma)$. Note that:
                \begin{equation*}
                \begin{split}
                    f(y)
                    & = f(y)\delta_y(y) \\
                    & = f(y)\delta(y)(y) + \sum_{x \neq y}f(x)\delta_x(y) \\
                    & = \sum_{x \in \Gamma}f(y)\delta_x(y).
                \end{split}
                \end{equation*}
            Note that $f(y)$ is just a scalar in $F$, hence an arbitrary element of $\bbF(\Gamma)$ looks like $\sum_{i = 1}^n a_i \delta_{x_i}$. Suppose then that $\sum_{i = 1}^n a_i \delta_{x_i} = 0_{\bbF(\Gamma)}$. We have that $\sum_{i = 1}^n a_i \delta_{x_i}(y) = 0_F$ for all $y \in \Gamma$. Thus:
                \begin{equation*}
                \begin{split}
                    0_F 
                    & = \sum_{i = 1}^n a_i \delta_{x_i}(x_j) \\
                    & = a_j.
                \end{split}
                \end{equation*}
            This establishes $\cX$ as a basis for $\bbF(\Gamma)$.

            Now suppose we have $t: \Gamma \rightarrow W$. Define $T: \bbF(\Gamma) \rightarrow W$ by:
                \begin{equation*}
                \begin{split}
                    T \left(\sum_{i = 1}^n a_i \delta_{x_i}\right) = \sum_{i =1}^n a_i t(\iota^{-1}(\delta_{x_i})).
                \end{split}
                \end{equation*}
            Because $\cX$ is a basis, this gives a well-defined linear map. It is unique because any linear map that agrees with $t$ on $\cX$ must agree with $T$ on $\bbF(\Gamma)$, establishing the proof.
        \end{proof}

        \begin{example}
            If $\Gamma = \bfR$, we can form $\bbF_\bfR(\bfR)$. An example of  an element of $\bbF_\bfR(\bfR)$ is $2 \cdot \pi + 3 \cdot 2$, where $\pi,2$ are basis elements and $2,3$ are scalars. Note that, from this construction, we cannot simplify this expression.
        \end{example}
    
        \begin{exercise}
            Show that if $\Gamma = \{x_1,...,x_n\}$, then $\bbF(\Gamma) \cong F^n$.
        \end{exercise}

\section{Extension of Scalars}
    Let $V$ be an $F$-vector space and $K/F$ an extension of fields. We can naturally consider $K$ as an $F$-vector space. As we did with complexification, we want to define a way to "multiply" vectors in $V$ by scalars in $K$. The way we define "multiplication" should be obvious: Let $a, a_1,a_2 \in K$, $c\in F$, and $v,v_1,v_2 \in V$. We want multiplication to satisfy:
        \begin{enumerate}[label = (\arabic*)]
            \item $(a_1 + a_2) \star v$;
            \item $a \star (v_1 + v_2) = a \star v_1 + a \star v_2$;
            \item $(ac) \star v = a \star (cv)$.
        \end{enumerate}
    We will construct a vector space that satisfies exactly this by constructing the \textit{tensor product} of $V$ with $K$.

    \begin{definition}
        Let $V$ be an $F$-vector space and $K/F$ be an extension of fields. Let $K \times V$ be the Cartesian product of $K$ and $V$ and define:
            \begin{equation*}
            \begin{split}
                \cA_1 &= \{(a_1 + a_2,v)-(a_1,v)-(a_2,v) \mid a_1,a_2 \in K, v\in V\}, \\
                \cA_2 &= \{(a,v_1 + v_2) - (a,v_1) - (a,v_2) \mid a \in K, v_1,v_2 \in V\}, \\
                \cA_3 &= \{(ca,v) - (a,cv) \mid c \in F, a \in K, v\in V\}, \\
                \cA_4 &= \{a_1(a_2,v) - (a_1 a_2,v) \mid a_1,a_2 \in K,v \in V\}.
            \end{split}
            \end{equation*}
        Define $\Rel_K(K \times V) = \Span_F(\cA_1,\cA_2,\cA_3,\cA_4)$. The \textui{tensor product} of $K$ and $V$ over $F$ is:
            \begin{equation*}
            \begin{split}
                K \otimes_F V = \bbF(K \times V)/\Rel_K(K \times V).
            \end{split}
            \end{equation*}
        For any arbitrary element $\sum_{\text{\tiny finite}}c_i \delta_{a_i,v_i} \in \bbF(K \times V)$, we denote the equivalence class \newline $\sum_{\text{\tiny finite}}c_i \delta_{a_i,v_i} + \Rel_K(K \times V)$ as:
            \begin{equation*}
            \begin{split}
                \sum_{\text{finite}}c_i(a_i \otimes v_i)
                & = \sum_{\text{finite}}c_i a_i(1 \otimes v_i) \\
                & = \sum_{\text{finite}}b_i \otimes v_i
            \end{split}
            \end{equation*}
        for some $b_i \in K$. An element of the form $a \otimes v$ is referred to as a \textui{pure tensor}. Both arbitrary elements of $K \otimes_F V$ and pure tensors admit the following properties:
            \begin{enumerate}[label = (\arabic*)]
                \item $(a_1 + a_2) \otimes v = a_1 \otimes v + a_2 \otimes v$ for all $a_1,a_2 \in K$, $v \in V$;
                \item $a \otimes (v_1 + v_2) = a \otimes v_1 + a \otimes v_2$ for all $a \in K$, $v_1,v_2 \in V$;
                \item $ca \otimes v = a \otimes cv$ for all $c \in F$, $a \in K$, and $v \in V$;
                \item $a_1(a_2 \otimes v) = (a_1 a_2)\otimes v$ for all $a_1,a_2 \in K$, $v \in V$.
            \end{enumerate}
    \end{definition}

    \begin{note}
        \phantom{a}
        \begin{enumerate}[label = (\arabic*)]
            \item An \textbf{arbitrary element of $K \otimes_F V$ is a finite sum.} It is a common mistake when working with tensor products to check things for pure tensors and not with arbitrary elements.
            \item Since $K \otimes_F V$ is a quotient space, there must be care in checking things are well-defined when working with tensor products. 
        \end{enumerate}
    \end{note}

    \begin{exercise}
        Show that $K \otimes_F V$ is a $K$-vector space. (Hint: $0 \otimes 0_V$ is the additive identity in $K \otimes_F V$).
    \end{exercise}

    \begin{proposition}\label{prop:basis-of-tensor}
        Let $K/F$ be a field extension and $V$ an $F$-vector space with basis $\cB = \{v_i\}_{i \in I}$. We have $\Span_K \left\{1 \otimes v_i\right\}_{i \in I} = K \otimes_F V$.
    \end{proposition}
        \begin{proof}
            Let $a \otimes v \in K\otimes_F V$. Write $v = \sum_{i=1}^n c_i v_i$ for some $c_i \in F$. We have:
                \begin{equation*}
                \begin{split}
                    a \otimes v 
                    & = a \otimes \left(\sum_{i=1}^n c_i v_i\right) \\
                    & = \sum_{i = 1}^n a \otimes c_i v_i \\
                    & = \sum_{i = 1}^n ac_i \otimes v_i \\
                    & = \sum_{i = 1}^n ac_i(1 \otimes v_i).
                \end{split}
                \end{equation*}
            From this, it follows that every pure tensor $a \otimes v$ is also in the span of $\{1 \otimes v_i\}_{i \in I}$. This gives that all finite sums of the form $\sum_{j \in I}a_j \otimes \tilde{v}_j$ are also in the span of $\{1 \otimes v_i\}_{i \in I}$. Hence $\Span_F \left\{1 \otimes v_i\right\}_{i \in I} = K \otimes_F V$.
        \end{proof}

    \begin{theorem}
        Let $K/F$ be an extension of fields, $V$ an $F$-vectorspace, and $\iota_V:V \rightarrow K \otimes_F V$ defined by $\iota_V (v) = 1 \otimes v$. Let $W$ be any $K$-vector space and $S \in \Hom_F(V,W)$. There is a unique $T \in \Hom_K(K \otimes_F V,W)$ so that $S = T \circ \iota_V$; i.e., the following diagram commutes:
            \begin{center}
                \begin{tikzcd}
                    V \arrow[r, "\iota_V"] \arrow[rd, "S"'] & K \otimes_F V \arrow[d, "T"] \\
                                                            & W                           
                \end{tikzcd}
            \end{center}
        Conversely, if $T \in \Hom_K(K \otimes_F V,W)$, then $T \circ \iota_V \in \Hom_F(V,W)$.
    \end{theorem}
        \begin{proof}
            Let $S \in \Hom_F(V,W)$. Recall we constructed $K \otimes_F V$ as a quotient of $\bbF(K \times V)$. Define:
                \begin{equation*}
                \begin{split}
                    t: K \times V \rightarrow W \mtext{by} (a,v) \mapsto aS(v)
                \end{split}
                \end{equation*}
            as a map of sets. Theorem~\ref{thm:free-vectorspace} tells us that $t$ extends to a map $T \in \Hom_K(\bbF(K \times V), W)$ such that $T((a,v)) = t((a,v))$. Since $T$ is linear:
                \begin{equation*}
                \begin{split}
                    T \left(\sum_{i \in I} c_i(a_i,v_i)\right)
                    & = \sum_{i \in I}T(c_i(a_i,v_i)) \\
                    & = \sum_{i \in I}c_i T((a_i,v_i)) \\
                    & = \sum_{i \in I}c_i t((a_i,v_i)) \\
                    & = \sum_{i \in I}c_i a_i S(v_i). \\
                \end{split}
                \end{equation*}
            We must check that $T$ is the zero map when restricted to $\Rel_K(K \times V)$. We have:
                \begin{equation*}
                \begin{split}
                    T((a+b,v)-(a,v)-(b,v))
                    & = T((a+b,v)) - T((a,v)) - T((b,v)) \\
                    & = (a+b)S(v) - aS(v) - bS(v) \\
                    & = 0_W.
                \end{split}
                \end{equation*}
            The rest of the relations are left as an exercise. Thus we have $T \in \Hom_K(K \otimes_F V , W)$ defined by $T \left(\sum_{i \in I}c_i(a_i \otimes v_i)\right) = \sum_{i \in I}c_i a_i S(v_i)$. To see that the diagram commutes, observe that:
                \begin{equation*}
                \begin{split}
                    T(\iota_V(v)) = T(1 \otimes v) = 1 \cdot S(v) = S(v).
                \end{split}
                \end{equation*}
            From Proposition~\ref{prop:basis-of-tensor}, we saw that $K \otimes_F V$ is spanned by elements of the form $1 \otimes v$. Hence any linear map on $K \otimes_F V$ is determined by the image of these elements. Since $T(1 \otimes v) = S(v)$, we get $T$ is uniquely determined by $S$.

            The converse statement that for any $T \in \Hom_K(K \otimes_F V)$ one has $T \circ \iota_V \in \Hom_F(V,W)$ is left as an exercise.
        \end{proof}

    \begin{exercise}
        Complete the proof that $T$ vanishes on all the relations.
    \end{exercise}

    \begin{exercise}
        Given $T \in \Hom_K(K \otimes_F V,W)$, show $T \circ \iota_V \in \Hom_F(V,W)$.
    \end{exercise}

    \begin{proposition}
        Let $K/F$ be an extension of fields. Then $K \otimes_F F \cong K$ as $K$-vector spaces. 
    \end{proposition}
        \begin{proof}
            There is a natural inclusion map $i: F \rightarrow K$. Let $\iota:F \rightarrow K \otimes_F F$. By the universal property we obtain a unique $K$-linear map $T: K \otimes_F F \rightarrow K$ so that the following diagram commutes:
                \begin{center}
                    \begin{tikzcd}
                        F \arrow[rd, "i"', hook] \arrow[r, "\iota"] & K \otimes_F F \arrow[d, "T"] \\ & K                           
                    \end{tikzcd}
                \end{center}
            We see that $T(1 \otimes x) = i(x)= x$. Since $T$ is $K$-linear this completely determines $T$ because, for $\sum_{i \in I}a_i \otimes x_i \in K \otimes_F F$, we have:
                \begin{equation*}
                \begin{split}
                    T \left(\sum_{i \in I}a_i \otimes x_i\right)
                    & = \sum_{i \in I}T(a_i \otimes x_i) \\
                    & = \sum_{i \in I}T(a_i(1 \otimes x_i)) \\
                    & = \sum_{i \in I} a_i T(1 \otimes x_i) \\
                    & = \sum_{i \in I}a_i x_i.
                \end{split}
                \end{equation*}
            If we show $T$ has an inverse map, then we obtain an isomorphism. Let $S:K \rightarrow K \otimes_F F$ defined by $y \mapsto y \otimes 1$. Let $a \in K$, and $y_1, y_2 \in K$. Then:
                \begin{equation*}
                \begin{split}
                    S(y_1 + a y_2)
                    & = ... 
                \end{split}
                \end{equation*}
            Hence $S \in \Hom_K(K,K \otimes_F F)$. Since $S,T$ are linear, it is enough to check that they are inverses with pure tensors. Observe that:
                \begin{equation*}
                \begin{split}
                    T(S(y)) &= T(y \otimes 1) = y T(1 \otimes 1) = y \\
                    S(T(a \otimes x)) &= S(a T(1 \otimes x)) = S(ax) = ax \otimes 1 = a \otimes x.
                \end{split}
                \end{equation*}
            Thus $T^{-1} = S$, and so $K \otimes_F F  \cong K$ as $K$-vector spaces.
        \end{proof}

    \begin{example}
        From the previous section, we can now see that $\bfR_\bfC = \bfC \otimes_\bfR \bfR \cong \bfC$.
    \end{example}

    \begin{example}
        It is not always obvious that an element of $K \otimes_F F$ is nonzero. Take for example $\bfZ/2\bfZ \otimes_\bfZ \bfZ$. We have that $1 \otimes 2 = 2 \otimes 1 = 0_{\bfZ/2\bfZ} \otimes 1 = 0_{\bfZ/2\bfZ \otimes_\bfZ \bfZ}$.
    \end{example}

    \begin{exercise}
        Show that $V_\bfC \cong \bfC \otimes_\bfR V$.
    \end{exercise}

    \begin{proposition}
        Let $K/F$ be an extension of fields and $V$ an $F$-vector space with $\dim_F V = n$. Then $K \otimes_F V \cong K^n$ as $K$-vector spaces.
    \end{proposition}
        \begin{proof}
            We want a $K$-linear map $K \otimes_F V \rightarrow K^n$. Take $\cB = \{v_1,...,v_n\}$ to be a basis for $V$ and $\cE_n = \{e_1,...,e_n\}$ the standard basis for $K^n$. Define a map $t:V \rightarrow K^n$ by $t(v_i) = e_i$. Since this map is defined on a basis, it extends to an $F$-linear map. So $t \in \Hom_F(V,K^n)$. The universal property gives $T \in \Hom_K(K \otimes_F V, K^n)$ so that $T(1 \otimes v_i) = e_i$. We will show that $T$ has an inverse. Define $S \in \Hom_K(K^n,K \otimes_F V)$ by $S(e_i) = 1 \otimes v_i$. These are clearly inverse maps, so $K \otimes_F V \cong K^n$. Moreover, since $S$ is invertible and the $e_i$ are a basis, $\{S(e_i)\}_{i =1}^n$ gives a basis of $K \otimes_F V$; i.e., $\{1 \otimes v_i\}_{i = 1}^n$ is a basis.
        \end{proof}

    \begin{proposition}
        Let $K/F$ be an extension of fields and $V$ an $F$-vector space. Let $\cB = \{v_i\}_{i \in I}$ be an $F$-basis of $V$. We have $\cB_K = \{1 \otimes v_i\}_{i \in I}$ is a basis of $K \otimes_F V$.
    \end{proposition}
        \begin{proof}
            We saw in Proposition~\ref{prop:basis-of-tensor} that $\cB_K$ spans $K \otimes_F V$. Suppose there exists a linear dependence $\sum_{i \in I}c_i(1 \otimes v_i) = 0_{K \otimes_F V}$. Given $(b,v) \in K \times V$, write $(b,\sum_{i \in I}a_i v_i)$ for some $a_i \in F$. Fix $i_0 \in I$ and define:
                \begin{equation*}
                \begin{split}
                    t_{i_0}: V \rightarrow K
                \end{split}
                \end{equation*}
            by $t_{i_0}(v) = t_{i_0}(\sum_{i \in I}a_i v_i) = a_{i_0}$. One can check that $t_{i_0} \in \Hom_F(V,K)$. The universal property extends this to $T_{i_0} \in \Hom_K(K \otimes_F V, K)$ so that $T_{i_0}(1 \otimes v) = t_{i_0}(v) = a_{i_0}$. Recall that $\sum_{i \in I}c_i(1 \otimes v_i) = 0_{K \otimes_F V}$. Observe that:
                \begin{equation*}
                \begin{split}
                    0_K 
                    &= T_{i_0}(0_{K \otimes_F V}) \\
                    & = T_{i_0}\left(\sum_{i \in I}c_i (1 \otimes v_i)\right) \\
                    & = \sum_{i \in I}c_i T_{i_0}(1 \otimes v_i) \\
                    & = \sum_{i \in I}c_i t_{i_0}(v_i) \\
                    & = c_{i_0}.
                \end{split}
                \end{equation*}
            Since $i_0 \in I$ was arbitrary, we have that $c_i = 0$ for all $i \in I$; i.e., $\cB_K$ is linearly independent. Thus $\cB_K$ is a basis of $K \otimes_F V$.
        \end{proof}

    \begin{theorem}
        Let $K/F$ be an extension of fields and $V,W$ both $F$-vector spaces. Given $T \in \Hom_F(V,W)$, there is a unique map $T_K \in \Hom_K(K \otimes_F V, K \otimes_F W)$ so that the following diagram commutes:
            \begin{center}
                \begin{tikzcd}
                    V \arrow[r, "T"] \arrow[d, "\iota_V"', hook] & W \arrow[d, "\iota_W", hook] \\
                    K\otimes_F V \arrow[r, "T_K"']               & K\otimes_F W                
                    \end{tikzcd}
            \end{center}
    \end{theorem}
        \begin{proof}
            Define $t:V \rightarrow K \otimes_F W$ by $t(v) = 1 \otimes T(v)$. It can be shown that $t \in \Hom_F(V, K\otimes_F W)$. The universal property extends this to a unique map $T_K \in \Hom_K(K \otimes_F V, K\otimes_F W)$ so that $t = T_K \circ \iota_V$. Let $v \in V$. We have that $\iota_W(T(v)) = 1 \otimes T(v) = t(v) = (T_K \circ \iota_V)(v)$, meaning the diagram commutes.
        \end{proof}
    
    \begin{exercise}
        Let $V$ be an $\bfR$-vector space. We have $\bfC \otimes_\bfR V \cong V_\bfC$.
    \end{exercise}

\section{Tensor Products of Vector Spaces}
    \begin{definition}
        Let $U,V,W$ be $F$-vector spaces. If $t:V \times W \rightarrow U$ satisfies:
            \begin{enumerate}[label = (\arabic*)]
                \item ${t}\left(v_1+v_2, w\right)={t}\left(v_1, w\right)+{t}\left(v_2, w\right)$;
                \item ${t}\left(v, w_1+w_2\right)={t}\left(v, w_1\right)+{t}\left(v, w_2\right)$;
                \item ${ct}(v, w)={t}({c} v, w)={t}(v, {c} w)$;
            \end{enumerate}
        we call $t$ a \textui{bilinear map}. The collection of bilinear maps is denoted $\Hom_F(V,W;U)$. If $t \in \Hom_F(V,V;F)$, then we say $t$ is a \textui{bilinear form}.
    \end{definition}

    \begin{example}
        \phantom{a}
        \begin{enumerate}
            \item The standard dot product $\bfR^n \times \bfR^n \rightarrow \bfR$ is a bilinear form.
            \item The standard cross-product in $\bfR^3 \times \bfR^3 \rightarrow \bfR^3$ is a bilinear map.
        \end{enumerate}
    \end{example}
    
\end{document}
%%%%%%%%%%%%%%%%%%%%%%%%%%%%%%%%%%%%%%%%%%%%%%%%%%%%%%%%%%%%%%%%%%%%%%%%%