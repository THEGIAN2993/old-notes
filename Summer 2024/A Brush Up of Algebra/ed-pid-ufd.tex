\chapter{Euclidean Domains, PIDs, UFDs}\label{chapter:introduction}

\pagenumbering{arabic}

\section{Euclidean Domains}
    \begin{definition}
        Let $R$ be an integral domain. Any function $N:R \rightarrow \bfZ^+ \cup \{0\}$ with $N(0)=0$ is called a \textui{norm} on the integral domain $R$. If $N(a) > 0$ for $a \neq 0$ define $N$ to be a \textui{positive norm}.
    \end{definition}

    \begin{definition}
        The integral domain $R$ is said to be a \textui{Euclidean Domain} (or possess a \textui{Division Algorithm}) if there is a norm $N$ on $R$ such that for any two elements $a$ and $b$ of $R$ with $b \neq 0$  there exist elements $q$ and $r$ in $R$ with 
            \begin{equation*}
            \begin{split}
                a = qb + r \quad \text{with} \hspace{4pt} r=0 \hspace{4pt} \text{or} \hspace{4pt} N(r) < N(b).
            \end{split}
            \end{equation*}
        The element $q$ is called the \textui{quotient} and the element $r$ is called the $remainder$ of the division. 
    \end{definition}

    \begin{example}[Euclidean Algorithm]\label{example:euclidean-algorithm}
        Let $a$ and $b$ be any two elements of the Euclidean domain $R$. By successive "divisions" (these actually are divisions in the field of fractions of $R$) we can write
            \begin{equation*}
            \begin{split}
                a &= q_0 b + r_0 \\
                b &= q_1 r_0 + r_1 \\
                r_0 &= q_2 r_1 + r_2 \\
                & \vdots \\
                r_{n-2} &= q_{n} r_{n-1} + r_{n} \\
                r_{n-1} & = q_{n+1} r_{n}
            \end{split}
            \end{equation*}
        where $r_n$ is the last nonzero remainder. Such an $r_n$ exists since $N(b) > N(r_0) > N(r_1) > ... > N(r_n)$ is a decreasing sequence of nonnegative integers if the remainders are nonzero, and such a sequence cannot continue indefinitely. Note also that there is no guarentee that these elements are unique.
    \end{example}

    \begin{example}
        \phantom{a}
        \begin{enumerate}[label = (\arabic*)]
            \item Fields are trivial examples of Euclidean Domains where any norm will satisfy the defining condition (e.g., $N(a) = 0$ for all $a$). This is because for every $a,b$ with $b\neq 0$ we have $a = qb + 0$, where $q = ab^{-1}$.
            \item The integers $\bfZ$ are a Euclidean Domain with norm given by $N(a) = |a|$, the usual absolute value.
            \item If $F$ is a field, then the polynomial ring $F[x]$ is a Euclidean Domain with norm given by $N(p(x)) = \deg{p(x)}$. The Division Algorithm for polynomials is simply "long division" of polynomials. The proof is very similar to that for $\bfZ$ and is given in the next chapter. We will prove in Section~\ref{sec:2} that $R[x]$ is not a Euclidean Domain if $R$ is not a field.
        \end{enumerate}
    \end{example}

    \begin{proposition}\label{prop:1.1.1}
        Every ideal in a Euclidean Domain is principle. More precisely, if $I$ is any nonzero ideal in the Euclidean Domain $R$ then $I = (d)$, where $d$ is any nonzero element of $I$ of minimum norm.
    \end{proposition}
        \begin{proof}
            If $I$ is the zero ideal there is nothing to prove. Otherwise let $d \in I$ be any nonzero element of minimum norm (such a $d$ exists since the set $\{N(a) \mid a \in I\}$ has a minimum element by the well-ordering of $\bfZ$). Clearly $(d) \subseteq I$ since $d$ is an element of $I$. To show the reverse inclusion let $a \in I$ and use the Division Algorithm to write $a = qd + r$ with $r = 0$ or $N(r) < N(d)$. Then $r = a - qd$ and note that $a \in I$ and $qd \in I$, so $r$ is an element of $I$. By the minimality of the norm of $d$, it must be the case that $r = 0$. Hence $a = qd \in (d)$, showing $I \subseteq (d)$ which establishes the proposition that $I = (d)$.
        \end{proof}

    \begin{example}
        Let $R = \bfZ[x]$. Since the ideal $(2,x)$ is not principle, it follows that the ring $\bfZ[x]$ of polynomials with integer coefficients is not a Euclidean Domain.
    \end{example}

    \begin{definition}
        Let $R$ be a commutative ring and let $a,b \in R$ with $b \neq 0$.
        \begin{enumerate}[label = (\arabic*)]
            \item $a$ is said to be a \textui{multiple} of $b$ if there exists an element $x \in R$ with $a = bx$. In this case $b$ is said to \textui{divide} $a$ or be a \textui{divisor} of $a$, written $b \mid a$.
            \item A \textui{greatest common divisor} of $a$ and $b$ is a nonzero element $d$ such that 
                \begin{enumerate}[label = (\roman*)]
                    \item $d \mid a$ and $d \mid b$, and
                    \item if $d' \mid a$ and $d' \mid b$, then $d' \mid d$.
                \end{enumerate}
            A greatest common divisor of $a$ and $b$ will be denoted by $\gcd{(a,b)}$, or (abusing the notation) simply $(a,b)$.
        \end{enumerate}
    \end{definition}

    \begin{definition}
        If $I$ is the ideal of $R$ generated by $a$ and $b$ (that is, $I = (a,b)$), then $d$ is the greatest common divisor of $a$ and $b$ if
            \begin{enumerate}[label = (\roman*)]
                \item I is contained in the principial ideal $(d)$, and
                \item if $(d')$ is any principical ideal containing $I$ then $(d) \subseteq (d')$.
            \end{enumerate}
    \end{definition}

    \begin{proposition}
        If $a$ and $b$ are nonzero elements in the commutative ring $R$ such that the ideal generated by $a$ and $b$ is a principal ideal $(d)$, then $d$ is a a greatest common divisor of $a$ and $b$.
    \end{proposition}
        \begin{proof}
            This follows directly from the previous definition.
        \end{proof}
    
    \begin{proposition}
        Let $R$ be an integral domain. If two elements $d$ and $d'$ of $R$ generate the same principal ideal; i.e. $(d) = (d')$, then $d' = ud$ for some unit $u \in R$. In particular, if $d$ and $d'$ are both greatest common divisors of $a$ and $b$, then $d' = ud$ for some unit $u$.
    \end{proposition}
        \begin{proof}
            If either $d$ or $d'$ are $0$ then we are done. Assume $d$ and $d'$ are nonzero. Since $d \in (d')$ there is some $x \in R$ such that $d = xd'$. Since $d' \in (d)$ there is some $y \in R$ such that $d' = yd$. Thus $d = xyd$ and so $d(1-xy) = 0$. Since $d \neq 0$, it must be the case that $xy = 1$, that is, both $x$ and $y$ are units. This proves the first assertion.

            The second assertion follows from the first since any two greatest common divisors of $a$ and $b$ generate the same principle ideal (they divide eachother).
        \end{proof}
    \newpage
    \begin{theorem}
        Let $R$ be a Euclidean Domain and let $a$ and $b$ be nonzero elements of $R$. Let $d = r_n$ be the last nonzero remainder in the Euclidean Algorithm for $a$ and $b$ described in Example~\ref{example:euclidean-algorithm}. Then
            \begin{enumerate}[label = (\arabic*)]
                \item $d$ is the greatest common divisor of $a$ and $b$, and 
                \item the principal ideal $(d)$ is the ideal generated by $a$ and $b$. In particular, $d$ can be written as an $R$-linear combination of $a$ and $b$; i.e., there are elements $x$ and $y$ in $R$ such that
                    \begin{equation*}
                    \begin{split}
                        d = ax + by.
                    \end{split}
                    \end{equation*}
            \end{enumerate}
    \end{theorem}
        \begin{proof}
            By Proposition~\ref{prop:1.1.1}, the ideal generated by $a$ and $b$ is principal so $a,b$ do have a greatest common divisor, namely any element which generates the (principal) ideal $(a,b)$. Both parts of the theorem will follow once we show $d = r_n$ generates this ideal; i.e., once we show that
                \begin{enumerate}[label = (\roman*)]
                    \item $d \mid a$ and $d \mid b$ (which means $(a,b) \subseteq (d)$)
                    \item $d$ is an $R$-linear combination of $a$ and $b$ (which means $(d) \subseteq (a,b)$.)
                \end{enumerate}
            
            To prove that $d$ divides both $a$ and $b$, simply keep track of the divisibilities in the Euclidean Algorithm. Recall the following set of equations from Example~\ref{example:euclidean-algorithm}
            \begin{equation*}
                \begin{split}
                    a &= q_0 b + r_0 \quad\quad\quad\quad (0) \\
                    b &= q_1 r_0 + r_1 \quad\quad\quad\quad (1) \\
                    r_0 &= q_2 r_1 + r_2 \quad\quad\quad\quad (2) \\
                    & \vdots \\
                    r_{k-1} &= q_{k+1}r_k + r_{k+1} \quad\quad\quad\quad (k+1) \\
                    & \vdots \\
                    r_{n-2} &= q_{n} r_{n-1} + r_{n} \quad\quad\quad\quad (n) \\
                    r_{n-1} & = q_{n+1} r_{n} \quad\quad\quad\quad (n+1)
                \end{split}
                \end{equation*}
            We proceed with induction with $n$ as the base case. Equation $(n+1)$ gives $r_n \mid r_{n-1}$ and clearly $r_n \mid r_n$. Assume $r_n \mid r_{k+1}$ and $r_n \mid r_{k}$ as our inductive hypothesis. By Equation $(k+1)$ we see that $r_n$ divides both terms on the right hand side \textemdash hence $r_n \mid r_{k-1}$. From Equation (1) $r_n \mid b$ and from Equation (0) $r_n \mid a$, which establishes $(i)$.
        \end{proof}
