%%%%%%%PACKAGES%%%%%%%
    \documentclass[11pt,twoside,openany]{memoir}
    \usepackage[T1]{fontenc}
    \usepackage[utf8]{inputenc}
    \usepackage{titlesec}
    \usepackage{anyfontsize}
    \usepackage{fancybox}
    \usepackage[dvipsnames,svgnames,x11names,hyperref]{xcolor}
    \usepackage{enumerate}
    \usepackage{comment}
    \usepackage{amsfonts}
    \usepackage{amsthm}
    \usepackage{amsmath}
    \usepackage{amssymb}
    \usepackage{hyperref}
    \usepackage{fullpage}
    \usepackage{bm}
    \usepackage{cprotect}
    \usepackage{calligra}
    \usepackage{emptypage}
    \usepackage{titleps}
    \usepackage{microtype}
    \usepackage{float}
    \usepackage{ocgx}
    \usepackage{appendix}
    \usepackage{graphicx}
    \usepackage{pdfcomment}
    \usepackage{enumitem}
    \usepackage{mathtools}
    \usepackage{tikz-cd}
    \usepackage{relsize}
    \usepackage[font=footnotesize,labelfont=bf]{caption}
    \usepackage{changepage}
    \usepackage{xcolor}
    \usepackage{ulem}
    \usepackage{marginnote}
        \newcommand*{\mnote}[1]{ % <----------
        \checkoddpage
        \ifoddpage
            \marginparmargin{left}
        \else
            \marginparmargin{right}
        \fi
            \marginnote{\tiny \textcolor{oorange}{#1}}
        }
    \usepackage{tgbonum}
    \usepackage{datetime}
        \newdateformat{specialdate}{\THEYEAR\ \monthname\ \THEDAY}
    \usepackage[margin=0.9in]{geometry}
        \setlength{\voffset}{-0.4in}
        \setlength{\headsep}{30pt}
    \usepackage{fancyhdr}
        \fancyhf{}
        \pagestyle{fancy}
        \cfoot{\footnotesize \thepage}
        \fancyhead[R]{\footnotesize \rightmark}
        \fancyhead[L]{\footnotesize \leftmark}
    \usepackage[T1]{fontenc}% http://ctan.org/pkg/fontenc
    \usepackage[outline]{contour}% http://ctan.org/pkg/contour
        \renewcommand{\arraystretch}{1.5}
        \contourlength{0.4pt}
        \contournumber{10}%
    \usepackage{letterspace}







    

%%%%%%%%%%%%%%%%%%%%%%
%%%%%%%%MACROS%%%%%%%%
%%%%%%%%%%%%%%%%%%%%%%
    %to make the correct symbol for Sha
%\newcommand\cyr{%
%\renewcommand\rmdefault{wncyr}%
%\renewcommand\sfdefault{wncyss}%
%\renewcommand\encodingdefault{OT2}%
%\normalfont \selectfont} \DeclareTextFontCommand{\textcyr}{\cyr}


\DeclareMathOperator{\ab}{ab}
\newcommand{\absgal}{\G_{\bbQ}}
\DeclareMathOperator{\ad}{ad}
\DeclareMathOperator{\adj}{adj}
\DeclareMathOperator{\alg}{alg}
\DeclareMathOperator{\Alt}{Alt}
\DeclareMathOperator{\Ann}{Ann}
\DeclareMathOperator{\arith}{arith}
\DeclareMathOperator{\Aut}{Aut}
\DeclareMathOperator{\Be}{B}
\DeclareMathOperator{\Bd}{Bd}
\DeclareMathOperator{\card}{card}
\DeclareMathOperator{\Char}{char}
\DeclareMathOperator{\csp}{csp}
\DeclareMathOperator{\codim}{codim}
\DeclareMathOperator{\coker}{coker}
\DeclareMathOperator{\coh}{H}
\DeclareMathOperator{\compl}{compl}
\DeclareMathOperator{\conj}{conj}
\DeclareMathOperator{\cont}{cont}
\DeclareMathOperator{\crys}{crys}
\DeclareMathOperator{\Crys}{Crys}
\DeclareMathOperator{\cusp}{cusp}
\DeclareMathOperator{\diag}{diag}
\DeclareMathOperator{\diam}{diam}
\DeclareMathOperator{\Dom}{Dom}
\DeclareMathOperator{\disc}{disc}
\DeclareMathOperator{\dist}{dist}
\DeclareMathOperator{\dR}{dR}
\DeclareMathOperator{\Eis}{Eis}
\DeclareMathOperator{\End}{End}
\DeclareMathOperator{\ev}{ev}
\DeclareMathOperator{\eval}{eval}
\DeclareMathOperator{\Eq}{Eq}
\DeclareMathOperator{\Ext}{Ext}
\DeclareMathOperator{\Fil}{Fil}
\DeclareMathOperator{\Fitt}{Fitt}
\DeclareMathOperator{\Frob}{Frob}
\DeclareMathOperator{\G}{G}
\DeclareMathOperator{\Gal}{Gal}
\DeclareMathOperator{\GL}{GL}
\DeclareMathOperator{\Gr}{Gr}
\DeclareMathOperator{\Graph}{Graph}
\DeclareMathOperator{\GSp}{GSp}
\DeclareMathOperator{\GUn}{GU}
\DeclareMathOperator{\Hom}{Hom}
\DeclareMathOperator{\id}{id}
\DeclareMathOperator{\Id}{Id}
\DeclareMathOperator{\Ik}{Ik}
\DeclareMathOperator{\IM}{Im}
\DeclareMathOperator{\Image}{im}
\DeclareMathOperator{\Ind}{Ind}
\DeclareMathOperator{\Inf}{inf}
\DeclareMathOperator{\Isom}{Isom}
\DeclareMathOperator{\J}{J}
\DeclareMathOperator{\Jac}{Jac}
\DeclareMathOperator{\lcm}{lcm}
\DeclareMathOperator{\length}{length}
\DeclareMathOperator*{\limit}{limit}
\DeclareMathOperator{\Log}{Log}
\DeclareMathOperator{\M}{M}
\DeclareMathOperator{\Mat}{Mat}
\DeclareMathOperator{\N}{N}
\DeclareMathOperator{\Nm}{Nm}
\DeclareMathOperator{\NIk}{N-Ik}
\DeclareMathOperator{\NSK}{N-SK}
\DeclareMathOperator{\new}{new}
\DeclareMathOperator{\obj}{obj}
\DeclareMathOperator{\old}{old}
\DeclareMathOperator{\ord}{ord}
\DeclareMathOperator{\Or}{O}
\DeclareMathOperator{\op}{op}
\DeclareMathOperator{\PGL}{PGL}
\DeclareMathOperator{\PGSp}{PGSp}
\DeclareMathOperator{\rank}{rank}
\DeclareMathOperator{\Ran}{Ran}
\DeclareMathOperator{\Rel}{Rel}
\DeclareMathOperator{\Real}{Re}
\DeclareMathOperator{\RES}{res}
\DeclareMathOperator{\Res}{Res}
%\DeclareMathOperator{\Sha}{\textcyr{Sh}}
\DeclareMathOperator{\Sel}{Sel}
\DeclareMathOperator{\semi}{ss}
\DeclareMathOperator{\sgn}{sign}
\DeclareMathOperator{\SK}{SK}
\DeclareMathOperator{\SL}{SL}
\DeclareMathOperator{\SO}{SO}
\DeclareMathOperator{\Sp}{Sp}
\DeclareMathOperator{\Span}{span}
\DeclareMathOperator{\Spec}{Spec}
\DeclareMathOperator{\spin}{spin}
\DeclareMathOperator{\st}{st}
\DeclareMathOperator{\St}{St}
\DeclareMathOperator{\SUn}{SU}
\DeclareMathOperator{\supp}{supp}
\DeclareMathOperator{\Sup}{sup}
\DeclareMathOperator{\Sym}{Sym}
\DeclareMathOperator{\Tam}{Tam}
\DeclareMathOperator{\tors}{tors}
\DeclareMathOperator{\tr}{tr}
\DeclareMathOperator{\Tr}{Tr}
\DeclareMathOperator{\un}{un}
\DeclareMathOperator{\Un}{U}
\DeclareMathOperator{\val}{val}
\DeclareMathOperator{\vol}{vol}

\DeclareMathOperator{\Sets}{S \mkern1.04mu e \mkern1.04mu t \mkern1.04mu s}
    \newcommand{\cSets}{\scalebox{1.02}{\contour{black}{$\Sets$}}}
    
\DeclareMathOperator{\Groups}{G \mkern1.04mu r \mkern1.04mu o \mkern1.04mu u \mkern1.04mu p \mkern1.04mu s}
    \newcommand{\cGroups}{\scalebox{1.02}{\contour{black}{$\Groups$}}}

\DeclareMathOperator{\TTop}{T \mkern1.04mu o \mkern1.04mu p}
    \newcommand{\cTop}{\scalebox{1.02}{\contour{black}{$\TTop$}}}

\DeclareMathOperator{\Htp}{H \mkern1.04mu t \mkern1.04mu p}
    \newcommand{\cHtp}{\scalebox{1.02}{\contour{black}{$\Htp$}}}

\DeclareMathOperator{\Mod}{M \mkern1.04mu o \mkern1.04mu d}
    \newcommand{\cMod}{\scalebox{1.02}{\contour{black}{$\Mod$}}}

\DeclareMathOperator{\Ab}{A \mkern1.04mu b}
    \newcommand{\cAb}{\scalebox{1.02}{\contour{black}{$\Ab$}}}

\DeclareMathOperator{\Rings}{R \mkern1.04mu i \mkern1.04mu n \mkern1.04mu g \mkern1.04mu s}
    \newcommand{\cRings}{\scalebox{1.02}{\contour{black}{$\Rings$}}}

\DeclareMathOperator{\ComRings}{C \mkern1.04mu o \mkern1.04mu m \mkern1.04mu R \mkern1.04mu i \mkern1.04mu n \mkern1.04mu g \mkern1.04mu s}
    \newcommand{\cComRings}{\scalebox{1.05}{\contour{black}{$\ComRings$}}}

\DeclareMathOperator{\hHom}{H \mkern1.04mu o \mkern1.04mu m}
    \newcommand{\cHom}{\scalebox{1.02}{\contour{black}{$\hHom$}}}

         %  \item $\cGroups$
          %  \item $\cTop$
          %  \item $\cHtp$
          %  \item $\cMod$




\renewcommand{\k}{\kappa}
\newcommand{\Ff}{F_{f}}
%\newcommand{\ts}{\,^{t}\!}


%Mathcal

\newcommand{\cA}{\mathcal{A}}
\newcommand{\cB}{\mathcal{B}}
\newcommand{\cC}{\mathcal{C}}
\newcommand{\cD}{\mathcal{D}}
\newcommand{\cE}{\mathcal{E}}
\newcommand{\cF}{\mathcal{F}}
\newcommand{\cG}{\mathcal{G}}
\newcommand{\cH}{\mathcal{H}}
\newcommand{\cI}{\mathcal{I}}
\newcommand{\cJ}{\mathcal{J}}
\newcommand{\cK}{\mathcal{K}}
\newcommand{\cL}{\mathcal{L}}
\newcommand{\cM}{\mathcal{M}}
\newcommand{\cN}{\mathcal{N}}
\newcommand{\cO}{\mathcal{O}}
\newcommand{\cP}{\mathcal{P}}
\newcommand{\cQ}{\mathcal{Q}}
\newcommand{\cR}{\mathcal{R}}
\newcommand{\cS}{\mathcal{S}}
\newcommand{\cT}{\mathcal{T}}
\newcommand{\cU}{\mathcal{U}}
\newcommand{\cV}{\mathcal{V}}
\newcommand{\cW}{\mathcal{W}}
\newcommand{\cX}{\mathcal{X}}
\newcommand{\cY}{\mathcal{Y}}
\newcommand{\cZ}{\mathcal{Z}}


%mathfrak (missing \fi)

\newcommand{\fa}{\mathfrak{a}}
\newcommand{\fA}{\mathfrak{A}}
\newcommand{\fb}{\mathfrak{b}}
\newcommand{\fB}{\mathfrak{B}}
\newcommand{\fc}{\mathfrak{c}}
\newcommand{\fC}{\mathfrak{C}}
\newcommand{\fd}{\mathfrak{d}}
\newcommand{\fD}{\mathfrak{D}}
\newcommand{\fe}{\mathfrak{e}}
\newcommand{\fE}{\mathfrak{E}}
\newcommand{\ff}{\mathfrak{f}}
\newcommand{\fF}{\mathfrak{F}}
\newcommand{\fg}{\mathfrak{g}}
\newcommand{\fG}{\mathfrak{G}}
\newcommand{\fh}{\mathfrak{h}}
\newcommand{\fH}{\mathfrak{H}}
\newcommand{\fI}{\mathfrak{I}}
\newcommand{\fj}{\mathfrak{j}}
\newcommand{\fJ}{\mathfrak{J}}
\newcommand{\fk}{\mathfrak{k}}
\newcommand{\fK}{\mathfrak{K}}
\newcommand{\fl}{\mathfrak{l}}
\newcommand{\fL}{\mathfrak{L}}
\newcommand{\fm}{\mathfrak{m}}
\newcommand{\fM}{\mathfrak{M}}
\newcommand{\fn}{\mathfrak{n}}
\newcommand{\fN}{\mathfrak{N}}
\newcommand{\fo}{\mathfrak{o}}
\newcommand{\fO}{\mathfrak{O}}
\newcommand{\fp}{\mathfrak{p}}
\newcommand{\fP}{\mathfrak{P}}
\newcommand{\fq}{\mathfrak{q}}
\newcommand{\fQ}{\mathfrak{Q}}
\newcommand{\fr}{\mathfrak{r}}
\newcommand{\fR}{\mathfrak{R}}
\newcommand{\fs}{\mathfrak{s}}
\newcommand{\fS}{\mathfrak{S}}
\newcommand{\ft}{\mathfrak{t}}
\newcommand{\fT}{\mathfrak{T}}
\newcommand{\fu}{\mathfrak{u}}
\newcommand{\fU}{\mathfrak{U}}
\newcommand{\fv}{\mathfrak{v}}
\newcommand{\fV}{\mathfrak{V}}
\newcommand{\fw}{\mathfrak{w}}
\newcommand{\fW}{\mathfrak{W}}
\newcommand{\fx}{\mathfrak{x}}
\newcommand{\fX}{\mathfrak{X}}
\newcommand{\fy}{\mathfrak{y}}
\newcommand{\fY}{\mathfrak{Y}}
\newcommand{\fz}{\mathfrak{z}}
\newcommand{\fZ}{\mathfrak{Z}}


%mathbf
\newcommand{\bfA}{\mathbf{A}}
\newcommand{\bfB}{\mathbf{B}}
\newcommand{\bfC}{\mathbf{C}}
\newcommand{\bfD}{\mathbf{D}}
\newcommand{\bfE}{\mathbf{E}}
\newcommand{\bfF}{\mathbf{F}}
\newcommand{\bfG}{\mathbf{G}}
\newcommand{\bfH}{\mathbf{H}}
\newcommand{\bfI}{\mathbf{I}}
\newcommand{\bfJ}{\mathbf{J}}
\newcommand{\bfK}{\mathbf{K}}
\newcommand{\bfL}{\mathbf{L}}
\newcommand{\bfM}{\mathbf{M}}
\newcommand{\bfN}{\mathbf{N}}
\newcommand{\bfO}{\mathbf{O}}
\newcommand{\bfP}{\mathbf{P}}
\newcommand{\bfQ}{\mathbf{Q}}
\newcommand{\bfR}{\mathbf{R}}
\newcommand{\bfS}{\mathbf{S}}
\newcommand{\bfT}{\mathbf{T}}
\newcommand{\bfU}{\mathbf{U}}
\newcommand{\bfV}{\mathbf{V}}
\newcommand{\bfW}{\mathbf{W}}
\newcommand{\bfX}{\mathbf{X}}
\newcommand{\bfY}{\mathbf{Y}}
\newcommand{\bfZ}{\mathbf{Z}}

\newcommand{\bfa}{\mathbf{a}}
\newcommand{\bfb}{\mathbf{b}}
\newcommand{\bfc}{\mathbf{c}}
\newcommand{\bfd}{\mathbf{d}}
\newcommand{\bfe}{\mathbf{e}}
\newcommand{\bff}{\mathbf{f}}
\newcommand{\bfg}{\mathbf{g}}
\newcommand{\bfh}{\mathbf{h}}
\newcommand{\bfi}{\mathbf{i}}
\newcommand{\bfj}{\mathbf{j}}
\newcommand{\bfk}{\mathbf{k}}
\newcommand{\bfl}{\mathbf{l}}
\newcommand{\bfm}{\mathbf{m}}
\newcommand{\bfn}{\mathbf{n}}
\newcommand{\bfo}{\mathbf{o}}
\newcommand{\bfp}{\mathbf{p}}
\newcommand{\bfq}{\mathbf{q}}
\newcommand{\bfr}{\mathbf{r}}
\newcommand{\bfs}{\mathbf{s}}
\newcommand{\bft}{\mathbf{t}}
\newcommand{\bfu}{\mathbf{u}}
\newcommand{\bfv}{\mathbf{v}}
\newcommand{\bfw}{\mathbf{w}}
\newcommand{\bfx}{\mathbf{x}}
\newcommand{\bfy}{\mathbf{y}}
\newcommand{\bfz}{\mathbf{z}}

%blackboard bold

\newcommand{\bbA}{\mathbb{A}}
\newcommand{\bbB}{\mathbb{B}}
\newcommand{\bbC}{\mathbb{C}}
\newcommand{\bbD}{\mathbb{D}}
\newcommand{\bbE}{\mathbb{E}}
\newcommand{\bbF}{\mathbb{F}}
\newcommand{\bbG}{\mathbb{G}}
\newcommand{\bbH}{\mathbb{H}}
\newcommand{\bbI}{\mathbb{I}}
\newcommand{\bbJ}{\mathbb{J}}
\newcommand{\bbK}{\mathbb{K}}
\newcommand{\bbL}{\mathbb{L}}
\newcommand{\bbM}{\mathbb{M}}
\newcommand{\bbN}{\mathbb{N}}
\newcommand{\bbO}{\mathbb{O}}
\newcommand{\bbP}{\mathbb{P}}
\newcommand{\bbQ}{\mathbb{Q}}
\newcommand{\bbR}{\mathbb{R}}
\newcommand{\bbS}{\mathbb{S}}
\newcommand{\bbT}{\mathbb{T}}
\newcommand{\bbU}{\mathbb{U}}
\newcommand{\bbV}{\mathbb{V}}
\newcommand{\bbW}{\mathbb{W}}
\newcommand{\bbX}{\mathbb{X}}
\newcommand{\bbY}{\mathbb{Y}}
\newcommand{\bbZ}{\mathbb{Z}}
\newcommand{\jota}{\jmath}

\newcommand{\bmat}{\left( \begin{matrix}}
\newcommand{\emat}{\end{matrix} \right)}

\newcommand{\pmat}{\left( \begin{smallmatrix}}
\newcommand{\epmat}{\end{smallmatrix} \right)}

\newcommand{\lat}{\mathscr{L}}
\newcommand{\mat}[4]{\begin{pmatrix}{#1}&{#2}\\{#3}&{#4}\end{pmatrix}}
\newcommand{\ov}[1]{\overline{#1}}
\newcommand{\res}[1]{\underset{#1}{\RES}\,}
\newcommand{\up}{\upsilon}

\newcommand{\tac}{\textasteriskcentered}

%mahesh macros
\newcommand{\tm}{\textrm}

%Comments
\newcommand{\com}[1]{\vspace{5 mm}\par \noindent
\marginpar{\textsc{Comment}} \framebox{\begin{minipage}[c]{0.95
\textwidth} \tt #1 \end{minipage}}\vspace{5 mm}\par}

\newcommand{\Bmu}{\mbox{$\raisebox{-0.59ex}
  {$l$}\hspace{-0.18em}\mu\hspace{-0.88em}\raisebox{-0.98ex}{\scalebox{2}
  {$\color{white}.$}}\hspace{-0.416em}\raisebox{+0.88ex}
  {$\color{white}.$}\hspace{0.46em}$}{}}  %need graphicx and xcolor. this produces blackboard bold mu 

\newcommand{\hooktwoheadrightarrow}{%
  \hookrightarrow\mathrel{\mspace{-15mu}}\rightarrow
}

\makeatletter
\newcommand{\xhooktwoheadrightarrow}[2][]{%
  \lhook\joinrel
  \ext@arrow 0359\rightarrowfill@ {#1}{#2}%
  \mathrel{\mspace{-15mu}}\rightarrow
}
\makeatother

\renewcommand{\geq}{\geqslant}
\renewcommand{\leq}{\leqslant}
\newcommand{\midd}{\hspace{4pt}\middle|\hspace{4pt}}
    
    \newcommand{\bone}{\mathbf{1}}
    \newcommand{\sign}{\mathrm{sign}}
    \newcommand{\eps}{\varepsilon}
    \newcommand{\textui}[1]{\uline{\textit{#1}}}
    
    %\newcommand{\ov}{\overline}
    %\newcommand{\un}{\underline}
    \newcommand{\fin}{\mathrm{fin}}
    
    \newcommand{\chnum}{\titleformat
    {\chapter} % command
    [display] % shape
    {\centering} % format
    {\Huge \color{black} \shadowbox{\thechapter}} % label
    {-0.5em} % sep (space between the number and title)
    {\LARGE \color{black} \underline} % before-code
    }
    
    \newcommand{\chunnum}{\titleformat
    {\chapter} % command
    [display] % shape
    {} % format
    {} % label
    {0em} % sep
    { \begin{flushright} \begin{tabular}{r}  \Huge \color{black}
    } % before-code
    [
    \end{tabular} \end{flushright} \normalsize
    ] % after-code
    }

\newcommand{\nl}{\newline \mbox{}}

\newcommand{\h}[1]{\hspace{#1pt}}

\newcommand{\littletaller}{\mathchoice{\vphantom{\big|}}{}{}{}}
\newcommand\restr[2]{{% we make the whole thing an ordinary symbol
  \left.\kern-\nulldelimiterspace % automatically resize the bar with \right
  #1 % the function
  \littletaller % pretend it's a little taller at normal size
  \right|_{#2} % this is the delimiter
  }}

\newcommand{\mtext}[1]{\hspace{6pt}\text{#1}\hspace{6pt}}

\newcommand{\lnorm}{\left\lVert}
\newcommand{\rnorm}{\right\rVert}

\newcommand{\ds}{\displaystyle}
\newcommand{\ts}{\textstyle}

%This adds a "front cover" page.
%{\thispagestyle{empty}
%\vspace*{\fill}
%\begin{tabular}{l}
%\begin{tabular}{l}
%\includegraphics[scale=0.24]{oxy-logo.png}
%\end{tabular} \\
%\begin{tabular}{l}
%\Large \color{black} Module Theory, Linear Algebra, and Homological Algebra \\ \Large \color{black} Gianluca Crescenzo
%\end{tabular}
%\end{tabular}
%\newpage

\newcommand{\sfrac}[2]{{}^{#1}\mskip -5mu/\mskip -3mu_{#2}}


\makeatletter
\newcommand*{\da@rightarrow}{\mathchar"0\hexnumber@\symAMSa 4B }
\newcommand*{\da@leftarrow}{\mathchar"0\hexnumber@\symAMSa 4C }
\newcommand*{\xdashrightarrow}[2][]{%
  \mathrel{%
    \mathpalette{\da@xarrow{#1}{#2}{}\da@rightarrow{\,}{}}{}%
  }%
}
\newcommand{\xdashleftarrow}[2][]{%
  \mathrel{%
    \mathpalette{\da@xarrow{#1}{#2}\da@leftarrow{}{}{\,}}{}%
  }%
}
\newcommand*{\da@xarrow}[7]{%
  % #1: below
  % #2: above
  % #3: arrow left
  % #4: arrow right
  % #5: space left 
  % #6: space right
  % #7: math style 
  \sbox0{$\ifx#7\scriptstyle\scriptscriptstyle\else\scriptstyle\fi#5#1#6\m@th$}%
  \sbox2{$\ifx#7\scriptstyle\scriptscriptstyle\else\scriptstyle\fi#5#2#6\m@th$}%
  \sbox4{$#7\dabar@\m@th$}%
  \dimen@=\wd0 %
  \ifdim\wd2 >\dimen@
    \dimen@=\wd2 %   
  \fi
  \count@=2 %
  \def\da@bars{\dabar@\dabar@}%
  \@whiledim\count@\wd4<\dimen@\do{%
    \advance\count@\@ne
    \expandafter\def\expandafter\da@bars\expandafter{%
      \da@bars
      \dabar@ 
    }%
  }%  
  \mathrel{#3}%
  \mathrel{%   
    \mathop{\da@bars}\limits
    \ifx\\#1\\%
    \else
      _{\copy0}%
    \fi
    \ifx\\#2\\%
    \else
      ^{\copy2}%
    \fi
  }%   
  \mathrel{#4}%
}
\makeatother


    \newcommand{\TBC}{\textbf{TO BE CONTINUED}}
    \theoremstyle{plain}
    \newtheorem{theorem}{Theorem}[section]
    \newtheorem{proposition}[theorem]{Proposition}
    \newtheorem{corollary}[theorem]{Corollary}
    \newtheorem{lemma}[theorem]{Lemma}
    \newtheorem{open}[theorem]{Open problem}
    \newtheorem{fact}[theorem]{Fact}
    
    \theoremstyle{definition}
    \newtheorem{definition}[theorem]{Definition}
    \newtheorem*{definition*}{Definition}
    \newtheorem{example}{Example}[section]
    \newtheorem{notation}[theorem]{Notation}
    \newtheorem{exercise}{Exercise}[chapter]
    \newtheorem{practice}[exercise]{Practice}
    \newtheorem{challenge}[exercise]{Challenge}
    \newtheorem{homework}{Homework}
    \newtheorem{note}{Note}[section]
    
    \theoremstyle{remark}
    \newtheorem{remark}[theorem]{Remark}
    \newtheorem*{noproof}{Proof omitted}
    \numberwithin{equation}{section}
    
    \newenvironment{solution}[1]{\noindent \textbf{#1}:}{}
    
    \newcommand{\NN}{\mathbf{N}}
    \newcommand{\ZZ}{\mathbf{Z}}
    \newcommand{\QQ}{\mathbf{Q}}
    \newcommand{\RR}{\mathbf{R}}
    \newcommand{\CC}{\mathbf{C}}
    \newcommand{\HH}{\mathbf{H}}
    \newcommand{\KK}{\mathbf{K}}
    \newcommand{\FF}{\mathbf{F}}
    
    \newcommand{\bRR}{\overline{\RR}}
    \newcommand{\bRRp}{\overline{\RR}_{\geq 0}}
    
    \renewcommand{\geq}{\geqslant}
    \renewcommand{\leq}{\leqslant}
    
    \newcommand{\cA}{\mathcal{A}}
    \newcommand{\cB}{\mathcal{B}}
    \newcommand{\cC}{\mathcal{C}}
    \newcommand{\cF}{\mathcal{F}}
    \newcommand{\cL}{\mathcal{L}}
    \newcommand{\cM}{\mathcal{M}}
    \newcommand{\cN}{\mathcal{N}}
    \newcommand{\cU}{\mathcal{U}}
    \newcommand{\bone}{\mathbf{1}}
    \newcommand{\sign}{\mathrm{sign}}
    \newcommand{\eps}{\varepsilon}
    \newcommand{\textui}[1]{\uline{\textit{#1}}}
    
    %\newcommand{\ov}{\overline}
    %\newcommand{\un}{\underline}
    \newcommand{\fin}{\mathrm{fin}}
    
    \newcommand{\chnum}{\titleformat
    {\chapter} % command
    [display] % shape
    {} % format
    {} % label
    {-8em} % sep
    { \begin{flushright} \begin{tabular}{r} \fontsize{55}{20}\selectfont \color{black} \shadowbox{\thechapter} \\ \Huge \color{black}
    } % before-code
    [
    \end{tabular} \end{flushright} \normalsize
    ] % after-code
    }
    
    \newcommand{\chunnum}{\titleformat
    {\chapter} % command
    [display] % shape
    {} % format
    {} % label
    {0em} % sep
    { \begin{flushright} \begin{tabular}{r}  \Huge \color{black}
    } % before-code
    [
    \end{tabular} \end{flushright} \normalsize
    ] % after-code
    }





    
%%%%%%%%%%%%%%%%%%%%%%
%%%%DOCUMENT SETUP%%%%
%%%%%%%%%%%%%%%%%%%%%%
    \setsecnumdepth{subsection}
    \definecolor{bluey}{RGB}{21, 80, 234}
    \definecolor{darkgreen}{rgb}{0, 0.5976, 0}
    \hypersetup{pdfauthor=Gianluca Crescenzo, pdftitle=Rotman's Notes, pdfstartview=FitH, colorlinks=true, linkcolor=darkgreen, citecolor=darkgreen}
    
    \begin{document}
    
    %This adds a "front cover" page.
    %{\thispagestyle{empty}
    %\vspace*{\fill}
    %\begin{tabular}{l}
    %\begin{tabular}{l}
    %\includegraphics[scale=0.24]{oxy-logo.png}
    %\end{tabular} \\
    %\begin{tabular}{l}
    %\Large \color{black} Module Theory, Linear Algebra, and Homological Algebra \\ \Large \color{black} Gianluca Crescenzo
    %\end{tabular}
    %\end{tabular}
    
    %\newpage
    \pagenumbering{roman}
    \tableofcontents
    
    \chunnum
    \vfill
    \specialdate
    Last update: \today
    \chnum

    \chapter{Introduction}\label{chapter:introduction}

\pagenumbering{arabic}

\section{Categories and Functors}\label{sec:categories-and-functors}
    \begin{definition}\label{def:class}
        A \textui{class} is a collection of sets (or sometimes other mathematical objects) that can be unambiguously defined by a property that all its members share.
    \end{definition}

    \begin{definition}\label{def:categories}
        A \textui{category} $\mathcal{C}$ consists of three ingredients: a class $\obj({\mathcal{C})}$ of \textui{objects}, a set of \textui{morphisms} $\Hom(A,B)$ for every ordered pair $(A,B)$ of objects, and \textui{composition} $\Hom{(A,B)} \times \Hom{(B,C)} \rightarrow \Hom{(A,C)}$, denoted by
            \begin{equation*}
            \begin{split}
                (f,g) \mapsto g\circ f,
            \end{split}
            \end{equation*}
        for every ordered tripled $A,B,C$ of objects. These ingredients are subject to the following axioms:
            \begin{enumerate}[label = (\arabic*)]
                \item The $\Hom{}$ sets are pairwise disjoint; i.e., each $f \in \Hom{(A,B)}$ has a unique \textui{domain} A and a unique \textui{target} B;
                \item for each object $A$, there is an \textui{identity morphism} $1_A \in \Hom{(A,A)}$ such that $f \circ 1_A = f$ and $1_B \circ f = f$ for all $f: A \rightarrow B$;
                \item composition is associative: given morphisms $A \xrightarrow{f} B \xrightarrow{g} C \xrightarrow{h} D$, then 
                    \begin{equation*}
                    \begin{split}
                        h\circ(g \circ f)=(h\circ g)\circ f.
                    \end{split}
                    \end{equation*}
            \end{enumerate}
    \end{definition}

    \begin{example}
        \begin{enumerate}[label = (\arabic*)]
            \item The category $\cSets$ has its objects as sets, morphisms as functions, and composition the usual composition of functions.
            \item The category $\cGroups$ has its objects as groups, morphisms as homomorphisms, and the usual composition of functions (homomorphisms are functions). One must verify that identity maps are homomorphisms and the composition of homomorphisms is also a homomorphism.
            \item An ordered set $X$ can be regarded as a category whose objects are elements of $X$ and whose $\Hom$ sets are:
                \begin{equation*}
                \begin{split}
                    \Hom(x,y) = \begin{cases} \emptyset, & x \leq y \\ \{\iota_y ^ x\}, & x \leq y \end{cases}.
                \end{split}
                \end{equation*}
            Note that $1_x = \iota_x^x$ by reflexivity, and composition follows from that fact that $\leq$ is transitive.
            \item If $X$ is a topological space with a topology $\cT$, then $\cT$ forms a category whose objects are its open sets and morphisms inclusion maps.
            \item The category $\cAb$ has its objects as abelian groups, its morphisms as homomorphisms, and composition as the usual composition of functions.
            \item The category $\cRings$ has its objects as rings, morphisms as ring homomorphisms, and composition as the usual composition of functions. We assume that all rings $R \in \obj(\cRings)$ are unital.
            \item The category $\cComRings$ has its objects as commutative rings, morphisms as ring homomorphisms, and composition as the usual composition of functions.
            \item The category $_R\cMod$ has its objects as left $R$-modules (where $R$ is a ring), its morphisms as $R$-module homomorphisms, and composition as the usual composition of functions. The $\Hom$ sets are denoted $\Hom_R(A,B)$. If $R = \bfZ$, then $_\bfZ\cMod = \cAb$, as abelian groups are $\bfZ$-modules. There is also a category of right $R$-modules denoted $\cMod_R$.
        \end{enumerate}
    \end{example}

    \begin{definition}
        A category $\cC$ is \textui{discrete} if its only morphisms are identity morphisms.
    \end{definition}

    \begin{definition}
        Let $\cC$ be a category. A category $\cS$ is a \textui{subcategory} of $\cC$ if:
            \begin{enumerate}[label = (\arabic*)]
                \item $\obj(\cS) \subseteq \obj(\cC)$;
                \item $\Hom_\cS(A,B) \subseteq \Hom_\cC(A,B)$
                \item If $f \in \Hom_\cS(A,B)$ and $g \in \Hom_S(B,C)$, then $g \circ f \in \Hom_\cS(A,C)$ is equal to $g \circ f \in \Hom_\cC(A,C)$.
                \item If $A \in \obj(\cS)$, then $1_A \in \Hom_S(A,A)$ is equal to $1_A \in \Hom_\cC(A,A)$.
            \end{enumerate}
        We say $\cS$ is a \textui{full subcategory} if, for all $A,B \in \obj(\cS)$, we have $\Hom_\cS(A,B) = \Hom_\cC(A,B)$.
    \end{definition}

    \begin{example}
        \phantom{a}
        \begin{enumerate}[label = (\arabic*)]
            \item The category of finite sets forms a full subcategory of $\cSets$.
            \item The category whose objects are sets and whose morphisms are bijections forms a non-full subcategory of $\cSets$.
        \end{enumerate}
    \end{example}

    \chapter{Hom and Tensor}\label{chapter:hom-tensor}

\section{Constructs in \textsubscript{R}Mod}
    \begin{note}
        Recall from Exercise~\ref{exer:1-7} that 
            \begin{equation*}
            \begin{split}
                \End_\bfZ(M) = \{\text{homomorphisms}\hspace{4pt}\varphi:M \rightarrow M\}
            \end{split} 
            \end{equation*}
        is a ring under pointwise addition ($\varphi+\psi:m \mapsto \varphi(m) + \psi(m)$) and composition as multiplication.
    \end{note}

    \begin{definition}\label{def:representations}
        A \textui{representation} of a ring $R$ is a ring homomorphism $\varphi:R \rightarrow \End_\bfZ(M)$ for some abelian group $M$.
    \end{definition}

    \begin{proposition}
        Let $R$ be a ring and let $M$ be an abelian group. If $\varphi:R \rightarrow \End_\bfZ{(M)}$ is a representation, define $\sigma:R \times M \rightarrow M$ by $\sigma(r,m) = \varphi_r(m)$, where we write $\varphi(r) = \varphi_r$; then $\sigma $ is a scalar multiplier making $M$ into a left $R$-module. Conversely, if $M$ is a left $R$-module, then the function $\psi:R \rightarrow \End_\bfZ{(M)}$, given by $\psi(r): m \mapsto rm$, is a representation.
    \end{proposition}
        \begin{proof}
        

        \end{proof}

    \begin{definition}
        A functor $T: {}_{R}\cMod \rightarrow \cAb$ of either variance is called an \textui{additive functor} if, for every pair of $R$-module homomorphisms $f,g:A \rightarrow B$, we have
            \begin{equation*}
            \begin{split}
                T(f+g) = T(f) + T(g).
            \end{split}
            \end{equation*}
    \end{definition}

    \begin{lemma}\label{lemma:hom-ab-group}
        If $A,B \in \obj{({}_{R}\cMod)}$, then the set $\Hom_R{(A,B)}$ is an abelian group. Moreover, if $p:A' \rightarrow A$ and $q: B \rightarrow B'$ are $R$-module homomorphisms, then
            \begin{equation*}
            \begin{split}
                (\varphi+\psi)p = \varphi p + \psi p \hspace{8pt} \text{and} \hspace{8pt} q(\varphi+\psi) = q\varphi + q\psi.
            \end{split}
            \end{equation*} 
    \end{lemma}
        \begin{proof}
            Let $\varphi,\psi \in \Hom_R{(A,B)}$ and $r\in R$, $x,y\in A$. Observe that 
                \begin{equation*}
                \begin{split}
                    (\varphi+\psi)(x+y) 
                    & = \varphi(x+y)+\psi(x+y) \\
                    & = \varphi(x)+\varphi(y) + \psi(x) + \psi(y) \\
                    & = \varphi(x) + \psi(x) + \varphi(y) + \psi(y) \\
                    & = (\varphi+\psi)(x) + (\varphi+\psi)(y).
                \end{split}
                \end{equation*}
            Hence $\varphi+\psi$ is an $R$-module homomorphism. The identity element of $\Hom_R{(A,B)}$ is the zero-map : $(\varphi+0_{AB})(x) = \varphi(x) + 0_{AB}(x) = \varphi(x) + 0_B = \varphi(x)$ ($(0_{AB} + \varphi)(x)$ holds similarly). Given any $R$-module homomorphism $\varphi$, the inverse of $\varphi$ is $-\varphi:x \mapsto -(\varphi(x))$: observe that $(\varphi + (-\varphi))(x) = \varphi(x) + (-\varphi)(x) = \varphi(x) - \varphi(x) = 0_B$. It is routine to show that addition is associative, and likewise $(\varphi+\psi)(x) = \varphi(x) + \psi(x) = \psi(x) + \varphi(x) = (\psi+\varphi)(x)$. Hence $\Hom_R{(A,B)}$ is an additive abelian group.

            Let $a' \in A'$ and $b \in B$. Observe that 
                \begin{equation*}
                \begin{split}
                    (\varphi+\psi)(p)(a')
                    & = (\varphi+\psi)(p(a')) \\
                    & = \varphi(p(a')) + \psi(p(a')) \\
                    & = (\varphi p)(a') + (\psi p)(a')\\
                    & = (\varphi p + \psi p)(a')
                \end{split}
                \end{equation*}
            and
                \begin{equation*}
                \begin{split}
                    (q)(\varphi+g)(b)
                    & = q((\varphi+g)(b))\\
                    & = q(\varphi(b) + g(b))\\
                    & = q(\varphi(b)) + q(g(b))\\
                    & = (q\varphi)(b) + (q\psi)(b)\\
                    & = (q\varphi + q\psi)(b).
                \end{split}
                \end{equation*}
            This establishes the lemma.
        \end{proof}

        \begin{proposition}\label{prop:hom-additive-functor}
            Let $R$ be a ring, and let $A,B,B'$ be left $R$-modules.
            \begin{enumerate}[label = (\arabic*)]
                \item $\Hom_R{(A,\square)}$ is an additive functor ${}_{R}\cMod \rightarrow \cAb$.
            \end{enumerate}
        \end{proposition}
            \begin{proof}
                Lemma~\ref{lemma:hom-ab-group} says that $\Hom_R{(A,B)}$ is an abelian group. This satisfies axiom (1). Define $\Hom_R{(A,q)}: \Hom_R{(A,B)} \rightarrow \Hom_R{(A,B')}$ by $f \mapsto qf$, this clearly satisfies axiom (2).Let $p:B' \rightarrow B''$ be an $R$-module homomorphism. We'd like to show that the following two functions are equivalent:
                    \begin{equation*}
                    \begin{split}
                        \Hom_R{(A,pq)}:\Hom_R{(A,B)} \rightarrow \Hom_R{(A,B'')} \\
                        \Hom_R{(A,p)}\Hom_R{(A,q)}:\Hom_R{(A,B)} \rightarrow \Hom_R{(A,B'')}
                    \end{split}
                    \end{equation*}
                If $f \in \Hom_R{(A,B)}$, then $\Hom_R{(A,pq)}:f \mapsto (pq)f$. On the other hand, associativity of composition gives that $\Hom_R{(A,p)}\Hom_R{(A,q)}:f \mapsto qf \mapsto p(qf) = (pq)f$, as desired for axiom (3). If $1_B : B \rightarrow B$ is the identity map, then
                    \begin{equation*}
                    \begin{split}
                        \Hom_R{(A,1_B)}:f \mapsto 1_B f = f
                    \end{split}
                    \end{equation*}
                for all $f \in \Hom_R{(A,B)}$\textemdash hence $\Hom_R{(A,1_B)} = 1_{\Hom_R{(A,B)}}$ satisfying axiom (4). Finally, Lemma~\ref{lemma:hom-ab-group} also showed that $\Hom_R{(A,q)}$ is additive: $\Hom_R{(A,q)}:f+g \mapsto q(f+g) = qf + qg$. This establishes the proposition.
            \end{proof}

        \begin{proposition}
            Let $R$ be a ring and let $A, A', and B$ be left $R$-modules.
                \begin{enumerate}[label = (\arabic*)]
                    \item $\Hom_R(\square, B)$ is an additive (contravariant) functor ${}_{R}\cMod \rightarrow \cAb$.
                \end{enumerate}
        \end{proposition}
            \begin{proof}
                Similar to the proof of Proposition~\ref{prop:hom-additive-functor}
            \end{proof}

        \begin{proposition}
            Let $T:{}_{R}\cMod \rightarrow \cAb$ be an additive functor of either varience.
                \begin{enumerate}[label = (\arabic*)]
                    \item If $0_{AB}:A \rightarrow B$ is the zero map, then $T(0) = 0$.
                    \item $T(\{0\}) = \{0\}$
                \end{enumerate}
        \end{proposition}

\section{d and f}
Throughout this section all rings contain a $1$.
    \begin{definition}
        \phantom{a}
        \begin{enumerate}[label = (\arabic*)]
            \item The pair of homomorphisms $X \xrightarrow{\alpha} Y \xrightarrow{\beta} Z$ is said to be \textui{exact} (at Y) if $\Image{\alpha} = \ker{\beta}$.
            \item A sequence $... \rightarrow X_{n-1} \rightarrow X_n \rightarrow X_{n+1} \rightarrow ...$ of homomorphisms is said to be an \textui{exact sequnce} if it is exact at every $X_n$ between a pair of homomorphisms.
        \end{enumerate}
    \end{definition}

    \begin{proposition}
        Let $A$, $B$, and $C$ be $R$-modules over some ring $R$. Then
            \begin{enumerate}[label = (\arabic*)]
                \item The sequence $0 \rightarrow A \xrightarrow{\psi} B$ is exact (at A) if and only if $\psi$ is injective.
                \item The sequence $B \xrightarrow{\varphi} C \rightarrow 0$ is exact (at C) if and only if $\varphi$ is surjective.
            \end{enumerate}
    \end{proposition}
        \begin{proof}
            The (uniquely defined) homomorphism $0 \rightarrow A$ has image $0$ in $A$. This will be the kernel of $\psi$ if and only if $\psi$ is injective.

            Similarly, the kernel of the (uniquely defined) zero homomorphism $C \rightarrow 0$ is all of $C$, which is the image of $\varphi$ if and only if $\varphi$ is surjective.
        \end{proof}
    
    \end{document}
%%%%%%%%%%%%%%%%%%%%%%%%%%%%%%%%%%%%%%%%%%%%%%%%%%%%%%%%%%%%%%%%%%%%%%%%%