\documentclass[11pt,twoside,openany]{memoir}
%\usepackage{tgpagella} % text only
%\usepackage{mathpazo}  % math & text
\usepackage[T1]{fontenc}
\usepackage[hidelinks]{hyperref}
\usepackage{amsmath}
\usepackage{amsthm}
\usepackage{amssymb}
\usepackage{mathtools}
%\usepackage{newpxtext}
%\usepackage{eulerpx}
%\usepackage{eucal}
\usepackage{datetime}
    \newdateformat{specialdate}{\THEYEAR\ \monthname\ \THEDAY}
\usepackage[margin=1in]{geometry}
\usepackage{fancyhdr}
    \fancyhf{}
    \pagestyle{fancy}
    \cfoot{\scriptsize \thepage}
    \fancyhead[R]{\scriptsize \rightmark}
    \fancyhead[L]{\scriptsize \leftmark}
    \renewcommand{\headrulewidth}{0pt}
    \renewcommand{\footrulewidth}{0pt} % if you also want to remove the footer rule
\usepackage{thmtools}
    \declaretheoremstyle[
        spaceabove=10pt,
        spacebelow=10pt,
        headfont=\normalfont\bfseries,
        notefont=\mdseries, notebraces={(}{)},
        bodyfont=\normalfont,
        postheadspace=0.5em
        %qed=\qedsymbol
        ]{defs}

    \declaretheoremstyle[ 
        spaceabove=10pt, % space above the theorem
        spacebelow=10pt,
        headfont=\normalfont\bfseries,
        bodyfont=\normalfont\itshape,
        postheadspace=0.5em
        ]{thmstyle}
    
    \declaretheorem[
        style=thmstyle,
        numberwithin=section
    ]{theorem}

    \declaretheorem[
        style=thmstyle,
        sibling=theorem,
    ]{proposition}

    \declaretheorem[
        style=thmstyle,
        sibling=theorem,
    ]{lemma}

    \declaretheorem[
        style=thmstyle,
        sibling=theorem,
    ]{corollary}

    \declaretheorem[
        numberwithin=section,
        style=defs,
    ]{example}

    \declaretheorem[
        numberwithin=section,
        style=defs,
    ]{definition}

    \declaretheorem[
        style=defs,
        numbered=unless unique,
    ]{problem}

    \declaretheorem[
        numbered=unless unique,
        shaded={rulecolor=black,
    rulewidth=1pt, bgcolor={rgb}{1,1,1}}
    ]{axiom}

    \declaretheorem[numberwithin=section,style=defs]{note}
    \declaretheorem[numbered=unless unique,style=defs]{question}
    \declaretheorem[numbered=no,style=defs]{recall}
    \declaretheorem[numbered=no,style=remark]{answer}
    \declaretheorem[numbered=no,style=remark]{solution}

    \declaretheorem[numbered=no,style=defs]{remark}
\usepackage{enumitem}
\usepackage{titlesec}
    \titleformat{\chapter}[display]
    {\bfseries\LARGE\raggedright}
    {Chapter {\thechapter}}
    {1ex minus .1ex}
    {\Huge}
    \titlespacing{\chapter}
    {3pc}{*3}{40pt}[3pc]

    \titleformat{\section}[block]
    {\normalfont\bfseries\Large}
    {\S\ \thesection.}{.5em}{}[]
    \titlespacing{\section}
    {0pt}{3ex plus .1ex minus .2ex}{3ex plus .1ex minus .2ex}
\usepackage[utf8x]{inputenc}
\usepackage{tikz}
\usepackage{tikz-cd}
\usepackage{wasysym}
\usepackage{pgf}
\usepackage{tcolorbox}
\usepackage{fancyvrb}
\usepackage{listings}

\linespread{1}
%to make the correct symbol for Sha
%\newcommand\cyr{%
%\renewcommand\rmdefault{wncyr}%
%\renewcommand\sfdefault{wncyss}%
%\renewcommand\encodingdefault{OT2}%
%\normalfont \selectfont} \DeclareTextFontCommand{\textcyr}{\cyr}


\DeclareMathOperator{\ab}{ab}
\newcommand{\absgal}{\G_{\bbQ}}
\DeclareMathOperator{\ad}{ad}
\DeclareMathOperator{\adj}{adj}
\DeclareMathOperator{\alg}{alg}
\DeclareMathOperator{\Alt}{Alt}
\DeclareMathOperator{\Ann}{Ann}
\DeclareMathOperator{\arith}{arith}
\DeclareMathOperator{\Aut}{Aut}
\DeclareMathOperator{\Be}{B}
\DeclareMathOperator{\Bd}{Bd}
\DeclareMathOperator{\card}{card}
\DeclareMathOperator{\Char}{char}
\DeclareMathOperator{\csp}{csp}
\DeclareMathOperator{\codim}{codim}
\DeclareMathOperator{\coker}{coker}
\DeclareMathOperator{\coh}{H}
\DeclareMathOperator{\compl}{compl}
\DeclareMathOperator{\conj}{conj}
\DeclareMathOperator{\cont}{cont}
\DeclareMathOperator{\crys}{crys}
\DeclareMathOperator{\Crys}{Crys}
\DeclareMathOperator{\cusp}{cusp}
\DeclareMathOperator{\diag}{diag}
\DeclareMathOperator{\diam}{diam}
\DeclareMathOperator{\Dom}{Dom}
\DeclareMathOperator{\disc}{disc}
\DeclareMathOperator{\dist}{dist}
\DeclareMathOperator{\dR}{dR}
\DeclareMathOperator{\Eis}{Eis}
\DeclareMathOperator{\End}{End}
\DeclareMathOperator{\ev}{ev}
\DeclareMathOperator{\eval}{eval}
\DeclareMathOperator{\Eq}{Eq}
\DeclareMathOperator{\Ext}{Ext}
\DeclareMathOperator{\Fil}{Fil}
\DeclareMathOperator{\Fitt}{Fitt}
\DeclareMathOperator{\Frob}{Frob}
\DeclareMathOperator{\G}{G}
\DeclareMathOperator{\Gal}{Gal}
\DeclareMathOperator{\GL}{GL}
\DeclareMathOperator{\Gr}{Gr}
\DeclareMathOperator{\Graph}{Graph}
\DeclareMathOperator{\GSp}{GSp}
\DeclareMathOperator{\GUn}{GU}
\DeclareMathOperator{\Hom}{Hom}
\DeclareMathOperator{\id}{id}
\DeclareMathOperator{\Id}{Id}
\DeclareMathOperator{\Ik}{Ik}
\DeclareMathOperator{\IM}{Im}
\DeclareMathOperator{\Image}{im}
\DeclareMathOperator{\Ind}{Ind}
\DeclareMathOperator{\Inf}{inf}
\DeclareMathOperator{\Isom}{Isom}
\DeclareMathOperator{\J}{J}
\DeclareMathOperator{\Jac}{Jac}
\DeclareMathOperator{\lcm}{lcm}
\DeclareMathOperator{\length}{length}
\DeclareMathOperator*{\limit}{limit}
\DeclareMathOperator{\Log}{Log}
\DeclareMathOperator{\M}{M}
\DeclareMathOperator{\Mat}{Mat}
\DeclareMathOperator{\N}{N}
\DeclareMathOperator{\Nm}{Nm}
\DeclareMathOperator{\NIk}{N-Ik}
\DeclareMathOperator{\NSK}{N-SK}
\DeclareMathOperator{\new}{new}
\DeclareMathOperator{\obj}{obj}
\DeclareMathOperator{\old}{old}
\DeclareMathOperator{\ord}{ord}
\DeclareMathOperator{\Or}{O}
\DeclareMathOperator{\op}{op}
\DeclareMathOperator{\PGL}{PGL}
\DeclareMathOperator{\PGSp}{PGSp}
\DeclareMathOperator{\rank}{rank}
\DeclareMathOperator{\Ran}{Ran}
\DeclareMathOperator{\Rel}{Rel}
\DeclareMathOperator{\Real}{Re}
\DeclareMathOperator{\RES}{res}
\DeclareMathOperator{\Res}{Res}
%\DeclareMathOperator{\Sha}{\textcyr{Sh}}
\DeclareMathOperator{\Sel}{Sel}
\DeclareMathOperator{\semi}{ss}
\DeclareMathOperator{\sgn}{sign}
\DeclareMathOperator{\SK}{SK}
\DeclareMathOperator{\SL}{SL}
\DeclareMathOperator{\SO}{SO}
\DeclareMathOperator{\Sp}{Sp}
\DeclareMathOperator{\Span}{span}
\DeclareMathOperator{\Spec}{Spec}
\DeclareMathOperator{\spin}{spin}
\DeclareMathOperator{\st}{st}
\DeclareMathOperator{\St}{St}
\DeclareMathOperator{\SUn}{SU}
\DeclareMathOperator{\supp}{supp}
\DeclareMathOperator{\Sup}{sup}
\DeclareMathOperator{\Sym}{Sym}
\DeclareMathOperator{\Tam}{Tam}
\DeclareMathOperator{\tors}{tors}
\DeclareMathOperator{\tr}{tr}
\DeclareMathOperator{\Tr}{Tr}
\DeclareMathOperator{\un}{un}
\DeclareMathOperator{\Un}{U}
\DeclareMathOperator{\val}{val}
\DeclareMathOperator{\vol}{vol}

\DeclareMathOperator{\Sets}{S \mkern1.04mu e \mkern1.04mu t \mkern1.04mu s}
    \newcommand{\cSets}{\scalebox{1.02}{\contour{black}{$\Sets$}}}
    
\DeclareMathOperator{\Groups}{G \mkern1.04mu r \mkern1.04mu o \mkern1.04mu u \mkern1.04mu p \mkern1.04mu s}
    \newcommand{\cGroups}{\scalebox{1.02}{\contour{black}{$\Groups$}}}

\DeclareMathOperator{\TTop}{T \mkern1.04mu o \mkern1.04mu p}
    \newcommand{\cTop}{\scalebox{1.02}{\contour{black}{$\TTop$}}}

\DeclareMathOperator{\Htp}{H \mkern1.04mu t \mkern1.04mu p}
    \newcommand{\cHtp}{\scalebox{1.02}{\contour{black}{$\Htp$}}}

\DeclareMathOperator{\Mod}{M \mkern1.04mu o \mkern1.04mu d}
    \newcommand{\cMod}{\scalebox{1.02}{\contour{black}{$\Mod$}}}

\DeclareMathOperator{\Ab}{A \mkern1.04mu b}
    \newcommand{\cAb}{\scalebox{1.02}{\contour{black}{$\Ab$}}}

\DeclareMathOperator{\Rings}{R \mkern1.04mu i \mkern1.04mu n \mkern1.04mu g \mkern1.04mu s}
    \newcommand{\cRings}{\scalebox{1.02}{\contour{black}{$\Rings$}}}

\DeclareMathOperator{\ComRings}{C \mkern1.04mu o \mkern1.04mu m \mkern1.04mu R \mkern1.04mu i \mkern1.04mu n \mkern1.04mu g \mkern1.04mu s}
    \newcommand{\cComRings}{\scalebox{1.05}{\contour{black}{$\ComRings$}}}

\DeclareMathOperator{\hHom}{H \mkern1.04mu o \mkern1.04mu m}
    \newcommand{\cHom}{\scalebox{1.02}{\contour{black}{$\hHom$}}}

         %  \item $\cGroups$
          %  \item $\cTop$
          %  \item $\cHtp$
          %  \item $\cMod$




\renewcommand{\k}{\kappa}
\newcommand{\Ff}{F_{f}}
%\newcommand{\ts}{\,^{t}\!}


%Mathcal

\newcommand{\cA}{\mathcal{A}}
\newcommand{\cB}{\mathcal{B}}
\newcommand{\cC}{\mathcal{C}}
\newcommand{\cD}{\mathcal{D}}
\newcommand{\cE}{\mathcal{E}}
\newcommand{\cF}{\mathcal{F}}
\newcommand{\cG}{\mathcal{G}}
\newcommand{\cH}{\mathcal{H}}
\newcommand{\cI}{\mathcal{I}}
\newcommand{\cJ}{\mathcal{J}}
\newcommand{\cK}{\mathcal{K}}
\newcommand{\cL}{\mathcal{L}}
\newcommand{\cM}{\mathcal{M}}
\newcommand{\cN}{\mathcal{N}}
\newcommand{\cO}{\mathcal{O}}
\newcommand{\cP}{\mathcal{P}}
\newcommand{\cQ}{\mathcal{Q}}
\newcommand{\cR}{\mathcal{R}}
\newcommand{\cS}{\mathcal{S}}
\newcommand{\cT}{\mathcal{T}}
\newcommand{\cU}{\mathcal{U}}
\newcommand{\cV}{\mathcal{V}}
\newcommand{\cW}{\mathcal{W}}
\newcommand{\cX}{\mathcal{X}}
\newcommand{\cY}{\mathcal{Y}}
\newcommand{\cZ}{\mathcal{Z}}


%mathfrak (missing \fi)

\newcommand{\fa}{\mathfrak{a}}
\newcommand{\fA}{\mathfrak{A}}
\newcommand{\fb}{\mathfrak{b}}
\newcommand{\fB}{\mathfrak{B}}
\newcommand{\fc}{\mathfrak{c}}
\newcommand{\fC}{\mathfrak{C}}
\newcommand{\fd}{\mathfrak{d}}
\newcommand{\fD}{\mathfrak{D}}
\newcommand{\fe}{\mathfrak{e}}
\newcommand{\fE}{\mathfrak{E}}
\newcommand{\ff}{\mathfrak{f}}
\newcommand{\fF}{\mathfrak{F}}
\newcommand{\fg}{\mathfrak{g}}
\newcommand{\fG}{\mathfrak{G}}
\newcommand{\fh}{\mathfrak{h}}
\newcommand{\fH}{\mathfrak{H}}
\newcommand{\fI}{\mathfrak{I}}
\newcommand{\fj}{\mathfrak{j}}
\newcommand{\fJ}{\mathfrak{J}}
\newcommand{\fk}{\mathfrak{k}}
\newcommand{\fK}{\mathfrak{K}}
\newcommand{\fl}{\mathfrak{l}}
\newcommand{\fL}{\mathfrak{L}}
\newcommand{\fm}{\mathfrak{m}}
\newcommand{\fM}{\mathfrak{M}}
\newcommand{\fn}{\mathfrak{n}}
\newcommand{\fN}{\mathfrak{N}}
\newcommand{\fo}{\mathfrak{o}}
\newcommand{\fO}{\mathfrak{O}}
\newcommand{\fp}{\mathfrak{p}}
\newcommand{\fP}{\mathfrak{P}}
\newcommand{\fq}{\mathfrak{q}}
\newcommand{\fQ}{\mathfrak{Q}}
\newcommand{\fr}{\mathfrak{r}}
\newcommand{\fR}{\mathfrak{R}}
\newcommand{\fs}{\mathfrak{s}}
\newcommand{\fS}{\mathfrak{S}}
\newcommand{\ft}{\mathfrak{t}}
\newcommand{\fT}{\mathfrak{T}}
\newcommand{\fu}{\mathfrak{u}}
\newcommand{\fU}{\mathfrak{U}}
\newcommand{\fv}{\mathfrak{v}}
\newcommand{\fV}{\mathfrak{V}}
\newcommand{\fw}{\mathfrak{w}}
\newcommand{\fW}{\mathfrak{W}}
\newcommand{\fx}{\mathfrak{x}}
\newcommand{\fX}{\mathfrak{X}}
\newcommand{\fy}{\mathfrak{y}}
\newcommand{\fY}{\mathfrak{Y}}
\newcommand{\fz}{\mathfrak{z}}
\newcommand{\fZ}{\mathfrak{Z}}


%mathbf
\newcommand{\bfA}{\mathbf{A}}
\newcommand{\bfB}{\mathbf{B}}
\newcommand{\bfC}{\mathbf{C}}
\newcommand{\bfD}{\mathbf{D}}
\newcommand{\bfE}{\mathbf{E}}
\newcommand{\bfF}{\mathbf{F}}
\newcommand{\bfG}{\mathbf{G}}
\newcommand{\bfH}{\mathbf{H}}
\newcommand{\bfI}{\mathbf{I}}
\newcommand{\bfJ}{\mathbf{J}}
\newcommand{\bfK}{\mathbf{K}}
\newcommand{\bfL}{\mathbf{L}}
\newcommand{\bfM}{\mathbf{M}}
\newcommand{\bfN}{\mathbf{N}}
\newcommand{\bfO}{\mathbf{O}}
\newcommand{\bfP}{\mathbf{P}}
\newcommand{\bfQ}{\mathbf{Q}}
\newcommand{\bfR}{\mathbf{R}}
\newcommand{\bfS}{\mathbf{S}}
\newcommand{\bfT}{\mathbf{T}}
\newcommand{\bfU}{\mathbf{U}}
\newcommand{\bfV}{\mathbf{V}}
\newcommand{\bfW}{\mathbf{W}}
\newcommand{\bfX}{\mathbf{X}}
\newcommand{\bfY}{\mathbf{Y}}
\newcommand{\bfZ}{\mathbf{Z}}

\newcommand{\bfa}{\mathbf{a}}
\newcommand{\bfb}{\mathbf{b}}
\newcommand{\bfc}{\mathbf{c}}
\newcommand{\bfd}{\mathbf{d}}
\newcommand{\bfe}{\mathbf{e}}
\newcommand{\bff}{\mathbf{f}}
\newcommand{\bfg}{\mathbf{g}}
\newcommand{\bfh}{\mathbf{h}}
\newcommand{\bfi}{\mathbf{i}}
\newcommand{\bfj}{\mathbf{j}}
\newcommand{\bfk}{\mathbf{k}}
\newcommand{\bfl}{\mathbf{l}}
\newcommand{\bfm}{\mathbf{m}}
\newcommand{\bfn}{\mathbf{n}}
\newcommand{\bfo}{\mathbf{o}}
\newcommand{\bfp}{\mathbf{p}}
\newcommand{\bfq}{\mathbf{q}}
\newcommand{\bfr}{\mathbf{r}}
\newcommand{\bfs}{\mathbf{s}}
\newcommand{\bft}{\mathbf{t}}
\newcommand{\bfu}{\mathbf{u}}
\newcommand{\bfv}{\mathbf{v}}
\newcommand{\bfw}{\mathbf{w}}
\newcommand{\bfx}{\mathbf{x}}
\newcommand{\bfy}{\mathbf{y}}
\newcommand{\bfz}{\mathbf{z}}

%blackboard bold

\newcommand{\bbA}{\mathbb{A}}
\newcommand{\bbB}{\mathbb{B}}
\newcommand{\bbC}{\mathbb{C}}
\newcommand{\bbD}{\mathbb{D}}
\newcommand{\bbE}{\mathbb{E}}
\newcommand{\bbF}{\mathbb{F}}
\newcommand{\bbG}{\mathbb{G}}
\newcommand{\bbH}{\mathbb{H}}
\newcommand{\bbI}{\mathbb{I}}
\newcommand{\bbJ}{\mathbb{J}}
\newcommand{\bbK}{\mathbb{K}}
\newcommand{\bbL}{\mathbb{L}}
\newcommand{\bbM}{\mathbb{M}}
\newcommand{\bbN}{\mathbb{N}}
\newcommand{\bbO}{\mathbb{O}}
\newcommand{\bbP}{\mathbb{P}}
\newcommand{\bbQ}{\mathbb{Q}}
\newcommand{\bbR}{\mathbb{R}}
\newcommand{\bbS}{\mathbb{S}}
\newcommand{\bbT}{\mathbb{T}}
\newcommand{\bbU}{\mathbb{U}}
\newcommand{\bbV}{\mathbb{V}}
\newcommand{\bbW}{\mathbb{W}}
\newcommand{\bbX}{\mathbb{X}}
\newcommand{\bbY}{\mathbb{Y}}
\newcommand{\bbZ}{\mathbb{Z}}
\newcommand{\jota}{\jmath}

\newcommand{\bmat}{\left( \begin{matrix}}
\newcommand{\emat}{\end{matrix} \right)}

\newcommand{\pmat}{\left( \begin{smallmatrix}}
\newcommand{\epmat}{\end{smallmatrix} \right)}

\newcommand{\lat}{\mathscr{L}}
\newcommand{\mat}[4]{\begin{pmatrix}{#1}&{#2}\\{#3}&{#4}\end{pmatrix}}
\newcommand{\ov}[1]{\overline{#1}}
\newcommand{\res}[1]{\underset{#1}{\RES}\,}
\newcommand{\up}{\upsilon}

\newcommand{\tac}{\textasteriskcentered}

%mahesh macros
\newcommand{\tm}{\textrm}

%Comments
\newcommand{\com}[1]{\vspace{5 mm}\par \noindent
\marginpar{\textsc{Comment}} \framebox{\begin{minipage}[c]{0.95
\textwidth} \tt #1 \end{minipage}}\vspace{5 mm}\par}

\newcommand{\Bmu}{\mbox{$\raisebox{-0.59ex}
  {$l$}\hspace{-0.18em}\mu\hspace{-0.88em}\raisebox{-0.98ex}{\scalebox{2}
  {$\color{white}.$}}\hspace{-0.416em}\raisebox{+0.88ex}
  {$\color{white}.$}\hspace{0.46em}$}{}}  %need graphicx and xcolor. this produces blackboard bold mu 

\newcommand{\hooktwoheadrightarrow}{%
  \hookrightarrow\mathrel{\mspace{-15mu}}\rightarrow
}

\makeatletter
\newcommand{\xhooktwoheadrightarrow}[2][]{%
  \lhook\joinrel
  \ext@arrow 0359\rightarrowfill@ {#1}{#2}%
  \mathrel{\mspace{-15mu}}\rightarrow
}
\makeatother

\renewcommand{\geq}{\geqslant}
\renewcommand{\leq}{\leqslant}
\newcommand{\midd}{\hspace{4pt}\middle|\hspace{4pt}}
    
    \newcommand{\bone}{\mathbf{1}}
    \newcommand{\sign}{\mathrm{sign}}
    \newcommand{\eps}{\varepsilon}
    \newcommand{\textui}[1]{\uline{\textit{#1}}}
    
    %\newcommand{\ov}{\overline}
    %\newcommand{\un}{\underline}
    \newcommand{\fin}{\mathrm{fin}}
    
    \newcommand{\chnum}{\titleformat
    {\chapter} % command
    [display] % shape
    {\centering} % format
    {\Huge \color{black} \shadowbox{\thechapter}} % label
    {-0.5em} % sep (space between the number and title)
    {\LARGE \color{black} \underline} % before-code
    }
    
    \newcommand{\chunnum}{\titleformat
    {\chapter} % command
    [display] % shape
    {} % format
    {} % label
    {0em} % sep
    { \begin{flushright} \begin{tabular}{r}  \Huge \color{black}
    } % before-code
    [
    \end{tabular} \end{flushright} \normalsize
    ] % after-code
    }

\newcommand{\nl}{\newline \mbox{}}

\newcommand{\h}[1]{\hspace{#1pt}}

\newcommand{\littletaller}{\mathchoice{\vphantom{\big|}}{}{}{}}
\newcommand\restr[2]{{% we make the whole thing an ordinary symbol
  \left.\kern-\nulldelimiterspace % automatically resize the bar with \right
  #1 % the function
  \littletaller % pretend it's a little taller at normal size
  \right|_{#2} % this is the delimiter
  }}

\newcommand{\mtext}[1]{\hspace{6pt}\text{#1}\hspace{6pt}}

\newcommand{\lnorm}{\left\lVert}
\newcommand{\rnorm}{\right\rVert}

\newcommand{\ds}{\displaystyle}
\newcommand{\ts}{\textstyle}

%This adds a "front cover" page.
%{\thispagestyle{empty}
%\vspace*{\fill}
%\begin{tabular}{l}
%\begin{tabular}{l}
%\includegraphics[scale=0.24]{oxy-logo.png}
%\end{tabular} \\
%\begin{tabular}{l}
%\Large \color{black} Module Theory, Linear Algebra, and Homological Algebra \\ \Large \color{black} Gianluca Crescenzo
%\end{tabular}
%\end{tabular}
%\newpage

\newcommand{\sfrac}[2]{{}^{#1}\mskip -5mu/\mskip -3mu_{#2}}


\makeatletter
\newcommand*{\da@rightarrow}{\mathchar"0\hexnumber@\symAMSa 4B }
\newcommand*{\da@leftarrow}{\mathchar"0\hexnumber@\symAMSa 4C }
\newcommand*{\xdashrightarrow}[2][]{%
  \mathrel{%
    \mathpalette{\da@xarrow{#1}{#2}{}\da@rightarrow{\,}{}}{}%
  }%
}
\newcommand{\xdashleftarrow}[2][]{%
  \mathrel{%
    \mathpalette{\da@xarrow{#1}{#2}\da@leftarrow{}{}{\,}}{}%
  }%
}
\newcommand*{\da@xarrow}[7]{%
  % #1: below
  % #2: above
  % #3: arrow left
  % #4: arrow right
  % #5: space left 
  % #6: space right
  % #7: math style 
  \sbox0{$\ifx#7\scriptstyle\scriptscriptstyle\else\scriptstyle\fi#5#1#6\m@th$}%
  \sbox2{$\ifx#7\scriptstyle\scriptscriptstyle\else\scriptstyle\fi#5#2#6\m@th$}%
  \sbox4{$#7\dabar@\m@th$}%
  \dimen@=\wd0 %
  \ifdim\wd2 >\dimen@
    \dimen@=\wd2 %   
  \fi
  \count@=2 %
  \def\da@bars{\dabar@\dabar@}%
  \@whiledim\count@\wd4<\dimen@\do{%
    \advance\count@\@ne
    \expandafter\def\expandafter\da@bars\expandafter{%
      \da@bars
      \dabar@ 
    }%
  }%  
  \mathrel{#3}%
  \mathrel{%   
    \mathop{\da@bars}\limits
    \ifx\\#1\\%
    \else
      _{\copy0}%
    \fi
    \ifx\\#2\\%
    \else
      ^{\copy2}%
    \fi
  }%   
  \mathrel{#4}%
}
\makeatother


\begin{document}
\begin{center}
{\large Econ 272 \\[0.1in]Homework 5 \\[0.1in]}
{Name:} {\underline{Gianluca Crescenzo\hspace*{2in}}}\\[0.15in]
\end{center}
\vspace{4pt}
%%%%%%%%%%%%%%%%%%%%%%%%%%%%%%%%%%%%%%%%%%%%%%%%%%%%%%%%%%%%%
\begin{question}
    We will test the hypothesis of whether married women earn more or less on the labour market.
    \begin{enumerate}[label = (\alph*),itemsep=1pt,topsep=3pt]
        \item Estimate the following fully saturated regression specification:
            \begin{equation*}
            \begin{split}
                \text{Wage}_i = \beta_0 + \beta_1 \text{Married}_i + \beta_2 \text{Female}_i + \beta_3 (\text{Female}_i\cdot \text{Married}_i) + \epsilon_i. 
            \end{split}
            \end{equation*}
        Use your Stata estimates to interpret $\beta_0$ and $\beta_3$.
            {\color{blue} \begin{solution}
                Running the regression gives:
                \begin{Verbatim}
--------------------------------------------------------------------------------
          wage | Coefficient  Std. err.      t    P>|t|     [95% conf. interval]
---------------+----------------------------------------------------------------
       married |   42988.52   1347.147    31.91   0.000      40348.1    45628.94
        female |  -12441.77   1507.421    -8.25   0.000    -15396.33   -9487.216
married_female |  -32428.07   1911.748   -16.96   0.000     -36175.1   -28681.03
         _cons |   76068.82   1091.899    69.67   0.000     73928.69    78208.94
--------------------------------------------------------------------------------                    
                \end{Verbatim}
                $\widehat{\beta_0}$ is the average wage of single men. $\widehat{\beta_3}$ is how much the effect of marriage for women differs from the effect of marriage for men.
            \end{solution}}

        \item An important omitted variable is the time worked by an individual. Define $\text{WkWork}_i$ as weeks worked in the year. Discuss how the estimate of $\beta_1$ would change if you include $\text{WkWork}_i$ in your estimation in $(a)$. Consider the two cases: one where $\text{Corr}(\text{Married}_i, \text{WkWork}_i) > 0$ in the same, and second where $\text{Corr}(\text{Married}_i, \text{WkWork}_i) < 0$. Discuss how the estimated $\beta_1$ would change in each case.
            {\color{blue} \begin{solution}
                If there is a positive correlation, omitting $\text{WkWork}_i$ biases $\beta_1$ upward. Adding $\text{WkWork}_i$ to the regression will reduce the estimated marriage effect. If there is a negative correlation, omitting $\text{WkWork}_i$ biases $\beta_1$ downward. Adding $\text{WkWork}_i$ will increase the estimated marriage effect.
            \end{solution}}

        \item Now estimate this in the data by included weeks worked in your estimation (wkswork1). Would you say the estimate of $\beta_1$ in (a) was biased upwards, or downwards? What does this mean about the sign of $\text{Corr}(\text{Married}_i, \text{WkWork}_i)$? Verify this in the data.
            {\color{blue} \begin{solution}
                Running the regression gives:
                \begin{Verbatim}
--------------------------------------------------------------------------------
          wage | Coefficient  Std. err.      t    P>|t|     [95% conf. interval]
---------------+----------------------------------------------------------------
       married |   36375.78   1321.461    27.53   0.000      33785.7    38965.85
        female |  -12671.93   1471.949    -8.61   0.000    -15556.96   -9786.898
married_female |  -28534.23   1868.226   -15.27   0.000    -32195.97    -24872.5
      wkswork1 |   1904.327   36.28418    52.48   0.000      1833.21    1975.444
         _cons |  -10659.59   1966.592    -5.42   0.000    -14514.12   -6805.057
--------------------------------------------------------------------------------
                
                \end{Verbatim}
                So part (a) was biased upwards. This also means $\text{Corr}(\text{Married}_i, \text{WkWork}_i) > 0$.
            \end{solution}}

        \item A second omitted variable is whether the worker has a college degree or not. How do the $\beta_2$ and $\beta_3$ coefficients change upon including this variable? Use your regression coefficients and the omitted variable bias formula to infer whether women in the labor market are more likely to have a college degree.
            {\color{blue} \begin{solution}
                
            \end{solution}}

        \item Expand your set of covariates to include the workers gender (female), age (age), and age-squared (sq\_age). Also add in controls for worker location, geography, and industry. You can use the global macros for this. Report and interpret the coefficient estimates for each of $\widehat{\beta_1}$, $\widehat{\beta_2}$, and $\widehat{\beta_3}$
            {\color{blue} \begin{solution}
                Running the regressions gives:
                \begin{Verbatim}
---------------------------------------------------------------------------------
           wage | Coefficient  Std. err.      t    P>|t|     [95% conf. interval]
----------------+----------------------------------------------------------------
        married |   30408.47   1330.407    22.86   0.000     27800.86    33016.07
         female |  -19937.97   1480.498   -13.47   0.000    -22839.76   -17036.19
 married_female |  -26237.66   1839.744   -14.26   0.000    -29843.57   -22631.75

                ...

          _cons |  -94117.24    7603.57   -12.38   0.000    -109020.3   -79214.19
---------------------------------------------------------------------------------
                \end{Verbatim}     
                For $\beta_1$: holding all other variables constant, a married man earns about \$30,408 more per year than a comparable single man. For $\beta_2$: holding all else constant, a single woman earns about \$19,938 less per year than a comparable single man. For $\beta_3$: holding all else constant, the wage effect of marriage for women is \$26,238 lower than the wage effect of marriage for men.
            \end{solution}}

        
        \item Use the \textit{test} command to Stata to test the following null hypothesis: $H_0: \beta_1 + \beta_3 = 0$. Can you reject the null at the $1\%$ level? What information is conveyed by $\beta_1 + \beta_3$?
            {\color{blue} \begin{solution}
                Using the test command gave:
                \begin{Verbatim}
                    ( 1)  married + married_female = 0

                    F(  1, 56417) =    9.95
                            Prob > F =    0.0016             
                \end{Verbatim}
            Since the $p$-value is below 0.01, we can reject the null hypothesis. $\beta_1 + \beta_3$ gives the overall effect of marriage for women.
            \end{solution}}
        
        \item Now re-estimate the same regression, but after taking the natural log of the outcome variable. Based on your results, is there a gap in earnings for married females, relative to married males?
            {\color{blue} \begin{solution}
                Running the regression gives:
                \begin{Verbatim}
---------------------------------------------------------------------------------
        ln_wage | Coefficient  Std. err.      t    P>|t|     [95% conf. interval]
----------------+----------------------------------------------------------------
        married |   .2948742   .0101798    28.97   0.000     .2749216    .3148267
         female |  -.2099117   .0113078   -18.56   0.000    -.2320751   -.1877482
 married_female |  -.2258532   .0140448   -16.08   0.000    -.2533811   -.1983253

            ...

          _cons |   7.529791   .0593628   126.84   0.000     7.413439    7.646142
---------------------------------------------------------------------------------
                
                \end{Verbatim}
            There is a large gap in earning for married females relative to married males. 
            \end{solution}}
        
        \item Estimate the following regression:
            \begin{equation*}
            \begin{split}
                \ln\text{Wage}_i = \beta_0 + \beta_1 \text{Female}_i + \beta_2 \text{Age}_i + \beta_3 \text{Married}_i + \beta_4 (\text{Age}_i \cdot \text{Married}_i) + \beta_5 (\text{Age}_i \cdot \text{Female}_i)+ \delta X_i + \epsilon_i.
            \end{split}
            \end{equation*}
        In (g) include all the controls from (d), as well as College, but exclude age-squared, and the interaction term $\text{Married}_i \cdot \text{Female}_i$. Interpret the $\beta_1$ and $\beta_4$ coefficients. Based on the coefficient estimates for $\beta_4$ and $\beta_5$, are there differential returns for female workers for an additional year of experience (assuming age and experience to be equivalent)?
            {\color{blue} \begin{solution}
                Running the regression gives:
                \begin{Verbatim}
---------------------------------------------------------------------------------
        ln_wage | Coefficient  Std. err.      t    P>|t|     [95% conf. interval]
----------------+----------------------------------------------------------------
         female |   -.189308   .0265723    -7.12   0.000      -.24139    -.137226
            age |   .0097006   .0005617    17.27   0.000     .0085997    .0108014
        married |   .2739934   .0267897    10.23   0.000     .2214854    .3265014
    age_married |  -.0032269   .0006104    -5.29   0.000    -.0044233   -.0020305
     age_female |  -.0035948   .0005943    -6.05   0.000    -.0047595     -.00243
       
                ...

          _cons |   8.146538   .0314906   258.70   0.000     8.084816    8.208259
---------------------------------------------------------------------------------   
                \end{Verbatim}

                $\beta_1$ shows that a female worker has a wage that is 18.9\% lower than a comparable male. $\beta_4$ shows how the return to an additional year of age changes when a worker is married. Since $\beta_5 < 0$, there are differential returns, as each additional year of experience raises female log wages less than it does for similar males.
            \end{solution}}

        \item Based on the coefficients in (g), how much more would a female worker expect to earn from working 1 additional year? Using the test command in Stata, can you reject the null hypothesis: $H_0:\beta_2 + \beta_4 = 0$ with 95\% confidence?
            {\color{blue} \begin{solution}
                Note that $\beta_2 + \beta_5 = 0.0061058$. For an unmarried female, each additional year of experience is associated with about a 0.61\% increase in wages. The \textit{test} command gives:
            \begin{Verbatim}
                    F(  1, 51267) =  182.97
                            Prob > F =    0.0000
            \end{Verbatim}
            \end{solution}}
            Thus we can reject the null hypothesis.
    \end{enumerate}
\end{question}
%%%%%%%%%%%%%%%%%%%%%%%%%%%%%%%%%%%%%%%%%%%%%%%%%%%%%%%%%%%%%
\begin{question}
    We want to assess whether women in the labor force in metropolitan areas are more or less likely to finish college. Consider the fully saturated specification:
        \begin{equation*}
        \begin{split}
            P(\text{College}_i = 1) = \beta_0 + \beta_1 \text{Female}_i + \beta_2 \text{Metro}_i + \beta_3 \text{Female}_i \cdot \text{Metro}_i + \epsilon_i.
        \end{split}
        \end{equation*}
    Estimate this in the data to assess whether women in the labor market in metropolitan areas have a higher or lower likelihood of finishing college. What information is provided by the coefficients $\beta_0$ and $\beta_2$?
\end{question}
    {\color{blue} \begin{solution}
        Stata gives:
            \begin{Verbatim}
------------------------------------------------------------------------------
     high_ed | Coefficient  Std. err.      t    P>|t|     [95% conf. interval]
-------------+----------------------------------------------------------------
      female |   .0678869   .0111446     6.09   0.000     .0460434    .0897305
       metro |   .0289121   .0028355    10.20   0.000     .0233546    .0344696
female_metro |  -.0076668   .0040858    -1.88   0.061     -.015675    .0003415
       _cons |   .2687816   .0077311    34.77   0.000     .2536286    .2839346
------------------------------------------------------------------------------
            \end{Verbatim}
        $\beta_0$ is the predicted probability that $\text{College}_i = 1$ for men who are not in a metropolitan area. $\beta_2$ is how the predicted probability that $\text{College}_i = 1$ changes if a man instead lives in a metropolitan area.
    \end{solution}}
%%%%%%%%%%%%%%%%%%%%%%%%%%%%%%%%%%%%%%%%%%%%%%%%%%%%%%%%%%%%%
\begin{question}
    Consider the government thinking about a stimulus plan to boost the economy. A key part of the stimulus plan is to issue checks to households with a high propensity to consume. The government considers testing the hypothesis that families with more children have greater spending propensity. The population regression functions is:
        \begin{equation*}
        \begin{split}
            \ln(\text{Consumption}_i) = \beta_0 + \beta_1 \ln(\text{Income}_i) 
            &+ \beta_2 \ln(\text{Income}_i)\cdot \text{OneChild}_i \\
            &+ \beta_3 \ln(\text{Income}_i)\cdot \text{TwoChild}_i \\
            &+ \beta_4 \ln(\text{Income}_i)\cdot \text{ThreeChild}_i \\
            &+ \beta_5 \text{OneChild}_i \\
            &+ \beta_5 \text{TwoChild}_i \\
            &+ \beta_5 \text{ThreeChild}_i  + \delta X_i + \epsilon_i.\\
        \end{split}
        \end{equation*}
    $\text{Income}_i$ refers to the annual income of household $i$, and $\text{Consumption}_i$ is the households consumption. OneChild, TwoChild, and ThreeChild are binary variables. Assume the following regression coefficients:
        \begin{equation*}
        \begin{split}
            \widehat{\beta_1} = 0.11, \h5 \text{SE}(\widehat{\beta_1}) = .089 \\
            \widehat{\beta_2} = 0.31, \h5 \text{SE}(\widehat{\beta_2}) = .067 \\
            \widehat{\beta_3} = 0.61, \h5 \text{SE}(\widehat{\beta_3}) = .143 \\
            \widehat{\beta_4} = 0.21, \h5 \text{SE}(\widehat{\beta_4}) = .158 \\
        \end{split}
        \end{equation*}
    \begin{enumerate}[label = (\alph*),itemsep=1pt,topsep=3pt]
        \item Do changes in income have a large or small impact on consumption changes for families with no children?
            {\color{blue} \begin{solution}
                For every 1\% increase in income, there is a 0.11\% increase in consumption. So income has a very small impact on consumption changes for families with no children.
            \end{solution}}

        \item Interpret the $\beta_2$ coefficient.
            {\color{blue} \begin{solution}
                Having exactly one child increases the consumption-income elasticity by 0.31 relative to households with no children.
            \end{solution}}

        \item Based on the above evidence, as a policymaker, which type of families would you target when considering a stimulus package?
            {\color{blue} \begin{solution}
                Since $\widehat{\beta_3}$, corresponding to families with two children, has the highest consumption-income elasticity, I would target those families when considering a stimulus package.
            \end{solution}}
    \end{enumerate}
\end{question}
%%%%%%%%%%%%%%%%%%%%%%%%%%%%%%%%%%%%%%%%%%%%%%%%%%%%%%%%%%%%%

\end{document}