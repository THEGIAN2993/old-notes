\documentclass[11pt,twoside,openany]{memoir}
%\usepackage{mlmodern}
%\usepackage{tgpagella} % text only
%\usepackage{mathpazo}  % math & text
\usepackage[T1]{fontenc}
\usepackage[hidelinks]{hyperref}
\usepackage{amsmath}
\usepackage{amsthm}
\usepackage{amssymb}
%\usepackage{mathtools}
%\renewcommand*{\mathbf}[1]{\varmathbb{#1}}
%\usepackage{newpxtext}
%\usepackage{eulerpx}
%\usepackage{eucal}
\usepackage{datetime}
    \newdateformat{specialdate}{\THEYEAR\ \monthname\ \THEDAY}
\usepackage[margin=1in]{geometry}
\usepackage{fancyhdr}
    \fancyhf{}
    \pagestyle{fancy}
    \cfoot{\scriptsize \thepage}
    \fancyhead[R]{\scriptsize \rightmark}
    \fancyhead[L]{\scriptsize \leftmark}
    \renewcommand{\headrulewidth}{0pt}
    \renewcommand{\footrulewidth}{0pt} % if you also want to remove the footer rule
\usepackage{thmtools}
    \declaretheoremstyle[
        spaceabove=10pt,
        spacebelow=10pt,
        headfont=\normalfont\bfseries,
        notefont=\mdseries, notebraces={(}{)},
        bodyfont=\normalfont,
        postheadspace=0.5em
        %qed=\qedsymbol
        ]{defs}

    \declaretheoremstyle[ 
        spaceabove=10pt, % space above the theorem
        spacebelow=10pt,
        headfont=\normalfont\bfseries,
        bodyfont=\normalfont\itshape,
        postheadspace=0.5em
        ]{thmstyle}
    
    \declaretheorem[
        style=thmstyle,
        numberwithin=section
    ]{theorem}

    \declaretheorem[
        style=thmstyle,
        sibling=theorem,
    ]{proposition}

    \declaretheorem[
        style=thmstyle,
        sibling=theorem,
    ]{lemma}

    \declaretheorem[
        style=thmstyle,
        sibling=theorem,
    ]{corollary}

    \declaretheorem[
        numberwithin=section,
        style=defs,
    ]{example}

    \declaretheorem[
        numberwithin=section,
        style=defs,
    ]{definition}

    \declaretheorem[
        style=defs,
        numbered=unless unique,
    ]{exercise}

    \declaretheorem[
        numbered=unless unique,
        shaded={rulecolor=black,
    rulewidth=1pt, bgcolor={rgb}{1,1,1}}
    ]{axiom}

    \declaretheorem[numberwithin=section,style=defs]{note}
    \declaretheorem[style=defs]{question}
    \declaretheorem[numbered=no,style=defs]{recall}
    \declaretheorem[numbered=no,style=remark]{answer}
    \declaretheorem[numbered=no,style=remark]{solution}

    \declaretheorem[numbered=no,style=defs]{remark}
\usepackage{enumitem}
\usepackage{titlesec}
    \titleformat{\chapter}[display]
    {\bfseries\LARGE\raggedright}
    {Chapter {\thechapter}}
    {1ex minus .1ex}
    {\Huge}
    \titlespacing{\chapter}
    {3pc}{*3}{40pt}[3pc]

    \titleformat{\section}[block]
    {\normalfont\bfseries\Large}
    {\S\ \thesection.}{.5em}{}[]
    \titlespacing{\section}
    {0pt}{3ex plus .1ex minus .2ex}{3ex plus .1ex minus .2ex}
\usepackage[utf8x]{inputenc}
\usepackage{tikz}
\usepackage{tikz-cd}
\usepackage{wasysym}
\renewcommand{\int}{\varint}
\usepackage{fancyvrb}
\usepackage{tcolorbox}
\usepackage{listings}

\linespread{1}
%to make the correct symbol for Sha
%\newcommand\cyr{%
%\renewcommand\rmdefault{wncyr}%
%\renewcommand\sfdefault{wncyss}%
%\renewcommand\encodingdefault{OT2}%
%\normalfont \selectfont} \DeclareTextFontCommand{\textcyr}{\cyr}


\DeclareMathOperator{\ab}{ab}
\newcommand{\absgal}{\G_{\bbQ}}
\DeclareMathOperator{\ad}{ad}
\DeclareMathOperator{\adj}{adj}
\DeclareMathOperator{\alg}{alg}
\DeclareMathOperator{\Alt}{Alt}
\DeclareMathOperator{\Ann}{Ann}
\DeclareMathOperator{\arith}{arith}
\DeclareMathOperator{\Aut}{Aut}
\DeclareMathOperator{\Be}{B}
\DeclareMathOperator{\Bd}{Bd}
\DeclareMathOperator{\card}{card}
\DeclareMathOperator{\Char}{char}
\DeclareMathOperator{\csp}{csp}
\DeclareMathOperator{\codim}{codim}
\DeclareMathOperator{\coker}{coker}
\DeclareMathOperator{\coh}{H}
\DeclareMathOperator{\compl}{compl}
\DeclareMathOperator{\conj}{conj}
\DeclareMathOperator{\cont}{cont}
\DeclareMathOperator{\crys}{crys}
\DeclareMathOperator{\Crys}{Crys}
\DeclareMathOperator{\cusp}{cusp}
\DeclareMathOperator{\diag}{diag}
\DeclareMathOperator{\diam}{diam}
\DeclareMathOperator{\Dom}{Dom}
\DeclareMathOperator{\disc}{disc}
\DeclareMathOperator{\dist}{dist}
\DeclareMathOperator{\dR}{dR}
\DeclareMathOperator{\Eis}{Eis}
\DeclareMathOperator{\End}{End}
\DeclareMathOperator{\ev}{ev}
\DeclareMathOperator{\eval}{eval}
\DeclareMathOperator{\Eq}{Eq}
\DeclareMathOperator{\Ext}{Ext}
\DeclareMathOperator{\Fil}{Fil}
\DeclareMathOperator{\Fitt}{Fitt}
\DeclareMathOperator{\Frob}{Frob}
\DeclareMathOperator{\G}{G}
\DeclareMathOperator{\Gal}{Gal}
\DeclareMathOperator{\GL}{GL}
\DeclareMathOperator{\Gr}{Gr}
\DeclareMathOperator{\Graph}{Graph}
\DeclareMathOperator{\GSp}{GSp}
\DeclareMathOperator{\GUn}{GU}
\DeclareMathOperator{\Hom}{Hom}
\DeclareMathOperator{\id}{id}
\DeclareMathOperator{\Id}{Id}
\DeclareMathOperator{\Ik}{Ik}
\DeclareMathOperator{\IM}{Im}
\DeclareMathOperator{\Image}{im}
\DeclareMathOperator{\Ind}{Ind}
\DeclareMathOperator{\Inf}{inf}
\DeclareMathOperator{\Isom}{Isom}
\DeclareMathOperator{\J}{J}
\DeclareMathOperator{\Jac}{Jac}
\DeclareMathOperator{\lcm}{lcm}
\DeclareMathOperator{\length}{length}
\DeclareMathOperator*{\limit}{limit}
\DeclareMathOperator{\Log}{Log}
\DeclareMathOperator{\M}{M}
\DeclareMathOperator{\Mat}{Mat}
\DeclareMathOperator{\N}{N}
\DeclareMathOperator{\Nm}{Nm}
\DeclareMathOperator{\NIk}{N-Ik}
\DeclareMathOperator{\NSK}{N-SK}
\DeclareMathOperator{\new}{new}
\DeclareMathOperator{\obj}{obj}
\DeclareMathOperator{\old}{old}
\DeclareMathOperator{\ord}{ord}
\DeclareMathOperator{\Or}{O}
\DeclareMathOperator{\op}{op}
\DeclareMathOperator{\PGL}{PGL}
\DeclareMathOperator{\PGSp}{PGSp}
\DeclareMathOperator{\rank}{rank}
\DeclareMathOperator{\Ran}{Ran}
\DeclareMathOperator{\Rel}{Rel}
\DeclareMathOperator{\Real}{Re}
\DeclareMathOperator{\RES}{res}
\DeclareMathOperator{\Res}{Res}
%\DeclareMathOperator{\Sha}{\textcyr{Sh}}
\DeclareMathOperator{\Sel}{Sel}
\DeclareMathOperator{\semi}{ss}
\DeclareMathOperator{\sgn}{sign}
\DeclareMathOperator{\SK}{SK}
\DeclareMathOperator{\SL}{SL}
\DeclareMathOperator{\SO}{SO}
\DeclareMathOperator{\Sp}{Sp}
\DeclareMathOperator{\Span}{span}
\DeclareMathOperator{\Spec}{Spec}
\DeclareMathOperator{\spin}{spin}
\DeclareMathOperator{\st}{st}
\DeclareMathOperator{\St}{St}
\DeclareMathOperator{\SUn}{SU}
\DeclareMathOperator{\supp}{supp}
\DeclareMathOperator{\Sup}{sup}
\DeclareMathOperator{\Sym}{Sym}
\DeclareMathOperator{\Tam}{Tam}
\DeclareMathOperator{\tors}{tors}
\DeclareMathOperator{\tr}{tr}
\DeclareMathOperator{\Tr}{Tr}
\DeclareMathOperator{\un}{un}
\DeclareMathOperator{\Un}{U}
\DeclareMathOperator{\val}{val}
\DeclareMathOperator{\vol}{vol}

\DeclareMathOperator{\Sets}{S \mkern1.04mu e \mkern1.04mu t \mkern1.04mu s}
    \newcommand{\cSets}{\scalebox{1.02}{\contour{black}{$\Sets$}}}
    
\DeclareMathOperator{\Groups}{G \mkern1.04mu r \mkern1.04mu o \mkern1.04mu u \mkern1.04mu p \mkern1.04mu s}
    \newcommand{\cGroups}{\scalebox{1.02}{\contour{black}{$\Groups$}}}

\DeclareMathOperator{\TTop}{T \mkern1.04mu o \mkern1.04mu p}
    \newcommand{\cTop}{\scalebox{1.02}{\contour{black}{$\TTop$}}}

\DeclareMathOperator{\Htp}{H \mkern1.04mu t \mkern1.04mu p}
    \newcommand{\cHtp}{\scalebox{1.02}{\contour{black}{$\Htp$}}}

\DeclareMathOperator{\Mod}{M \mkern1.04mu o \mkern1.04mu d}
    \newcommand{\cMod}{\scalebox{1.02}{\contour{black}{$\Mod$}}}

\DeclareMathOperator{\Ab}{A \mkern1.04mu b}
    \newcommand{\cAb}{\scalebox{1.02}{\contour{black}{$\Ab$}}}

\DeclareMathOperator{\Rings}{R \mkern1.04mu i \mkern1.04mu n \mkern1.04mu g \mkern1.04mu s}
    \newcommand{\cRings}{\scalebox{1.02}{\contour{black}{$\Rings$}}}

\DeclareMathOperator{\ComRings}{C \mkern1.04mu o \mkern1.04mu m \mkern1.04mu R \mkern1.04mu i \mkern1.04mu n \mkern1.04mu g \mkern1.04mu s}
    \newcommand{\cComRings}{\scalebox{1.05}{\contour{black}{$\ComRings$}}}

\DeclareMathOperator{\hHom}{H \mkern1.04mu o \mkern1.04mu m}
    \newcommand{\cHom}{\scalebox{1.02}{\contour{black}{$\hHom$}}}

         %  \item $\cGroups$
          %  \item $\cTop$
          %  \item $\cHtp$
          %  \item $\cMod$




\renewcommand{\k}{\kappa}
\newcommand{\Ff}{F_{f}}
%\newcommand{\ts}{\,^{t}\!}


%Mathcal

\newcommand{\cA}{\mathcal{A}}
\newcommand{\cB}{\mathcal{B}}
\newcommand{\cC}{\mathcal{C}}
\newcommand{\cD}{\mathcal{D}}
\newcommand{\cE}{\mathcal{E}}
\newcommand{\cF}{\mathcal{F}}
\newcommand{\cG}{\mathcal{G}}
\newcommand{\cH}{\mathcal{H}}
\newcommand{\cI}{\mathcal{I}}
\newcommand{\cJ}{\mathcal{J}}
\newcommand{\cK}{\mathcal{K}}
\newcommand{\cL}{\mathcal{L}}
\newcommand{\cM}{\mathcal{M}}
\newcommand{\cN}{\mathcal{N}}
\newcommand{\cO}{\mathcal{O}}
\newcommand{\cP}{\mathcal{P}}
\newcommand{\cQ}{\mathcal{Q}}
\newcommand{\cR}{\mathcal{R}}
\newcommand{\cS}{\mathcal{S}}
\newcommand{\cT}{\mathcal{T}}
\newcommand{\cU}{\mathcal{U}}
\newcommand{\cV}{\mathcal{V}}
\newcommand{\cW}{\mathcal{W}}
\newcommand{\cX}{\mathcal{X}}
\newcommand{\cY}{\mathcal{Y}}
\newcommand{\cZ}{\mathcal{Z}}


%mathfrak (missing \fi)

\newcommand{\fa}{\mathfrak{a}}
\newcommand{\fA}{\mathfrak{A}}
\newcommand{\fb}{\mathfrak{b}}
\newcommand{\fB}{\mathfrak{B}}
\newcommand{\fc}{\mathfrak{c}}
\newcommand{\fC}{\mathfrak{C}}
\newcommand{\fd}{\mathfrak{d}}
\newcommand{\fD}{\mathfrak{D}}
\newcommand{\fe}{\mathfrak{e}}
\newcommand{\fE}{\mathfrak{E}}
\newcommand{\ff}{\mathfrak{f}}
\newcommand{\fF}{\mathfrak{F}}
\newcommand{\fg}{\mathfrak{g}}
\newcommand{\fG}{\mathfrak{G}}
\newcommand{\fh}{\mathfrak{h}}
\newcommand{\fH}{\mathfrak{H}}
\newcommand{\fI}{\mathfrak{I}}
\newcommand{\fj}{\mathfrak{j}}
\newcommand{\fJ}{\mathfrak{J}}
\newcommand{\fk}{\mathfrak{k}}
\newcommand{\fK}{\mathfrak{K}}
\newcommand{\fl}{\mathfrak{l}}
\newcommand{\fL}{\mathfrak{L}}
\newcommand{\fm}{\mathfrak{m}}
\newcommand{\fM}{\mathfrak{M}}
\newcommand{\fn}{\mathfrak{n}}
\newcommand{\fN}{\mathfrak{N}}
\newcommand{\fo}{\mathfrak{o}}
\newcommand{\fO}{\mathfrak{O}}
\newcommand{\fp}{\mathfrak{p}}
\newcommand{\fP}{\mathfrak{P}}
\newcommand{\fq}{\mathfrak{q}}
\newcommand{\fQ}{\mathfrak{Q}}
\newcommand{\fr}{\mathfrak{r}}
\newcommand{\fR}{\mathfrak{R}}
\newcommand{\fs}{\mathfrak{s}}
\newcommand{\fS}{\mathfrak{S}}
\newcommand{\ft}{\mathfrak{t}}
\newcommand{\fT}{\mathfrak{T}}
\newcommand{\fu}{\mathfrak{u}}
\newcommand{\fU}{\mathfrak{U}}
\newcommand{\fv}{\mathfrak{v}}
\newcommand{\fV}{\mathfrak{V}}
\newcommand{\fw}{\mathfrak{w}}
\newcommand{\fW}{\mathfrak{W}}
\newcommand{\fx}{\mathfrak{x}}
\newcommand{\fX}{\mathfrak{X}}
\newcommand{\fy}{\mathfrak{y}}
\newcommand{\fY}{\mathfrak{Y}}
\newcommand{\fz}{\mathfrak{z}}
\newcommand{\fZ}{\mathfrak{Z}}


%mathbf
\newcommand{\bfA}{\mathbf{A}}
\newcommand{\bfB}{\mathbf{B}}
\newcommand{\bfC}{\mathbf{C}}
\newcommand{\bfD}{\mathbf{D}}
\newcommand{\bfE}{\mathbf{E}}
\newcommand{\bfF}{\mathbf{F}}
\newcommand{\bfG}{\mathbf{G}}
\newcommand{\bfH}{\mathbf{H}}
\newcommand{\bfI}{\mathbf{I}}
\newcommand{\bfJ}{\mathbf{J}}
\newcommand{\bfK}{\mathbf{K}}
\newcommand{\bfL}{\mathbf{L}}
\newcommand{\bfM}{\mathbf{M}}
\newcommand{\bfN}{\mathbf{N}}
\newcommand{\bfO}{\mathbf{O}}
\newcommand{\bfP}{\mathbf{P}}
\newcommand{\bfQ}{\mathbf{Q}}
\newcommand{\bfR}{\mathbf{R}}
\newcommand{\bfS}{\mathbf{S}}
\newcommand{\bfT}{\mathbf{T}}
\newcommand{\bfU}{\mathbf{U}}
\newcommand{\bfV}{\mathbf{V}}
\newcommand{\bfW}{\mathbf{W}}
\newcommand{\bfX}{\mathbf{X}}
\newcommand{\bfY}{\mathbf{Y}}
\newcommand{\bfZ}{\mathbf{Z}}

\newcommand{\bfa}{\mathbf{a}}
\newcommand{\bfb}{\mathbf{b}}
\newcommand{\bfc}{\mathbf{c}}
\newcommand{\bfd}{\mathbf{d}}
\newcommand{\bfe}{\mathbf{e}}
\newcommand{\bff}{\mathbf{f}}
\newcommand{\bfg}{\mathbf{g}}
\newcommand{\bfh}{\mathbf{h}}
\newcommand{\bfi}{\mathbf{i}}
\newcommand{\bfj}{\mathbf{j}}
\newcommand{\bfk}{\mathbf{k}}
\newcommand{\bfl}{\mathbf{l}}
\newcommand{\bfm}{\mathbf{m}}
\newcommand{\bfn}{\mathbf{n}}
\newcommand{\bfo}{\mathbf{o}}
\newcommand{\bfp}{\mathbf{p}}
\newcommand{\bfq}{\mathbf{q}}
\newcommand{\bfr}{\mathbf{r}}
\newcommand{\bfs}{\mathbf{s}}
\newcommand{\bft}{\mathbf{t}}
\newcommand{\bfu}{\mathbf{u}}
\newcommand{\bfv}{\mathbf{v}}
\newcommand{\bfw}{\mathbf{w}}
\newcommand{\bfx}{\mathbf{x}}
\newcommand{\bfy}{\mathbf{y}}
\newcommand{\bfz}{\mathbf{z}}

%blackboard bold

\newcommand{\bbA}{\mathbb{A}}
\newcommand{\bbB}{\mathbb{B}}
\newcommand{\bbC}{\mathbb{C}}
\newcommand{\bbD}{\mathbb{D}}
\newcommand{\bbE}{\mathbb{E}}
\newcommand{\bbF}{\mathbb{F}}
\newcommand{\bbG}{\mathbb{G}}
\newcommand{\bbH}{\mathbb{H}}
\newcommand{\bbI}{\mathbb{I}}
\newcommand{\bbJ}{\mathbb{J}}
\newcommand{\bbK}{\mathbb{K}}
\newcommand{\bbL}{\mathbb{L}}
\newcommand{\bbM}{\mathbb{M}}
\newcommand{\bbN}{\mathbb{N}}
\newcommand{\bbO}{\mathbb{O}}
\newcommand{\bbP}{\mathbb{P}}
\newcommand{\bbQ}{\mathbb{Q}}
\newcommand{\bbR}{\mathbb{R}}
\newcommand{\bbS}{\mathbb{S}}
\newcommand{\bbT}{\mathbb{T}}
\newcommand{\bbU}{\mathbb{U}}
\newcommand{\bbV}{\mathbb{V}}
\newcommand{\bbW}{\mathbb{W}}
\newcommand{\bbX}{\mathbb{X}}
\newcommand{\bbY}{\mathbb{Y}}
\newcommand{\bbZ}{\mathbb{Z}}
\newcommand{\jota}{\jmath}

\newcommand{\bmat}{\left( \begin{matrix}}
\newcommand{\emat}{\end{matrix} \right)}

\newcommand{\pmat}{\left( \begin{smallmatrix}}
\newcommand{\epmat}{\end{smallmatrix} \right)}

\newcommand{\lat}{\mathscr{L}}
\newcommand{\mat}[4]{\begin{pmatrix}{#1}&{#2}\\{#3}&{#4}\end{pmatrix}}
\newcommand{\ov}[1]{\overline{#1}}
\newcommand{\res}[1]{\underset{#1}{\RES}\,}
\newcommand{\up}{\upsilon}

\newcommand{\tac}{\textasteriskcentered}

%mahesh macros
\newcommand{\tm}{\textrm}

%Comments
\newcommand{\com}[1]{\vspace{5 mm}\par \noindent
\marginpar{\textsc{Comment}} \framebox{\begin{minipage}[c]{0.95
\textwidth} \tt #1 \end{minipage}}\vspace{5 mm}\par}

\newcommand{\Bmu}{\mbox{$\raisebox{-0.59ex}
  {$l$}\hspace{-0.18em}\mu\hspace{-0.88em}\raisebox{-0.98ex}{\scalebox{2}
  {$\color{white}.$}}\hspace{-0.416em}\raisebox{+0.88ex}
  {$\color{white}.$}\hspace{0.46em}$}{}}  %need graphicx and xcolor. this produces blackboard bold mu 

\newcommand{\hooktwoheadrightarrow}{%
  \hookrightarrow\mathrel{\mspace{-15mu}}\rightarrow
}

\makeatletter
\newcommand{\xhooktwoheadrightarrow}[2][]{%
  \lhook\joinrel
  \ext@arrow 0359\rightarrowfill@ {#1}{#2}%
  \mathrel{\mspace{-15mu}}\rightarrow
}
\makeatother

\renewcommand{\geq}{\geqslant}
\renewcommand{\leq}{\leqslant}
\newcommand{\midd}{\hspace{4pt}\middle|\hspace{4pt}}
    
    \newcommand{\bone}{\mathbf{1}}
    \newcommand{\sign}{\mathrm{sign}}
    \newcommand{\eps}{\varepsilon}
    \newcommand{\textui}[1]{\uline{\textit{#1}}}
    
    %\newcommand{\ov}{\overline}
    %\newcommand{\un}{\underline}
    \newcommand{\fin}{\mathrm{fin}}
    
    \newcommand{\chnum}{\titleformat
    {\chapter} % command
    [display] % shape
    {\centering} % format
    {\Huge \color{black} \shadowbox{\thechapter}} % label
    {-0.5em} % sep (space between the number and title)
    {\LARGE \color{black} \underline} % before-code
    }
    
    \newcommand{\chunnum}{\titleformat
    {\chapter} % command
    [display] % shape
    {} % format
    {} % label
    {0em} % sep
    { \begin{flushright} \begin{tabular}{r}  \Huge \color{black}
    } % before-code
    [
    \end{tabular} \end{flushright} \normalsize
    ] % after-code
    }

\newcommand{\nl}{\newline \mbox{}}

\newcommand{\h}[1]{\hspace{#1pt}}

\newcommand{\littletaller}{\mathchoice{\vphantom{\big|}}{}{}{}}
\newcommand\restr[2]{{% we make the whole thing an ordinary symbol
  \left.\kern-\nulldelimiterspace % automatically resize the bar with \right
  #1 % the function
  \littletaller % pretend it's a little taller at normal size
  \right|_{#2} % this is the delimiter
  }}

\newcommand{\mtext}[1]{\hspace{6pt}\text{#1}\hspace{6pt}}

\newcommand{\lnorm}{\left\lVert}
\newcommand{\rnorm}{\right\rVert}

\newcommand{\ds}{\displaystyle}
\newcommand{\ts}{\textstyle}

%This adds a "front cover" page.
%{\thispagestyle{empty}
%\vspace*{\fill}
%\begin{tabular}{l}
%\begin{tabular}{l}
%\includegraphics[scale=0.24]{oxy-logo.png}
%\end{tabular} \\
%\begin{tabular}{l}
%\Large \color{black} Module Theory, Linear Algebra, and Homological Algebra \\ \Large \color{black} Gianluca Crescenzo
%\end{tabular}
%\end{tabular}
%\newpage

\newcommand{\sfrac}[2]{{}^{#1}\mskip -5mu/\mskip -3mu_{#2}}


\makeatletter
\newcommand*{\da@rightarrow}{\mathchar"0\hexnumber@\symAMSa 4B }
\newcommand*{\da@leftarrow}{\mathchar"0\hexnumber@\symAMSa 4C }
\newcommand*{\xdashrightarrow}[2][]{%
  \mathrel{%
    \mathpalette{\da@xarrow{#1}{#2}{}\da@rightarrow{\,}{}}{}%
  }%
}
\newcommand{\xdashleftarrow}[2][]{%
  \mathrel{%
    \mathpalette{\da@xarrow{#1}{#2}\da@leftarrow{}{}{\,}}{}%
  }%
}
\newcommand*{\da@xarrow}[7]{%
  % #1: below
  % #2: above
  % #3: arrow left
  % #4: arrow right
  % #5: space left 
  % #6: space right
  % #7: math style 
  \sbox0{$\ifx#7\scriptstyle\scriptscriptstyle\else\scriptstyle\fi#5#1#6\m@th$}%
  \sbox2{$\ifx#7\scriptstyle\scriptscriptstyle\else\scriptstyle\fi#5#2#6\m@th$}%
  \sbox4{$#7\dabar@\m@th$}%
  \dimen@=\wd0 %
  \ifdim\wd2 >\dimen@
    \dimen@=\wd2 %   
  \fi
  \count@=2 %
  \def\da@bars{\dabar@\dabar@}%
  \@whiledim\count@\wd4<\dimen@\do{%
    \advance\count@\@ne
    \expandafter\def\expandafter\da@bars\expandafter{%
      \da@bars
      \dabar@ 
    }%
  }%  
  \mathrel{#3}%
  \mathrel{%   
    \mathop{\da@bars}\limits
    \ifx\\#1\\%
    \else
      _{\copy0}%
    \fi
    \ifx\\#2\\%
    \else
      ^{\copy2}%
    \fi
  }%   
  \mathrel{#4}%
}
\makeatother


\begin{document}
\begin{center}
{\large Econ 272 \\[0.1in]Homework 6 \\[0.1in]}
{Name:} {\underline{Gianluca Crescenzo\hspace*{2in}}}\\[0.15in]
\end{center}
\vspace{4pt}
%%%%%%%%%%%%%%%%%%%%%%%%%%%%%%%%%%%%%%%%%%%%%%%%%%%%%%%%%%%%%
\begin{question}
    \phantom{a}
    \begin{enumerate}[label = (\alph*),itemsep=1pt,topsep=3pt]
        %%%%%%%%%%%%%%%%%%%%%%%%%%%%%%%%%%%%%%%%%%%%%%%%%%%%%%%%%%%%%
        \item Summarize the total number of workers (\textit{nototalworker}) and total ouput (\textit{total\_output}) first without weighting by the multiplier (\textit{mult}), and then after weighting using the multiplier.
        
        Does the sample average change upon including the multiplier weights? Based on the two sample averages, what information can you gain about which type of firms are "over-sampled"?
            {\color{blue} \begin{solution}
                Stata gave the following output:
                    \begin{Verbatim}[fontsize=\footnotesize]
. sum nototalworker total_output

    Variable |        Obs        Mean    Std. dev.       Min        Max
-------------+---------------------------------------------------------
nototalwor~r |      6,898    290.7061    741.6751          0      21853
total_output |      6,898    121.1889     964.988   .0001354   37129.21

. 
. sum nototalworker total_output [aw = mult]

    Variable |     Obs      Weight        Mean   Std. dev.       Min        Max
-------------+-----------------------------------------------------------------
nototalwor~r |   6,898       15698    143.7318   510.0754          0      21853
total_output |   6,898       15698    60.27405   642.4071   .0001354   37129.21

                    \end{Verbatim}
                We can see that the sample average did change upon including the multiplier weights. Since the average total number of workers \textit{decreased} after including the weights, larger firms were over-sampled.
            \end{solution}}
        %%%%%%%%%%%%%%%%%%%%%%%%%%%%%%%%%%%%%%%%%%%%%%%%%%%%%%%%%%%%%
        \item We will explore the hypothesis of whether firm profits are affected by the presence of banks. The population regression function is:
            \begin{equation*}
            \begin{split}
                \text{ShProfit}_{id} = \beta_0 + \beta_1 \text{BranchPC}_d + \delta \mathbf{X}_{id} + \epsilon_{id}.
            \end{split}
            \end{equation*}
        $\text{ShProfit}$ is the profits earned by firm $i$, divided by the total firm assets. $\text{BranchPC}$ is bank branches per million population in the discrict in which firm $i$ is located. $\mathbf{X}$ denotes covariates. Include the following covariates: fixed assets (\textit{avg\_nfa}), total workers (\textit{nototalworker}), raw materials (\textit{avg\_raw\_mat}), a quadratic in age (\textit{age} and \textit{sq\_age}), whether the firm is publicly listed (\textit{listed}), and whether the firm is an importer (\textit{importer}). Report the sign and statistical significance of $\beta_1$.
            \newpage
            {\color{blue} \begin{solution}
                Here is my Stata output:
                    \begin{Verbatim}[fontsize=\footnotesize]
                        
-------------------------------------------------------------------------------
     sh_proft | Coefficient  Std. err.      t    P>|t|     [95% conf. interval]
--------------+----------------------------------------------------------------
    branch_pc |  -.0001441   .0000911    -1.58   0.114    -.0003227    .0000344
      avg_nfa |  -5.39e-06   .0000264    -0.20   0.838    -.0000571    .0000464
nototalworker |   4.26e-07   9.21e-06     0.05   0.963    -.0000176    .0000185
  avg_raw_mat |   9.18e-06   .0001936     0.05   0.962    -.0003704    .0003888
          age |  -.0017386   .0006052    -2.87   0.004    -.0029249   -.0005523
       sq_age |   .0000185   6.45e-06     2.87   0.004     5.85e-06    .0000311
       listed |  -.0296402   .0154571    -1.92   0.055    -.0599409    .0006604
     importer |   .0336938   .0107916     3.12   0.002      .012539    .0548487
        _cons |    .113839    .013368     8.52   0.000     .0876336    .1400444
-------------------------------------------------------------------------------
                    \end{Verbatim}
                We can see that $\widehat{\beta_0} = -0.0001441$. With a $p$-value of 0.114, our value is not statistically significant at the 5\% level.
            \end{solution}}
        %%%%%%%%%%%%%%%%%%%%%%%%%%%%%%%%%%%%%%%%%%%%%%%%%%%%%%%%%%%%%
        \item Now re-run the specification in (b), but winsorize profits at the top and bottom 1\%. How has the $\beta_1$ coefficient changes in terms of sign and statistical significant upon winsorizing?
            {\color{blue} \begin{solution}
                Observe that:
                \begin{Verbatim}[fontsize=\footnotesize]
. winsor sh_profit, p(0.01) generate(shprofit_w)

. 
. reg shprofit_w branch_pc avg_nfa nototalworker avg_raw_mat age sq_age listed imp
> orter [aw = mult]
(sum of wgt is 15,665)

.

-------------------------------------------------------------------------------
   shprofit_w | Coefficient  Std. err.      t    P>|t|     [95% conf. interval]
--------------+----------------------------------------------------------------
    branch_pc |  -.0000955   .0000505    -1.89   0.059    -.0001944    3.44e-06
      avg_nfa |  -3.16e-06   .0000146    -0.22   0.829    -.0000318    .0000255
nototalworker |   2.45e-06   5.10e-06     0.48   0.631    -7.56e-06    .0000125
  avg_raw_mat |  -7.05e-06   .0001073    -0.07   0.948    -.0002173    .0002032
          age |  -.0010709   .0003353    -3.19   0.001    -.0017282   -.0004137
       sq_age |   .0000104   3.57e-06     2.91   0.004     3.39e-06    .0000174
       listed |  -.0276464   .0085633    -3.23   0.001     -.044433   -.0108597
     importer |   .0332783   .0059786     5.57   0.000     .0215585    .0449982
        _cons |   .0897713   .0074059    12.12   0.000     .0752534    .1042892
-------------------------------------------------------------------------------
                \end{Verbatim}
                The sign has not changed. Although the coefficient is still not statistically significant, it is much closer than before.
            \end{solution}}
        %%%%%%%%%%%%%%%%%%%%%%%%%%%%%%%%%%%%%%%%%%%%%%%%%%%%%%%%%%%%%
        \item Would it be appropriate to use a natural log transformation of the $\text{ShProfit}$? How would the regression change if we apply a log transformation.
            {\color{blue} \begin{solution}
                If we were to consider doing a log-transform, first note that the $\beta_1$ coefficient represents a percentage effect rather than an absolute-change effect. Since $\text{ShProfit}$ is a ratio, it might be beneficial to consider it's log-transform, but only if all the values are strictly positive.
            \end{solution}}
        %%%%%%%%%%%%%%%%%%%%%%%%%%%%%%%%%%%%%%%%%%%%%%%%%%%%%%%%%%%%%
        \item Set up and test the hypothesis of whether profits are significantly different for firms which are both small (less than 20 workers) and young (less than 10 years of age). These firms are classified by the binary variable \textit{small\_young}. Include all covariates from (b) except for \textit{age} and \textit{sq\_age}, and also include $\text{BranchPC}$ in the estimation. Do small and young firms report significantly different profits than small and old firms? 
            {\color{blue} \begin{solution}
                From the table:
                \begin{Verbatim}[fontsize=\footnotesize]
-------------------------------------------------------------------------------
    sh_profit | Coefficient  Std. err.      t    P>|t|     [95% conf. interval]
--------------+----------------------------------------------------------------
    branch_pc |  -.0001615   .0000909    -1.78   0.076    -.0003398    .0000167
      avg_nfa |  -4.30e-06   .0000264    -0.16   0.871    -.0000561    .0000475
nototalworker |   6.76e-07   9.17e-06     0.07   0.941    -.0000173    .0000187
  avg_raw_mat |   2.96e-06   .0001937     0.02   0.988    -.0003767    .0003826
       listed |  -.0321119   .0154248    -2.08   0.037    -.0623494   -.0018745
     importer |   .0333153   .0107932     3.09   0.002     .0121572    .0544733
  small_young |  -.0341555   .0193569    -1.76   0.078    -.0721009      .00379
        _cons |   .0894562   .0092007     9.72   0.000       .07142    .1074924
-------------------------------------------------------------------------------
                \end{Verbatim}
                We can see there is a -0.034 difference in $\text{ShProfit}$. This is not statistically significant at the 5\% level.
            \end{solution}}
        %%%%%%%%%%%%%%%%%%%%%%%%%%%%%%%%%%%%%%%%%%%%%%%%%%%%%%%%%%%%%
        \item Now consider the following population regression function:
            \begin{equation*}
            \begin{split}
                \text{Output}_{id} = \beta_0 + \beta_1\text{BranchPC}_d + \delta \mathbf{X}_{id} + \epsilon_{id}.
            \end{split}
            \end{equation*}
        $\text{Output}$ refers to total manufacturing output (\textit{total\_output}). Report the sign of the estimated $\beta_1$ coefficient and its statistical significance. Use all of the controls specific in (b).
            {\color{blue} \begin{solution}
                Running the regression gave the following table:
                \begin{Verbatim}[fontsize=\footnotesize]
-------------------------------------------------------------------------------
 total_output | Coefficient  Std. err.      t    P>|t|     [95% conf. interval]
--------------+----------------------------------------------------------------
    branch_pc |   .0276864   .1098674     0.25   0.801    -.1876877    .2430606
      avg_nfa |   .0136459   .0318454     0.43   0.668    -.0487809    .0760727
nototalworker |    -.00101   .0111142    -0.09   0.928    -.0227972    .0207773
  avg_raw_mat |   16.09031   .2335668    68.89   0.000     15.63245    16.54818
          age |   .6200993   .7299844     0.85   0.396    -.8108959    2.051095
       sq_age |  -.0044073   .0077835    -0.57   0.571    -.0196654    .0108508
       listed |    19.3078   18.64495     1.04   0.300    -17.24207    55.85768
     importer |  -19.76467   13.01727    -1.52   0.129    -45.28254      5.7532
        _cons |  -18.49531   16.12504    -1.15   0.251    -50.10538    13.11475
-------------------------------------------------------------------------------
                \end{Verbatim}
                The sign of $\beta_1$ is positive, but it is not statistically significant at the 5\% or 10\% levels.
            \end{solution}}
        %%%%%%%%%%%%%%%%%%%%%%%%%%%%%%%%%%%%%%%%%%%%%%%%%%%%%%%%%%%%%
        \newpage
        \item Create a new variable \textit{woutput} which is total output, but winsorized at the top 1\%. Re-estimate the population regression function in (f), but using the winsorized outcome varable \textit{woutput}. How does the coefficient on $\beta_1$ change in terms of sign and statistical significance, relative to (f)?
            {\color{blue} \begin{solution}
                Stata outputted:
                \begin{Verbatim}[fontsize=\footnotesize]
-------------------------------------------------------------------------------
      woutput | Coefficient  Std. err.      t    P>|t|     [95% conf. interval]
--------------+----------------------------------------------------------------
    branch_pc |  -.1082417   .0280508    -3.86   0.000      -.16323   -.0532535
      avg_nfa |   -.101354   .0081306   -12.47   0.000    -.1172925   -.0854155
nototalworker |   .0891258   .0028376    31.41   0.000     .0835632    .0946884
  avg_raw_mat |   1.785863   .0596331    29.95   0.000     1.668963    1.902762
          age |    .419517   .1863759     2.25   0.024     .0541625    .7848715
       sq_age |  -.0049028   .0019873    -2.47   0.014    -.0087984   -.0010072
       listed |   81.48681   4.760335    17.12   0.000     72.15508    90.81854
     importer |   51.66862   3.323503    15.55   0.000     45.15353    58.18372
        _cons |   12.30627   4.116964     2.99   0.003     4.235749    20.37679
-------------------------------------------------------------------------------
                \end{Verbatim}
            After winsorizing, $\widehat{\beta_1}$ is now negative and statistically significant.
            \end{solution}}
        %%%%%%%%%%%%%%%%%%%%%%%%%%%%%%%%%%%%%%%%%%%%%%%%%%%%%%%%%%%%%
        \item Re-estimate the population regression function in (f) but using the natural log of total output as the outcome variable (take the natural log of the non-winsorized value of output). Based on the estimated $\beta_1$, would you say bank branches have a large or small impact on manufacturing output?
            {\color{blue} \begin{solution}
                From the table:
                \begin{Verbatim}
-------------------------------------------------------------------------------
ln_total_ou~t | Coefficient  Std. err.      t    P>|t|     [95% conf. interval]
--------------+----------------------------------------------------------------
    branch_pc |  -.0024923   .0004974    -5.01   0.000    -.0034673   -.0015172
      avg_nfa |  -.0016166   .0001442   -11.21   0.000    -.0018992    -.001334
nototalworker |   .0011478   .0000503    22.81   0.000     .0010492    .0012464
  avg_raw_mat |   .0122164   .0010574    11.55   0.000     .0101436    .0142892
          age |  -.0151372   .0033047    -4.58   0.000    -.0216155   -.0086589
       sq_age |   .0001422   .0000352     4.04   0.000     .0000732    .0002113
       listed |   1.541427   .0844081    18.26   0.000     1.375961    1.706893
     importer |   1.702065   .0589308    28.88   0.000     1.586542    1.817588
        _cons |   1.635389   .0730001    22.40   0.000     1.492286    1.778492
-------------------------------------------------------------------------------
                \end{Verbatim}
                we can see that bank branches have a small and negative impact on manufacturing output based on the estimated coefficient. A 1-unit increase in $\text{BranchPC}$; i.e., one additional bank branch per million population, results in a change of -0.24923\% to total output. 
            \end{solution}}
        %%%%%%%%%%%%%%%%%%%%%%%%%%%%%%%%%%%%%%%%%%%%%%%%%%%%%%%%%%%%%
        \item Based on the estimates of $\beta_1$ in (g) and (h), how different are your results from winsorizing, as opposed to taking the natural log of the outcome variable? When comparing the estimate of $\beta_1$ between (f), (g), and (h), what do you think winsorizing does to the regression estimates? (Consider both the estimated value of $\beta_1$ and the standard error of the $\beta_1$ coefficient.)
            {\color{blue} \begin{solution}
                Both (g) and (h) resulted in $\beta_1$ coefficients which were statistically significant, whereas the coefficient in (f) was not statistically significant. Comparing winsorizing and taking the natural log of the outcome variable, there was a large change in the standard error.
            \end{solution}}
    \end{enumerate}
\end{question}
%%%%%%%%%%%%%%%%%%%%%%%%%%%%%%%%%%%%%%%%%%%%%%%%%%%%%%%%%%%%%
\begin{question}
    \phantom{a}
    \begin{enumerate}[label = (\alph*),itemsep=1pt,topsep=3pt]
        %%%%%%%%%%%%%%%%%%%%%%%%%%%%%%%%%%%%%%%%%%%%%%%%%%%%%%%%%%%%%
        \item Consider the following population regression function to test whether capital investment responds to bank branches. Remember to weight all of your regressions using \textit{mult}.
            \begin{equation*}
            \begin{split}
                P(\text{Capex} = 1)_id = \beta_0 + \beta_1\text{BranchPC}_d + \delta \mathbf{X}_d + \epsilon_{id}.
            \end{split}
            \end{equation*}
        $P(\text{Capex} = 1)$ is a binary equaling to 1 if the firm undertook any capital investment (\textit{pcapex}). In $\mathbf{X}$, include the binary variables \textit{listed}, \textit{small}, \textit{young}, and \textit{importer}. Does bank branches have a large or small impact on capital investment?
            {\color{blue} \begin{solution}
                From the table:

                \begin{Verbatim}
------------------------------------------------------------------------------
      pcapex | Coefficient  Std. err.      t    P>|t|     [95% conf. interval]
-------------+----------------------------------------------------------------
   branch_pc |   .0001537   .0001338     1.15   0.251    -.0001086     .000416
      listed |  -.0586969   .0226346    -2.59   0.010    -.1030678   -.0143261
       small |  -.1291336   .0120588   -10.71   0.000    -.1527726   -.1054945
       young |  -.0320557   .0205018    -1.56   0.118    -.0722455    .0081341
    importer |   .0769497   .0160723     4.79   0.000     .0454431    .1084563
       _cons |   .3688157   .0148515    24.83   0.000     .3397022    .3979293
------------------------------------------------------------------------------
                \end{Verbatim}
                we can see that bank branches has a small impact on capital investment. 
            \end{solution}}
        %%%%%%%%%%%%%%%%%%%%%%%%%%%%%%%%%%%%%%%%%%%%%%%%%%%%%%%%%%%%%
        \item Test whether the impact of bank branches on the likelihood of making any capital investment differs from across small and alrge firms. Estimate the following regression specification:
            \begin{equation*}
            \begin{split}
                P(\text{Capex} = 1)_id = \beta_0 + \beta_1\text{BranchPC}_d + \beta_2 (\text{BranchPC}_d \cdot \text{Small}_{id}) + \delta \mathbf{X}_d + \epsilon_{id}.
            \end{split}
            \end{equation*}
        Interpret the $\beta_1$ and $\beta_2$ coefficients. Can you reject the null hypothesis $H_0:\beta_1 + \beta_2 = 0$? What information is provided by $\beta_1 + \beta_2$.
            {\color{blue} \begin{solution}
                Stata gave the following table:
                \begin{Verbatim}
------------------------------------------------------------------------------
      pcapex | Coefficient  Std. err.      t    P>|t|     [95% conf. interval]
-------------+----------------------------------------------------------------
   branch_pc |   .0003948   .0001919     2.06   0.040     .0000187    .0007709
branch_small |  -.0004689   .0002674    -1.75   0.080    -.0009931    .0000553
      listed |  -.0596855   .0226382    -2.64   0.008    -.1040634   -.0153076
       small |  -.0875657   .0265934    -3.29   0.001    -.1396971   -.0354344
       young |  -.0323447   .0204993    -1.58   0.115    -.0725298    .0078403
    importer |   .0771489   .0160702     4.80   0.000     .0456462    .1086515
       _cons |   .3479739   .0190196    18.30   0.000     .3106897    .3852581
------------------------------------------------------------------------------                    
                \end{Verbatim}
                $\beta_1$ is the effect of $\text{BranchPC}$ on the probability of capital investment for large firms. $\beta_2$ tells us how the effect of $\text{BranchPC}$ differs for small firms, relative to large firms. Running the test command in Stata outputted:
                    \begin{Verbatim}
 ( 1)  branch_pc + branch_small = 0

       F(  1,  6874) =    0.16
            Prob > F =    0.6912
                    \end{Verbatim}
                So we cannot reject the null-hypothesis. $\beta_1 + \beta_2$ is the total effect for small firms.
            \end{solution}}
    \end{enumerate}
\end{question}
%%%%%%%%%%%%%%%%%%%%%%%%%%%%%%%%%%%%%%%%%%%%%%%%%%%%%%%%%%%%%
\begin{question}
    \phantom{a}
    \begin{enumerate}[label = (\alph*),itemsep=1pt,topsep=3pt]
        \item
        \item 
    \end{enumerate}
\end{question}

\end{document}