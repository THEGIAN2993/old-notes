\chapter{Measure and Integration}

\section{The Riemann-Stieltjes Integral}
\iffalse
    \begin{definition}
        Let $A$ be a set.
        \begin{enumerate}[label = (\arabic*),itemsep=1pt,topsep=3pt]
            \item A \textit{partition of $A$} is a collection $\{A_i\}_{i \in I}$ such that $A = \bigsqcup_{i \in I}A_i$. The collection of partitions of $A$ is denoted $\fB_A$.
            \item A refinement of a partition $P \in \fB_A$ is a partition $Q \in \fB_A$ such that $P \subseteq Q$.
            \item The \textit{common refinement} of two partitions $P_1,P_2 \in \fB_A$ is $P_1 \cup P_2$.
        \end{enumerate}
    \end{definition}

    \begin{example}
        In Proposition~\ref{prop:partition-open}, we showed that any open subset $U$ of $\bfR$ can be written as the disjoint union of a countable number of open intervals. Moreover, Wacław Sierpinski proved that if $X$ is an interval and $\{X_i\}_{i=1}^\infty$ is a countable partition of closed sets, then \textit{at most} one of the sets is non-empty. In many cases, the partitions of an interval (or any set for that matter) are nontrivial, containing any number of combinations of closed intervals, open intervals, and intervals which are neither open nor closed.
    \end{example}

    \begin{definition}
        \phantom{a}
        \begin{enumerate}[label = (\arabic*),itemsep=1pt,topsep=3pt]
            \item Let $[a,b] \subseteq \bfR$ and $\{x_0,x_1,...,x_n\}$ an ordered subset of $[a,b]$. The \textit{partition generated by $\{x_0,x_1,...,x_n\}$} is the partition:
            \begin{equation*}
            \begin{split}
                [x_0,x_1) \cup [x_1,x_2) \cup ... \cup [x_{n-2},x_{n-1}) \cup [x_{n-1},x_n],
            \end{split}
            \end{equation*}
            where $x_0 := a$ and $x_n := b$.

            \item The set of all partitions generated by a finite set is denoted $\overline{fB_{[a,b]}}$

            \item Given a partition $P$ generated by $\{x_0,...,x_n\}$, the \textit{change in $x_i$} is defined as $\Delta x_i = x_i - x_{i-1}$. The \textit{mesh} of $P$ is defined as $\lnorm P \rnorm = \max_{i = 1}^n \Delta x_i$.
        \end{enumerate}
    \end{definition}

    Many resources online define partitions of intervals differently than above. This has been an attempt to reconcile the two very separate notions of combinatorial/topological partitions, and "partitions of intervals." It must be emphasized that:
        \begin{equation*}
        \begin{split}
            [x_0,x_1] \cup (x_1,x_2) \cup [x_2,x_3] \cup ... \cup [x_{n-1},x_n]
        \end{split}
        \end{equation*}
    is equally suitable for defining what a "partition generated by a set of points" should be. The next definition will make it clear that, as long as our partition does not contain any degenerate intervals, it does not matter how we rearrange the open, closed, or sets which are neither open nor closed.

    \fi
    \begin{definition}
        Let $[a,b] \subseteq \bfR$.
        \begin{enumerate}[label = (\arabic*),itemsep=1pt,topsep=3pt]
            \item A \textit{partition} $P$ of the interval $[a,b]$ is a finite set of points $\{x_0,...,x_n\} \subseteq [a,b]$ such that:
                \begin{equation*}
                \begin{split}
                    a = x_0 < x_1 < ... , x_{n-1} < x_n = b.
                \end{split}
                \end{equation*}
            The set of partitions of an interval $[a,b]$ is denoted $\fB_{[a,b]}$.
            \item A refinement of a partition $P$ is a partition $Q$ such that $P \subseteq Q$. In this case, we say that $Q$ is \textit{finer} than $P$. Given two partitions $P_1,P_2 \subseteq [a,b]$, we call the union $P_1 \cup P_2$ their \textit{common refinement}.
            \item The \textit{change in $x_i$} is denoted $\Delta x_i := x_{i} - x_{i-1}$.
            \item The \textit{mesh} of a partition $P$ is denoted $\lnorm P \rnorm := \max_{i = 1}^n \Delta x_i$.
        \end{enumerate}
    \end{definition}

    \begin{definition}
        Let $\alpha$ be a monotonically increasing function on $[a,b]$, $f$ a bounded function on $[a,b]$ and $P=\{x_0,...,x_n\}$ a partition of $[a,b]$. We define:
            \begin{enumerate}[label = (\arabic*),itemsep=1pt,topsep=3pt]
                \item $\Delta \alpha_j := \alpha(x_j) - \alpha(x_{j-1})$;
                \item $M_j(f) = \sup_{x \in [x_{j-1},x_j]}f(x)$; and
                \item $m_j(f) = \inf_{x \in [x_{j-1},x_j]}f(x)$.
            \end{enumerate}
        We call the numbers:
            \begin{equation*}
            \begin{split}
                U(P,f,\alpha) = \sum_{j = 1}^n M_j(f)\Delta \alpha_j, \h9\h9 L(P,f,\alpha) = \sum_{j = 1}^n m_j(f)\Delta \alpha_j, 
            \end{split}
            \end{equation*}
        the \textit{upper (Riemann-Stieltjes) sum} and, respectively, the \textit{lower (Riemann-Stieltjes) sum} of $f$ over the partition $P$ with respect to $\alpha$.
    \end{definition}

    \iffalse
    Since $f$ is defined on all of $[a,b]$, it is not necessary that we define $M_j(f)$ strictly between the partition which it was defined. It is sufficient to consider the closed set $[x_{j-1},x_j]$.
    \fi

    \begin{proposition}
        The sets $\{U(P,f,\alpha) \mid P \in \fB_{[a,b]}\}$ and $\{L(P,f,\alpha) \mid P \in \fB_{[a,b]}\}$ are bounded subsets of $\bfR$.
    \end{proposition}
        \begin{proof}
            Since $f$ is bounded, we can find $m,M \in \bfR$ such that $m  \leq f(x) \leq M$ for all $x \in [a,b]$. Then for any partition $P$ we have $m \leq m_j(f) \leq M_j(f) \leq M$. Multiplying $\sum_{j = 1}^n\Delta \alpha_j$ through this inequality gives $m(\alpha(b) - \alpha(a)) \leq L(P,f,\alpha) \leq U(P,f,\alpha) \leq M(\alpha(b) - \alpha(a))$. Since $P$ was arbitrary, the above sets are bounded.
        \end{proof}

    \begin{definition}
        \phantom{a}
            \begin{enumerate}[label = (\arabic*),itemsep=1pt,topsep=3pt]
                \item The \textit{upper Riemannn-Stieltjes integral} of $f$ over $[a,b]$ with respect to $\alpha$ is:
                    \begin{equation*}
                    \begin{split}
                        \overline{\int_{a}^{b}}f d\alpha =\inf_{P \in \fB_{[a,b]}}U(P,f,\alpha).
                    \end{split}
                    \end{equation*}
                \item The \textit{lower Riemannn-Stieltjes integral} of $f$ over $[a,b]$ with respect to $\alpha$ is:
                \begin{equation*}
                \begin{split}
                    \underline{\int_{a}^{b}}f d\alpha =\sup_{P \in \fB_{[a,b]}}L(P,f,\alpha).
                    \end{split}
                    \end{equation*}

                \item We say that $f$ is \textit{Riemann-Stieltjes integrable} on $[a,b]$ with respect to $\alpha$, and write $f \in \cR(\alpha)[a,b]$, provided that:
                    \begin{equation*}
                    \begin{split}
                        \overline{\int_{a}^{b}}f d\alpha = \underline{\int_{a}^{b}}f d\alpha .
                    \end{split}
                    \end{equation*}
                In this case, the common value of the upper and lower Riemann-Stieltjes integrals is called the \textit{Riemann-Stieltjes integral} of $f$ over $[a,b]$ with respect to $\alpha$ and is denoted:
                    \begin{equation*}
                    \begin{split}
                        \int_a^b f d\alpha.
                    \end{split}
                    \end{equation*}

                \item When the function $\alpha$ is the identity function; i.e., $\alpha(x) = x$, we redefine the following notation:
                    \begin{enumerate}[label = (\roman*),itemsep=6pt,topsep=3pt]
                        \item $\ds U(P,f) := U(P,f,\alpha)$;
                        \item $\ds L(P,f) := L(P,f,\alpha)$;
                        \item $\ds \overline{\int_a^b} f d \alpha := \overline{\int_a^b} f d x$;
                        \item $\ds \underline{\int_a^b} f d \alpha := \underline{\int_a^b} f d x$;
                        \item $\cR(\alpha)[a,b] := \cR[a,b]$.
                    \end{enumerate}
            \end{enumerate}

        \begin{definition}
            When the function $\alpha$
        \end{definition}
    \end{definition}