\documentclass[10pt,twoside,openany]{memoir}
%\usepackage{mlmodern}
%\usepackage{tgpagella} % text only
%\usepackage{mathpazo}  % math & text
\usepackage[T1]{fontenc}
\usepackage[hidelinks]{hyperref}
\usepackage{amsmath}
\usepackage[fixamsmath]{mathtools}  % Extension to amsmath
\usepackage{amsthm}
\usepackage{amssymb}
\renewcommand*{\mathbf}[1]{\varmathbb{#1}}
\usepackage{newpxtext}
\usepackage{eulerpx}
\usepackage{eucal}
\usepackage{datetime}
    \newdateformat{specialdate}{\THEYEAR\ \monthname\ \THEDAY}
\usepackage[margin=1.5in]{geometry}
\usepackage{fancyhdr}
    \fancyhf{}
    \pagestyle{fancy}
    \cfoot{\scriptsize \thepage}
    \fancyhead[R]{\scalebox{0.7}{\rightmark}}
    \fancyhead[L]{\scalebox{0.7}{\leftmark}}
\usepackage{thmtools}
    \declaretheoremstyle[
        spaceabove=10pt,
        spacebelow=10pt,
        headfont=\normalfont\bfseries,
        notefont=\mdseries, notebraces={(}{)},
        bodyfont=\normalfont,
        postheadspace=0.5em
        %qed=\qedsymbol
        ]{defs}

    \declaretheoremstyle[ 
        spaceabove=10pt, % space above the theorem
        spacebelow=10pt,
        headfont=\normalfont\bfseries,
        bodyfont=\normalfont\itshape,
        postheadspace=0.5em
        ]{thmstyle}
    
    \declaretheorem[
        style=thmstyle,
        numberwithin=section
    ]{theorem}

    \declaretheorem[
        style=thmstyle,
        sibling=theorem,
    ]{proposition}

    \declaretheorem[
        style=thmstyle,
        sibling=theorem,
    ]{lemma}

    \declaretheorem[
        style=thmstyle,
        sibling=theorem,
    ]{corollary}

    \declaretheorem[
        numberwithin=section,
        style=defs,
    ]{example}

    \declaretheorem[
        numberwithin=section,
        style=defs,
    ]{definition}

    \declaretheorem[
        style=defs,
        sibling=theorem,
        numberwithin=section,
    ]{exercise}

    \declaretheorem[
        numbered=unless unique,
        shaded={rulecolor=black,
    rulewidth=1pt, bgcolor={rgb}{1,1,1}}
    ]{axiom}

    \declaretheorem[numberwithin=section,style=defs]{note}
    \declaretheorem[numbered=no,style=defs]{question}
    \declaretheorem[numbered=no,style=defs]{recall}
    \declaretheorem[numbered=no,style=remark]{answer}
    \declaretheorem[numbered=no,style=remark]{solution}
    \declaretheorem[numbered=no,style=defs]{remark}
\usepackage{enumitem}
\usepackage{titlesec}
    \titleformat{\chapter}[display]
    {\bfseries\Huge\raggedright}
    {Chapter {\thechapter}}
    {1ex minus .1ex}
    {\HUGE}
    \titlespacing{\chapter}
    {3pc}{*3}{40pt}[3pc]

    \titleformat{\section}[block]
    {\normalfont\bfseries\LARGE}
    {\S\ \thesection.}{.5em}{}[]
    \titlespacing{\section}
    {0pt}{3ex plus .1ex minus .2ex}{3ex plus .1ex minus .2ex}
\usepackage[utf8x]{inputenc}
\usepackage{tikz}
\usepackage{tikz-cd}
\usepackage{wasysym}
\linespread{1.00}
%%%%%%%%%%%%%%%%%%%%%%%%%%%%%%%%%%%%%%%%%%%%%%%%%%%%%%%%%%%%%
%%%%%%%%%%%%%%%%%%%%%%%%%%%%%%%%%%%%%%%%%%%%%%%%%%%%%%%%%%%%%
\input{/Users/gcrescenzo/Documents/School/LaTeX Documents/latex-class-notes/makros.tex}
\begin{document}
%This adds a "front cover" page.
%{\thispagestyle{empty}
%\vspace*{\fill}
%\begin{tabular}{l}
%\begin{tabular}{l}
%\includegraphics[scale=0.24]{oxy-logo.png}
%\end{tabular} \\
%\begin{tabular}{l}
%\Large \color{black} Module Theory, Linear Algebra, and Homological Algebra \\ \Large \color{black} Gianluca Crescenzo
%\end{tabular}
%\end{tabular}
%\newpage
    \pagenumbering{roman}
    \tableofcontents

    \chapter*{Preface}
    These are the notes I took for a second semester analysis course. \nl
    
    \noindent \textbf{Urgent Things That I Need To Fix}
    \begin{enumerate}[label = (\arabic*),itemsep=1pt,topsep=3pt]
        \item Any corollary, lemma, proposition, example, or theorem with three asterisks (***) means I don't understand the proof, or there is no proof altogether. There are 18 instances of these throughout the notes.
        \item Turn every "Exercise" into a proposition and do the proofs. Any proposition whose proof is an "exercise" needs to be filled in.
        \item The end of $\S$~\ref{sec:conv-seq}~\nameref{sec:conv-seq} is supposed to include one or two propositions related to the distance of an element $x \in X$ to a set $A \subseteq X$.
        \item $\S$~\ref{sec:cantor-set}~\nameref{sec:cantor-set} is incomplete. 
        \item The notes on applications of meager sets and the \nameref{thm:baires-theorem} need to be included at the end of $\S$~\ref{sec:completeness}~\nameref{sec:completeness}.
        \item $\S$~\ref{sec:top-of-metric-spaces}~\nameref{sec:top-of-metric-spaces}, $\S$~\ref{sec:conv-seq}~\nameref{sec:conv-seq}, and $\S$~\ref{sec:continuity}~\nameref{sec:continuity} are kind of messy. Can't put my finger on why but it doesn't feel as nice as some other parts of these notes.
    \end{enumerate}

    \noindent \textbf{Less Urgent Things That I'm Probably Gonna Do Instead}
    \begin{enumerate}[label = (\arabic*),itemsep=1pt,topsep=3pt]
        \item Clean up $\S$~\ref{sec:completeness}~\nameref{sec:completeness} and $\S$~\ref{sec:compactness}~\nameref{sec:compactness}. Finish $\S$~\ref{sec:connectedness}~\nameref{sec:connectedness}.
        \item Start Chapter 3. First section will be on Riemann Integration, will probably copy most of it from Bartle's \textit{An Introduction to Real Analysis}. Second section will be on measure theory. Third section will be on Lebesgue integration.
    \end{enumerate}

    \noindent \textbf{Things That I'm Really Happy With}
    \begin{enumerate}[label = (\arabic*),itemsep=1pt,topsep=3pt]
        \item Besides some stuff about inner-products, Chapter 1 is almost perfect.
    \end{enumerate}


    \vfill
    \specialdate
    Last update: \today

    \include{normed-spaces-algebras}
    \include{metrics}
    \include{measure-and-integration}
    %\appendix
    %\include{seq-fncts-inf-series}
\end{document}