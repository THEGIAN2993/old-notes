\documentclass[11pt,twoside,openany]{memoir}
\usepackage{mlmodern}
%\usepackage{tgpagella} % text only
%\usepackage{mathpazo}  % math & text
\usepackage[T1]{fontenc}
\usepackage[hidelinks]{hyperref}
\usepackage{amsmath}
\usepackage{amsthm}
\usepackage{amssymb}
\usepackage{mathtools}
%\renewcommand*{\mathbf}[1]{\varmathbb{#1}}
%\usepackage{newpxtext}
%\usepackage{eulerpx}
%\usepackage{eucal}
\usepackage{datetime}
    \newdateformat{specialdate}{\THEYEAR\ \monthname\ \THEDAY}
\usepackage[margin=1in]{geometry}
\usepackage{fancyhdr}
    \fancyhf{}
    \pagestyle{fancy}
    \cfoot{\scriptsize \thepage}
    \fancyhead[R]{\scriptsize \rightmark}
    \fancyhead[L]{\scriptsize \leftmark}
    \renewcommand{\headrulewidth}{0pt}
    \renewcommand{\footrulewidth}{0pt} % if you also want to remove the footer rule
\usepackage{thmtools}
    \declaretheoremstyle[
        spaceabove=10pt,
        spacebelow=10pt,
        headfont=\normalfont\bfseries,
        notefont=\mdseries, notebraces={(}{)},
        bodyfont=\normalfont,
        postheadspace=0.5em
        %qed=\qedsymbol
        ]{defs}

    \declaretheoremstyle[ 
        spaceabove=10pt, % space above the theorem
        spacebelow=10pt,
        headfont=\normalfont\bfseries,
        bodyfont=\normalfont\itshape,
        postheadspace=0.5em
        ]{thmstyle}
    
    \declaretheorem[
        style=thmstyle,
        numberwithin=section
    ]{theorem}

    \declaretheorem[
        style=thmstyle,
        sibling=theorem,
    ]{proposition}

    \declaretheorem[
        style=thmstyle,
        sibling=theorem,
    ]{lemma}

    \declaretheorem[
        style=thmstyle,
        sibling=theorem,
    ]{corollary}

    \declaretheorem[
        numberwithin=section,
        style=defs,
    ]{example}

    \declaretheorem[
        numberwithin=section,
        style=defs,
    ]{definition}

    \declaretheorem[
        style=defs,
        numbered=unless unique,
    ]{exercise}

    \declaretheorem[
        numbered=unless unique,
        shaded={rulecolor=black,
    rulewidth=1pt, bgcolor={rgb}{1,1,1}}
    ]{axiom}

    \declaretheorem[numberwithin=section,style=defs]{note}
    \declaretheorem[numbered=no,style=defs]{question}
    \declaretheorem[numbered=no,style=defs]{recall}
    \declaretheorem[numbered=no,style=remark]{answer}
    \declaretheorem[numbered=no,style=remark]{solution}

    \declaretheorem[numbered=no,style=defs]{remark}
\usepackage{enumitem}
\usepackage{titlesec}
    \titleformat{\chapter}[display]
    {\bfseries\LARGE\raggedright}
    {Chapter {\thechapter}}
    {1ex minus .1ex}
    {\Huge}
    \titlespacing{\chapter}
    {3pc}{*3}{40pt}[3pc]

    \titleformat{\section}[block]
    {\normalfont\bfseries\Large}
    {\S\ \thesection.}{.5em}{}[]
    \titlespacing{\section}
    {0pt}{3ex plus .1ex minus .2ex}{3ex plus .1ex minus .2ex}
\usepackage[utf8x]{inputenc}
\usepackage{tikz}
\usepackage{tikz-cd}
\usepackage{wasysym}
\renewcommand{\int}{\varint}

\linespread{0.95}
%to make the correct symbol for Sha
%\newcommand\cyr{%
%\renewcommand\rmdefault{wncyr}%
%\renewcommand\sfdefault{wncyss}%
%\renewcommand\encodingdefault{OT2}%
%\normalfont \selectfont} \DeclareTextFontCommand{\textcyr}{\cyr}


\DeclareMathOperator{\ab}{ab}
\newcommand{\absgal}{\G_{\bbQ}}
\DeclareMathOperator{\ad}{ad}
\DeclareMathOperator{\adj}{adj}
\DeclareMathOperator{\alg}{alg}
\DeclareMathOperator{\Alt}{Alt}
\DeclareMathOperator{\Ann}{Ann}
\DeclareMathOperator{\arith}{arith}
\DeclareMathOperator{\Aut}{Aut}
\DeclareMathOperator{\Be}{B}
\DeclareMathOperator{\Bd}{Bd}
\DeclareMathOperator{\card}{card}
\DeclareMathOperator{\Char}{char}
\DeclareMathOperator{\csp}{csp}
\DeclareMathOperator{\codim}{codim}
\DeclareMathOperator{\coker}{coker}
\DeclareMathOperator{\coh}{H}
\DeclareMathOperator{\compl}{compl}
\DeclareMathOperator{\conj}{conj}
\DeclareMathOperator{\cont}{cont}
\DeclareMathOperator{\crys}{crys}
\DeclareMathOperator{\Crys}{Crys}
\DeclareMathOperator{\cusp}{cusp}
\DeclareMathOperator{\diag}{diag}
\DeclareMathOperator{\diam}{diam}
\DeclareMathOperator{\Dom}{Dom}
\DeclareMathOperator{\disc}{disc}
\DeclareMathOperator{\dist}{dist}
\DeclareMathOperator{\dR}{dR}
\DeclareMathOperator{\Eis}{Eis}
\DeclareMathOperator{\End}{End}
\DeclareMathOperator{\ev}{ev}
\DeclareMathOperator{\eval}{eval}
\DeclareMathOperator{\Eq}{Eq}
\DeclareMathOperator{\Ext}{Ext}
\DeclareMathOperator{\Fil}{Fil}
\DeclareMathOperator{\Fitt}{Fitt}
\DeclareMathOperator{\Frob}{Frob}
\DeclareMathOperator{\G}{G}
\DeclareMathOperator{\Gal}{Gal}
\DeclareMathOperator{\GL}{GL}
\DeclareMathOperator{\Gr}{Gr}
\DeclareMathOperator{\Graph}{Graph}
\DeclareMathOperator{\GSp}{GSp}
\DeclareMathOperator{\GUn}{GU}
\DeclareMathOperator{\Hom}{Hom}
\DeclareMathOperator{\id}{id}
\DeclareMathOperator{\Id}{Id}
\DeclareMathOperator{\Ik}{Ik}
\DeclareMathOperator{\IM}{Im}
\DeclareMathOperator{\Image}{im}
\DeclareMathOperator{\Ind}{Ind}
\DeclareMathOperator{\Inf}{inf}
\DeclareMathOperator{\Isom}{Isom}
\DeclareMathOperator{\J}{J}
\DeclareMathOperator{\Jac}{Jac}
\DeclareMathOperator{\lcm}{lcm}
\DeclareMathOperator{\length}{length}
\DeclareMathOperator*{\limit}{limit}
\DeclareMathOperator{\Log}{Log}
\DeclareMathOperator{\M}{M}
\DeclareMathOperator{\Mat}{Mat}
\DeclareMathOperator{\N}{N}
\DeclareMathOperator{\Nm}{Nm}
\DeclareMathOperator{\NIk}{N-Ik}
\DeclareMathOperator{\NSK}{N-SK}
\DeclareMathOperator{\new}{new}
\DeclareMathOperator{\obj}{obj}
\DeclareMathOperator{\old}{old}
\DeclareMathOperator{\ord}{ord}
\DeclareMathOperator{\Or}{O}
\DeclareMathOperator{\op}{op}
\DeclareMathOperator{\PGL}{PGL}
\DeclareMathOperator{\PGSp}{PGSp}
\DeclareMathOperator{\rank}{rank}
\DeclareMathOperator{\Ran}{Ran}
\DeclareMathOperator{\Rel}{Rel}
\DeclareMathOperator{\Real}{Re}
\DeclareMathOperator{\RES}{res}
\DeclareMathOperator{\Res}{Res}
%\DeclareMathOperator{\Sha}{\textcyr{Sh}}
\DeclareMathOperator{\Sel}{Sel}
\DeclareMathOperator{\semi}{ss}
\DeclareMathOperator{\sgn}{sign}
\DeclareMathOperator{\SK}{SK}
\DeclareMathOperator{\SL}{SL}
\DeclareMathOperator{\SO}{SO}
\DeclareMathOperator{\Sp}{Sp}
\DeclareMathOperator{\Span}{span}
\DeclareMathOperator{\Spec}{Spec}
\DeclareMathOperator{\spin}{spin}
\DeclareMathOperator{\st}{st}
\DeclareMathOperator{\St}{St}
\DeclareMathOperator{\SUn}{SU}
\DeclareMathOperator{\supp}{supp}
\DeclareMathOperator{\Sup}{sup}
\DeclareMathOperator{\Sym}{Sym}
\DeclareMathOperator{\Tam}{Tam}
\DeclareMathOperator{\tors}{tors}
\DeclareMathOperator{\tr}{tr}
\DeclareMathOperator{\Tr}{Tr}
\DeclareMathOperator{\un}{un}
\DeclareMathOperator{\Un}{U}
\DeclareMathOperator{\val}{val}
\DeclareMathOperator{\vol}{vol}

\DeclareMathOperator{\Sets}{S \mkern1.04mu e \mkern1.04mu t \mkern1.04mu s}
    \newcommand{\cSets}{\scalebox{1.02}{\contour{black}{$\Sets$}}}
    
\DeclareMathOperator{\Groups}{G \mkern1.04mu r \mkern1.04mu o \mkern1.04mu u \mkern1.04mu p \mkern1.04mu s}
    \newcommand{\cGroups}{\scalebox{1.02}{\contour{black}{$\Groups$}}}

\DeclareMathOperator{\TTop}{T \mkern1.04mu o \mkern1.04mu p}
    \newcommand{\cTop}{\scalebox{1.02}{\contour{black}{$\TTop$}}}

\DeclareMathOperator{\Htp}{H \mkern1.04mu t \mkern1.04mu p}
    \newcommand{\cHtp}{\scalebox{1.02}{\contour{black}{$\Htp$}}}

\DeclareMathOperator{\Mod}{M \mkern1.04mu o \mkern1.04mu d}
    \newcommand{\cMod}{\scalebox{1.02}{\contour{black}{$\Mod$}}}

\DeclareMathOperator{\Ab}{A \mkern1.04mu b}
    \newcommand{\cAb}{\scalebox{1.02}{\contour{black}{$\Ab$}}}

\DeclareMathOperator{\Rings}{R \mkern1.04mu i \mkern1.04mu n \mkern1.04mu g \mkern1.04mu s}
    \newcommand{\cRings}{\scalebox{1.02}{\contour{black}{$\Rings$}}}

\DeclareMathOperator{\ComRings}{C \mkern1.04mu o \mkern1.04mu m \mkern1.04mu R \mkern1.04mu i \mkern1.04mu n \mkern1.04mu g \mkern1.04mu s}
    \newcommand{\cComRings}{\scalebox{1.05}{\contour{black}{$\ComRings$}}}

\DeclareMathOperator{\hHom}{H \mkern1.04mu o \mkern1.04mu m}
    \newcommand{\cHom}{\scalebox{1.02}{\contour{black}{$\hHom$}}}

         %  \item $\cGroups$
          %  \item $\cTop$
          %  \item $\cHtp$
          %  \item $\cMod$




\renewcommand{\k}{\kappa}
\newcommand{\Ff}{F_{f}}
%\newcommand{\ts}{\,^{t}\!}


%Mathcal

\newcommand{\cA}{\mathcal{A}}
\newcommand{\cB}{\mathcal{B}}
\newcommand{\cC}{\mathcal{C}}
\newcommand{\cD}{\mathcal{D}}
\newcommand{\cE}{\mathcal{E}}
\newcommand{\cF}{\mathcal{F}}
\newcommand{\cG}{\mathcal{G}}
\newcommand{\cH}{\mathcal{H}}
\newcommand{\cI}{\mathcal{I}}
\newcommand{\cJ}{\mathcal{J}}
\newcommand{\cK}{\mathcal{K}}
\newcommand{\cL}{\mathcal{L}}
\newcommand{\cM}{\mathcal{M}}
\newcommand{\cN}{\mathcal{N}}
\newcommand{\cO}{\mathcal{O}}
\newcommand{\cP}{\mathcal{P}}
\newcommand{\cQ}{\mathcal{Q}}
\newcommand{\cR}{\mathcal{R}}
\newcommand{\cS}{\mathcal{S}}
\newcommand{\cT}{\mathcal{T}}
\newcommand{\cU}{\mathcal{U}}
\newcommand{\cV}{\mathcal{V}}
\newcommand{\cW}{\mathcal{W}}
\newcommand{\cX}{\mathcal{X}}
\newcommand{\cY}{\mathcal{Y}}
\newcommand{\cZ}{\mathcal{Z}}


%mathfrak (missing \fi)

\newcommand{\fa}{\mathfrak{a}}
\newcommand{\fA}{\mathfrak{A}}
\newcommand{\fb}{\mathfrak{b}}
\newcommand{\fB}{\mathfrak{B}}
\newcommand{\fc}{\mathfrak{c}}
\newcommand{\fC}{\mathfrak{C}}
\newcommand{\fd}{\mathfrak{d}}
\newcommand{\fD}{\mathfrak{D}}
\newcommand{\fe}{\mathfrak{e}}
\newcommand{\fE}{\mathfrak{E}}
\newcommand{\ff}{\mathfrak{f}}
\newcommand{\fF}{\mathfrak{F}}
\newcommand{\fg}{\mathfrak{g}}
\newcommand{\fG}{\mathfrak{G}}
\newcommand{\fh}{\mathfrak{h}}
\newcommand{\fH}{\mathfrak{H}}
\newcommand{\fI}{\mathfrak{I}}
\newcommand{\fj}{\mathfrak{j}}
\newcommand{\fJ}{\mathfrak{J}}
\newcommand{\fk}{\mathfrak{k}}
\newcommand{\fK}{\mathfrak{K}}
\newcommand{\fl}{\mathfrak{l}}
\newcommand{\fL}{\mathfrak{L}}
\newcommand{\fm}{\mathfrak{m}}
\newcommand{\fM}{\mathfrak{M}}
\newcommand{\fn}{\mathfrak{n}}
\newcommand{\fN}{\mathfrak{N}}
\newcommand{\fo}{\mathfrak{o}}
\newcommand{\fO}{\mathfrak{O}}
\newcommand{\fp}{\mathfrak{p}}
\newcommand{\fP}{\mathfrak{P}}
\newcommand{\fq}{\mathfrak{q}}
\newcommand{\fQ}{\mathfrak{Q}}
\newcommand{\fr}{\mathfrak{r}}
\newcommand{\fR}{\mathfrak{R}}
\newcommand{\fs}{\mathfrak{s}}
\newcommand{\fS}{\mathfrak{S}}
\newcommand{\ft}{\mathfrak{t}}
\newcommand{\fT}{\mathfrak{T}}
\newcommand{\fu}{\mathfrak{u}}
\newcommand{\fU}{\mathfrak{U}}
\newcommand{\fv}{\mathfrak{v}}
\newcommand{\fV}{\mathfrak{V}}
\newcommand{\fw}{\mathfrak{w}}
\newcommand{\fW}{\mathfrak{W}}
\newcommand{\fx}{\mathfrak{x}}
\newcommand{\fX}{\mathfrak{X}}
\newcommand{\fy}{\mathfrak{y}}
\newcommand{\fY}{\mathfrak{Y}}
\newcommand{\fz}{\mathfrak{z}}
\newcommand{\fZ}{\mathfrak{Z}}


%mathbf
\newcommand{\bfA}{\mathbf{A}}
\newcommand{\bfB}{\mathbf{B}}
\newcommand{\bfC}{\mathbf{C}}
\newcommand{\bfD}{\mathbf{D}}
\newcommand{\bfE}{\mathbf{E}}
\newcommand{\bfF}{\mathbf{F}}
\newcommand{\bfG}{\mathbf{G}}
\newcommand{\bfH}{\mathbf{H}}
\newcommand{\bfI}{\mathbf{I}}
\newcommand{\bfJ}{\mathbf{J}}
\newcommand{\bfK}{\mathbf{K}}
\newcommand{\bfL}{\mathbf{L}}
\newcommand{\bfM}{\mathbf{M}}
\newcommand{\bfN}{\mathbf{N}}
\newcommand{\bfO}{\mathbf{O}}
\newcommand{\bfP}{\mathbf{P}}
\newcommand{\bfQ}{\mathbf{Q}}
\newcommand{\bfR}{\mathbf{R}}
\newcommand{\bfS}{\mathbf{S}}
\newcommand{\bfT}{\mathbf{T}}
\newcommand{\bfU}{\mathbf{U}}
\newcommand{\bfV}{\mathbf{V}}
\newcommand{\bfW}{\mathbf{W}}
\newcommand{\bfX}{\mathbf{X}}
\newcommand{\bfY}{\mathbf{Y}}
\newcommand{\bfZ}{\mathbf{Z}}

\newcommand{\bfa}{\mathbf{a}}
\newcommand{\bfb}{\mathbf{b}}
\newcommand{\bfc}{\mathbf{c}}
\newcommand{\bfd}{\mathbf{d}}
\newcommand{\bfe}{\mathbf{e}}
\newcommand{\bff}{\mathbf{f}}
\newcommand{\bfg}{\mathbf{g}}
\newcommand{\bfh}{\mathbf{h}}
\newcommand{\bfi}{\mathbf{i}}
\newcommand{\bfj}{\mathbf{j}}
\newcommand{\bfk}{\mathbf{k}}
\newcommand{\bfl}{\mathbf{l}}
\newcommand{\bfm}{\mathbf{m}}
\newcommand{\bfn}{\mathbf{n}}
\newcommand{\bfo}{\mathbf{o}}
\newcommand{\bfp}{\mathbf{p}}
\newcommand{\bfq}{\mathbf{q}}
\newcommand{\bfr}{\mathbf{r}}
\newcommand{\bfs}{\mathbf{s}}
\newcommand{\bft}{\mathbf{t}}
\newcommand{\bfu}{\mathbf{u}}
\newcommand{\bfv}{\mathbf{v}}
\newcommand{\bfw}{\mathbf{w}}
\newcommand{\bfx}{\mathbf{x}}
\newcommand{\bfy}{\mathbf{y}}
\newcommand{\bfz}{\mathbf{z}}

%blackboard bold

\newcommand{\bbA}{\mathbb{A}}
\newcommand{\bbB}{\mathbb{B}}
\newcommand{\bbC}{\mathbb{C}}
\newcommand{\bbD}{\mathbb{D}}
\newcommand{\bbE}{\mathbb{E}}
\newcommand{\bbF}{\mathbb{F}}
\newcommand{\bbG}{\mathbb{G}}
\newcommand{\bbH}{\mathbb{H}}
\newcommand{\bbI}{\mathbb{I}}
\newcommand{\bbJ}{\mathbb{J}}
\newcommand{\bbK}{\mathbb{K}}
\newcommand{\bbL}{\mathbb{L}}
\newcommand{\bbM}{\mathbb{M}}
\newcommand{\bbN}{\mathbb{N}}
\newcommand{\bbO}{\mathbb{O}}
\newcommand{\bbP}{\mathbb{P}}
\newcommand{\bbQ}{\mathbb{Q}}
\newcommand{\bbR}{\mathbb{R}}
\newcommand{\bbS}{\mathbb{S}}
\newcommand{\bbT}{\mathbb{T}}
\newcommand{\bbU}{\mathbb{U}}
\newcommand{\bbV}{\mathbb{V}}
\newcommand{\bbW}{\mathbb{W}}
\newcommand{\bbX}{\mathbb{X}}
\newcommand{\bbY}{\mathbb{Y}}
\newcommand{\bbZ}{\mathbb{Z}}
\newcommand{\jota}{\jmath}

\newcommand{\bmat}{\left( \begin{matrix}}
\newcommand{\emat}{\end{matrix} \right)}

\newcommand{\pmat}{\left( \begin{smallmatrix}}
\newcommand{\epmat}{\end{smallmatrix} \right)}

\newcommand{\lat}{\mathscr{L}}
\newcommand{\mat}[4]{\begin{pmatrix}{#1}&{#2}\\{#3}&{#4}\end{pmatrix}}
\newcommand{\ov}[1]{\overline{#1}}
\newcommand{\res}[1]{\underset{#1}{\RES}\,}
\newcommand{\up}{\upsilon}

\newcommand{\tac}{\textasteriskcentered}

%mahesh macros
\newcommand{\tm}{\textrm}

%Comments
\newcommand{\com}[1]{\vspace{5 mm}\par \noindent
\marginpar{\textsc{Comment}} \framebox{\begin{minipage}[c]{0.95
\textwidth} \tt #1 \end{minipage}}\vspace{5 mm}\par}

\newcommand{\Bmu}{\mbox{$\raisebox{-0.59ex}
  {$l$}\hspace{-0.18em}\mu\hspace{-0.88em}\raisebox{-0.98ex}{\scalebox{2}
  {$\color{white}.$}}\hspace{-0.416em}\raisebox{+0.88ex}
  {$\color{white}.$}\hspace{0.46em}$}{}}  %need graphicx and xcolor. this produces blackboard bold mu 

\newcommand{\hooktwoheadrightarrow}{%
  \hookrightarrow\mathrel{\mspace{-15mu}}\rightarrow
}

\makeatletter
\newcommand{\xhooktwoheadrightarrow}[2][]{%
  \lhook\joinrel
  \ext@arrow 0359\rightarrowfill@ {#1}{#2}%
  \mathrel{\mspace{-15mu}}\rightarrow
}
\makeatother

\renewcommand{\geq}{\geqslant}
\renewcommand{\leq}{\leqslant}
\newcommand{\midd}{\hspace{4pt}\middle|\hspace{4pt}}
    
    \newcommand{\bone}{\mathbf{1}}
    \newcommand{\sign}{\mathrm{sign}}
    \newcommand{\eps}{\varepsilon}
    \newcommand{\textui}[1]{\uline{\textit{#1}}}
    
    %\newcommand{\ov}{\overline}
    %\newcommand{\un}{\underline}
    \newcommand{\fin}{\mathrm{fin}}
    
    \newcommand{\chnum}{\titleformat
    {\chapter} % command
    [display] % shape
    {\centering} % format
    {\Huge \color{black} \shadowbox{\thechapter}} % label
    {-0.5em} % sep (space between the number and title)
    {\LARGE \color{black} \underline} % before-code
    }
    
    \newcommand{\chunnum}{\titleformat
    {\chapter} % command
    [display] % shape
    {} % format
    {} % label
    {0em} % sep
    { \begin{flushright} \begin{tabular}{r}  \Huge \color{black}
    } % before-code
    [
    \end{tabular} \end{flushright} \normalsize
    ] % after-code
    }

\newcommand{\nl}{\newline \mbox{}}

\newcommand{\h}[1]{\hspace{#1pt}}

\newcommand{\littletaller}{\mathchoice{\vphantom{\big|}}{}{}{}}
\newcommand\restr[2]{{% we make the whole thing an ordinary symbol
  \left.\kern-\nulldelimiterspace % automatically resize the bar with \right
  #1 % the function
  \littletaller % pretend it's a little taller at normal size
  \right|_{#2} % this is the delimiter
  }}

\newcommand{\mtext}[1]{\hspace{6pt}\text{#1}\hspace{6pt}}

\newcommand{\lnorm}{\left\lVert}
\newcommand{\rnorm}{\right\rVert}

\newcommand{\ds}{\displaystyle}
\newcommand{\ts}{\textstyle}

%This adds a "front cover" page.
%{\thispagestyle{empty}
%\vspace*{\fill}
%\begin{tabular}{l}
%\begin{tabular}{l}
%\includegraphics[scale=0.24]{oxy-logo.png}
%\end{tabular} \\
%\begin{tabular}{l}
%\Large \color{black} Module Theory, Linear Algebra, and Homological Algebra \\ \Large \color{black} Gianluca Crescenzo
%\end{tabular}
%\end{tabular}
%\newpage

\newcommand{\sfrac}[2]{{}^{#1}\mskip -5mu/\mskip -3mu_{#2}}


\makeatletter
\newcommand*{\da@rightarrow}{\mathchar"0\hexnumber@\symAMSa 4B }
\newcommand*{\da@leftarrow}{\mathchar"0\hexnumber@\symAMSa 4C }
\newcommand*{\xdashrightarrow}[2][]{%
  \mathrel{%
    \mathpalette{\da@xarrow{#1}{#2}{}\da@rightarrow{\,}{}}{}%
  }%
}
\newcommand{\xdashleftarrow}[2][]{%
  \mathrel{%
    \mathpalette{\da@xarrow{#1}{#2}\da@leftarrow{}{}{\,}}{}%
  }%
}
\newcommand*{\da@xarrow}[7]{%
  % #1: below
  % #2: above
  % #3: arrow left
  % #4: arrow right
  % #5: space left 
  % #6: space right
  % #7: math style 
  \sbox0{$\ifx#7\scriptstyle\scriptscriptstyle\else\scriptstyle\fi#5#1#6\m@th$}%
  \sbox2{$\ifx#7\scriptstyle\scriptscriptstyle\else\scriptstyle\fi#5#2#6\m@th$}%
  \sbox4{$#7\dabar@\m@th$}%
  \dimen@=\wd0 %
  \ifdim\wd2 >\dimen@
    \dimen@=\wd2 %   
  \fi
  \count@=2 %
  \def\da@bars{\dabar@\dabar@}%
  \@whiledim\count@\wd4<\dimen@\do{%
    \advance\count@\@ne
    \expandafter\def\expandafter\da@bars\expandafter{%
      \da@bars
      \dabar@ 
    }%
  }%  
  \mathrel{#3}%
  \mathrel{%   
    \mathop{\da@bars}\limits
    \ifx\\#1\\%
    \else
      _{\copy0}%
    \fi
    \ifx\\#2\\%
    \else
      ^{\copy2}%
    \fi
  }%   
  \mathrel{#4}%
}
\makeatother


\begin{document}
\begin{center}
{\large Math 397 \\[0.1in]Homework 2 \\[0.1in]}
{Name:} {\underline{Gianluca Crescenzo\hspace*{2in}}}\\[0.15in]
\end{center}
\vspace{4pt}
%%%%%%%%%%%%%%%%%%%%%%%%%%%%%%%%%%%%%%%%%%%%%%%%%%%%%%%%%%%%%
    \addtocounter{exercise}{4}
    \begin{exercise}
        Let $\lambda = (\lambda_k)_k$ belong to $\ell_\infty$. Show that the map:
            \begin{equation*}
            \begin{split}
                D_\lambda: \ell_2 \rightarrow \ell_2 \mtext{defined by} D_\lambda((\xi_k)_k) = (\lambda_k \xi_k)_k
            \end{split}
            \end{equation*}
        is well-defined, linear, and bounded with $\lnorm D_\lambda \rnorm_\text{op} = \lnorm \lambda \rnorm_\infty$.
    \end{exercise}
        \begin{proof}
            Let $(x_k)_k,(y_k)_k \in \ell_2$ with $(x_k)_k = (y_k)_k$. Then $x_k = y_k$ for all $k \in \bfN$. We have:
                \begin{equation*}
                \begin{split}
                    D_\lambda((x_k)_k)
                    & = (\lambda_k x_k)_k \\
                    & = (\lambda_k y_k)_k \\
                    & = D_\lambda((y_k)_k).
                \end{split}
                \end{equation*}
            Thus $D_\lambda$ is well-defined. Let $\alpha \in F$. Observe that:
                \begin{equation*}
                \begin{split}
                    D_\lambda\Bigl((x_k)_k + \alpha(y_k)_k\Bigr) 
                    & = D_\lambda \Bigl((x_k + \alpha y_k)_k \Bigr) \\
                    & = \Bigl(\lambda_k (x_k + \alpha y_k)\Bigr)_k \\
                    & = (\lambda_k x_k + \alpha \lambda_k y_k)_k \\
                    & = D_\lambda((x_k)_k) + \alpha D_\lambda((y_k)_k).
                \end{split}
                \end{equation*}
            Whence $D_\lambda$ is linear. It only remains to show that $D_\lambda$ is bounded. We have:
                \begin{equation*}
                \begin{split}
                    \lnorm D_\lambda \rnorm_\text{op}
                    & = \sup_{(x_k)_k \in B_{\ell_2}} \lnorm (\lambda_k x_k)_k \rnorm_{\ell_2} \\
                    & = \sup_{(x_k)_k \in B_{\ell_2}} \left( \sum_{k = 1}^\infty |\lambda_k x_k|^2 \right)^\frac{1}{2} \\
                    & = \sup_{(x_k)_k \in B_{\ell_2}}\left( \sum_{k = 1}^\infty |\lambda_k|^2 |x_k|^2 \right)^\frac{1}{2} \\
                    & \leq \sup_{(x_k)_k \in B_{\ell_2}}\left( \sum_{k = 1}^\infty \left( \sup_{i = 1}^\infty |\lambda_i| \right)^2 |x_k|^2 \right)^\frac{1}{2} \\
                    & = \sup_{(x_k)_k \in B_{\ell_2}} \left( \sup_{i = 1}^\infty |\lambda_i| \cdot \left( \sum_{k = 1}^\infty |x_k|^2 \right)^\frac{1}{2} \right) \\
                    & = \sup_{i = 1}^\infty |\lambda_i| \cdot \sup_{(x_k)_k \in B_{\ell_2}}\left( \sum_{k = 1}^\infty |x_k|^2 \right)^\frac{1}{2} \\
                    & = \sup_{i = 1}^\infty |\lambda_i| \cdot \sup_{(x_k)_k \in B_{\ell_2}} \lnorm (x_k)_k \rnorm_{\ell_2} \\
                    & \leq \sup_{i = 1}^\infty |\lambda_i| \cdot 1 \\
                    & = \lnorm (\lambda_k)_k \rnorm_\infty < \infty.
                \end{split}
                \end{equation*}
        \end{proof}
%%%%%%%%%%%%%%%%%%%%%%%%%%%%%%%%%%%%%%%%%%%%%%%%%%%%%%%%%%%%%
    \begin{exercise}
        Consider the vector space $C\bigl([0,2\pi]\bigr)$ equipped with the pairing:
            \begin{equation*}
            \begin{split}
                \langle f,g \rangle := \frac{1}{2\pi}\int_0^{2\pi}f(t) \overline{g(t)}dt.
            \end{split}
            \end{equation*}
            \begin{enumerate}[label = (\arabic*),itemsep=1pt,topsep=3pt]
                \item Show that this pairing defines an inner product on $C\bigl([0,2\pi]\bigr)$.
                \item For $n \in \bfZ$ set $e_n(t):=\cos(nt) + i\sin(nt)$. Show that the family $\{e_n\}_{n \in \bfZ}$ is orthonormal.
            \end{enumerate}
    \end{exercise}
        \begin{proof}
            (1) We must first show that $\langle\cdot,\cdot \rangle$ defined as above is sesquilinear. We have:
                \begin{equation*}
                \begin{split}
                    \langle f, g_1 + \alpha g_2 \rangle
                    & = \frac{1}{2\pi} \int_0^{2\pi}f(t)\overline{(g_1 + \alpha g_2)(t)}dt \\
                    & = \frac{1}{2\pi} \int_0^{2\pi} f(t)\overline{g_1(t)}dt + \frac{\overline{\alpha}}{2\pi} \int_0^{2\pi}f(t)\overline{g_2(t)}dt \\
                    & = \langle f,g_1 \rangle + \overline{\alpha}\langle f,g_2 \rangle
                \end{split}
                \end{equation*}

                \begin{equation*}
                \begin{split}
                    \langle f_1 + \alpha f_2,g \rangle
                    & = \frac{1}{2\pi} \int_0^{2\pi}(f_1 + \alpha f_2)(t)\overline{g(t)}dt \\
                    & = \frac{1}{2\pi} \int_0^{2\pi} f_1(t)\overline{g(t)}dt + \frac{\alpha}{2\pi} \int_0^{2\pi} f_2(t)\overline{g(t)}dt \\
                    & = \langle f_1,g \rangle + \alpha \langle f_2,g \rangle.
                \end{split}
                \end{equation*}
            The map $\langle \cdot, \cdot \rangle$ is Hermitian because:
                \begin{equation*}
                \begin{split}
                    \overline{\langle g,f \rangle}
                    & = \overline{\frac{1}{2\pi} \int_0^{2\pi}g(t)\overline{f(t)}dt} \\
                    & = \frac{1}{2\pi} \int_0^{2\pi} \overline{g(t)}f(t)dt \\
                    & = \langle f,g \rangle.
                \end{split}
                \end{equation*}
            It remains to show that $\langle \cdot,\cdot \rangle$ is positive-definite. Observe that:
                \begin{equation*}
                \begin{split}
                    \langle f,f \rangle
                    & = \frac{1}{2\pi} \int_0^{2\pi}f(t)\overline{f(t)}dt \\
                    & = \frac{1}{2\pi} \int_0^{2\pi} |f(t)|^2 dt \\
                    & > 0.
                \end{split}
                \end{equation*}
            Thus $\Bigl(C\bigl([0,2\pi]\bigr),\langle \cdot,\cdot \rangle\Bigr)$ is an inner product space.

            (2) Observe that:
                \begin{equation*}
                \begin{split}
                    \lnorm e_n \rnorm_2 
                    & = \langle e_n,e_n \rangle^\frac{1}{2} \\
                    & = \left( \frac{1}{2\pi} \int_0^{2\pi}(\cos(nt) + i\sin(nt))(\cos(nt)-i\sin(nt))dt \right)^\frac{1}{2} \\
                    & = \left( \frac{1}{2\pi}\int_0^{2\pi}(\cos^2(nt) + \sin^2(nt))dt \right)^\frac{1}{2} \\
                    & = \left( \frac{1}{2\pi} \int_0^{2\pi}1 dt \right)^\frac{1}{2} \\
                    & = 1.
                \end{split}
                \end{equation*}
            Thus $\{e_n\}_{n \in \bfZ}$ is a family of unit vectors. We also have:
                \begin{equation*}
                \begin{split}
                    \langle e_j,e_k \rangle
                    & = \left( \frac{1}{2\pi} \int_0^{2\pi} \left( \cos(jt) + i\sin(jt) \right)\left( \cos(kt) + i\sin(kt) \right)dt \right) \\
                    & = \left( \frac{1}{2\pi} \int_0^{2\pi} \left( \cos(jt)\cos(kt) + i\sin(jt)\cos(kt) - i\sin(kt)\cos(jt) + \sin(jt)\sin(kt) \right) dt \right) \\
                    & = \left( \frac{1}{2\pi} \int_0^{2\pi} \left( \cos((j-k)t) + i\sin((j-k)t) \right) dt \right) \\
                    & = \frac{1}{\sqrt{2\pi}} \cdot \frac{1}{j-k} \left( \sin((j-k)t) - i \cos((j-k)t) \Big|_0^{2\pi}\right) \\
                    & = \frac{1}{\sqrt{2\pi}} \cdot \frac{1}{j-k} \left( -i + i \right) \\
                    & = 0.
                \end{split}
                \end{equation*}
            Thus the family $\{e_n\}_{n \in \bfZ}$ is orthonormal.
        \end{proof}
%%%%%%%%%%%%%%%%%%%%%%%%%%%%%%%%%%%%%%%%%%%%%%%%%%%%%%%%%%%
    \newpage
    \begin{exercise}
        Let $V$ be any normed space, $p \in [1,\infty]$, and suppose $T: \ell_p^n \rightarrow V$ is linear. Show that $T$ is bounded.
    \end{exercise}
        \begin{proof}
            Let $\fB = \{e_1,...,e_n\}$ be a basis for $\ell_p^n$ with $\lnorm e_i \rnorm_p = 1$ for all $i$. Observe that:
                \begin{equation*}
                \begin{split}
                    \lnorm T \rnorm_\text{op} 
                    & = \sup_{v \in B_{\ell_p^n}} \lnorm T(v) \rnorm_p \\
                    & = \sup_{e_i \in B_{\ell_p^n}} \lnorm \sum_{i = 1}^n \alpha_iT(e_i) \rnorm_p \\
                    & \leq \sum_{i = 1}^n  |\alpha_i|\sup_{e_i \in B_{\ell_p^n}}\lnorm T(e_i) \rnorm_p \\
                    & = \sum_{i = 1}^n  |\alpha_i|\max_{e_i \in B_{\ell_p^n}}\lnorm T(e_i) \rnorm_p \\
                    & = \sum_{i = 1}^n |\alpha_i|\lnorm T(e_M) \rnorm_p \\
                    & < \infty.
                \end{split}
                \end{equation*}
            Thus $T$ is bounded.
        \end{proof}
%%%%%%%%%%%%%%%%%%%%%%%%%%%%%%%%%%%%%%%%%%%%%%%%%%%%%%%%%%%%%
    \addtocounter{exercise}{1}
    \begin{exercise}
        Let $V$ be an infinite-dimensional normed space. Show that there is a linear functional $\varphi:V \rightarrow F$ that is unbounded.
    \end{exercise}
        \begin{proof}
            Let $\fB$ be a basis for $V$. Since the cardinality of $\fB$ is infinite, we have that $\bfN \hookrightarrow \fB$. Let $\{v_1,v_2,...\} \subseteq \fB$. Define $\varphi:\fB \rightarrow F$ by: 
                \begin{equation*}
                \begin{split}
                    \varphi(v) = \begin{cases}
                        n,& v \in \{v_1,v_2,...\} \\ 0, & v \not\in \{v_1,v_2,...\}
                    \end{cases}
                \end{split}
                \end{equation*}
            Since this is a map of basis elements, there exists a unique linear map $\varphi_F: V \rightarrow F$ with $\varphi_F(v) = \varphi(v)$ for all $v \in \fB$. Whence:
                \begin{equation*}
                \begin{split}
                    \lnorm \varphi_F \rnorm_\text{op}
                    & = \sup_{v \in B_V}\lnorm \varphi_F(v) \rnorm \\
                    & \geq \sup_{n = 1}^\infty \lnorm \varphi_F(v_n) \rnorm \\
                    & = \sup_{n = 1}^\infty n \\
                    & = \infty.
                \end{split}
                \end{equation*}
            Thus $\varphi_F:V \rightarrow F$ is unbounded.
        \end{proof}
%%%%%%%%%%%%%%%%%%%%%%%%%%%%%%%%%%%%%%%%%%%%%%%%%%%%%%%%%%%%%
\end{document}