\documentclass[11pt,twoside,openany]{memoir}
\usepackage{mlmodern}
%\usepackage{tgpagella} % text only
%\usepackage{mathpazo}  % math & text
\usepackage[T1]{fontenc}
\usepackage[hidelinks]{hyperref}
\usepackage{amsmath}
\usepackage{amsthm}
\usepackage{amssymb}
\usepackage{mathtools}
%\usepackage{newpxtext}
\usepackage{eulerpx}
\usepackage{eucal}
\usepackage{datetime}
    \newdateformat{specialdate}{\THEYEAR\ \monthname\ \THEDAY}
\usepackage[margin=1in]{geometry}
\usepackage{fancyhdr}
    \fancyhf{}
    \pagestyle{fancy}
    \cfoot{\scriptsize \thepage}
    \fancyhead[R]{\scriptsize \rightmark}
    \fancyhead[L]{\scriptsize \leftmark}
    \renewcommand{\headrulewidth}{0pt}
    \renewcommand{\footrulewidth}{0pt} % if you also want to remove the footer rule
\usepackage{thmtools}
    \declaretheoremstyle[
        spaceabove=10pt,
        spacebelow=10pt,
        headfont=\normalfont\bfseries,
        notefont=\mdseries, notebraces={(}{)},
        bodyfont=\normalfont,
        postheadspace=0.5em
        %qed=\qedsymbol
        ]{defs}

    \declaretheoremstyle[ 
        spaceabove=10pt, % space above the theorem
        spacebelow=10pt,
        headfont=\normalfont\bfseries,
        bodyfont=\normalfont\itshape,
        postheadspace=0.5em
        ]{thmstyle}
    
    \declaretheorem[
        style=thmstyle,
        numberwithin=section
    ]{theorem}

    \declaretheorem[
        style=thmstyle,
        sibling=theorem,
    ]{proposition}

    \declaretheorem[
        style=thmstyle,
        sibling=theorem,
    ]{lemma}

    \declaretheorem[
        style=thmstyle,
        sibling=theorem,
    ]{corollary}

    \declaretheorem[
        numberwithin=section,
        style=defs,
    ]{example}

    \declaretheorem[
        numberwithin=section,
        style=defs,
    ]{definition}

    \declaretheorem[
        style=defs,
        numbered=unless unique,
    ]{exercise}

    \declaretheorem[
        numbered=unless unique,
        shaded={rulecolor=black,
    rulewidth=1pt, bgcolor={rgb}{1,1,1}}
    ]{axiom}

    \declaretheorem[numberwithin=section,style=defs]{note}
    \declaretheorem[numbered=no,style=defs]{question}
    \declaretheorem[numbered=no,style=defs]{recall}
    \declaretheorem[numbered=no,style=remark]{answer}
    \declaretheorem[numbered=no,style=remark]{solution}

    \declaretheorem[numbered=no,style=defs]{remark}
\usepackage{enumitem}
\usepackage{titlesec}
    \titleformat{\chapter}[display]
    {\bfseries\LARGE\raggedright}
    {Chapter {\thechapter}}
    {1ex minus .1ex}
    {\Huge}
    \titlespacing{\chapter}
    {3pc}{*3}{40pt}[3pc]

    \titleformat{\section}[block]
    {\normalfont\bfseries\Large}
    {\S\ \thesection.}{.5em}{}[]
    \titlespacing{\section}
    {0pt}{3ex plus .1ex minus .2ex}{3ex plus .1ex minus .2ex}
\usepackage[utf8x]{inputenc}
\usepackage{tikz}
\usepackage{tikz-cd}
\usepackage{wasysym}
\renewcommand{\int}{\varint}

\linespread{0.95}
%to make the correct symbol for Sha
%\newcommand\cyr{%
%\renewcommand\rmdefault{wncyr}%
%\renewcommand\sfdefault{wncyss}%
%\renewcommand\encodingdefault{OT2}%
%\normalfont \selectfont} \DeclareTextFontCommand{\textcyr}{\cyr}


\DeclareMathOperator{\ab}{ab}
\newcommand{\absgal}{\G_{\bbQ}}
\DeclareMathOperator{\ad}{ad}
\DeclareMathOperator{\adj}{adj}
\DeclareMathOperator{\alg}{alg}
\DeclareMathOperator{\Alt}{Alt}
\DeclareMathOperator{\Ann}{Ann}
\DeclareMathOperator{\arith}{arith}
\DeclareMathOperator{\Aut}{Aut}
\DeclareMathOperator{\Be}{B}
\DeclareMathOperator{\Bd}{Bd}
\DeclareMathOperator{\card}{card}
\DeclareMathOperator{\Char}{char}
\DeclareMathOperator{\csp}{csp}
\DeclareMathOperator{\codim}{codim}
\DeclareMathOperator{\coker}{coker}
\DeclareMathOperator{\coh}{H}
\DeclareMathOperator{\compl}{compl}
\DeclareMathOperator{\conj}{conj}
\DeclareMathOperator{\cont}{cont}
\DeclareMathOperator{\crys}{crys}
\DeclareMathOperator{\Crys}{Crys}
\DeclareMathOperator{\cusp}{cusp}
\DeclareMathOperator{\diag}{diag}
\DeclareMathOperator{\diam}{diam}
\DeclareMathOperator{\Dom}{Dom}
\DeclareMathOperator{\disc}{disc}
\DeclareMathOperator{\dist}{dist}
\DeclareMathOperator{\dR}{dR}
\DeclareMathOperator{\Eis}{Eis}
\DeclareMathOperator{\End}{End}
\DeclareMathOperator{\ev}{ev}
\DeclareMathOperator{\eval}{eval}
\DeclareMathOperator{\Eq}{Eq}
\DeclareMathOperator{\Ext}{Ext}
\DeclareMathOperator{\Fil}{Fil}
\DeclareMathOperator{\Fitt}{Fitt}
\DeclareMathOperator{\Frob}{Frob}
\DeclareMathOperator{\G}{G}
\DeclareMathOperator{\Gal}{Gal}
\DeclareMathOperator{\GL}{GL}
\DeclareMathOperator{\Gr}{Gr}
\DeclareMathOperator{\Graph}{Graph}
\DeclareMathOperator{\GSp}{GSp}
\DeclareMathOperator{\GUn}{GU}
\DeclareMathOperator{\Hom}{Hom}
\DeclareMathOperator{\id}{id}
\DeclareMathOperator{\Id}{Id}
\DeclareMathOperator{\Ik}{Ik}
\DeclareMathOperator{\IM}{Im}
\DeclareMathOperator{\Image}{im}
\DeclareMathOperator{\Ind}{Ind}
\DeclareMathOperator{\Inf}{inf}
\DeclareMathOperator{\Isom}{Isom}
\DeclareMathOperator{\J}{J}
\DeclareMathOperator{\Jac}{Jac}
\DeclareMathOperator{\lcm}{lcm}
\DeclareMathOperator{\length}{length}
\DeclareMathOperator*{\limit}{limit}
\DeclareMathOperator{\Log}{Log}
\DeclareMathOperator{\M}{M}
\DeclareMathOperator{\Mat}{Mat}
\DeclareMathOperator{\N}{N}
\DeclareMathOperator{\Nm}{Nm}
\DeclareMathOperator{\NIk}{N-Ik}
\DeclareMathOperator{\NSK}{N-SK}
\DeclareMathOperator{\new}{new}
\DeclareMathOperator{\obj}{obj}
\DeclareMathOperator{\old}{old}
\DeclareMathOperator{\ord}{ord}
\DeclareMathOperator{\Or}{O}
\DeclareMathOperator{\op}{op}
\DeclareMathOperator{\PGL}{PGL}
\DeclareMathOperator{\PGSp}{PGSp}
\DeclareMathOperator{\rank}{rank}
\DeclareMathOperator{\Ran}{Ran}
\DeclareMathOperator{\Rel}{Rel}
\DeclareMathOperator{\Real}{Re}
\DeclareMathOperator{\RES}{res}
\DeclareMathOperator{\Res}{Res}
%\DeclareMathOperator{\Sha}{\textcyr{Sh}}
\DeclareMathOperator{\Sel}{Sel}
\DeclareMathOperator{\semi}{ss}
\DeclareMathOperator{\sgn}{sign}
\DeclareMathOperator{\SK}{SK}
\DeclareMathOperator{\SL}{SL}
\DeclareMathOperator{\SO}{SO}
\DeclareMathOperator{\Sp}{Sp}
\DeclareMathOperator{\Span}{span}
\DeclareMathOperator{\Spec}{Spec}
\DeclareMathOperator{\spin}{spin}
\DeclareMathOperator{\st}{st}
\DeclareMathOperator{\St}{St}
\DeclareMathOperator{\SUn}{SU}
\DeclareMathOperator{\supp}{supp}
\DeclareMathOperator{\Sup}{sup}
\DeclareMathOperator{\Sym}{Sym}
\DeclareMathOperator{\Tam}{Tam}
\DeclareMathOperator{\tors}{tors}
\DeclareMathOperator{\tr}{tr}
\DeclareMathOperator{\Tr}{Tr}
\DeclareMathOperator{\un}{un}
\DeclareMathOperator{\Un}{U}
\DeclareMathOperator{\val}{val}
\DeclareMathOperator{\vol}{vol}

\DeclareMathOperator{\Sets}{S \mkern1.04mu e \mkern1.04mu t \mkern1.04mu s}
    \newcommand{\cSets}{\scalebox{1.02}{\contour{black}{$\Sets$}}}
    
\DeclareMathOperator{\Groups}{G \mkern1.04mu r \mkern1.04mu o \mkern1.04mu u \mkern1.04mu p \mkern1.04mu s}
    \newcommand{\cGroups}{\scalebox{1.02}{\contour{black}{$\Groups$}}}

\DeclareMathOperator{\TTop}{T \mkern1.04mu o \mkern1.04mu p}
    \newcommand{\cTop}{\scalebox{1.02}{\contour{black}{$\TTop$}}}

\DeclareMathOperator{\Htp}{H \mkern1.04mu t \mkern1.04mu p}
    \newcommand{\cHtp}{\scalebox{1.02}{\contour{black}{$\Htp$}}}

\DeclareMathOperator{\Mod}{M \mkern1.04mu o \mkern1.04mu d}
    \newcommand{\cMod}{\scalebox{1.02}{\contour{black}{$\Mod$}}}

\DeclareMathOperator{\Ab}{A \mkern1.04mu b}
    \newcommand{\cAb}{\scalebox{1.02}{\contour{black}{$\Ab$}}}

\DeclareMathOperator{\Rings}{R \mkern1.04mu i \mkern1.04mu n \mkern1.04mu g \mkern1.04mu s}
    \newcommand{\cRings}{\scalebox{1.02}{\contour{black}{$\Rings$}}}

\DeclareMathOperator{\ComRings}{C \mkern1.04mu o \mkern1.04mu m \mkern1.04mu R \mkern1.04mu i \mkern1.04mu n \mkern1.04mu g \mkern1.04mu s}
    \newcommand{\cComRings}{\scalebox{1.05}{\contour{black}{$\ComRings$}}}

\DeclareMathOperator{\hHom}{H \mkern1.04mu o \mkern1.04mu m}
    \newcommand{\cHom}{\scalebox{1.02}{\contour{black}{$\hHom$}}}

         %  \item $\cGroups$
          %  \item $\cTop$
          %  \item $\cHtp$
          %  \item $\cMod$




\renewcommand{\k}{\kappa}
\newcommand{\Ff}{F_{f}}
%\newcommand{\ts}{\,^{t}\!}


%Mathcal

\newcommand{\cA}{\mathcal{A}}
\newcommand{\cB}{\mathcal{B}}
\newcommand{\cC}{\mathcal{C}}
\newcommand{\cD}{\mathcal{D}}
\newcommand{\cE}{\mathcal{E}}
\newcommand{\cF}{\mathcal{F}}
\newcommand{\cG}{\mathcal{G}}
\newcommand{\cH}{\mathcal{H}}
\newcommand{\cI}{\mathcal{I}}
\newcommand{\cJ}{\mathcal{J}}
\newcommand{\cK}{\mathcal{K}}
\newcommand{\cL}{\mathcal{L}}
\newcommand{\cM}{\mathcal{M}}
\newcommand{\cN}{\mathcal{N}}
\newcommand{\cO}{\mathcal{O}}
\newcommand{\cP}{\mathcal{P}}
\newcommand{\cQ}{\mathcal{Q}}
\newcommand{\cR}{\mathcal{R}}
\newcommand{\cS}{\mathcal{S}}
\newcommand{\cT}{\mathcal{T}}
\newcommand{\cU}{\mathcal{U}}
\newcommand{\cV}{\mathcal{V}}
\newcommand{\cW}{\mathcal{W}}
\newcommand{\cX}{\mathcal{X}}
\newcommand{\cY}{\mathcal{Y}}
\newcommand{\cZ}{\mathcal{Z}}


%mathfrak (missing \fi)

\newcommand{\fa}{\mathfrak{a}}
\newcommand{\fA}{\mathfrak{A}}
\newcommand{\fb}{\mathfrak{b}}
\newcommand{\fB}{\mathfrak{B}}
\newcommand{\fc}{\mathfrak{c}}
\newcommand{\fC}{\mathfrak{C}}
\newcommand{\fd}{\mathfrak{d}}
\newcommand{\fD}{\mathfrak{D}}
\newcommand{\fe}{\mathfrak{e}}
\newcommand{\fE}{\mathfrak{E}}
\newcommand{\ff}{\mathfrak{f}}
\newcommand{\fF}{\mathfrak{F}}
\newcommand{\fg}{\mathfrak{g}}
\newcommand{\fG}{\mathfrak{G}}
\newcommand{\fh}{\mathfrak{h}}
\newcommand{\fH}{\mathfrak{H}}
\newcommand{\fI}{\mathfrak{I}}
\newcommand{\fj}{\mathfrak{j}}
\newcommand{\fJ}{\mathfrak{J}}
\newcommand{\fk}{\mathfrak{k}}
\newcommand{\fK}{\mathfrak{K}}
\newcommand{\fl}{\mathfrak{l}}
\newcommand{\fL}{\mathfrak{L}}
\newcommand{\fm}{\mathfrak{m}}
\newcommand{\fM}{\mathfrak{M}}
\newcommand{\fn}{\mathfrak{n}}
\newcommand{\fN}{\mathfrak{N}}
\newcommand{\fo}{\mathfrak{o}}
\newcommand{\fO}{\mathfrak{O}}
\newcommand{\fp}{\mathfrak{p}}
\newcommand{\fP}{\mathfrak{P}}
\newcommand{\fq}{\mathfrak{q}}
\newcommand{\fQ}{\mathfrak{Q}}
\newcommand{\fr}{\mathfrak{r}}
\newcommand{\fR}{\mathfrak{R}}
\newcommand{\fs}{\mathfrak{s}}
\newcommand{\fS}{\mathfrak{S}}
\newcommand{\ft}{\mathfrak{t}}
\newcommand{\fT}{\mathfrak{T}}
\newcommand{\fu}{\mathfrak{u}}
\newcommand{\fU}{\mathfrak{U}}
\newcommand{\fv}{\mathfrak{v}}
\newcommand{\fV}{\mathfrak{V}}
\newcommand{\fw}{\mathfrak{w}}
\newcommand{\fW}{\mathfrak{W}}
\newcommand{\fx}{\mathfrak{x}}
\newcommand{\fX}{\mathfrak{X}}
\newcommand{\fy}{\mathfrak{y}}
\newcommand{\fY}{\mathfrak{Y}}
\newcommand{\fz}{\mathfrak{z}}
\newcommand{\fZ}{\mathfrak{Z}}


%mathbf
\newcommand{\bfA}{\mathbf{A}}
\newcommand{\bfB}{\mathbf{B}}
\newcommand{\bfC}{\mathbf{C}}
\newcommand{\bfD}{\mathbf{D}}
\newcommand{\bfE}{\mathbf{E}}
\newcommand{\bfF}{\mathbf{F}}
\newcommand{\bfG}{\mathbf{G}}
\newcommand{\bfH}{\mathbf{H}}
\newcommand{\bfI}{\mathbf{I}}
\newcommand{\bfJ}{\mathbf{J}}
\newcommand{\bfK}{\mathbf{K}}
\newcommand{\bfL}{\mathbf{L}}
\newcommand{\bfM}{\mathbf{M}}
\newcommand{\bfN}{\mathbf{N}}
\newcommand{\bfO}{\mathbf{O}}
\newcommand{\bfP}{\mathbf{P}}
\newcommand{\bfQ}{\mathbf{Q}}
\newcommand{\bfR}{\mathbf{R}}
\newcommand{\bfS}{\mathbf{S}}
\newcommand{\bfT}{\mathbf{T}}
\newcommand{\bfU}{\mathbf{U}}
\newcommand{\bfV}{\mathbf{V}}
\newcommand{\bfW}{\mathbf{W}}
\newcommand{\bfX}{\mathbf{X}}
\newcommand{\bfY}{\mathbf{Y}}
\newcommand{\bfZ}{\mathbf{Z}}

\newcommand{\bfa}{\mathbf{a}}
\newcommand{\bfb}{\mathbf{b}}
\newcommand{\bfc}{\mathbf{c}}
\newcommand{\bfd}{\mathbf{d}}
\newcommand{\bfe}{\mathbf{e}}
\newcommand{\bff}{\mathbf{f}}
\newcommand{\bfg}{\mathbf{g}}
\newcommand{\bfh}{\mathbf{h}}
\newcommand{\bfi}{\mathbf{i}}
\newcommand{\bfj}{\mathbf{j}}
\newcommand{\bfk}{\mathbf{k}}
\newcommand{\bfl}{\mathbf{l}}
\newcommand{\bfm}{\mathbf{m}}
\newcommand{\bfn}{\mathbf{n}}
\newcommand{\bfo}{\mathbf{o}}
\newcommand{\bfp}{\mathbf{p}}
\newcommand{\bfq}{\mathbf{q}}
\newcommand{\bfr}{\mathbf{r}}
\newcommand{\bfs}{\mathbf{s}}
\newcommand{\bft}{\mathbf{t}}
\newcommand{\bfu}{\mathbf{u}}
\newcommand{\bfv}{\mathbf{v}}
\newcommand{\bfw}{\mathbf{w}}
\newcommand{\bfx}{\mathbf{x}}
\newcommand{\bfy}{\mathbf{y}}
\newcommand{\bfz}{\mathbf{z}}

%blackboard bold

\newcommand{\bbA}{\mathbb{A}}
\newcommand{\bbB}{\mathbb{B}}
\newcommand{\bbC}{\mathbb{C}}
\newcommand{\bbD}{\mathbb{D}}
\newcommand{\bbE}{\mathbb{E}}
\newcommand{\bbF}{\mathbb{F}}
\newcommand{\bbG}{\mathbb{G}}
\newcommand{\bbH}{\mathbb{H}}
\newcommand{\bbI}{\mathbb{I}}
\newcommand{\bbJ}{\mathbb{J}}
\newcommand{\bbK}{\mathbb{K}}
\newcommand{\bbL}{\mathbb{L}}
\newcommand{\bbM}{\mathbb{M}}
\newcommand{\bbN}{\mathbb{N}}
\newcommand{\bbO}{\mathbb{O}}
\newcommand{\bbP}{\mathbb{P}}
\newcommand{\bbQ}{\mathbb{Q}}
\newcommand{\bbR}{\mathbb{R}}
\newcommand{\bbS}{\mathbb{S}}
\newcommand{\bbT}{\mathbb{T}}
\newcommand{\bbU}{\mathbb{U}}
\newcommand{\bbV}{\mathbb{V}}
\newcommand{\bbW}{\mathbb{W}}
\newcommand{\bbX}{\mathbb{X}}
\newcommand{\bbY}{\mathbb{Y}}
\newcommand{\bbZ}{\mathbb{Z}}
\newcommand{\jota}{\jmath}

\newcommand{\bmat}{\left( \begin{matrix}}
\newcommand{\emat}{\end{matrix} \right)}

\newcommand{\pmat}{\left( \begin{smallmatrix}}
\newcommand{\epmat}{\end{smallmatrix} \right)}

\newcommand{\lat}{\mathscr{L}}
\newcommand{\mat}[4]{\begin{pmatrix}{#1}&{#2}\\{#3}&{#4}\end{pmatrix}}
\newcommand{\ov}[1]{\overline{#1}}
\newcommand{\res}[1]{\underset{#1}{\RES}\,}
\newcommand{\up}{\upsilon}

\newcommand{\tac}{\textasteriskcentered}

%mahesh macros
\newcommand{\tm}{\textrm}

%Comments
\newcommand{\com}[1]{\vspace{5 mm}\par \noindent
\marginpar{\textsc{Comment}} \framebox{\begin{minipage}[c]{0.95
\textwidth} \tt #1 \end{minipage}}\vspace{5 mm}\par}

\newcommand{\Bmu}{\mbox{$\raisebox{-0.59ex}
  {$l$}\hspace{-0.18em}\mu\hspace{-0.88em}\raisebox{-0.98ex}{\scalebox{2}
  {$\color{white}.$}}\hspace{-0.416em}\raisebox{+0.88ex}
  {$\color{white}.$}\hspace{0.46em}$}{}}  %need graphicx and xcolor. this produces blackboard bold mu 

\newcommand{\hooktwoheadrightarrow}{%
  \hookrightarrow\mathrel{\mspace{-15mu}}\rightarrow
}

\makeatletter
\newcommand{\xhooktwoheadrightarrow}[2][]{%
  \lhook\joinrel
  \ext@arrow 0359\rightarrowfill@ {#1}{#2}%
  \mathrel{\mspace{-15mu}}\rightarrow
}
\makeatother

\renewcommand{\geq}{\geqslant}
\renewcommand{\leq}{\leqslant}
\newcommand{\midd}{\hspace{4pt}\middle|\hspace{4pt}}
    
    \newcommand{\bone}{\mathbf{1}}
    \newcommand{\sign}{\mathrm{sign}}
    \newcommand{\eps}{\varepsilon}
    \newcommand{\textui}[1]{\uline{\textit{#1}}}
    
    %\newcommand{\ov}{\overline}
    %\newcommand{\un}{\underline}
    \newcommand{\fin}{\mathrm{fin}}
    
    \newcommand{\chnum}{\titleformat
    {\chapter} % command
    [display] % shape
    {\centering} % format
    {\Huge \color{black} \shadowbox{\thechapter}} % label
    {-0.5em} % sep (space between the number and title)
    {\LARGE \color{black} \underline} % before-code
    }
    
    \newcommand{\chunnum}{\titleformat
    {\chapter} % command
    [display] % shape
    {} % format
    {} % label
    {0em} % sep
    { \begin{flushright} \begin{tabular}{r}  \Huge \color{black}
    } % before-code
    [
    \end{tabular} \end{flushright} \normalsize
    ] % after-code
    }

\newcommand{\nl}{\newline \mbox{}}

\newcommand{\h}[1]{\hspace{#1pt}}

\newcommand{\littletaller}{\mathchoice{\vphantom{\big|}}{}{}{}}
\newcommand\restr[2]{{% we make the whole thing an ordinary symbol
  \left.\kern-\nulldelimiterspace % automatically resize the bar with \right
  #1 % the function
  \littletaller % pretend it's a little taller at normal size
  \right|_{#2} % this is the delimiter
  }}

\newcommand{\mtext}[1]{\hspace{6pt}\text{#1}\hspace{6pt}}

\newcommand{\lnorm}{\left\lVert}
\newcommand{\rnorm}{\right\rVert}

\newcommand{\ds}{\displaystyle}
\newcommand{\ts}{\textstyle}

%This adds a "front cover" page.
%{\thispagestyle{empty}
%\vspace*{\fill}
%\begin{tabular}{l}
%\begin{tabular}{l}
%\includegraphics[scale=0.24]{oxy-logo.png}
%\end{tabular} \\
%\begin{tabular}{l}
%\Large \color{black} Module Theory, Linear Algebra, and Homological Algebra \\ \Large \color{black} Gianluca Crescenzo
%\end{tabular}
%\end{tabular}
%\newpage

\newcommand{\sfrac}[2]{{}^{#1}\mskip -5mu/\mskip -3mu_{#2}}


\makeatletter
\newcommand*{\da@rightarrow}{\mathchar"0\hexnumber@\symAMSa 4B }
\newcommand*{\da@leftarrow}{\mathchar"0\hexnumber@\symAMSa 4C }
\newcommand*{\xdashrightarrow}[2][]{%
  \mathrel{%
    \mathpalette{\da@xarrow{#1}{#2}{}\da@rightarrow{\,}{}}{}%
  }%
}
\newcommand{\xdashleftarrow}[2][]{%
  \mathrel{%
    \mathpalette{\da@xarrow{#1}{#2}\da@leftarrow{}{}{\,}}{}%
  }%
}
\newcommand*{\da@xarrow}[7]{%
  % #1: below
  % #2: above
  % #3: arrow left
  % #4: arrow right
  % #5: space left 
  % #6: space right
  % #7: math style 
  \sbox0{$\ifx#7\scriptstyle\scriptscriptstyle\else\scriptstyle\fi#5#1#6\m@th$}%
  \sbox2{$\ifx#7\scriptstyle\scriptscriptstyle\else\scriptstyle\fi#5#2#6\m@th$}%
  \sbox4{$#7\dabar@\m@th$}%
  \dimen@=\wd0 %
  \ifdim\wd2 >\dimen@
    \dimen@=\wd2 %   
  \fi
  \count@=2 %
  \def\da@bars{\dabar@\dabar@}%
  \@whiledim\count@\wd4<\dimen@\do{%
    \advance\count@\@ne
    \expandafter\def\expandafter\da@bars\expandafter{%
      \da@bars
      \dabar@ 
    }%
  }%  
  \mathrel{#3}%
  \mathrel{%   
    \mathop{\da@bars}\limits
    \ifx\\#1\\%
    \else
      _{\copy0}%
    \fi
    \ifx\\#2\\%
    \else
      ^{\copy2}%
    \fi
  }%   
  \mathrel{#4}%
}
\makeatother


\begin{document}
\begin{center}
{\large Math 397 \\[0.1in]Homework 1 \\[0.1in]}
{Name:} {\underline{Gianluca Crescenzo\hspace*{2in}}}\\[0.15in]
\end{center}
\vspace{4pt}
%%%%%%%%%%%%%%%%%%%%%%%%%%%%%%%%%%%%%%%%%%%%%%%%%%%%%%%%%%%%%
    \begin{exercise}
        Let $V$ be a vector space, and suppose $\{W_i\}_{i \in I}$ is a family of subspaces of $V$. 
            \begin{enumerate}[label = (\arabic*),itemsep=1pt,topsep=3pt]
                \item Show that $\bigcap_{i \in I}W_i$ is the largest subspace of $V$ contained in every $W_i$.
                \item Show that:
                    \begin{equation*}
                    \begin{split}
                        \sum_{i \in I}W_i = \left\{\sum_{i \in F}w_i \mid w_i \in W_i, \h2 F \subseteq I \h2\text{finite}  \right\}
                    \end{split}
                    \end{equation*}
                is the smallest subspace containing each $W_i$.
                \item We say that $V$ is the \textit{internal direct sum} of the family $\{W_i\}_{i \in I}$ and we write $V = \bigoplus_{i \in I}W_i$ if:
                    \begin{enumerate}[label = (\roman*),itemsep=1pt,topsep=3pt]
                        \item $V = \sum_{i \in I}W_i$;
                        \item For each $j \in I$, $W_j \cap \sum_{i \neq j}W_i n \{0\}$.
                    \end{enumerate}
                If $V = \bigoplus_{i \in I}W_i$, show that every $v \in V$ has a unique expression $v = \sum_{i \in F}w_i$ where $F \subseteq I$ is finite and $0 \neq w_i$ for each $w_i \in W_i$.
            \end{enumerate}
    \end{exercise}
        \begin{proof}
            (1) Let $U$ be a subspace of $V$ with $U \subseteq W_i$ for each $i \in I$. Then clearly $U \subseteq \bigcap_{i \in I}W_i$. \nl

            \noindent (2) Let $W = \sum_{i \in I}W_i$ and let $U$ be a subspace of $V$ with $U \supseteq W_i$ for each $i \in I$. If $x \in W$, then $x = \sum_{i \in I}w_i$. But since $W_i$ is a subspace, it is closed under addition. Whence $x \in W_i$ for each $i \in I$. By inclusion then, $x \in U$. Hence $W \subseteq U$. \nl

            \noindent (3) By the definition of internal direct sums $V = \sum_{i \in I}W_i$, whence each $v \in V$ can be written as $v = \sum_{i \in F}w_i$. It remains to show that this expression is unique. Suppose $v = \sum_{i \in F}w_i = \sum_{i \in F}u_i$ with $w_i,u_i \in W_i$. For each $j$ we have:
                \begin{equation*}
                \begin{split}
                    w_j - u_j = \ds \sum_{\mathclap{\substack{ i \in F \\ i \neq j }}} (w_i - u_i)
                \end{split}
                \end{equation*}
            But notice that $w_j - u_j \in W_j$ and $ \sum_{i \in F, i \neq j } (w_i - u_i) \in \sum_{i \neq j}W_i$. So $w_j - u_j \in W_j \cap \sum_{i \neq j}W_i$. By the definition of internal direct sums this gives $w_j - u_j = 0$, which simplifies to $w_j = u_j$. 
        \end{proof}
%%%%%%%%%%%%%%%%%%%%%%%%%%%%%%%%%%%%%%%%%%%%%%%%%%%%%%%%%%%%%
   \addtocounter{exercise}{1}
%%%%%%%%%%%%%%%%%%%%%%%%%%%%%%%%%%%%%%%%%%%%%%%%%%%%%% %%%%%%%
    \begin{exercise}
        Let $V$ be a vector space with subspaces $W_i \subseteq V$ for $i = 1,2$. If $W_1 \cup W_2 \subseteq V$ is a subspace, show that $W_1 \subseteq W_2$ or $W_2 \subseteq W_1$.
    \end{exercise}
        \begin{proof}
            Suppose towards contradiction $W_1 \not\subseteq W_2$ and $W_2 \not\subseteq W_1$. Then there exists $w_1 \in W_1 \setminus W_2$ and $w_2 \in W_2 \setminus W_1$. Let $v = w_1 + w_2$. Then $v \in W_1 \cup W_2$. But this means $w_2 = v - w_1 \in W_2$. Whence $w_1 \in W_2$, which is a contradiction.
        \end{proof}
%%%%%%%%%%%%%%%%%%%%%%%%%%%%%%%%%%%%%%%%%%%%%%%%%%%%%%%%%%%%%
    \begin{exercise}
        Let $V$ be a vector space over $F$ and suppose $W \subset V$ is a subspace.
        \begin{enumerate}[label = (\arabic*),itemsep=1pt,topsep=3pt]
            \item Show that the quotient space $V/W = \{[v]_w \mid v \in V\}$ is a vector space with operations:
                \begin{equation*}
                \begin{split}
                    [u]_W + [v]_W = [u + v]_W \h2 ; \h6 \alpha[v]_W = [\alpha v]_W.
                \end{split}
                \end{equation*}
            \item Suppose $\lnorm \cdot \rnorm$ is a norm on $V$. Show that:
            \begin{equation*}
            \begin{split}
                \lnorm [v]_W \rnorm_{V/W} := \inf_{w \in W}\lnorm v - w \rnorm
            \end{split}
            \end{equation*}
        is a seminorm.
        \end{enumerate}
    \end{exercise}
        \begin{proof}
            (1) Since $V$ is an abelian group and $W \subseteq V$ is normal, $V/W$ is an abelian group. It only remains to show that $\alpha[v]_W = [\alpha v]_W$ satisfies the vector space axioms. We have that:
                \begin{equation*}
                \begin{split}
                    \alpha \left( [u]_W + [v]_W \right)
                    & = \alpha [u+v]_W \\
                    & = [\alpha (u+v)]_W \\
                    & = [\alpha u + \alpha v]_W \\
                    & = [\alpha u]_W + [\alpha  v]_W,
                \end{split}
                \end{equation*}
                \begin{equation*}
                \begin{split}
                    \alpha \left( \beta [v]_W \right)
                    & = \alpha [\beta v]_W \\
                    & = [\alpha (\beta v)]_W \\
                    & = [(\alpha \beta)v]_W \\
                    & = (\alpha \beta)[v]_W,
                \end{split}
                \end{equation*}
                \begin{equation*}
                \begin{split}
                    1_F [v]_W 
                    & = [1_F v]_W \\
                    & = [v]_W.
                \end{split}
                \end{equation*}
            Whence $V/W$ is a vector space.

            (2) We must first show that $\lnorm \cdot \rnorm_{V/W} : V/W \rightarrow F$ is well-defined. Let $[v_1]_W  = [v_2]_W$. Then $v_2 - v_1 \in W$. Observe that:
                \begin{equation*}
                \begin{split}
                    \lnorm [v_1]_W \rnorm_{V/W}
                    & = \inf_{w \in W} \lnorm v_1 - w \rnorm \\
                    & = \inf_{w - (v_2 - v_1) \in W} \lnorm v_1 - \bigl(w - (v_2 - v_1)\bigr) \rnorm  \\
                    & = \inf_{w - (v_2 - v_1) \in W} \lnorm v_1 - w + v_2 - v_1 \rnorm \\
                    & = \inf_{w \in W}\lnorm v_2 - w\rnorm \\
                    & = \lnorm [v_2]_W \rnorm_{V/W}.
                \end{split}
                \end{equation*}
            We also have that:
                \begin{equation*}
                \begin{split}
                    \lnorm \alpha [v]_W \rnorm_{V/W}
                    & = \lnorm [\alpha v]_W \rnorm_{V/W} \\
                    & = \inf_{w \in W}\lnorm \alpha v - w \rnorm \\
                    & = \inf_{w' \in W} \lnorm \alpha v - \alpha w' \rnorm \\
                    & = \inf_{w' \in W} \lnorm \alpha (v-w') \rnorm \\
                    & = |\alpha| \inf_{w' \in W} \lnorm v - w' \rnorm \\
                    & = |\alpha| \lnorm [v]_W \rnorm_{V/W}.
                \end{split}
                \end{equation*}
            Whence $\lnorm \cdot \rnorm_{V/W}$ is homogenous. Finally, we can see that:
                \begin{equation*}
                \begin{split}
                    \lnorm [u]_W + [v]_W \rnorm_{V/W}
                    & = \lnorm [u+v]_W \rnorm_{V/W} \\
                    & = \inf_{w \in W} \lnorm u+v - w \rnorm \\
                    & = \inf_{w,w' \in W} \lnorm u + v - (w + w') \rnorm \\
                    & = \inf_{w,w' \in W} \lnorm u-w + v-w' \rnorm \\
                    & \leq \inf_{w,w' \in W} \left( \lnorm  u-w\rnorm +  \lnorm v - w' \rnorm\right) \\
                    & = \inf_{w \in W}\lnorm u-w \rnorm + \inf_{w' \in W} \lnorm v - w' \rnorm \\
                    & = \lnorm [u]_W \rnorm_{V/W} + \lnorm [v]_W \rnorm_{V/W}.
                \end{split}
                \end{equation*}
            Thus $\lnorm \cdot \rnorm_{V/W}$ is a seminorm.
        \end{proof}
%%%%%%%%%%%%%%%%%%%%%%%%%%%%%%%%%%%%%%%%%%%%%%%%%%%%%%%%%%%%%
    \begin{exercise}
        Show that the quantity:
            \begin{equation*}
            \begin{split}
                \lnorm f \rnorm_1 := \int_0^1 |f(t)|dt
            \end{split}
            \end{equation*}
        defines a norm on $C([0,1])$ with $\lnorm f \rnorm_1 \leq \lnorm f \rnorm_u$. Are $\lnorm \cdot \rnorm_1$ and $\lnorm \cdot \rnorm_u$ equivalent norms?
    \end{exercise}
        \begin{proof}
            $\lnorm \cdot \rnorm_1$ is homogenous because:
                \begin{equation*}
                \begin{split}
                    \lnorm \alpha f \rnorm_1
                    & = \int_0^1 |(\alpha f)(t)| dt \\
                    & = \int_0^1 |\alpha f(t)|dt \\
                    & = |\alpha|\int_0^1 |f(t)|dt \\
                    & = |\alpha| \lnorm f \rnorm_1.
                \end{split}
                \end{equation*}
            Note that $|f(t) + g(t)| \leq |f(t)| + |g(t)|$. Integrating both sides gives:
                \begin{equation*}
                \begin{split}
                    \int_0^1 |f(t) + g(t)|dt
                    & = \int_0^1 |(f+g)(t)|dt \\
                    & = \lnorm f+g \rnorm_1 \\
                    & \leq \int_0^1 \left( |f(t)| + |g(t)| \right)dt \\
                    & = \int_0^1 |f(t)|dt + \int_0^1 |g(t)|dt \\
                    & = \lnorm f \rnorm_1 +\lnorm g \rnorm_1.
                \end{split}
                \end{equation*}
            Whence our norm satisfies the triangle inequality. Now suppose $\lnorm \cdot \rnorm_1 = 0$. Then $\int_0^1 |f(t)|dt = 0$. Suppose $f \geq 0$ on $[0,1]$. Since $f$ is continuous, it is continuous at $f(c)$ for some $c \in [0,1]$. If $f(c) > 0$, then there exists $\delta >0$ such that $f(t) \geq \frac{f(c)}{2} > 0$ for all $t \in V_\delta(c)$. This gives:
                \begin{equation*}
                \begin{split}
                    0  =\int_0^1 f(t)dt \geq \int_{c - \delta}^{c+\delta}f(t)dt \geq \int_{c-\delta}^{c+\delta}\frac{f(c)}{2} = f(c) > 0.
                \end{split}
                \end{equation*}
            This is a contradiction. Since $c \in [0,1]$ was arbitrary, it must be that $f = 0$, satisfying positive-definiteness. Moreover, note that $|f(t)| \leq \sup_{t \in [0,1]}|f(t)|$. We have that $\int_0^1 |f(t)|dt \leq \int_0^1 \sup_{t \in [0,1]}|f(t)|dt$, which is equivalent to $\lnorm f \rnorm_1 \leq \int_0^1 \lnorm f \rnorm_u dt = \lnorm f \rnorm_u$.

            Suppose that $\lnorm \cdot \rnorm_1$ and $\lnorm \cdot \rnorm_u$ are equivalent. Then $\lnorm f \rnorm_u \leq c \lnorm f \rnorm_1$. Consider $g(t) = t^N$, where $N > c$. Then:
                \begin{equation*}
                \begin{split}
                    1 &= \sup_{t \in [0,1]}|t^N| \\
                    & \leq \int_0^1 |t^N| dt \\ 
                    & = \frac{c}{N} \\
                    & < 1,
                \end{split}
                \end{equation*}
            This is a contradiction, hence $\lnorm \cdot \rnorm_1$ and $\lnorm \cdot \rnorm_u$ are not equivalent.
        \end{proof}
%%%%%%%%%%%%%%%%%%%%%%%%%%%%%%%%%%%%%%%%%%%%%%%%%%%%%%%%%%%%%
    \begin{exercise}
        Show that all the $p$-norms $\lnorm \cdot \rnorm_p$ ($1 \leq p \leq \infty$) on $F^n$ are equivalent and if $1 \leq p \leq q \leq \infty$, then $\ell_p \subseteq \ell_q$.
    \end{exercise}
        \begin{proof}
            Let $x \in F^n$. We have that:
                \begin{equation*}
                \begin{split}
                    \lnorm x \rnorm_p &= \left( \sum_{i = 1}^n |x_i|^p \right)^\frac{1}{p} 
                     \leq \left( \sum_{i = 1}^n \left( \max_{i = 1}^n |x_i| \right)^p \right)^\frac{1}{p} 
                     = \left( \sum_{i = 1}^n \lnorm x \rnorm_\infty ^p \right)^\frac{1}{p} 
                     = n^p \lnorm x \rnorm_\infty.
                \end{split}
                \end{equation*}

                \begin{equation*}
                \begin{split}
                    \lnorm x \rnorm_\infty = \left( \left( \max_{i = 1}^n |x_i| \right)^p \right)^\frac{1}{p} \leq \left( \sum_{i = 1}^n |x_i|^p \right)^\frac{1}{p} = \lnorm x \rnorm_p.
                \end{split}
                \end{equation*}

                \begin{equation*}
                \begin{split}
                    \lnorm x \rnorm_\infty = \max_{i = 1}^n |x_i| \leq \sum_{i = 1}^n |x_i| = \lnorm x_i \rnorm_1.
                \end{split}
                \end{equation*}

                \begin{equation*}
                \begin{split}
                    \lnorm x \rnorm_1 = \sum_{i = 1}^n |x_i| \leq \sum_{i = 1}^n \max_{i =1}^n |x_i| = n \max_{i = 1}^n |x_i| = n \lnorm x \rnorm_\infty.
                \end{split}
                \end{equation*}
            From this, and since equivalent norms form an equivalence relation, all norms on $F^n$ are equivalent.

            Suppose $p = 1$ and $q = \infty$. Let $(x_n)_n \in \ell_1$. Then clearly $\sup_{i=1}^\infty |x_i| \leq \sum_{i = 1}^\infty |x_i| < \infty$. Whence $\ell_1 \subseteq \ell_\infty$. Now suppose $p,q < \infty$ with $p \leq q$. Let $(x_n)_n \in \ell_p$. Then $\sum_{n = 1}^\infty |x_n|^p < \infty$. In particular, $(x_n)_n \rightarrow 0$, which implies that $(|x_n|)_n \rightarrow 0$. From this, there exists $K \in \bfN$ large such that for all $n \geq K$, we have $0 \leq x_n < 1$. It then follows that the tail $\sum_{n \geq K}|x_n|^p$ converges. Whence:
                \begin{equation*}
                \begin{split}
                    \sum_{n \geq K} |x_n|^q \leq \sum_{n \geq K}|x_n|^p < \infty.
                \end{split}
                \end{equation*}
            Thus $\sum_{n = 1}^\infty |x_n|^q < \infty$, establishing that $(x_n)_n \in \ell_q$.


        \end{proof}
%%%%%%%%%%%%%%%%%%%%%%%%%%%%%%%%%%%%%%%%%%%%%%%%%%%%%%%%%%%%%
    \begin{exercise}
        Let $M_{m,n}(\bfC)$ denote the linear space of all $m \times n$ matrices with coefficients from $\bfC$. For $A \in M_{m,n}(\bfC)$, set:
            \begin{equation*}
            \begin{split}
                \lnorm A \rnorm_\text{op} := \sup_{\xi \in B_{\ell_2^n}}\lnorm A \xi \rnorm_{\ell_2^m}.
            \end{split}
            \end{equation*}
        Show that $\lnorm \cdot \rnorm_\text{op}$ is a norm on $M_{m,n}(\bfC)$.
    \end{exercise}
        \begin{proof}
            Observe that:
                \begin{equation*}
                \begin{split}
                    \lnorm \alpha A \rnorm_\text{op}
                    & = \sup_{\xi \in B_{\ell_2^n}}\lnorm (\alpha A)\xi \rnorm_{\ell_2^n} \\
                    & = \sup_{\xi \in B_{\ell_2^n}} \lnorm \alpha (A \xi) \rnorm_{\ell_2^n} \\
                    & = |\alpha| \sup_{\xi \in B_{\ell_2^n}}\lnorm A \xi \rnorm_{\ell_2^n} \\
                    & = \alpha \lnorm A \rnorm_\text{op}.
                \end{split}
                \end{equation*}
            Thus $\lnorm \cdot \rnorm_\text{op}$ is homogenous. We also have:
                \begin{equation*}
                \begin{split}
                    \lnorm A + B \rnorm_\text{op}
                    & = \sup_{\xi \in B_{\ell_2^n}}\lnorm (A + B)\xi \rnorm_{\ell_2^n} \\
                    & = \sup_{\xi \in B_{\ell_2^n}}\lnorm A\xi + B\xi \rnorm_{\ell_2^n} \\
                    & \leq \sup_{\xi \in B_{\ell_2^n}} \left( \lnorm  A \xi\rnorm_{\ell_2^n} + \lnorm B\xi \rnorm_{\ell_2^n} \right) \\
                    & = \sup_{\xi \in B_{\ell_2^n}}\lnorm A\xi \rnorm_{\ell_2^n} + \sup_{\xi \in B_{\ell_2^n}}\lnorm B \xi\rnorm_{\ell_2^n} \\
                    & = \lnorm A \rnorm_\text{op} + \lnorm B \rnorm_\text{op}. 
                \end{split}
                \end{equation*}
            Hence $\lnorm \cdot \rnorm_\text{op}$ satisfies the triangle inequality. Now suppose $\lnorm A \rnorm_\text{op} = 0$. Then $\sup_{\xi \in B_{\ell_2^n}}\lnorm A\xi \rnorm_{\ell_2^n} = 0$. Since $\lnorm \cdot \rnorm_{\ell_2^n}$ is positive definite, the set $\{\lnorm A\xi \rnorm_{\ell_2^n} \mid \xi \in B_{\ell_2^n}\}$ must only contain positive real numbers. Whence if the supremum of this set equals $0$, it must be that $\lnorm A \xi \rnorm_{\ell_2^n} = 0$ for all $\xi \in B_{\ell_2^n}$. Again, by the positive-definiteness of $\lnorm \cdot \rnorm_{\ell_2^n}$, we have that $A\xi = 0$ for all $\xi \in B_{\ell_2^n}$. Whence $A = 0$. This gives that $\lnorm \cdot \rnorm_\text{op}$ is a norm.
        \end{proof}
%%%%%%%%%%%%%%%%%%%%%%%%%%%%%%%%%%%%%%%%%%%%%%%%%%%%%%%%%%%%%
    \newpage
    \addtocounter{exercise}{2}
    \begin{exercise}
        Let $p$ be a semi-norm on a vector space $V$.
        \begin{enumerate}[label = (\arabic*),itemsep=1pt,topsep=3pt]
            \item Show that $N_p = \{w \in V \mid p(w) = 0\}$ is a subspace of $V$.
            \item We form the quotient vector space $V/N_p$. Show that
                \begin{equation*}
                \begin{split}
                    \lnorm [v]_{N_p} \rnorm_p := p(v)
                \end{split}
                \end{equation*}
                defines a norm on $V/N_p$.
            \item If $(E,\lnorm \cdot \rnorm)$ is a normed space and $T:V \rightarrow E$ is a linear map, show that $p(v) := \lnorm T(v) \rnorm$ is a semi-norm on $V$. In this case what is $N_p$?
        \end{enumerate}
    \end{exercise}
        \begin{proof}
            (1) Let $w_1,w_2 \in N_p$ and $\alpha \in F$. Then:
                \begin{equation*}
                \begin{split}
                    p(w_1 + \alpha w_2) \leq p(w_1) + |\alpha|p(w_2) = 0.
                \end{split}
                \end{equation*}
                Since $w_1 + \alpha w_2 \in N_p$, $N_p$ is a subspace.

            (2) We must first show that $\lnorm \cdot \rnorm_p$ is well-defined. Let $[v_1]_{N_p} = [v_2]_{N_p}$. Then $v_1 = v_2 + w$ for some $w \in N_p$. Then:
                \begin{equation*}
                \begin{split}
                    \lnorm [v_1]_{N_p} \rnorm_p 
                    & = p(v_1) \\
                    & = p(v_1 + w) \\
                    & \leq p(v_2) + p(w) \\
                    & = p(v_2) \\
                    & = \lnorm [v_2]_{N_p} \rnorm_p
                \end{split}
                \end{equation*}
            So $\lnorm [v_1]_{N_p} \rnorm_p \leq \lnorm [v_2]_{N_p} \rnorm_p$. But note that we also have $v_2 = v_1 + w$ for some $w \in N_p$. This will give $\lnorm [v_2]_{N_p} \rnorm_p \leq \lnorm [v_1]_{N_p} \rnorm_p$, whence by antisymmetry $\lnorm [v_1]_{N_p} \rnorm_p = \lnorm [v_2]_{N_p} \rnorm_p$. Now let $\alpha \in F$. Observe that:
                \begin{equation*}
                \begin{split}
                    \lnorm \alpha [v]_{N_p} \rnorm_p 
                    & = \lnorm [\alpha v]_{N_p} \rnorm_p  \\
                    & = p(\alpha v) \\
                    & = |\alpha| p(v) \\
                    & = |\alpha|\lnorm [v]_{N_p} \rnorm_p.
                \end{split}
                \end{equation*}
            Thus $\lnorm \cdot \rnorm_p$ satisfies homogeneity. The triangle inequality is also satisfied because:
                \begin{equation*}
                \begin{split}
                    \lnorm [v]_{N_p} + [w]_{N_p} \rnorm_P
                    & = \lnorm [v + w]_{N_p} \rnorm _p\\
                    & = p(v+w) \\
                    & \leq p(v) + p(w) \\
                    & = \lnorm [v]_{N_p} \rnorm_p +\lnorm [w]_{N_p} \rnorm_p.
                \end{split}
                \end{equation*}
            Suppose $\lnorm [v]_{N_p} \rnorm_p = 0$. Then $p(v) = 0$. But this means $v \in N_p$, whence $[v]_{N_p} = [0]_{N_p}$. Hence $\lnorm \cdot \rnorm_p$ is a norm.

            (3) We have that:
                \begin{equation*}
                \begin{split}
                    p(\alpha v)
                    & = \lnorm T(\alpha v) \rnorm \\
                    & = \lnorm \alpha T(v) \rnorm \\
                    & = |\alpha| \lnorm T(v) \rnorm \\
                    & = |\alpha| p(v).
                \end{split}
                \end{equation*}
            Thus $p$ satisfies homogeneity. We also get:
                \begin{equation*}
                \begin{split}
                    p(v+w)
                    & = \lnorm T(v+w) \rnorm \\
                    & = \lnorm T(v) + T(w) \rnorm \\
                    & \leq \lnorm T(v) \rnorm + \lnorm T(w) \rnorm \\
                    & = p(v) + p(w).
                \end{split}
                \end{equation*}
            Thus $p$ is a semi-norm. Observe that:
                \begin{equation*}
                \begin{split}
                    N_p 
                    & = \{v \in V \mid p(v) = 0\} \\
                    & = \{v \in V \mid \lnorm T(v) \rnorm = 0\} \\
                    & = \{v \in V \mid T(v) = 0\} \\
                    & = \ker(T).
                \end{split}
                \end{equation*}
        \end{proof}
%%%%%%%%%%%%%%%%%%%%%%%%%%%%%%%%%%%%%%%%%%%%%%%%%%%%%%%%%%%%%
\end{document}