\documentclass[11pt,twoside,openany]{memoir}
%\usepackage{mlmodern}
%\usepackage{tgpagella} % text only
%\usepackage{mathpazo}  % math & text
\usepackage[T1]{fontenc}
\usepackage[hidelinks]{hyperref}
\usepackage{amsmath}
\usepackage{amsthm}
\usepackage{amssymb}
%\usepackage{mathtools}
%\renewcommand*{\mathbf}[1]{\varmathbb{#1}}
%\usepackage{newpxtext}
%\usepackage{eulerpx}
%\usepackage{eucal}
\usepackage{datetime}
    \newdateformat{specialdate}{\THEYEAR\ \monthname\ \THEDAY}
\usepackage[margin=1in]{geometry}
\usepackage{fancyhdr}
    \fancyhf{}
    \pagestyle{fancy}
    \cfoot{\scriptsize \thepage}
    \fancyhead[R]{\scriptsize \rightmark}
    \fancyhead[L]{\scriptsize \leftmark}
    \renewcommand{\headrulewidth}{0pt}
    \renewcommand{\footrulewidth}{0pt} % if you also want to remove the footer rule
\usepackage{thmtools}
    \declaretheoremstyle[
        spaceabove=10pt,
        spacebelow=10pt,
        headfont=\normalfont\bfseries,
        notefont=\mdseries, notebraces={(}{)},
        bodyfont=\normalfont,
        postheadspace=0.5em
        %qed=\qedsymbol
        ]{defs}

    \declaretheoremstyle[ 
        spaceabove=10pt, % space above the theorem
        spacebelow=10pt,
        headfont=\normalfont\bfseries,
        bodyfont=\normalfont\itshape,
        postheadspace=0.5em
        ]{thmstyle}
    
    \declaretheorem[
        style=thmstyle,
        numberwithin=section
    ]{theorem}

    \declaretheorem[
        style=thmstyle,
        sibling=theorem,
    ]{proposition}

    \declaretheorem[
        style=thmstyle,
        sibling=theorem,
    ]{lemma}

    \declaretheorem[
        style=thmstyle,
        sibling=theorem,
    ]{corollary}

    \declaretheorem[
        numberwithin=section,
        style=defs,
    ]{example}

    \declaretheorem[
        numberwithin=section,
        style=defs,
    ]{definition}

    \declaretheorem[
        style=defs,
        numbered=unless unique,
    ]{exercise}

    \declaretheorem[
        numbered=unless unique,
        shaded={rulecolor=black,
    rulewidth=1pt, bgcolor={rgb}{1,1,1}}
    ]{axiom}

    \declaretheorem[numberwithin=section,style=defs]{note}
    \declaretheorem[numbered=no,style=defs]{question}
    \declaretheorem[numbered=no,style=defs]{recall}
    \declaretheorem[numbered=no,style=remark]{answer}
    \declaretheorem[numbered=no,style=remark]{solution}

    \declaretheorem[numbered=no,style=defs]{remark}
\usepackage{enumitem}
\usepackage{titlesec}
    \titleformat{\chapter}[display]
    {\bfseries\LARGE\raggedright}
    {Chapter {\thechapter}}
    {1ex minus .1ex}
    {\Huge}
    \titlespacing{\chapter}
    {3pc}{*3}{40pt}[3pc]

    \titleformat{\section}[block]
    {\normalfont\bfseries\Large}
    {\S\ \thesection.}{.5em}{}[]
    \titlespacing{\section}
    {0pt}{3ex plus .1ex minus .2ex}{3ex plus .1ex minus .2ex}
\usepackage[utf8x]{inputenc}
\usepackage{tikz}
\usepackage{tikz-cd}
\usepackage{wasysym}
\renewcommand{\int}{\varint}

\linespread{1}
%to make the correct symbol for Sha
%\newcommand\cyr{%
%\renewcommand\rmdefault{wncyr}%
%\renewcommand\sfdefault{wncyss}%
%\renewcommand\encodingdefault{OT2}%
%\normalfont \selectfont} \DeclareTextFontCommand{\textcyr}{\cyr}


\DeclareMathOperator{\ab}{ab}
\newcommand{\absgal}{\G_{\bbQ}}
\DeclareMathOperator{\ad}{ad}
\DeclareMathOperator{\adj}{adj}
\DeclareMathOperator{\alg}{alg}
\DeclareMathOperator{\Alt}{Alt}
\DeclareMathOperator{\Ann}{Ann}
\DeclareMathOperator{\arith}{arith}
\DeclareMathOperator{\Aut}{Aut}
\DeclareMathOperator{\Be}{B}
\DeclareMathOperator{\Bd}{Bd}
\DeclareMathOperator{\card}{card}
\DeclareMathOperator{\Char}{char}
\DeclareMathOperator{\csp}{csp}
\DeclareMathOperator{\codim}{codim}
\DeclareMathOperator{\coker}{coker}
\DeclareMathOperator{\coh}{H}
\DeclareMathOperator{\compl}{compl}
\DeclareMathOperator{\conj}{conj}
\DeclareMathOperator{\cont}{cont}
\DeclareMathOperator{\crys}{crys}
\DeclareMathOperator{\Crys}{Crys}
\DeclareMathOperator{\cusp}{cusp}
\DeclareMathOperator{\diag}{diag}
\DeclareMathOperator{\diam}{diam}
\DeclareMathOperator{\Dom}{Dom}
\DeclareMathOperator{\disc}{disc}
\DeclareMathOperator{\dist}{dist}
\DeclareMathOperator{\dR}{dR}
\DeclareMathOperator{\Eis}{Eis}
\DeclareMathOperator{\End}{End}
\DeclareMathOperator{\ev}{ev}
\DeclareMathOperator{\eval}{eval}
\DeclareMathOperator{\Eq}{Eq}
\DeclareMathOperator{\Ext}{Ext}
\DeclareMathOperator{\Fil}{Fil}
\DeclareMathOperator{\Fitt}{Fitt}
\DeclareMathOperator{\Frob}{Frob}
\DeclareMathOperator{\G}{G}
\DeclareMathOperator{\Gal}{Gal}
\DeclareMathOperator{\GL}{GL}
\DeclareMathOperator{\Gr}{Gr}
\DeclareMathOperator{\Graph}{Graph}
\DeclareMathOperator{\GSp}{GSp}
\DeclareMathOperator{\GUn}{GU}
\DeclareMathOperator{\Hom}{Hom}
\DeclareMathOperator{\id}{id}
\DeclareMathOperator{\Id}{Id}
\DeclareMathOperator{\Ik}{Ik}
\DeclareMathOperator{\IM}{Im}
\DeclareMathOperator{\Image}{im}
\DeclareMathOperator{\Ind}{Ind}
\DeclareMathOperator{\Inf}{inf}
\DeclareMathOperator{\Isom}{Isom}
\DeclareMathOperator{\J}{J}
\DeclareMathOperator{\Jac}{Jac}
\DeclareMathOperator{\lcm}{lcm}
\DeclareMathOperator{\length}{length}
\DeclareMathOperator*{\limit}{limit}
\DeclareMathOperator{\Log}{Log}
\DeclareMathOperator{\M}{M}
\DeclareMathOperator{\Mat}{Mat}
\DeclareMathOperator{\N}{N}
\DeclareMathOperator{\Nm}{Nm}
\DeclareMathOperator{\NIk}{N-Ik}
\DeclareMathOperator{\NSK}{N-SK}
\DeclareMathOperator{\new}{new}
\DeclareMathOperator{\obj}{obj}
\DeclareMathOperator{\old}{old}
\DeclareMathOperator{\ord}{ord}
\DeclareMathOperator{\Or}{O}
\DeclareMathOperator{\op}{op}
\DeclareMathOperator{\PGL}{PGL}
\DeclareMathOperator{\PGSp}{PGSp}
\DeclareMathOperator{\rank}{rank}
\DeclareMathOperator{\Ran}{Ran}
\DeclareMathOperator{\Rel}{Rel}
\DeclareMathOperator{\Real}{Re}
\DeclareMathOperator{\RES}{res}
\DeclareMathOperator{\Res}{Res}
%\DeclareMathOperator{\Sha}{\textcyr{Sh}}
\DeclareMathOperator{\Sel}{Sel}
\DeclareMathOperator{\semi}{ss}
\DeclareMathOperator{\sgn}{sign}
\DeclareMathOperator{\SK}{SK}
\DeclareMathOperator{\SL}{SL}
\DeclareMathOperator{\SO}{SO}
\DeclareMathOperator{\Sp}{Sp}
\DeclareMathOperator{\Span}{span}
\DeclareMathOperator{\Spec}{Spec}
\DeclareMathOperator{\spin}{spin}
\DeclareMathOperator{\st}{st}
\DeclareMathOperator{\St}{St}
\DeclareMathOperator{\SUn}{SU}
\DeclareMathOperator{\supp}{supp}
\DeclareMathOperator{\Sup}{sup}
\DeclareMathOperator{\Sym}{Sym}
\DeclareMathOperator{\Tam}{Tam}
\DeclareMathOperator{\tors}{tors}
\DeclareMathOperator{\tr}{tr}
\DeclareMathOperator{\Tr}{Tr}
\DeclareMathOperator{\un}{un}
\DeclareMathOperator{\Un}{U}
\DeclareMathOperator{\val}{val}
\DeclareMathOperator{\vol}{vol}

\DeclareMathOperator{\Sets}{S \mkern1.04mu e \mkern1.04mu t \mkern1.04mu s}
    \newcommand{\cSets}{\scalebox{1.02}{\contour{black}{$\Sets$}}}
    
\DeclareMathOperator{\Groups}{G \mkern1.04mu r \mkern1.04mu o \mkern1.04mu u \mkern1.04mu p \mkern1.04mu s}
    \newcommand{\cGroups}{\scalebox{1.02}{\contour{black}{$\Groups$}}}

\DeclareMathOperator{\TTop}{T \mkern1.04mu o \mkern1.04mu p}
    \newcommand{\cTop}{\scalebox{1.02}{\contour{black}{$\TTop$}}}

\DeclareMathOperator{\Htp}{H \mkern1.04mu t \mkern1.04mu p}
    \newcommand{\cHtp}{\scalebox{1.02}{\contour{black}{$\Htp$}}}

\DeclareMathOperator{\Mod}{M \mkern1.04mu o \mkern1.04mu d}
    \newcommand{\cMod}{\scalebox{1.02}{\contour{black}{$\Mod$}}}

\DeclareMathOperator{\Ab}{A \mkern1.04mu b}
    \newcommand{\cAb}{\scalebox{1.02}{\contour{black}{$\Ab$}}}

\DeclareMathOperator{\Rings}{R \mkern1.04mu i \mkern1.04mu n \mkern1.04mu g \mkern1.04mu s}
    \newcommand{\cRings}{\scalebox{1.02}{\contour{black}{$\Rings$}}}

\DeclareMathOperator{\ComRings}{C \mkern1.04mu o \mkern1.04mu m \mkern1.04mu R \mkern1.04mu i \mkern1.04mu n \mkern1.04mu g \mkern1.04mu s}
    \newcommand{\cComRings}{\scalebox{1.05}{\contour{black}{$\ComRings$}}}

\DeclareMathOperator{\hHom}{H \mkern1.04mu o \mkern1.04mu m}
    \newcommand{\cHom}{\scalebox{1.02}{\contour{black}{$\hHom$}}}

         %  \item $\cGroups$
          %  \item $\cTop$
          %  \item $\cHtp$
          %  \item $\cMod$




\renewcommand{\k}{\kappa}
\newcommand{\Ff}{F_{f}}
%\newcommand{\ts}{\,^{t}\!}


%Mathcal

\newcommand{\cA}{\mathcal{A}}
\newcommand{\cB}{\mathcal{B}}
\newcommand{\cC}{\mathcal{C}}
\newcommand{\cD}{\mathcal{D}}
\newcommand{\cE}{\mathcal{E}}
\newcommand{\cF}{\mathcal{F}}
\newcommand{\cG}{\mathcal{G}}
\newcommand{\cH}{\mathcal{H}}
\newcommand{\cI}{\mathcal{I}}
\newcommand{\cJ}{\mathcal{J}}
\newcommand{\cK}{\mathcal{K}}
\newcommand{\cL}{\mathcal{L}}
\newcommand{\cM}{\mathcal{M}}
\newcommand{\cN}{\mathcal{N}}
\newcommand{\cO}{\mathcal{O}}
\newcommand{\cP}{\mathcal{P}}
\newcommand{\cQ}{\mathcal{Q}}
\newcommand{\cR}{\mathcal{R}}
\newcommand{\cS}{\mathcal{S}}
\newcommand{\cT}{\mathcal{T}}
\newcommand{\cU}{\mathcal{U}}
\newcommand{\cV}{\mathcal{V}}
\newcommand{\cW}{\mathcal{W}}
\newcommand{\cX}{\mathcal{X}}
\newcommand{\cY}{\mathcal{Y}}
\newcommand{\cZ}{\mathcal{Z}}


%mathfrak (missing \fi)

\newcommand{\fa}{\mathfrak{a}}
\newcommand{\fA}{\mathfrak{A}}
\newcommand{\fb}{\mathfrak{b}}
\newcommand{\fB}{\mathfrak{B}}
\newcommand{\fc}{\mathfrak{c}}
\newcommand{\fC}{\mathfrak{C}}
\newcommand{\fd}{\mathfrak{d}}
\newcommand{\fD}{\mathfrak{D}}
\newcommand{\fe}{\mathfrak{e}}
\newcommand{\fE}{\mathfrak{E}}
\newcommand{\ff}{\mathfrak{f}}
\newcommand{\fF}{\mathfrak{F}}
\newcommand{\fg}{\mathfrak{g}}
\newcommand{\fG}{\mathfrak{G}}
\newcommand{\fh}{\mathfrak{h}}
\newcommand{\fH}{\mathfrak{H}}
\newcommand{\fI}{\mathfrak{I}}
\newcommand{\fj}{\mathfrak{j}}
\newcommand{\fJ}{\mathfrak{J}}
\newcommand{\fk}{\mathfrak{k}}
\newcommand{\fK}{\mathfrak{K}}
\newcommand{\fl}{\mathfrak{l}}
\newcommand{\fL}{\mathfrak{L}}
\newcommand{\fm}{\mathfrak{m}}
\newcommand{\fM}{\mathfrak{M}}
\newcommand{\fn}{\mathfrak{n}}
\newcommand{\fN}{\mathfrak{N}}
\newcommand{\fo}{\mathfrak{o}}
\newcommand{\fO}{\mathfrak{O}}
\newcommand{\fp}{\mathfrak{p}}
\newcommand{\fP}{\mathfrak{P}}
\newcommand{\fq}{\mathfrak{q}}
\newcommand{\fQ}{\mathfrak{Q}}
\newcommand{\fr}{\mathfrak{r}}
\newcommand{\fR}{\mathfrak{R}}
\newcommand{\fs}{\mathfrak{s}}
\newcommand{\fS}{\mathfrak{S}}
\newcommand{\ft}{\mathfrak{t}}
\newcommand{\fT}{\mathfrak{T}}
\newcommand{\fu}{\mathfrak{u}}
\newcommand{\fU}{\mathfrak{U}}
\newcommand{\fv}{\mathfrak{v}}
\newcommand{\fV}{\mathfrak{V}}
\newcommand{\fw}{\mathfrak{w}}
\newcommand{\fW}{\mathfrak{W}}
\newcommand{\fx}{\mathfrak{x}}
\newcommand{\fX}{\mathfrak{X}}
\newcommand{\fy}{\mathfrak{y}}
\newcommand{\fY}{\mathfrak{Y}}
\newcommand{\fz}{\mathfrak{z}}
\newcommand{\fZ}{\mathfrak{Z}}


%mathbf
\newcommand{\bfA}{\mathbf{A}}
\newcommand{\bfB}{\mathbf{B}}
\newcommand{\bfC}{\mathbf{C}}
\newcommand{\bfD}{\mathbf{D}}
\newcommand{\bfE}{\mathbf{E}}
\newcommand{\bfF}{\mathbf{F}}
\newcommand{\bfG}{\mathbf{G}}
\newcommand{\bfH}{\mathbf{H}}
\newcommand{\bfI}{\mathbf{I}}
\newcommand{\bfJ}{\mathbf{J}}
\newcommand{\bfK}{\mathbf{K}}
\newcommand{\bfL}{\mathbf{L}}
\newcommand{\bfM}{\mathbf{M}}
\newcommand{\bfN}{\mathbf{N}}
\newcommand{\bfO}{\mathbf{O}}
\newcommand{\bfP}{\mathbf{P}}
\newcommand{\bfQ}{\mathbf{Q}}
\newcommand{\bfR}{\mathbf{R}}
\newcommand{\bfS}{\mathbf{S}}
\newcommand{\bfT}{\mathbf{T}}
\newcommand{\bfU}{\mathbf{U}}
\newcommand{\bfV}{\mathbf{V}}
\newcommand{\bfW}{\mathbf{W}}
\newcommand{\bfX}{\mathbf{X}}
\newcommand{\bfY}{\mathbf{Y}}
\newcommand{\bfZ}{\mathbf{Z}}

\newcommand{\bfa}{\mathbf{a}}
\newcommand{\bfb}{\mathbf{b}}
\newcommand{\bfc}{\mathbf{c}}
\newcommand{\bfd}{\mathbf{d}}
\newcommand{\bfe}{\mathbf{e}}
\newcommand{\bff}{\mathbf{f}}
\newcommand{\bfg}{\mathbf{g}}
\newcommand{\bfh}{\mathbf{h}}
\newcommand{\bfi}{\mathbf{i}}
\newcommand{\bfj}{\mathbf{j}}
\newcommand{\bfk}{\mathbf{k}}
\newcommand{\bfl}{\mathbf{l}}
\newcommand{\bfm}{\mathbf{m}}
\newcommand{\bfn}{\mathbf{n}}
\newcommand{\bfo}{\mathbf{o}}
\newcommand{\bfp}{\mathbf{p}}
\newcommand{\bfq}{\mathbf{q}}
\newcommand{\bfr}{\mathbf{r}}
\newcommand{\bfs}{\mathbf{s}}
\newcommand{\bft}{\mathbf{t}}
\newcommand{\bfu}{\mathbf{u}}
\newcommand{\bfv}{\mathbf{v}}
\newcommand{\bfw}{\mathbf{w}}
\newcommand{\bfx}{\mathbf{x}}
\newcommand{\bfy}{\mathbf{y}}
\newcommand{\bfz}{\mathbf{z}}

%blackboard bold

\newcommand{\bbA}{\mathbb{A}}
\newcommand{\bbB}{\mathbb{B}}
\newcommand{\bbC}{\mathbb{C}}
\newcommand{\bbD}{\mathbb{D}}
\newcommand{\bbE}{\mathbb{E}}
\newcommand{\bbF}{\mathbb{F}}
\newcommand{\bbG}{\mathbb{G}}
\newcommand{\bbH}{\mathbb{H}}
\newcommand{\bbI}{\mathbb{I}}
\newcommand{\bbJ}{\mathbb{J}}
\newcommand{\bbK}{\mathbb{K}}
\newcommand{\bbL}{\mathbb{L}}
\newcommand{\bbM}{\mathbb{M}}
\newcommand{\bbN}{\mathbb{N}}
\newcommand{\bbO}{\mathbb{O}}
\newcommand{\bbP}{\mathbb{P}}
\newcommand{\bbQ}{\mathbb{Q}}
\newcommand{\bbR}{\mathbb{R}}
\newcommand{\bbS}{\mathbb{S}}
\newcommand{\bbT}{\mathbb{T}}
\newcommand{\bbU}{\mathbb{U}}
\newcommand{\bbV}{\mathbb{V}}
\newcommand{\bbW}{\mathbb{W}}
\newcommand{\bbX}{\mathbb{X}}
\newcommand{\bbY}{\mathbb{Y}}
\newcommand{\bbZ}{\mathbb{Z}}
\newcommand{\jota}{\jmath}

\newcommand{\bmat}{\left( \begin{matrix}}
\newcommand{\emat}{\end{matrix} \right)}

\newcommand{\pmat}{\left( \begin{smallmatrix}}
\newcommand{\epmat}{\end{smallmatrix} \right)}

\newcommand{\lat}{\mathscr{L}}
\newcommand{\mat}[4]{\begin{pmatrix}{#1}&{#2}\\{#3}&{#4}\end{pmatrix}}
\newcommand{\ov}[1]{\overline{#1}}
\newcommand{\res}[1]{\underset{#1}{\RES}\,}
\newcommand{\up}{\upsilon}

\newcommand{\tac}{\textasteriskcentered}

%mahesh macros
\newcommand{\tm}{\textrm}

%Comments
\newcommand{\com}[1]{\vspace{5 mm}\par \noindent
\marginpar{\textsc{Comment}} \framebox{\begin{minipage}[c]{0.95
\textwidth} \tt #1 \end{minipage}}\vspace{5 mm}\par}

\newcommand{\Bmu}{\mbox{$\raisebox{-0.59ex}
  {$l$}\hspace{-0.18em}\mu\hspace{-0.88em}\raisebox{-0.98ex}{\scalebox{2}
  {$\color{white}.$}}\hspace{-0.416em}\raisebox{+0.88ex}
  {$\color{white}.$}\hspace{0.46em}$}{}}  %need graphicx and xcolor. this produces blackboard bold mu 

\newcommand{\hooktwoheadrightarrow}{%
  \hookrightarrow\mathrel{\mspace{-15mu}}\rightarrow
}

\makeatletter
\newcommand{\xhooktwoheadrightarrow}[2][]{%
  \lhook\joinrel
  \ext@arrow 0359\rightarrowfill@ {#1}{#2}%
  \mathrel{\mspace{-15mu}}\rightarrow
}
\makeatother

\renewcommand{\geq}{\geqslant}
\renewcommand{\leq}{\leqslant}
\newcommand{\midd}{\hspace{4pt}\middle|\hspace{4pt}}
    
    \newcommand{\bone}{\mathbf{1}}
    \newcommand{\sign}{\mathrm{sign}}
    \newcommand{\eps}{\varepsilon}
    \newcommand{\textui}[1]{\uline{\textit{#1}}}
    
    %\newcommand{\ov}{\overline}
    %\newcommand{\un}{\underline}
    \newcommand{\fin}{\mathrm{fin}}
    
    \newcommand{\chnum}{\titleformat
    {\chapter} % command
    [display] % shape
    {\centering} % format
    {\Huge \color{black} \shadowbox{\thechapter}} % label
    {-0.5em} % sep (space between the number and title)
    {\LARGE \color{black} \underline} % before-code
    }
    
    \newcommand{\chunnum}{\titleformat
    {\chapter} % command
    [display] % shape
    {} % format
    {} % label
    {0em} % sep
    { \begin{flushright} \begin{tabular}{r}  \Huge \color{black}
    } % before-code
    [
    \end{tabular} \end{flushright} \normalsize
    ] % after-code
    }

\newcommand{\nl}{\newline \mbox{}}

\newcommand{\h}[1]{\hspace{#1pt}}

\newcommand{\littletaller}{\mathchoice{\vphantom{\big|}}{}{}{}}
\newcommand\restr[2]{{% we make the whole thing an ordinary symbol
  \left.\kern-\nulldelimiterspace % automatically resize the bar with \right
  #1 % the function
  \littletaller % pretend it's a little taller at normal size
  \right|_{#2} % this is the delimiter
  }}

\newcommand{\mtext}[1]{\hspace{6pt}\text{#1}\hspace{6pt}}

\newcommand{\lnorm}{\left\lVert}
\newcommand{\rnorm}{\right\rVert}

\newcommand{\ds}{\displaystyle}
\newcommand{\ts}{\textstyle}

%This adds a "front cover" page.
%{\thispagestyle{empty}
%\vspace*{\fill}
%\begin{tabular}{l}
%\begin{tabular}{l}
%\includegraphics[scale=0.24]{oxy-logo.png}
%\end{tabular} \\
%\begin{tabular}{l}
%\Large \color{black} Module Theory, Linear Algebra, and Homological Algebra \\ \Large \color{black} Gianluca Crescenzo
%\end{tabular}
%\end{tabular}
%\newpage

\newcommand{\sfrac}[2]{{}^{#1}\mskip -5mu/\mskip -3mu_{#2}}


\makeatletter
\newcommand*{\da@rightarrow}{\mathchar"0\hexnumber@\symAMSa 4B }
\newcommand*{\da@leftarrow}{\mathchar"0\hexnumber@\symAMSa 4C }
\newcommand*{\xdashrightarrow}[2][]{%
  \mathrel{%
    \mathpalette{\da@xarrow{#1}{#2}{}\da@rightarrow{\,}{}}{}%
  }%
}
\newcommand{\xdashleftarrow}[2][]{%
  \mathrel{%
    \mathpalette{\da@xarrow{#1}{#2}\da@leftarrow{}{}{\,}}{}%
  }%
}
\newcommand*{\da@xarrow}[7]{%
  % #1: below
  % #2: above
  % #3: arrow left
  % #4: arrow right
  % #5: space left 
  % #6: space right
  % #7: math style 
  \sbox0{$\ifx#7\scriptstyle\scriptscriptstyle\else\scriptstyle\fi#5#1#6\m@th$}%
  \sbox2{$\ifx#7\scriptstyle\scriptscriptstyle\else\scriptstyle\fi#5#2#6\m@th$}%
  \sbox4{$#7\dabar@\m@th$}%
  \dimen@=\wd0 %
  \ifdim\wd2 >\dimen@
    \dimen@=\wd2 %   
  \fi
  \count@=2 %
  \def\da@bars{\dabar@\dabar@}%
  \@whiledim\count@\wd4<\dimen@\do{%
    \advance\count@\@ne
    \expandafter\def\expandafter\da@bars\expandafter{%
      \da@bars
      \dabar@ 
    }%
  }%  
  \mathrel{#3}%
  \mathrel{%   
    \mathop{\da@bars}\limits
    \ifx\\#1\\%
    \else
      _{\copy0}%
    \fi
    \ifx\\#2\\%
    \else
      ^{\copy2}%
    \fi
  }%   
  \mathrel{#4}%
}
\makeatother


\begin{document}
\begin{center}
{\large Math 397 \\[0.1in]Homework 5 \\[0.1in]}
{Name:} {\underline{Gianluca Crescenzo\hspace*{2in}}}\\[0.15in]
\end{center}
\vspace{4pt}
%%%%%%%%%%%%%%%%%%%%%%%%%%%%%%%%%%%%%%%%%%%%%%%%%%%%%%%%%%%%%
    \begin{exercise}
        Show that $C_0(\bfR)$ is a Banach space.
    \end{exercise}
        \begin{proof}
            Note that $C_b(\bfR) \supseteq C_0(\bfR)$ is a Banach space. Let $(f_n)_n$ be a sequence in $C_0(\bfR)$ converging to $f \in C_b(\bfR)$. Let $\epsilon > 0$ and find $N$ large so that $\lnorm f - f_N \rnorm < \frac{\epsilon}{2}$. Since $f_N \in C_0(\bfR)$, find $M$ large so $|x| > M$ implies $|f_N(x)| < \frac{\epsilon}{2}$. For $|x| > M$ we have:
                \begin{equation*}
                \begin{split}
                    |f(x)|
                    & = |f(x) - f_N(x) + f_N(x)| \\
                    & \leq |f(x) - f_N(X)| + |f_N(x)| \\
                    & \leq \lnorm f - f_N \rnorm + |f_N(x)| \\
                    & < \frac{\epsilon}{2} + \frac{\epsilon}{2} \\
                    & = \epsilon.
                \end{split}
                \end{equation*}
            Thus $\limit_{|x| \rightarrow \infty} = 0$. Since $f \in C_0(\bfR)$, we have that $C_0(\bfR) \subseteq C_b(\bfR)$ is closed; i.e., it is complete.
        \end{proof}
%%%%%%%%%%%%%%%%%%%%%%%%%%%%%%%%%%%%%%%%%%%%%%%%%%%%%%%%%%%%%
    \begin{exercise}
        Show that $\ell_2$ is a Hilbert space.
    \end{exercise}
        \begin{proof}
            Let $(f_n)_n$ be $\lnorm \cdot \rnorm_{\ell_2}$-Cauchy. Let $\epsilon > 0$. Find $N_1$ large so $n,m \geq N_1$ implies $\lnorm f_n - f_m \rnorm_{\ell_2} < \epsilon$. Then:
                \begin{equation*}
                \begin{split}
                    |f_n(k) - f_m(k)|
                    & \leq \lnorm f_n - f_m \rnorm_{\ell_2} \\
                    & < \epsilon.
                \end{split}
                \end{equation*}
            So $(f_n(k))_n$ is Cauchy in $\bfC$. Since $\bfC$ is complete, define $f(k):= \limit_{n \rightarrow \infty} f_n(k)$. Claim: $f \in \ell_2$ and $\limit_{n \rightarrow \infty}f_n = f$. 
            
            We will first show that $f \in \ell_2$. Note that since $(f_n)_n$ is $\lnorm \cdot \rnorm_{\ell_2}$-Cauchy, it is bounded. For $K > 1$, observe that:
                \begin{equation*}
                \begin{split}
                    \sum_{j = 1}^K |f(j)|^2 
                    & = \sum_{j = 1}^K \bigl|\limit_{n \rightarrow \infty}f_n(j)\bigr|^2 \\
                    & = \limit_{n \rightarrow \infty} \sum_{j = 1}^K |f_n(j)|^2 \\
                    & \leq \sup_{n \geq 1} \lnorm f_n \rnorm_{\ell_2}^2 \\
                    & < \infty.
                \end{split}
                \end{equation*}
            Since $\left( \sum_{j = 1}^K |f(j)|^2 \right)_K$ is increasing and bounded above, the Monotone Convergence Theorem says its limit exists. This means:
                \begin{equation*}
                \begin{split}
                    \limit_{K \rightarrow \infty}\sum_{j = 1}^K |f(j)|^2 
                    & = \sum_{j = 1}^\infty |f(j)|^2 \\
                    & = \lnorm f \rnorm^2_{\ell_2} \\
                    & < \infty.
                \end{split}
                \end{equation*}
            Thus $f \in \ell_2$.

            We will now show that $f$ is the limit of our Cauchy sequence. With the same epsilon as before, find $N_2$ large so that $n,m \geq N_2$ implies $\lnorm f_n - f_m \rnorm_2 < \frac{\epsilon^2}{2}$. Then:
                \begin{equation*}
                \begin{split}
                    \sum_{j = 1}^K | f_n(j) - f_m(j)|
                    & \leq \lnorm f_n - f_m \rnorm_{\ell_2}^2 \\
                    & < \frac{\epsilon^2}{4}.
                \end{split}
                \end{equation*}
            Taking the limit as $m \rightarrow \infty$ and considering all $n \geq N_2$ gives $\sum_{j = 1}^K |f_n(j) - f(j)| \leq \frac{\epsilon^2}{4}$. Taking the limit as $K \rightarrow \infty$ gives:
                \begin{equation*}
                \begin{split}
                    \sum_{j = 1}^\infty |f_n(j) - f(j)|
                    & = \lnorm f_n - f \rnorm_{\ell_2}^2 \\
                    & \leq \frac{\epsilon^2}{4} \\
                    & < \epsilon^2.
                \end{split}
                \end{equation*}
            Square-rooting both sides establishes $(f_n)_n \rightarrow f$. Thus $\ell_2$ is a Banach space.

            Define $\langle \cdot,\cdot \rangle: \ell_2 \times \ell_2 \rightarrow \bfC$ by $\langle f,g \rangle = \sum_{k = 1}^\infty f(j)\overline{g(j)}$. We must first verify that this series exists. Note that:
                \begin{equation*}
                \begin{split}
                    \sum_{j = 1}^K |f(j) \overline{g(j)}| 
                    & = \sum_{j = 1}^K |f(j)||g(j)| \\
                    & \leq \left( \sum_{j = 1}^K |f(j)|^2 \right)^\frac{1}{2}\left( \sum_{j = 1}^K |g(j)|^2 \right)^\frac{1}{2} \\
                    & \leq \lnorm f \rnorm_{\ell_2} \lnorm g \rnorm_{\ell_2} \\
                    & < \infty.
                \end{split}
                \end{equation*}
            Since $\left( \sum_{j = 1}^K |f(j) \overline{g(j)}| \right)_K$ is increasing and bounded above, its limit exists by the Monotone Convergence Theorem. So $\sum_{j = 1}^\infty |f(j) \overline{g(j)}|$ converges. In particular, since $(\bfC, |\cdot|)$ is a Banach space, $\sum_{j = 1}^\infty f(j) \overline{g(j)}$ will converge.

            Let $f,g_1,g_2 \in \ell_2$ and $\alpha \in \bfC$. Observe that:
                \begin{equation*}
                \begin{split}
                    \langle f, g_1 + \alpha g_2 \rangle
                    & = \sum_{j=1}^\infty f(j) \overline{(g_1 + \alpha g_2)(j)} \\
                    & = \sum_{j = 1}^\infty f(j)\overline{g_1(j)} + \overline{\alpha}\sum_{j = 1}^\infty f(j)\overline{g_2(j)} \\
                    & = \langle f,g_1 \rangle + \overline{a}\langle f,g_2 \rangle.
                \end{split}
                \end{equation*}
            Now let $f_1,f_2,g \in \ell_2$ and $\alpha \in \bfC$. Observe that:
                \begin{equation*}
                \begin{split}
                    \langle f_1 + \alpha f_2,g \rangle
                    & = \sum_{j = 1}^\infty (f_1 + \alpha f_2)(j)\overline{g(j)} \\
                    & = \sum_{j = 1}^\infty f_1(j)\overline{g(j)} + \alpha \sum_{j = 1}^\infty f_2(j)\overline{g(j)} \\
                    & = \langle f_1,g \rangle + \alpha \langle f_2,g \rangle.
                \end{split}
                \end{equation*}
            Thus $\langle \cdot,\cdot \rangle$ is a sesquilinear form. Moreover, we can see:
                \begin{equation*}
                \begin{split}
                    \langle f,g \rangle
                    & = \sum_{j = 1}^\infty f(j)\overline{g(j)} \\
                    & = \overline{\sum_{j = 1}^\infty g(j)\overline{f(j)}} \\
                    & = \overline{\langle g,f \rangle}.
                \end{split}
                \end{equation*}
            Whence $\langle \cdot,\cdot \rangle$ is Hermitian. Finally, if $f \neq 0$, we have:
                \begin{equation*}
                \begin{split}
                    \langle f,f \rangle 
                    & = \sum_{j = 1}^\infty f(j)\overline{f(j)} \\
                    & = \sum_{j = 1}^\infty |f(j)|^2 \\
                    & > 0.
                \end{split}
                \end{equation*}
            Thus $\langle \cdot,\cdot \rangle$ is positive definite, establishing it as an inner-product. Thus $\ell_2$ is a Hilbert space.
        \end{proof}
%%%%%%%%%%%%%%%%%%%%%%%%%%%%%%%%%%%%%%%%%%%%%%%%%%%%%%%%%%%%%
    \newpage
    \begin{exercise}
        Suppose $(X,d)$ is a complete metric space and $(x_n)_n$ is a \textit{contractive} sequence in $X$, that is, there exists a $\theta \in (0,1)$ with $d(x_{n+1},x_n) \leq \theta d(x_n,x_{n-1})$. Show that $(x_n)_n$ is convergent.
    \end{exercise}
        \begin{proof}
            Note that $d(x_{n+1},x_n) \leq \theta^{n-1}d(x_2,x_1)$. Without loss of generality, for $n>m$ we have:
                \begin{equation*}
                \begin{split}
                    d(x_n,x_m)
                    & \leq d(x_{n},x_{n-1}) + d(x_{n-1},x_m) \\
                    & \vdots \\
                    & \leq d(x_n,x_{n-1}) + d(x_{n-1},x_{n-2}) + ... + d(x_{m+1},x_m) \\
                    & \leq \theta^{n-2}d(x_2,x_1) + \theta^{n-3}d(x_2,x_1) + ... + \theta^{m-1}d(x_2,x_1) \\
                    & = (\theta^{n-m-1} + \theta^{n-m-2} + ... + 1)\theta^{m-1}d(x_2,x_1) \\
                    & = \left( \frac{1 - \theta^{n-m}}{1 - \theta} \right)\theta^{m-1}d(x_2,x_1) \\
                    & \leq \left( \frac{1}{1-\theta} \right)\theta^{m-1}d(x_2,x_1) \\
                    & \rightarrow 0.
                \end{split}
                \end{equation*}
            Thus $(x_n)_n$ is Cauchy. Since $(X,d)$ is complete, $(x_n)_n$ converges.
        \end{proof}
%%%%%%%%%%%%%%%%%%%%%%%%%%%%%%%%%%%%%%%%%%%%%%%%%%%%%%%%%%%%%
    \begin{exercise}
        Let $(X,d)$ be a complete metric space and suppose $f:X \rightarrow X$ is a contractive map; i.e., for all $x,y \in X$ there is a $\theta \in (0,1)$ with:
            \begin{equation*}
            \begin{split}
                d(f(x),f(y)) \leq \theta d(x,y).
            \end{split}
            \end{equation*}
        Prove that $f$ has a unique fixed point.
    \end{exercise}
        \begin{proof}
            Define $(x_n)_n$ in $X$ by $x_n = f(x_{n-1})$. We can see:
                \begin{equation*}
                \begin{split}
                    d(x_{n+1},x_n) 
                    & = d(f(x_n),f(x_{n-1})) \\
                    & \leq \theta d(x_n,x_{n-1}).
                \end{split}
                \end{equation*}
            Thus $(x_n)_n$ is contractive. By Exercise 3, $(x_n)_n$ is convergent. Define $x := \limit_{n \rightarrow \infty }x_n$. Observe that:
                \begin{equation*}
                \begin{split}
                    x 
                    & = \limit_{n \rightarrow \infty}x_n \\
                    & = \limit_{n \rightarrow \infty}f(x_{n-1}) \\
                    & = f \left( \limit_{n \rightarrow \infty}x_{n-1} \right) \\
                    & = f(x).
                \end{split}
                \end{equation*}
            So $f$ admits a fixed point. Suppose $x' \in X$ is also a fixed point. Then:
                \begin{equation*}
                \begin{split}
                    d(x,x')
                    & =d(f(x),f(x')) \\
                    & \leq \theta d(x,x').
                \end{split}
                \end{equation*}
            Note that this only holds if $d(x,x') = 0$ Thus $x=x'$, establishing that $f$ admits a unique fixed point.
        \end{proof}
%%%%%%%%%%%%%%%%%%%%%%%%%%%%%%%%%%%%%%%%%%%%%%%%%%%%%%%%%%%%%
    \newpage
\addtocounter{exercise}{1}
    \begin{exercise}
        Let $T:V \rightarrow W$ be a continuous linear map between normed spaces which is bounded below, that is, there is a $C>0$ with $\lnorm Tv \rnorm \geq C \lnorm v \rnorm$ for all $v \in V$. If $V$ is complete, show that $\Image(T) \subseteq W$ is a closed subspace, and that $V \cong \Image(T)$ are uniformly isomorphic.
    \end{exercise}
        \begin{proof}
            Let $(T(v_n))_n$ be a sequence in $\Image(T)$ converging to $w \in W$. Given $\epsilon$, find $N$ large so that $n \geq M$ implies $\lnorm T(v_n)-w \rnorm < \frac{C\epsilon}{2}$. For $n,m \geq N$, observe that:
                \begin{equation*}
                \begin{split}
                    \lnorm v_n - v_m \rnorm
                    & \leq \frac{1}{C}\lnorm T(v_n - v_m) \rnorm \\
                    & = \frac{1}{C}\lnorm T(v_n) - T(v_m) \rnorm \\
                    & \leq \frac{1}{C}\lnorm T(v_n) - w \rnorm + \frac{1}{C}\lnorm w - T(v_m) \rnorm \\
                    & < \frac{\epsilon}{2} + \frac{\epsilon}{2} \\
                    & = \epsilon.
                \end{split}
                \end{equation*}
            Thus $(v_n)_n$ is Cauchy. Since $V$ is complete, let $v_0 := \limit_{n \rightarrow \infty}v_n$. Since $T$ is continuous, we can see $(T(v_n))_n \rightarrow T(v_0)$. It must be the case that $T(v_0) = w$; i.e., $w \in \Image(T)$. Thus $\Image(T)$ is a closed subspace.

            Since $T$ is continuous, there exists some $\alpha > 0$ such that $\lnorm Tv \rnorm \leq \alpha \lnorm  v \rnorm$. Clearly if $v = 0$, then $Tv = 0$, implying that $T$ is injective. Whence $V \cong \Image(T)$ as vector spaces. Since $T$ is continuous, it is uniformly continuous, so it remains to show that $T^{-1}:\Image(T) \rightarrow V$ (which exists) is also continuous. Let $w \in \Image(T)$, then there exists $v \in V$ with $T(v) = w$. Observe that:
                \begin{equation*}
                \begin{split}
                    \lnorm T^{-1} w \rnorm
                    & = \lnorm T^{-1}(T(v)) \rnorm \\
                    & = \lnorm v \rnorm \\
                    & \leq \frac{1}{C}\lnorm Tv \rnorm \\
                    & = \frac{1}{C}\lnorm w \rnorm.
                \end{split}
                \end{equation*}
            Thus $T$ is uniformism.
        \end{proof}
%%%%%%%%%%%%%%%%%%%%%%%%%%%%%%%%%%%%%%%%%%%%%%%%%%%%%%%%%%%%%
    \begin{exercise}
        Let $(X,d)$ and $(Y,\rho)$ be metric spaces with completions $(\widetilde{X},\iota_X)$ and $(\widetilde{Y},\iota_Y)$ respectively. If $f:X \rightarrow Y$ is an isometry, show that there is a unique isometry $\widetilde{f}:\widetilde{X} \rightarrow \widetilde{Y}$ that extends $f$, that is, the following diagram commutes:
            \begin{center}
                \begin{tikzcd}
                    \widetilde{X} \arrow[r, "\widetilde{f}"] & \widetilde{Y}           \\
                    X \arrow[r, "f"] \arrow[u, "\iota_x"]    & Y \arrow[u, "\iota_y"']
                    \end{tikzcd}
            \end{center}
    \end{exercise}
        \begin{proof}
            Define $\varphi:\iota(X) \rightarrow \widetilde{Y}$ by $\varphi(\iota(x)) = \iota_Y(f(x))$. Since $f$ and $\iota_Y$ are isometries, note that their composition $\iota_Y \circ f$ is also an isometry. This gives:
                \begin{equation*}
                \begin{split}
                    \rho(\varphi(\iota_X(x_1)),\varphi(\iota_X(x_2)))
                    & = \rho(\iota_Y(f(x_1)),\iota_Y(f(x_2))) \\
                    & = d(x_1,x_2).
                \end{split}
                \end{equation*}
            Since $\varphi$ is an isometry,  the unique uniformly continuous extension $\widetilde{f}:\widetilde{X} \rightarrow \widetilde{Y}$ is also an isometry.
        \end{proof}
%%%%%%%%%%%%%%%%%%%%%%%%%%%%%%%%%%%%%%%%%%%%%%%%%%%%%%%%%%%%%
    \newpage
    \begin{exercise}
        Let $V$ be a normed space, $W$ a Banach space, and $U \subseteq V$ a dense linear subspace. Moreover, let $T_0:U \rightarrow W$ be a bounded linear map. Show that there is a unique bounded linear map $T:V \rightarrow W$ that extends $T_0$, that is, $\restr{T}{U} = T_0$.
    \end{exercise}
        \begin{proof}
            Clearly $V$ is a metric space, $U\subseteq V$ is a dense subset, and $W$ is a complete metric space. So there exists a uniformly continuous map $T:V \rightarrow W$ with $T(v) = T_0(v)$ for all $v \in U$. Hence we only need to show $T$ is linear and bounded. Let $v,v' \in V$ and $\alpha \in F$. Let $(x_n)_n$ and $(y_n)_n$ be sequences in $U$ with $(x_n)_n \rightarrow v$ and $(y_n)_n \rightarrow v'$. Observe that:
                \begin{equation*}
                \begin{split}
                    T(v + \alpha v')
                    & = \limit_{n \rightarrow \infty} T_0(x_n + \alpha y_n) \\
                    & = \limit_{n \rightarrow \infty} T_0(x_n) + \alpha \limit_{n \rightarrow \infty} T_0(y_n) \\
                    & = T(v) + \alpha T(v').
                \end{split}
                \end{equation*}
            Thus $T$ is linear. To show $T$ is bounded, it suffices to show $\lnorm T \rnorm_{\op} = \lnorm T_0 \rnorm_{\op}$, since $T_0$ is bounded. Note that the composition $V \xrightarrow{\h6T\h6} W \xrightarrow{\lnorm \cdot \rnorm_W} F$ will be continuous and bounded, which means:
                \begin{equation*}
                \begin{split}
                   \lnorm T \rnorm_{\op} 
                   & = \sup_{v \in B_V}\lnorm T(v) \rnorm_W \\
                   & = \sup_{v \in B_U}\lnorm T(v) \rnorm_W \\
                   & = \sup_{v \in B_U}\lnorm T_0(v) \rnorm_W \\
                   & = \lnorm T_0 \rnorm_{\op}.
                \end{split}
                \end{equation*}
            Thus $T \in B(V,W)$.
        \end{proof}
%%%%%%%%%%%%%%%%%%%%%%%%%%%%%%%%%%%%%%%%%%%%%%%%%%%%%%%%%%%%%
    \begin{exercise}
        Let $X$ be a metric space. Show that the following are equivalent:
            \begin{enumerate}[label = (\arabic*),itemsep=1pt,topsep=3pt]
                \item Every meager set has empty interior.
                \item The complement of a meager set is dense.
            \end{enumerate}
        Moreover, show that these equivalent statements hold true if the metric space is complete.
    \end{exercise}
        \begin{proof}
            If $A \subseteq X$ is meager with $A^0 = \emptyset$, then $\overline{A^c} = (A^o)^c = \emptyset^c = X$. The converse is identical.

            Now suppose $X$ is a complete metric space. If $A \subseteq X$ is meager, then $A = \bigcup_{n \geq 1}A_n$ We will show that the complement of $A$ is dense. Clearly $\overline{\bigcap_{n \geq 1}A_n^c} \subseteq X$, so it remains to show the other direction of inclusion. Define $B_n = \overline{A_n}$. Clearly $A_n \subseteq B_n$, which implies that $A_n^c \supseteq B_n^c$ for each $n$. Whence $\bigcap_{n \geq 1}A_n^c \supseteq \bigcap_{n \geq 1}B_n^c$. Furthermore, $\overline{\bigcap_{n \geq 1}A_n^c} \supseteq \overline{\bigcap_{n \geq 1}B_n^c}$. Note that each $B_n^c$ is open and dense, so by Baire's theorem we have $\overline{\bigcap_{n \geq 1}B_n^c} = X$. Thus $\overline{\left(\bigcup_{n \geq 1}A_n\right)^c} = X$. Since (1) and (2) are equivalent, we've established that both statements hold true if $X$ is a complete metric space.
        \end{proof}
%%%%%%%%%%%%%%%%%%%%%%%%%%%%%%%%%%%%%%%%%%%%%%%%%%%%%%%%%%%%%
    \newpage
    \begin{exercise}
        Let $V$ be a normed space with linear basis $B$.
        \begin{enumerate}[label = (\arabic*),itemsep=1pt,topsep=3pt]
            \item If $W \subseteq V$ is a proper subspace, show that $W^o = \emptyset$.
            \item If $V$ is a Banach space, show that $B$ is uncountable. You may use the fact that finite-dimensional subspaces are always closed.
        \end{enumerate}
    \end{exercise}
        \begin{proof}
            Suppose towards contradiction that $W^o \neq \emptyset$. Then we can find some $v_0 \in W^o$. In particular, there exists $\delta > 0$ such that $U(v_0,\delta) \subseteq W$. Let $v \in V$. We can see that $\frac{\delta}{2}\frac{v}{\lnorm v \rnorm} +v_0 \in U(v_0,\delta) \subseteq W$, so for some $w \in W$ we have:
                \begin{equation*}
                \begin{split}
                    \frac{\delta}{2}\frac{v}{\lnorm v \rnorm} +v_0 = w.
                \end{split}
                \end{equation*}
            Solving for $v$ yields $v = \lnorm v \rnorm\frac{2}{\delta}(w - v_0)$. But $\lnorm v \rnorm\frac{2}{\delta}(w - v_0) \in W$, so $v \in W$, which contradicts $W \subseteq V$ being a proper subspace. Thus $W^o = \emptyset$.
            

            Suppose towards contradiction $B$ is countable, that is, $B = \{e_n \mid n \geq 1\}$. Then:
                \begin{equation*}
                \begin{split}
                    V
                    & = \Span \{e_n \mid n \geq 1\} \\
                    & = \bigcup_{n \geq 1}\Span\{e_1,...,e_n\}.
                \end{split}
                \end{equation*}
            Note that $\Span\{e_1,...,e_n\}$ is a finite \textit{and} proper subspace of $V$. This means $\overline{\Span\{e_1,e_2,...,e_n\}}^o = \emptyset$. But this contradicts Baire's theorem, since we've written $V$ \textemdash a complete normed space\textemdash as the countable union of nowhere dense sets. It must be the case that $B$ is uncountable.
        \end{proof}
%%%%%%%%%%%%%%%%%%%%%%%%%%%%%%%%%%%%%%%%%%%%%%%%%%%%%%%%%%%%%

\end{document}