\documentclass[11pt,twoside,openany]{memoir}
\usepackage{mlmodern}
%\usepackage{tgpagella} % text only
%\usepackage{mathpazo}  % math & text
\usepackage[T1]{fontenc}
\usepackage[hidelinks]{hyperref}
\usepackage{amsmath}
\usepackage{amsthm}
\usepackage{amssymb}
\usepackage{mathtools}
\usepackage{graphicx}
%\usepackage{newpxtext}
\usepackage{eulerpx}
\usepackage{eucal}
\usepackage{datetime}
    \newdateformat{specialdate}{\THEYEAR\ \monthname\ \THEDAY}
\usepackage[margin=1in]{geometry}
\usepackage{fancyhdr}
    \fancyhf{}
    \pagestyle{fancy}
    \cfoot{\scriptsize \thepage}
    \fancyhead[R]{\scriptsize \rightmark}
    \fancyhead[L]{\scriptsize \leftmark}
    \renewcommand{\headrulewidth}{0pt}
    \renewcommand{\footrulewidth}{0pt} % if you also want to remove the footer rule
\usepackage{thmtools}
    \declaretheoremstyle[
        spaceabove=10pt,
        spacebelow=10pt,
        headfont=\normalfont\bfseries,
        notefont=\mdseries, notebraces={(}{)},
        bodyfont=\normalfont,
        postheadspace=0.5em
        %qed=\qedsymbol
        ]{defs}

    \declaretheoremstyle[ 
        spaceabove=10pt, % space above the theorem
        spacebelow=10pt,
        headfont=\normalfont\bfseries,
        bodyfont=\normalfont\itshape,
        postheadspace=0.5em
        ]{thmstyle}
    
    \declaretheorem[
        style=thmstyle,
        numberwithin=section
    ]{theorem}

    \declaretheorem[
        style=thmstyle,
        sibling=theorem,
    ]{proposition}

    \declaretheorem[
        style=thmstyle,
        sibling=theorem,
    ]{lemma}

    \declaretheorem[
        style=thmstyle,
        sibling=theorem,
    ]{corollary}

    \declaretheorem[
        numberwithin=section,
        style=defs,
    ]{example}

    \declaretheorem[
        numberwithin=section,
        style=defs,
    ]{definition}

    \declaretheorem[
        style=defs,
        numbered=unless unique,
    ]{problem}

    \declaretheorem[
        numbered=unless unique,
        shaded={rulecolor=black,
    rulewidth=1pt, bgcolor={rgb}{1,1,1}}
    ]{axiom}

    \declaretheorem[numberwithin=section,style=defs]{note}
    \declaretheorem[numbered=no,style=defs]{question}
    \declaretheorem[numbered=no,style=defs]{recall}
    \declaretheorem[numbered=no,style=remark]{answer}
    \declaretheorem[numbered=no,style=remark]{solution}

    \declaretheorem[numbered=no,style=defs]{remark}
\usepackage{enumitem}
\usepackage{titlesec}
    \titleformat{\chapter}[display]
    {\bfseries\LARGE\raggedright}
    {Chapter {\thechapter}}
    {1ex minus .1ex}
    {\Huge}
    \titlespacing{\chapter}
    {3pc}{*3}{40pt}[3pc]

    \titleformat{\section}[block]
    {\normalfont\bfseries\Large}
    {\S\ \thesection.}{.5em}{}[]
    \titlespacing{\section}
    {0pt}{3ex plus .1ex minus .2ex}{3ex plus .1ex minus .2ex}
\usepackage[utf8x]{inputenc}
\usepackage{tikz}
\usepackage{tikz-cd}
\usepackage{wasysym}
\usepackage{pgf}
\usepackage{mmacells}
\usepackage{pythonhighlight}

\mmaDefineMathReplacement[≤]{<=}{\leq}
\mmaDefineMathReplacement[≥]{>=}{\geq}
\mmaDefineMathReplacement[≠]{!=}{\neq}
\mmaDefineMathReplacement[→]{->}{\to}[2]
\mmaDefineMathReplacement[⧴]{:>}{:\hspace{-.2em}\to}[2]
\mmaDefineMathReplacement{∉}{\notin}
\mmaDefineMathReplacement{∞}{\infty}
\mmaDefineMathReplacement{𝕕}{\mathbbm{d}}

\linespread{0.95}
%to make the correct symbol for Sha
%\newcommand\cyr{%
%\renewcommand\rmdefault{wncyr}%
%\renewcommand\sfdefault{wncyss}%
%\renewcommand\encodingdefault{OT2}%
%\normalfont \selectfont} \DeclareTextFontCommand{\textcyr}{\cyr}


\DeclareMathOperator{\ab}{ab}
\newcommand{\absgal}{\G_{\bbQ}}
\DeclareMathOperator{\ad}{ad}
\DeclareMathOperator{\adj}{adj}
\DeclareMathOperator{\alg}{alg}
\DeclareMathOperator{\Alt}{Alt}
\DeclareMathOperator{\Ann}{Ann}
\DeclareMathOperator{\arith}{arith}
\DeclareMathOperator{\Aut}{Aut}
\DeclareMathOperator{\Be}{B}
\DeclareMathOperator{\Bd}{Bd}
\DeclareMathOperator{\card}{card}
\DeclareMathOperator{\Char}{char}
\DeclareMathOperator{\csp}{csp}
\DeclareMathOperator{\codim}{codim}
\DeclareMathOperator{\coker}{coker}
\DeclareMathOperator{\coh}{H}
\DeclareMathOperator{\compl}{compl}
\DeclareMathOperator{\conj}{conj}
\DeclareMathOperator{\cont}{cont}
\DeclareMathOperator{\crys}{crys}
\DeclareMathOperator{\Crys}{Crys}
\DeclareMathOperator{\cusp}{cusp}
\DeclareMathOperator{\diag}{diag}
\DeclareMathOperator{\diam}{diam}
\DeclareMathOperator{\Dom}{Dom}
\DeclareMathOperator{\disc}{disc}
\DeclareMathOperator{\dist}{dist}
\DeclareMathOperator{\dR}{dR}
\DeclareMathOperator{\Eis}{Eis}
\DeclareMathOperator{\End}{End}
\DeclareMathOperator{\ev}{ev}
\DeclareMathOperator{\eval}{eval}
\DeclareMathOperator{\Eq}{Eq}
\DeclareMathOperator{\Ext}{Ext}
\DeclareMathOperator{\Fil}{Fil}
\DeclareMathOperator{\Fitt}{Fitt}
\DeclareMathOperator{\Frob}{Frob}
\DeclareMathOperator{\G}{G}
\DeclareMathOperator{\Gal}{Gal}
\DeclareMathOperator{\GL}{GL}
\DeclareMathOperator{\Gr}{Gr}
\DeclareMathOperator{\Graph}{Graph}
\DeclareMathOperator{\GSp}{GSp}
\DeclareMathOperator{\GUn}{GU}
\DeclareMathOperator{\Hom}{Hom}
\DeclareMathOperator{\id}{id}
\DeclareMathOperator{\Id}{Id}
\DeclareMathOperator{\Ik}{Ik}
\DeclareMathOperator{\IM}{Im}
\DeclareMathOperator{\Image}{im}
\DeclareMathOperator{\Ind}{Ind}
\DeclareMathOperator{\Inf}{inf}
\DeclareMathOperator{\Isom}{Isom}
\DeclareMathOperator{\J}{J}
\DeclareMathOperator{\Jac}{Jac}
\DeclareMathOperator{\lcm}{lcm}
\DeclareMathOperator{\length}{length}
\DeclareMathOperator*{\limit}{limit}
\DeclareMathOperator{\Log}{Log}
\DeclareMathOperator{\M}{M}
\DeclareMathOperator{\Mat}{Mat}
\DeclareMathOperator{\N}{N}
\DeclareMathOperator{\Nm}{Nm}
\DeclareMathOperator{\NIk}{N-Ik}
\DeclareMathOperator{\NSK}{N-SK}
\DeclareMathOperator{\new}{new}
\DeclareMathOperator{\obj}{obj}
\DeclareMathOperator{\old}{old}
\DeclareMathOperator{\ord}{ord}
\DeclareMathOperator{\Or}{O}
\DeclareMathOperator{\op}{op}
\DeclareMathOperator{\PGL}{PGL}
\DeclareMathOperator{\PGSp}{PGSp}
\DeclareMathOperator{\rank}{rank}
\DeclareMathOperator{\Ran}{Ran}
\DeclareMathOperator{\Rel}{Rel}
\DeclareMathOperator{\Real}{Re}
\DeclareMathOperator{\RES}{res}
\DeclareMathOperator{\Res}{Res}
%\DeclareMathOperator{\Sha}{\textcyr{Sh}}
\DeclareMathOperator{\Sel}{Sel}
\DeclareMathOperator{\semi}{ss}
\DeclareMathOperator{\sgn}{sign}
\DeclareMathOperator{\SK}{SK}
\DeclareMathOperator{\SL}{SL}
\DeclareMathOperator{\SO}{SO}
\DeclareMathOperator{\Sp}{Sp}
\DeclareMathOperator{\Span}{span}
\DeclareMathOperator{\Spec}{Spec}
\DeclareMathOperator{\spin}{spin}
\DeclareMathOperator{\st}{st}
\DeclareMathOperator{\St}{St}
\DeclareMathOperator{\SUn}{SU}
\DeclareMathOperator{\supp}{supp}
\DeclareMathOperator{\Sup}{sup}
\DeclareMathOperator{\Sym}{Sym}
\DeclareMathOperator{\Tam}{Tam}
\DeclareMathOperator{\tors}{tors}
\DeclareMathOperator{\tr}{tr}
\DeclareMathOperator{\Tr}{Tr}
\DeclareMathOperator{\un}{un}
\DeclareMathOperator{\Un}{U}
\DeclareMathOperator{\val}{val}
\DeclareMathOperator{\vol}{vol}

\DeclareMathOperator{\Sets}{S \mkern1.04mu e \mkern1.04mu t \mkern1.04mu s}
    \newcommand{\cSets}{\scalebox{1.02}{\contour{black}{$\Sets$}}}
    
\DeclareMathOperator{\Groups}{G \mkern1.04mu r \mkern1.04mu o \mkern1.04mu u \mkern1.04mu p \mkern1.04mu s}
    \newcommand{\cGroups}{\scalebox{1.02}{\contour{black}{$\Groups$}}}

\DeclareMathOperator{\TTop}{T \mkern1.04mu o \mkern1.04mu p}
    \newcommand{\cTop}{\scalebox{1.02}{\contour{black}{$\TTop$}}}

\DeclareMathOperator{\Htp}{H \mkern1.04mu t \mkern1.04mu p}
    \newcommand{\cHtp}{\scalebox{1.02}{\contour{black}{$\Htp$}}}

\DeclareMathOperator{\Mod}{M \mkern1.04mu o \mkern1.04mu d}
    \newcommand{\cMod}{\scalebox{1.02}{\contour{black}{$\Mod$}}}

\DeclareMathOperator{\Ab}{A \mkern1.04mu b}
    \newcommand{\cAb}{\scalebox{1.02}{\contour{black}{$\Ab$}}}

\DeclareMathOperator{\Rings}{R \mkern1.04mu i \mkern1.04mu n \mkern1.04mu g \mkern1.04mu s}
    \newcommand{\cRings}{\scalebox{1.02}{\contour{black}{$\Rings$}}}

\DeclareMathOperator{\ComRings}{C \mkern1.04mu o \mkern1.04mu m \mkern1.04mu R \mkern1.04mu i \mkern1.04mu n \mkern1.04mu g \mkern1.04mu s}
    \newcommand{\cComRings}{\scalebox{1.05}{\contour{black}{$\ComRings$}}}

\DeclareMathOperator{\hHom}{H \mkern1.04mu o \mkern1.04mu m}
    \newcommand{\cHom}{\scalebox{1.02}{\contour{black}{$\hHom$}}}

         %  \item $\cGroups$
          %  \item $\cTop$
          %  \item $\cHtp$
          %  \item $\cMod$




\renewcommand{\k}{\kappa}
\newcommand{\Ff}{F_{f}}
%\newcommand{\ts}{\,^{t}\!}


%Mathcal

\newcommand{\cA}{\mathcal{A}}
\newcommand{\cB}{\mathcal{B}}
\newcommand{\cC}{\mathcal{C}}
\newcommand{\cD}{\mathcal{D}}
\newcommand{\cE}{\mathcal{E}}
\newcommand{\cF}{\mathcal{F}}
\newcommand{\cG}{\mathcal{G}}
\newcommand{\cH}{\mathcal{H}}
\newcommand{\cI}{\mathcal{I}}
\newcommand{\cJ}{\mathcal{J}}
\newcommand{\cK}{\mathcal{K}}
\newcommand{\cL}{\mathcal{L}}
\newcommand{\cM}{\mathcal{M}}
\newcommand{\cN}{\mathcal{N}}
\newcommand{\cO}{\mathcal{O}}
\newcommand{\cP}{\mathcal{P}}
\newcommand{\cQ}{\mathcal{Q}}
\newcommand{\cR}{\mathcal{R}}
\newcommand{\cS}{\mathcal{S}}
\newcommand{\cT}{\mathcal{T}}
\newcommand{\cU}{\mathcal{U}}
\newcommand{\cV}{\mathcal{V}}
\newcommand{\cW}{\mathcal{W}}
\newcommand{\cX}{\mathcal{X}}
\newcommand{\cY}{\mathcal{Y}}
\newcommand{\cZ}{\mathcal{Z}}


%mathfrak (missing \fi)

\newcommand{\fa}{\mathfrak{a}}
\newcommand{\fA}{\mathfrak{A}}
\newcommand{\fb}{\mathfrak{b}}
\newcommand{\fB}{\mathfrak{B}}
\newcommand{\fc}{\mathfrak{c}}
\newcommand{\fC}{\mathfrak{C}}
\newcommand{\fd}{\mathfrak{d}}
\newcommand{\fD}{\mathfrak{D}}
\newcommand{\fe}{\mathfrak{e}}
\newcommand{\fE}{\mathfrak{E}}
\newcommand{\ff}{\mathfrak{f}}
\newcommand{\fF}{\mathfrak{F}}
\newcommand{\fg}{\mathfrak{g}}
\newcommand{\fG}{\mathfrak{G}}
\newcommand{\fh}{\mathfrak{h}}
\newcommand{\fH}{\mathfrak{H}}
\newcommand{\fI}{\mathfrak{I}}
\newcommand{\fj}{\mathfrak{j}}
\newcommand{\fJ}{\mathfrak{J}}
\newcommand{\fk}{\mathfrak{k}}
\newcommand{\fK}{\mathfrak{K}}
\newcommand{\fl}{\mathfrak{l}}
\newcommand{\fL}{\mathfrak{L}}
\newcommand{\fm}{\mathfrak{m}}
\newcommand{\fM}{\mathfrak{M}}
\newcommand{\fn}{\mathfrak{n}}
\newcommand{\fN}{\mathfrak{N}}
\newcommand{\fo}{\mathfrak{o}}
\newcommand{\fO}{\mathfrak{O}}
\newcommand{\fp}{\mathfrak{p}}
\newcommand{\fP}{\mathfrak{P}}
\newcommand{\fq}{\mathfrak{q}}
\newcommand{\fQ}{\mathfrak{Q}}
\newcommand{\fr}{\mathfrak{r}}
\newcommand{\fR}{\mathfrak{R}}
\newcommand{\fs}{\mathfrak{s}}
\newcommand{\fS}{\mathfrak{S}}
\newcommand{\ft}{\mathfrak{t}}
\newcommand{\fT}{\mathfrak{T}}
\newcommand{\fu}{\mathfrak{u}}
\newcommand{\fU}{\mathfrak{U}}
\newcommand{\fv}{\mathfrak{v}}
\newcommand{\fV}{\mathfrak{V}}
\newcommand{\fw}{\mathfrak{w}}
\newcommand{\fW}{\mathfrak{W}}
\newcommand{\fx}{\mathfrak{x}}
\newcommand{\fX}{\mathfrak{X}}
\newcommand{\fy}{\mathfrak{y}}
\newcommand{\fY}{\mathfrak{Y}}
\newcommand{\fz}{\mathfrak{z}}
\newcommand{\fZ}{\mathfrak{Z}}


%mathbf
\newcommand{\bfA}{\mathbf{A}}
\newcommand{\bfB}{\mathbf{B}}
\newcommand{\bfC}{\mathbf{C}}
\newcommand{\bfD}{\mathbf{D}}
\newcommand{\bfE}{\mathbf{E}}
\newcommand{\bfF}{\mathbf{F}}
\newcommand{\bfG}{\mathbf{G}}
\newcommand{\bfH}{\mathbf{H}}
\newcommand{\bfI}{\mathbf{I}}
\newcommand{\bfJ}{\mathbf{J}}
\newcommand{\bfK}{\mathbf{K}}
\newcommand{\bfL}{\mathbf{L}}
\newcommand{\bfM}{\mathbf{M}}
\newcommand{\bfN}{\mathbf{N}}
\newcommand{\bfO}{\mathbf{O}}
\newcommand{\bfP}{\mathbf{P}}
\newcommand{\bfQ}{\mathbf{Q}}
\newcommand{\bfR}{\mathbf{R}}
\newcommand{\bfS}{\mathbf{S}}
\newcommand{\bfT}{\mathbf{T}}
\newcommand{\bfU}{\mathbf{U}}
\newcommand{\bfV}{\mathbf{V}}
\newcommand{\bfW}{\mathbf{W}}
\newcommand{\bfX}{\mathbf{X}}
\newcommand{\bfY}{\mathbf{Y}}
\newcommand{\bfZ}{\mathbf{Z}}

\newcommand{\bfa}{\mathbf{a}}
\newcommand{\bfb}{\mathbf{b}}
\newcommand{\bfc}{\mathbf{c}}
\newcommand{\bfd}{\mathbf{d}}
\newcommand{\bfe}{\mathbf{e}}
\newcommand{\bff}{\mathbf{f}}
\newcommand{\bfg}{\mathbf{g}}
\newcommand{\bfh}{\mathbf{h}}
\newcommand{\bfi}{\mathbf{i}}
\newcommand{\bfj}{\mathbf{j}}
\newcommand{\bfk}{\mathbf{k}}
\newcommand{\bfl}{\mathbf{l}}
\newcommand{\bfm}{\mathbf{m}}
\newcommand{\bfn}{\mathbf{n}}
\newcommand{\bfo}{\mathbf{o}}
\newcommand{\bfp}{\mathbf{p}}
\newcommand{\bfq}{\mathbf{q}}
\newcommand{\bfr}{\mathbf{r}}
\newcommand{\bfs}{\mathbf{s}}
\newcommand{\bft}{\mathbf{t}}
\newcommand{\bfu}{\mathbf{u}}
\newcommand{\bfv}{\mathbf{v}}
\newcommand{\bfw}{\mathbf{w}}
\newcommand{\bfx}{\mathbf{x}}
\newcommand{\bfy}{\mathbf{y}}
\newcommand{\bfz}{\mathbf{z}}

%blackboard bold

\newcommand{\bbA}{\mathbb{A}}
\newcommand{\bbB}{\mathbb{B}}
\newcommand{\bbC}{\mathbb{C}}
\newcommand{\bbD}{\mathbb{D}}
\newcommand{\bbE}{\mathbb{E}}
\newcommand{\bbF}{\mathbb{F}}
\newcommand{\bbG}{\mathbb{G}}
\newcommand{\bbH}{\mathbb{H}}
\newcommand{\bbI}{\mathbb{I}}
\newcommand{\bbJ}{\mathbb{J}}
\newcommand{\bbK}{\mathbb{K}}
\newcommand{\bbL}{\mathbb{L}}
\newcommand{\bbM}{\mathbb{M}}
\newcommand{\bbN}{\mathbb{N}}
\newcommand{\bbO}{\mathbb{O}}
\newcommand{\bbP}{\mathbb{P}}
\newcommand{\bbQ}{\mathbb{Q}}
\newcommand{\bbR}{\mathbb{R}}
\newcommand{\bbS}{\mathbb{S}}
\newcommand{\bbT}{\mathbb{T}}
\newcommand{\bbU}{\mathbb{U}}
\newcommand{\bbV}{\mathbb{V}}
\newcommand{\bbW}{\mathbb{W}}
\newcommand{\bbX}{\mathbb{X}}
\newcommand{\bbY}{\mathbb{Y}}
\newcommand{\bbZ}{\mathbb{Z}}
\newcommand{\jota}{\jmath}

\newcommand{\bmat}{\left( \begin{matrix}}
\newcommand{\emat}{\end{matrix} \right)}

\newcommand{\pmat}{\left( \begin{smallmatrix}}
\newcommand{\epmat}{\end{smallmatrix} \right)}

\newcommand{\lat}{\mathscr{L}}
\newcommand{\mat}[4]{\begin{pmatrix}{#1}&{#2}\\{#3}&{#4}\end{pmatrix}}
\newcommand{\ov}[1]{\overline{#1}}
\newcommand{\res}[1]{\underset{#1}{\RES}\,}
\newcommand{\up}{\upsilon}

\newcommand{\tac}{\textasteriskcentered}

%mahesh macros
\newcommand{\tm}{\textrm}

%Comments
\newcommand{\com}[1]{\vspace{5 mm}\par \noindent
\marginpar{\textsc{Comment}} \framebox{\begin{minipage}[c]{0.95
\textwidth} \tt #1 \end{minipage}}\vspace{5 mm}\par}

\newcommand{\Bmu}{\mbox{$\raisebox{-0.59ex}
  {$l$}\hspace{-0.18em}\mu\hspace{-0.88em}\raisebox{-0.98ex}{\scalebox{2}
  {$\color{white}.$}}\hspace{-0.416em}\raisebox{+0.88ex}
  {$\color{white}.$}\hspace{0.46em}$}{}}  %need graphicx and xcolor. this produces blackboard bold mu 

\newcommand{\hooktwoheadrightarrow}{%
  \hookrightarrow\mathrel{\mspace{-15mu}}\rightarrow
}

\makeatletter
\newcommand{\xhooktwoheadrightarrow}[2][]{%
  \lhook\joinrel
  \ext@arrow 0359\rightarrowfill@ {#1}{#2}%
  \mathrel{\mspace{-15mu}}\rightarrow
}
\makeatother

\renewcommand{\geq}{\geqslant}
\renewcommand{\leq}{\leqslant}
\newcommand{\midd}{\hspace{4pt}\middle|\hspace{4pt}}
    
    \newcommand{\bone}{\mathbf{1}}
    \newcommand{\sign}{\mathrm{sign}}
    \newcommand{\eps}{\varepsilon}
    \newcommand{\textui}[1]{\uline{\textit{#1}}}
    
    %\newcommand{\ov}{\overline}
    %\newcommand{\un}{\underline}
    \newcommand{\fin}{\mathrm{fin}}
    
    \newcommand{\chnum}{\titleformat
    {\chapter} % command
    [display] % shape
    {\centering} % format
    {\Huge \color{black} \shadowbox{\thechapter}} % label
    {-0.5em} % sep (space between the number and title)
    {\LARGE \color{black} \underline} % before-code
    }
    
    \newcommand{\chunnum}{\titleformat
    {\chapter} % command
    [display] % shape
    {} % format
    {} % label
    {0em} % sep
    { \begin{flushright} \begin{tabular}{r}  \Huge \color{black}
    } % before-code
    [
    \end{tabular} \end{flushright} \normalsize
    ] % after-code
    }

\newcommand{\nl}{\newline \mbox{}}

\newcommand{\h}[1]{\hspace{#1pt}}

\newcommand{\littletaller}{\mathchoice{\vphantom{\big|}}{}{}{}}
\newcommand\restr[2]{{% we make the whole thing an ordinary symbol
  \left.\kern-\nulldelimiterspace % automatically resize the bar with \right
  #1 % the function
  \littletaller % pretend it's a little taller at normal size
  \right|_{#2} % this is the delimiter
  }}

\newcommand{\mtext}[1]{\hspace{6pt}\text{#1}\hspace{6pt}}

\newcommand{\lnorm}{\left\lVert}
\newcommand{\rnorm}{\right\rVert}

\newcommand{\ds}{\displaystyle}
\newcommand{\ts}{\textstyle}

%This adds a "front cover" page.
%{\thispagestyle{empty}
%\vspace*{\fill}
%\begin{tabular}{l}
%\begin{tabular}{l}
%\includegraphics[scale=0.24]{oxy-logo.png}
%\end{tabular} \\
%\begin{tabular}{l}
%\Large \color{black} Module Theory, Linear Algebra, and Homological Algebra \\ \Large \color{black} Gianluca Crescenzo
%\end{tabular}
%\end{tabular}
%\newpage

\newcommand{\sfrac}[2]{{}^{#1}\mskip -5mu/\mskip -3mu_{#2}}


\makeatletter
\newcommand*{\da@rightarrow}{\mathchar"0\hexnumber@\symAMSa 4B }
\newcommand*{\da@leftarrow}{\mathchar"0\hexnumber@\symAMSa 4C }
\newcommand*{\xdashrightarrow}[2][]{%
  \mathrel{%
    \mathpalette{\da@xarrow{#1}{#2}{}\da@rightarrow{\,}{}}{}%
  }%
}
\newcommand{\xdashleftarrow}[2][]{%
  \mathrel{%
    \mathpalette{\da@xarrow{#1}{#2}\da@leftarrow{}{}{\,}}{}%
  }%
}
\newcommand*{\da@xarrow}[7]{%
  % #1: below
  % #2: above
  % #3: arrow left
  % #4: arrow right
  % #5: space left 
  % #6: space right
  % #7: math style 
  \sbox0{$\ifx#7\scriptstyle\scriptscriptstyle\else\scriptstyle\fi#5#1#6\m@th$}%
  \sbox2{$\ifx#7\scriptstyle\scriptscriptstyle\else\scriptstyle\fi#5#2#6\m@th$}%
  \sbox4{$#7\dabar@\m@th$}%
  \dimen@=\wd0 %
  \ifdim\wd2 >\dimen@
    \dimen@=\wd2 %   
  \fi
  \count@=2 %
  \def\da@bars{\dabar@\dabar@}%
  \@whiledim\count@\wd4<\dimen@\do{%
    \advance\count@\@ne
    \expandafter\def\expandafter\da@bars\expandafter{%
      \da@bars
      \dabar@ 
    }%
  }%  
  \mathrel{#3}%
  \mathrel{%   
    \mathop{\da@bars}\limits
    \ifx\\#1\\%
    \else
      _{\copy0}%
    \fi
    \ifx\\#2\\%
    \else
      ^{\copy2}%
    \fi
  }%   
  \mathrel{#4}%
}
\makeatother


\begin{document}
\begin{center}
{\large Math 374 \\[0.1in]Homework 1 \\[0.1in]}
{Name:} {\underline{Gianluca Crescenzo\hspace*{2in}}}\\[0.15in]
\end{center}
\vspace{4pt}
%%%%%%%%%%%%%%%%%%%%%%%%%%%%%%%%%%%%%%%%%%%%%%%%%%%%%%%%%%%%%
    \begin{problem}
        Compute the decimal (base 10) value for the following binary numbers.
            \begin{enumerate}[label = (\arabic*),itemsep=1pt,topsep=3pt]
                \item $10101100111000 $
                \item $0.110110110 $
                \item $100101110.01100101 $
            \end{enumerate}
    \end{problem}
        \begin{solution}
            \begin{equation*}
            \begin{split}
                10101100111000_2
                & = 2^{13} + 2^{11} +2^9 + 2^8 + 2^5 + 2^4 + 2^3 \\
                & = 302.39453125_{10} \\
                &\phantom{=} \\
                0.110110110_2
                & = 2^{-1} + 2^{-2} + 2^{-4} + 2^{-5} + 2^{-7} + 2^{-8} \\
                & = 0.85546875_{10} \\
                &\phantom{=} \\
                100101110.01100101_2 
                & = 2^8 + 2^5 + 2^3 + 2^2 + 2^1 + 2^{-2} + 2^{-3} + 2^{-6} + 2^{-8} \\
                & = 302.39453125_{10}
            \end{split}
            \end{equation*}
        \end{solution}
%%%%%%%%%%%%%%%%%%%%%%%%%%%%%%%%%%%%%%%%%%%%%%%%%%%%%%%%%%%%%
    \begin{problem}
        Compute the binary form of the following decimal numbers. Write 10 digits to the right of the binary point.
            \begin{enumerate}[label = (\arabic*),itemsep=1pt,topsep=3pt]
                \item $1272025.3255$
                \item $\frac{3141592}{65}$
                \item $\frac{1}{7}$
            \end{enumerate}
    \end{problem}
        \begin{solution}
            I used Mathematica code for this part, because division on pen/paper/calculator was frustrating.
                \begin{enumerate}[label = (\arabic*),itemsep=1pt,topsep=3pt]
                    \item The integer part is:
\begin{mmaCell}[functionlocal=y]{Code}
Clear[n, f, q, r, bits, digit, s]
n = 1272025;
bits = "";
While[n > 0, {q, r} = QuotientRemainder[n, 2];
    bits = ToString[r] <> bits;
n = q;];
bits
\end{mmaCell}
\begin{mmaCell}{Output}
100110110100011011001
\end{mmaCell}
\newpage
                The fractional part is:
\begin{mmaCell}[functionlocal=y]{Code}
Clear[n, f, q, r, bits, digit, s]
f = 0.3255;
bits = "";
Do[s = 2*f;
    digit = Floor[s];
    bits = bits <> ToString[digit];
    f = s - digit; 
    , {10}];
bits
\end{mmaCell}
\begin{mmaCell}{Output}
0101001101
\end{mmaCell}
                Thus $1272025.3255_{10} = 100110110100011011001.0101001101_{2}$.

                \item Note that $\frac{3141592}{65} = 48332.1\overline{846153}$. It's probably safer to write this as a mixed fraction, since I'm not sure if truncating the fractional part will cause problems. Changing the code a bit gives us:
\begin{mmaCell}[functionlocal=y]{Code}
Clear[n, ibits, fbits, q, r, f, s, digit]
n = 3141592/65;
i = IntegerPart[n];
f = FractionalPart[n];
ibits = "";
fbits = "";
While[i > 0, 
    {q, r} = QuotientRemainder[i, 2];
    ibits = ToString[r] <> ibits;
    i = q;];
Do[s = 2*f;
    digit = Floor[s];
    fbits = fbits <> ToString[digit];
    f = s - digit;
    , {10}];
ibits <> "." <> fbits
\end{mmaCell}
\begin{mmaCell}{Output}
1011110011001100.0010111101
\end{mmaCell}
                    Whence $\frac{3141592}{65}_{10} = 1011110011001100.0010111101_2$.

                \item Computing $\frac{1}{7}$ isn't as frustrating. We can see that:
                    \begin{equation*}
                    \begin{split}
                        2 \ast \frac{1}{7} \\
                        2 \ast \frac{2}{7} &\h9\h9 r=0 \\
                        2 \ast \frac{4}{7} &\h9\h9 r=0 \\
                        2 \ast \frac{1}{7} &\h9\h9 r=1 \\
                        2 \ast \frac{2}{7} &\h9\h9 r=0 \\
                        2 \ast \frac{4}{7} &\h9\h9 r=0 \\
                        2 \ast \frac{1}{7} &\h9\h9 r=1 \\
                        \vdots
                    \end{split}
                    \end{equation*}
                    \end{enumerate}
                This repeats forever. Only writing 10 digits to the right of the binary point, we see that $\frac{1}{7}_{10} = 0.0010010010_2$.
        \end{solution}
%%%%%%%%%%%%%%%%%%%%%%%%%%%%%%%%%%%%%%%%%%%%%%%%%%%%%%%%%%%%%
    \begin{problem}
        Determine the Binary16 (half precision), Binary32 (single precision), and Binary64 (double precision) bit patterns for the number $\pi$. Express your answers in both binary and hexadecimal form.
    \end{problem}
        \begin{solution}
            Note that $\pi \approx 3.1415926535897932384 = 2^1 \cdot 1.57079632679489661922$. I made a lot of changes to the above code to make it work with floating point precision.
\begin{mmaCell}[functionlocal=y]{Code}
ClearAll[exponentBits, mantissaBits];

exponentBits[e_, bitLength_] := 
 Module[{ebits = "", q, r, localE = e},
  Do[{q, r} = QuotientRemainder[localE, 2];
   ebits = ToString[r] <> ebits;
   localE = q;, {bitLength}];
  ebits]
   
mantissaBits[f_, bitLength_] := 
 Module[{mbits = "", digit, s, localF = f},
  Do[s = 2*localF;
   digit = Floor[s];
   mbits = mbits <> ToString[digit];
   localF = s - digit;, {bitLength}];
  mbits]
   
ClearAll[Binary];
Binary[totalBits_, val_] := 
 Module[{sbit, eBits, mBits, bias, e, f, n, counter},
  Switch[totalBits,
   16, {eBits = 5; mBits = 10; bias = 15;},
   32, {eBits = 8; mBits = 23; bias = 127;},
   64, {eBits = 11; mBits = 52; bias = 1023;},
   128, {eBits = 15; mBits = 112; bias = 16383;},
   256, {eBits = 19; mBits = 236; bias = 262143;},
   _, Return["unsupported size"]];
   
  sbit = If[val < 0, "1", "0"]; 
  n = Abs[val];
  If[n < 2, e = bias;
   f = n - 1;,
   counter = 0;
   While[n >= 2, n = n/2;
    counter++;];
   e = counter + bias;
   f = n - 1;];
  sbit <> exponentBits[e, eBits] <> mantissaBits[f, mBits]]
\end{mmaCell}




            For Binary16, our exponent is going to be $e = 1 + 15 = 16_{10} = 10000_{2}$, our mantissa is going to be $.57079632679489661922_{10} \approx 1001001000_2$, and our sign bit is going to be $0$. Binary32 and Binary64 follow similarly.
\begin{mmaCell}[functionlocal=y]{Code}
Binary[16, N[Pi, 23]]
Binary[32, N[Pi, 23]]
Binary[64, N[Pi, 23]]
\end{mmaCell}

\begin{mmaCell}{Output}
0100001001001000
\end{mmaCell}
\begin{mmaCell}{Output}
01000000010010010000111111011010
\end{mmaCell}
\begin{mmaCell}{Output}
0100000000001001001000011111101101010100010001000010110100011000
\end{mmaCell}

            Splitting the bit string into groups of four allows us to express our answer in hexademical. So:
                \begin{equation*}
                \begin{split}
                    \text{Binary}16: \pi_{10} &\approx 4248_{16} \\
                    \text{Binary}32: \pi_{10} &\approx 40490\text{FDA}_{16} \\
                    \text{Binary}64: \pi_{10} &\approx 400921\text{FB}54442\text{D}18_{16} \\
                \end{split}
                \end{equation*}
        \end{solution}
%%%%%%%%%%%%%%%%%%%%%%%%%%%%%%%%%%%%%%%%%%%%%%%%%%%%%%%%%%%%%
    \begin{problem}
        Determine the Binary64, Binary128, and Binary256 bit patterns for the number $\frac{127}{128}$. Express your answer in both binary and hexadecimal form.
    \end{problem}
        \begin{solution}
            Note that $\frac{127}{128} = 0.9921875 = 2^{-1} \cdot 1.984375$. I adjusted the previous code so that it can account for values less than 1.
            
\begin{mmaCell}[functionlocal=y]{Code}
sbit = If[val < 0, "1", "0"];
n = Abs[val];
counter = 0;
While[n >= 2, n = n/2;
  counter++;];
While[n < 1, n = 2*n;
  counter--;];
e = counter + bias;
f = n - 1;
sbit <> exponentBits[e, eBits] <> mantissaBits[f, mBits]
\end{mmaCell}
\begin{mmaCell}[functionlocal=y]{Code}
Binary[64, 127/128]
Binary[128, 127/128]
Binary[256, 127/128]
\end{mmaCell}
\begin{mmaCell}{Output}
0011111111101111110000000000000000000000000000000000000000000000
\end{mmaCell}
\begin{mmaCell}{Output}
0011111111111110111111000000000000000000000000000000000000000000000000
    0000000000000000000000000000000000000000000000000000000000
\end{mmaCell}
\begin{mmaCell}{Output}
0011111111111111111011111100000000000000000000000000000000000000000000
    000000000000000000000000000000000000000000000000000000000000000000
    000000000000000000000000000000000000000000000000000000000000000000
    000000000000000000000000000000000000000000000000000000
\end{mmaCell}
            We can now easily see that:
                \begin{equation*}
                \begin{split}
                    \text{Binary}16:\h5 \frac{127}{128}_{10}\h5 &= 3\text{FEFC}{\underbrace{0...0}_{11}} \\
                    \text{Binary}32:\h5 \frac{127}{128}_{10}\h5 &=3\text{FFEFC}{\underbrace{0...0}_{26}} \\
                    \text{Binary}64:\h5 \frac{127}{128}_{10}\h5 &= 3\text{FFFEFC}{\underbrace{0...0}_{57}} \\
                \end{split}
                \end{equation*}
        \end{solution}
%%%%%%%%%%%%%%%%%%%%%%%%%%%%%%%%%%%%%%%%%%%%%%%%%%%%%%%%%%%%%
    \begin{problem}
        Determine the largest positive double precision number $x_1$ and the next largest positive double precision number $x_2$. What is the difference between these two numbers?
    \end{problem}
        \begin{solution}
            Since $e = 11111111111_2$ is used to represent $\infty$, the largest positive double precision number is:
                \begin{equation*}
                \begin{split}
                    x_1 &= 0\h4 11111111110 \h4 1111111111111111111111111111111111111111 1111111111111_2 \\
                    & = (-1)^0 \cdot 2^{2046-1023} \cdot \left( 1 + \left(\frac{1}{2} + \frac{1}{4} + ... + 2^{-52}\right) \right) \\
                    & =  2^{1023} \cdot(1 + (1 - 2^{-53}))\\
                    & = 2^{1023} \cdot (2 - 2^{-53})_{10} \\
                    & \approx 1.7976931348623158... \times 10^{308}.
                \end{split}
                \end{equation*}
            The next largest positive double precision number, $x_2$, would be:
                \begin{equation*}
                \begin{split}
                    x_2 &= 0\h4 11111111110 \h4 1111111111111111111111111111111111111111111111111110_2 \\
                    & = 2^{1023} \cdot (2  - 2^{-52}) \\
                    & \approx 1.7976931348623157... \times 10^{308}.
                \end{split}
                \end{equation*}
            So:
                \begin{equation*}
                \begin{split}
                    x_1 - x_2 
                    & = 2^{1023} \cdot (2 - 2^{-53}) - 2^{1023} \cdot (2  - 2^{-52}) \\
                    & = 2^{1023} \cdot (2 - 2^{-53} - 2 + 2^{-52}) \\
                    & = 2^{1023} \cdot (2^{-52} - 2^{-53}) \\
                    & \approx 9.9792015476735990... \times 10^{291}.
                \end{split}
                \end{equation*}
        \end{solution}
%%%%%%%%%%%%%%%%%%%%%%%%%%%%%%%%%%%%%%%%%%%%%%%%%%%%%%%%%%%%%
    \begin{problem}
        Determine the smallest positive double precision number $x_1$ and the next smallest positive double precision number $x_2$. What is the difference between these two numbers?
    \end{problem}
        \begin{solution}
           Since $e = 00000000000_2$ is used to represent subnormal numbers, $x_1$ in base 10 is going to have a slightly different formula. Instead of the mantissa being preceded by a 1, it is instead preceded by a 0. So we have:
                \begin{equation*}
                \begin{split}
                    x_1 &= 0\h4 00000000000 \h4 0000000000000000000000000000000000000000000000000001 \\
                    & = (-1)^0 \cdot 2^{1-1023} \cdot (0 + 2^{-52}) \\
                    & = 2^{-1074}.
                \end{split}
                \end{equation*}
            The next smallest positive double precision number, $x_2$, would be:
                \begin{equation*}
                \begin{split}
                    x_2 & = 
                    0 \h4 00000000000 \h4 0000000000000000000000000000000000000000000000000010 \\
                    & = 2^{-1022}\cdot (0 + 2^{-51}) \\
                    & = 2^{-1073}
                \end{split}
                \end{equation*}
            So:
                \begin{equation*}
                \begin{split}
                    x_1 - x_2 &= 2^{-1074} - 2^{-1073} \\
                    &\approx -4.9406564584124659... \times 10^{-324}
                \end{split}
                \end{equation*}
        \end{solution}
%%%%%%%%%%%%%%%%%%%%%%%%%%%%%%%%%%%%%%%%%%%%%%%%%%%%%%%%%%%%%
    \begin{problem} 
        In class, we saw that the way two binary numbers are added, was that for each bit we employed the logic functions:
            \begin{equation*}
            \begin{split}
                s_i &= x_i \oplus y_i \oplus \text{cin}_i \\
                \text{cout}_i & = (x_i \land y_i) \lor (x_i \land \text{cin}_i) \lor (y_i \land \text{cin}_i).
            \end{split}
            \end{equation*}
        Create the logic functions $s_i(x_i,y_i,\text{bin}_i)$ and $\text{bout}_i (x_i,y_i,\text{bin}_i)$, for performing bit by bit subtractions, where $\text{bin}_i$ and $\text{bout}_i$ are "borrow" bits.
    \end{problem}
        \begin{solution}
            From the table:
            \begin{center}
                \begin{tabular}{c c c | c c}
                    \hline
                    $x$ & $y$ & $b_{in}$ & $s_i$ & $b_{out}$ \\
                    \hline
                    0 & 0 & 0 & 0 & 0 \\
                    0 & 0 & 1 & 1 & 1 \\
                    0 & 1 & 0 & 1 & 1 \\
                    0 & 1 & 1 & 0 & 1 \\
                    1 & 0 & 0 & 1 & 0 \\
                    1 & 0 & 1 & 0 & 0 \\
                    1 & 1 & 0 & 0 & 0 \\
                    1 & 1 & 1 & 1 & 1 \\
                    \hline
                    \end{tabular}
            \end{center}
            We have that $s_i (x_i, y_i, b_{\text{in}}) = x_i \oplus y_i \oplus b_{\text{in}}$ and $b_\text{out}(x_i,y_i,b_\text{in}) = (\overline {x_i} \land y_i) \lor (\overline{x_i} \land b_\text{in}) \lor (y_i \land b_\text{in})$
        \end{solution}
%%%%%%%%%%%%%%%%%%%%%%%%%%%%%%%%%%%%%%%%%%%%%%%%%%%%%%%%%%%%%
    \begin{problem}
        Choose three prime number algorithms to locate all prime numbers from 2 to $n$. Run each of these for values of $n = 1000,2000,5000,10000,20000,50000$. For each algorithm, plot the time $T(n)$ v.s. $n$ on a scatter plot. Comment on the relative efficiency of each of your three algorithms.
    \end{problem}
        \begin{solution}
            I switched to Python, since Mathematica was too slow. Primelist3 was the fastest, followed by primelist2. Primelist1 was much slower.
            \begin{center}
                \includegraphics[scale=.6]{Figure_1}
            \end{center}
            \newpage Here is the code that I used:
                \begin{python}
import time
import matplotlib.pyplot as plt

def primelist1(n):
    if n < 2:
        return
    print(2)
    for j in range(3, n):
        isprime = True
        for i in range(2, j-1):
            if j % i == 0:
                isprime = False
        if isprime:
            print(j)

def primelist2(n):
    if n < 2:
        return
    print(2)
    for j in range(3, n):
        isprime = True
        for i in range(2, int(j**0.5)):
            if j % i == 0:
                isprime = False
        if isprime:
            print(j)

def primelist3(n):
    if n < 2:
        return
    print(2)
    for j in range(3, n):
        isprime = True
        for i in range(2, int(j**0.5)):
            if j % i == 0:
                isprime = False
                break
        if isprime:
            print(j)

n_values = [1000, 2000, 5000, 10000, 20000, 50000]

timingData1 = []
timingData2 = []
timingData3 = []

for n in n_values:
    start = time.time()
    primelist1(n)
    end = time.time()
    timingData1.append(end - start)

    start = time.time()
    primelist2(n)
    end = time.time()
    timingData2.append(end - start)

    start = time.time()
    primelist3(n)
    end = time.time()
    timingData3.append(end - start)

fig, axs = plt.subplots(2, 2, figsize=(12, 10))

axs[0,0].scatter(n_values, timingData1, color='red', marker='o')
axs[0,0].set_title('primelist1')
axs[0,0].set_xlabel('n')
axs[0,0].set_ylabel('T(n) (s)')
axs[0,0].grid(True)

axs[0,1].scatter(n_values, timingData2, color='blue', marker='s')
axs[0,1].set_title('primelist2')
axs[0,1].set_xlabel('n')
axs[0,1].set_ylabel('T(n) (s)')
axs[0,1].grid(True)

axs[1,0].scatter(n_values, timingData3, color='green', marker='^')
axs[1,0].set_title('primelist3')
axs[1,0].set_xlabel('n')
axs[1,0].set_ylabel('T(n) (s)')
axs[1,0].grid(True)

axs[1,1].scatter(n_values, timingData1, color='red', marker='o', label='primelist1')
axs[1,1].scatter(n_values, timingData2, color='blue', marker='s', label='primelist2')
axs[1,1].scatter(n_values, timingData3, color='green', marker='^', label='primelist3')
axs[1,1].set_title('All Primelist Timings')
axs[1,1].set_xlabel('n')
axs[1,1].set_ylabel('T(n) (s)')
axs[1,1].legend()
axs[1,1].grid(True)

plt.tight_layout()
plt.show()
                \end{python}

        \end{solution}
%%%%%%%%%%%%%%%%%%%%%%%%%%%%%%%%%%%%%%%%%%%%%%%%%%%%%%%%%%%%%
    \newpage
    \begin{problem}
        Repeat the question above for three prime number algorithms ...except that now, each $T(n)$ represents the average of times over five similar trials. Create the same three scatter plots as you did above. Do you notice any difference? If so, explain what just happened. 
    \end{problem}
        \begin{solution}
            The averages look very similar to the previous question.
            \begin{center}
                \includegraphics[scale=.6]{Figure_2}
            \end{center}

            I made the following changes to the above code:
            \begin{python}
n_values = [1000, 2000, 5000, 10000, 20000, 50000]

num_trials = 5

for n in n_values:
total_time_1 = 0.0
total_time_2 = 0.0
total_time_3 = 0.0

for _ in range(num_trials):
    start = time.time()
    primelist1(n)
    end = time.time()
    total_time_1 += (end - start)
    
    start = time.time()
    primelist2(n)
    end = time.time()
    total_time_2 += (end - start)
    
    start = time.time()
    primelist3(n)
    end = time.time()
    total_time_3 += (end - start)

avg_time_1 = total_time_1 / num_trials
avg_time_2 = total_time_2 / num_trials
avg_time_3 = total_time_3 / num_trials

timingData1.append(avg_time_1)
timingData2.append(avg_time_2)
timingData3.append(avg_time_3)
            \end{python}
        \end{solution}
%%%%%%%%%%%%%%%%%%%%%%%%%%%%%%%%%%%%%%%%%%%%%%%%%%%%%%%%%%%%%
\end{document}